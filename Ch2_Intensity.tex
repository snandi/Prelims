\chapter{Fluorescence Intensity Profiles}

Show examples of fluorescence intensity profiles—human (MM52), M. florum,  BACs
\begin{itemize}
\item Noise considerations (select a set of noise metrics—explain them, justify them and use them consistently throughout the manuscript) that stem from: stretching (show analysis that links stretching with fluorescence signal information content (this curve may look parabolic…..), amount of dye, staing techniques….How fluorescence signals are measured and pre-processed METHODS SECTION (INCA, etc.)
\item Dealing with noise (Introduction)—Systematic vs. random effects. How a signal consensus making approach deals must deal with types of noise through preprocessing (automated ranking of fluorescence signal QUAILTY (mathematical, morphological analysis): uneven stretching, dispersed stretching within an interval dataset, extraneous noise (“fluorescent garbage,” nearby molecules , crossing molecules, etc.). 
\item The Consensus algorithm- Outline algorithm, describe in detail, pseudocode???……..examples of synthetic data, data comparisons using previously. Explain how variants of this algorithm/analysis have been used in other applications…
\item Preprocessing to improve consensus signal making— Describe both mathematical and computational algorithms and workflow. Comparison and analysis of examples drawn from human, synthetic, M. florum, and BAC data, showing how more confident fluoroscans emerge from considered filtering. Use noise metrics to prove this point.
\end{itemize}


