%\documentclass[10pt,dvipsnames,table, handout]{beamer} % To printout the slides without the animations
\documentclass[10pt,dvipsnames,table]{beamer} 
%\usetheme{Luebeck} 
%\usetheme{Madrid} 
%\usetheme{Marburg} 
%\usetheme{Warsaw} 
\usetheme{CambridgeUS}
%\setbeamercolor{structure}{fg=cyan!90!white}
%\setbeamercolor{normal text}{fg=white, bg=black}

%%%%%%%%%%%%%%%%%%%%%%%%%%%%%%%%%%%%%%%%%%%%%%%%%%%%%%%%%%%%%%%%%%%%%%
%% Input header file 
%%%%%%%%%%%%%%%%%%%%%%%%%%%%%%%%%%%%%%%%%%%%%%%%%%%%%%%%%%%%%%%%%%%%%%
../../../TexScripts/HeaderfileTexSlides.tex

\logo{\includegraphics[scale=0.4]{uwlogo_web_sm_fl_wht.png}}
%\logo{\includegraphics[width=\beamer@sidebarwidth,height=\beamer@headheight]{uwlogo_web_sm_fl_wht.png}}
%%%%%%%%%%%%%%%%%%%%%%%%%%%%%%%%%%%%%%%%%%%%%%%%%%%%%%%%%%%%%%%%%%%%%%
%% TITLE PAGE 
%%%%%%%%%%%%%%%%%%%%%%%%%%%%%%%%%%%%%%%%%%%%%%%%%%%%%%%%%%%%%%%%%%%%%%

\DeclarePairedDelimiter\ceil{\lceil}{\rceil}
\title[Statistics for fluoroscanning]{Statistical methods for analyzing images of fluorochrome stained DNA molecules}
\author{Subhrangshu Nandi}
\institute[Prelim exam]{Preliminary Exam \\
Department of Statistics \\
University of Wisconsin-Madison}
\date{March 1, 2016}

\begin{document}
\setlength{\baselineskip}{16truept}
\setbeamertemplate{logo}{}

\frame{\maketitle}

%%%%%%%%%%%%%% Slide 1 %%%%%%%%%%%%%%
\begin{frame}
\frametitle{Background: Genome sequencing}
Genome sequencing \footnote{http://www.genomenewsnetwork.org/} is figuring out the order of DNA nucleotides, or bases that make up an organism's DNA. The human genome is made up of over {\underline{3 billion}} of these nucleic acids. A DNA sequence that has been translated from life's {\emph{chemical}} alphabet into our alphabet of written letters might look like this:
\begin{figure}[H]
\includegraphics[scale=0.48]{Images/Image_Sequence_1.jpg}
\end{figure}
\pause
\begin{figure}[H]
\includegraphics[scale=0.7]{Images/Image_DNA.jpg} \footnote{http://www.differencebetween.info/difference-between-gene-and-genome}
\end{figure}

\note{}
\end{frame}
%%%%%%%%%%%%%%%%%%%%%%%%%%%%%%%%%%%%%

%%%%%%%%%%%%%% Slide x %%%%%%%%%%%%%%
\begin{frame}
\frametitle{Sequencing Technologies evolving \footnote{\tiny{Wikipedia}} }
\vspace{-1cm}
\begin{columns}[t]
\begin{column}{0.6\textwidth}
\begin{itemize} 

\item {\bf{Pac Bio}}
\begin{itemize}
\item {\footnotesize{Avg 14 kb read length 
\item only 87\% accuracy
\item moderate throughput
\item cost: \_\_\_\_\_ per human genome }}

\end{itemize}

\item {\bf{Ion Torrent Sequencing}} 
\begin{itemize}
\item {\footnotesize{up to 400 bp read length
\item homopolymer errors
\item cost: \_\_\_\_\_ per human genome }}
\end{itemize}

\item {\bf{Illumina sequencing}} 
\begin{itemize}
\item {\footnotesize{up to 300 bp read length
\item very expensive equipment
\item more than 99\% single read accuracy 
\item cost: \_\_\_\_\_ per human genome }}
\end{itemize}

\item {\bf{Sanger sequencing}} 
\begin{itemize}
\item {\footnotesize{up to 900 bp read length
\item more than 99\% single read accuracy 
\item impractical for large sequencing projects
\item cost: \_\_\_\_\_ per human genome }}
\end{itemize}

\end{itemize}
\end{column}

\begin{column}{0.4\textwidth}
\begin{center}
\begin{figure}[t]
\includegraphics[scale=0.35]{Images/sequencingtechnology.pdf} \footnote{\tiny{https://www.broadinstitute.org/ \\blog/celebrating-fruits-human-genome-sequence}}
\end{figure}
\end{center}
%{\bf{Pac Bio}} Avg 14 kb read length; \textcolor{red}{only 87\% accuracy; moderate throughput} \\
%{\bf{Ion Torrent Sequencing}} \textcolor{red}{up to 400 bp read length; \$32 mil for 1 genome}

\begin{itemize}
{\small{
\item Human genome: 32 billion bp

\item Reference genome is incomplete
 
\item Need a parsimonious system to analyze the human population
}}
\end{itemize}
\end{column}

\end{columns}
\note{}
\end{frame}
%%%%%%%%%%%%%%%%%%%%%%%%%%%%%%%%%%%%%

%%%%%%%%%%%%%% Slide x %%%%%%%%%%%%%%
\begin{frame}
\frametitle{Motivation: mp3 representation of a Genome}
\vspace{-0.8cm}
\begin{columns}[t]
\begin{column}{0.6\textwidth}
\begin{itemize}
\item Lower resolution: of the order of 400 - 600 bp
\item Simple to obtain
\item Contains the important structural information about a genome
\end{itemize}
\end{column}

\begin{column}{0.4\textwidth}
\begin{figure}[t]
\includegraphics[scale=0.1]{Images/mp3_wav.jpg} \footnote{\tiny{http://coderzen.blogspot.com/ \\ 2014/03/wav-file-vs-mp3-file.html}}
\end{figure}

\end{column}

\end{columns}
\begin{figure}[H]
\includegraphics[scale=0.6]{Plots/chr13_frag1000_consensusOnly.pdf}
\end{figure}

\note{}
\end{frame}
%%%%%%%%%%%%%%%%%%%%%%%%%%%%%%%%%%%%%

%%%%%%%%%%%%%% Slide x %%%%%%%%%%%%%%
\begin{frame}
\frametitle{Background: Genome sequencing}

\note{}
\end{frame}
%%%%%%%%%%%%%%%%%%%%%%%%%%%%%%%%%%%%%







%%%%%%%%%%%%%% Slide 1 %%%%%%%%%%%%%%
\begin{frame}
\frametitle{Iterated Registration}
Similarity index:
\[ \rho(f_i, f_j) = \frac{\int _{S_{ij}}f'_i(s)f'_j(s) ds}{\int _{S_{ij}}f'_i(s)^2 ds \int _{S_{ij}}f'_j(s)^2 ds} \]
where, $S_{ij} = S_i \cap S_j$, the intersection of Sobolev spaces formed by the functions $f_i, f_j$.\\

Let $\phi$ denote the consensus intensity profile for a given set of curves. The objective is to maximize the average similarity to the consensus, i.e., maximize
\[ \bar{\rho}_{\phi, N} = \frac{1}{N} \sum \limits_{j = 1}^{N} \rho(\phi, f_j)\]
Stop iteration when $|\bar{\rho}_{\phi, N (t+1)} - \bar{\rho}_{\phi, N (t)} | < \epsilon$
\end{frame}


%%%%%%%%%%%%%% Slide 4 %%%%%%%%%%%%%%
\begin{frame}
\frametitle{Simultaneous Registration and Clustering}
\begin{enumerate}
\item Eliminate outliers by using ``data-depth''
\item Check if multiple clusters is better than one $\bar{\rho}_{\{\phi_1, \phi_2\}, N} > \bar{\rho}_{\phi, N}$
\item Model based, or k-means to get initial clusters, based on pairwise distances between curves
\item Estimate the medians, and ajust the clusters based on distance between medians and cluster members
\item Register to the median, re-evaluate the distances between medians and cluster members
\item Iterate until not much improvement in overall similarity index
\item Goal: To maximize $\bar{\rho}_{\{\phi_1, \phi_2\}, N}$
\end{enumerate}
\end{frame}

%%%%%%%%%%%%%% Slide 5 %%%%%%%%%%%%%%
\begin{frame}
\frametitle{Functional Data Depth}
The univariate depth of the point $x_i(t)$ is given by
\[ D_n(x_i(t)) = 1 - \left| \frac{1}{2} - F_{n,t} (x_i (t)) \right| \]
where, $F_{n,t} (x_i (t)) = \frac{1}{n}\sum\limits_{k=1}^{n}\Ind\{ x_k(t) \leq x_i(t)\}$ is the empirical cdf at $t \in [a,b]$ \\
Then, Fraiman and Muniz functional data depth is given by
\[ \text{FMD}_n(x_i) = \int\limits_a^b D_n(x_i(t)) dt\] \\
To detect outliers:
\begin{itemize}
\item Obtain the functional depths $\text{FMD}_n(x_1), \dots, \text{FMD}_n(x_n)$
\item Let $x_{i_1}, \dots, x_{i_k}$ be the $k$ curves such that $\text{FMD}_n (x_{i_k}) \leq C$, for a given cutoff $C$. Then, assume that $x_{i_1}, \dots, x_{i_k}$ are outliers and delete them from the sample.
\item Repeat until no more outliers
\item {\emph{But, how do you estimate the cutoff $C$}}
\end{itemize}
\end{frame}

%%%%%%%%%%%%%% Slide 6 %%%%%%%%%%%%%%
\begin{frame}
\frametitle{Outliers using Functional Data Depth}
\begin{itemize}
\item Obtain the functional depths $\text{FMD}_n(x_1), \dots, \text{FMD}_n(x_n)$
\item Obtain $B$ standard bootstrap samples of size $n$ from the dataset of curves obtained after deleting the $\alpha \%$ less deepest curves. The bootstrap samples are denoted by $x_i^b$, for $i = 1, \dots, n$ and $b = 1, \dots, B$.
\item For each bootstrap set $b = 1, \dots, B$, obtain $C^b$ as the empirical $1\%$ percentile of the distribution of the depths, $D(x_i^b), i = 1, \dots, n$.
\item Take $C$ as the median values of $C^b, b = 1, \dots, B$.
\end{itemize}
\end{frame}


\end{document}

