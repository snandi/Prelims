\documentclass[10pt,dvipsnames,table]{beamer} 
%\usetheme{Luebeck} 
%\usetheme{Madrid} 
%\usetheme{Marburg} 
%\usetheme{Warsaw} 
\usetheme{CambridgeUS}
%\setbeamercolor{structure}{fg=cyan!90!white}
%\setbeamercolor{normal text}{fg=white, bg=black}
\setbeamercolor{block title}{bg=red!80,fg=white}

%%%%%%%%%%%%%%%%%%%%%%%%%%%%%%%%%%%%%%%%%%%%%%%%%%%%%%%%%%%%%%%%%%%%%%
%% Input header file 
%%%%%%%%%%%%%%%%%%%%%%%%%%%%%%%%%%%%%%%%%%%%%%%%%%%%%%%%%%%%%%%%%%%%%%
../../../TexScripts/HeaderfileTexSlides.tex

\logo{\includegraphics[scale=0.4]{uwlogo_web_sm_fl_wht.png}}
%\logo{\includegraphics[width=\beamer@sidebarwidth,height=\beamer@headheight]{uwlogo_web_sm_fl_wht.png}}
%%%%%%%%%%%%%%%%%%%%%%%%%%%%%%%%%%%%%%%%%%%%%%%%%%%%%%%%%%%%%%%%%%%%%%
%% TITLE PAGE 
%%%%%%%%%%%%%%%%%%%%%%%%%%%%%%%%%%%%%%%%%%%%%%%%%%%%%%%%%%%%%%%%%%%%%%

\DeclarePairedDelimiter\ceil{\lceil}{\rceil}
\title[Alignment]{Alignment of molecules in nanocoding}
\author{Subhrangshu Nandi}
\institute[Prelim exam]{Preliminary Exam \\
Department of Statistics \\
University of Wisconsin-Madison}
\date{March 1, 2016}

\begin{document}
\setlength{\baselineskip}{16truept}
\setbeamertemplate{logo}{}

%\frame{\maketitle}

%%%%%%%%%%%%%% Slide x %%%%%%%%%%%%%%
\begin{frame}
\frametitle{Nanocoding - alignment of molecules to reference}
\vspace{-1cm}
\begin{figure}[t]
\includegraphics[scale=0.6]{Images/NMaps_Aligned.png}
\hspace{1cm}
\includegraphics[scale=0.2]{Images/BarcodeImage.jpg}
\caption{Alignment of molecules to reference \footnote{Kounovsky, et al, 2013}}
\end{figure}

{\emph{M. florum}} reference map: \\
{\footnotesize{81.6 18.7 59.4 13.9 9.0 5.0 12.3 10.2 15.0 25.4 3.9 20.9 15.6 10.2 9.5 11.1 4.5 
13.7 26.3 38.3 2.1 31.0 19.1 3.6 32.2 41.3 9.8}} \textcolor{Red}{16.4 15.7 6.3 36.5 18.3} {\footnotesize{32.1 21.0 
3.3 14.3 51.3 16.0 17.9}} \\
\vspace{0.5cm}
Molecule {\small{$2064486\_0\_11$}}: {\bf{31.24 \textcolor{Red}{16.80  15.77  6.25  35.08  18.12}}}

\note{
The sequence of lengths between punctates is called a nanocode, akin to barcode\\
It is unique to molecule, and it can be used to determine which location on the genome the molecule is from\\
Example, below is a reference map from a bacteria genome, and a nanocoded molecule
}

\end{frame}
%%%%%%%%%%%%%%%%%%%%%%%%%%%%%%%%%%%%%

\end{document}
