% abstract.tex
%
% This file has the abstract for the withesis style documentation
%
% Eric Benedict, Aug 2000
%
% It is provided without warranty on an AS IS basis.

\noindent       % Don't indent this paragraph.
The Human Genome Project (HGP), completed in 2003, is considered one of the greatest accomplishments of exploration in history of science. Since then thousands of genomes have been sequenced. However, no individual human genome has been annotated to completion. Nanocoding \cite{Jo_etal_2007_PNAS} (PNAS, 2007), developed by Laboratory of Molecular and Computational Genomics (LMCG), UW Madison , is a novel system for physically mapping genomes, using measurements of single DNA molecules to construct a high-resolution genome-wide restriction map, whose representation of genome structure complements genome sequences to yield biological insight. Staining the DNA molecules with cyanine dyes and imaging them is a critical step of nanocoding. It turns out that the quantum yield of the fluorescence intensity of these stained molecules are sequence dependent. In fact, for YO complexed with GC-rich DNA sequences the quantum yield are about twice as large as for YO complexed with AT-rich sequences. Hence, regions with distinct sequence compositions should exhibit unique fluorescence intensity profiles. Establishing the fluorescence intensity profiles of a genome would provide invaluable insights into its sequence compositions without having to sequence it. We name this technique ``Fluoroscanning''. Imaged DNA molecules from the same region on a genome should exhibit similar intensity profiles, unless there has been a modification in the underlying genomic sequence. Fluoroscanning can be used to identify heterozygotes and detect large scale structural variations as a result of cancer or other diseases.  

%\vspace*{0.5em}
%\noindent       % Don't indent this paragraph.

