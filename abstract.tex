% abstract.tex
%
% This file has the abstract for the withesis style documentation
%
% Eric Benedict, Aug 2000
%
% It is provided without warranty on an AS IS basis.

\noindent       % Don't indent this paragraph.
The Human Genome Project (HGP), completed in 2003, is considered one of the greatest accomplishments of exploration in history of science. Since then thousands of genomes have been sequenced. However, no individual human genome has been annotated to completion. Nanocoding \cite{Jo_etal_2007_PNAS} (PNAS, 2007), developed by Laboratory of Molecular and Computational Genomics (LMCG), UW Madison, is a novel system for physically mapping genomes. It uses measurements of single DNA molecules to construct a high-resolution genome-wide restriction map, whose representation of genome structure complements genome sequences to yield biological insight. Staining the DNA molecules with cyanine dyes and imaging them are essential procedural requirements of nanocoding. The quantum yield of the fluorescence intensity of these stained molecules are DNA sequence dependent. In fact, for dye complexed with GC-rich DNA sequences the quantum yield are about twice as large as for those complexed with AT-rich sequences. Hence, regions with distinct genomic sequence compositions should exhibit unique fluorescence intensity profiles. Establishing the fluorescence intensity profiles of a genome would provide invaluable insights into its sequence compositions without having to sequence it. This is akin to producing a mp3 version of a wave file, an approximation of the true sequence, yet preserving the salient characteristics. This newly discovered technique is named ``Fluoroscanning''. Fluoroscanning can be applied to analyzing whole genomes. For example, imaged DNA molecules from the same region on a genome should exhibit similar intensity profiles, unless there has been a modification in the underlying genomic sequence. Fluoroscanning can also be used to identify heterozygotes and detect large scale structural variations as a result of cancer or other diseases. In the quest of establishing fluoroscanning as the next generation precision genomics we need to address some interesting statistical and computational challenges. In this thesis we apply some aspects of functional data analysis, like curve registration, to establish reproducibility and uniqueness of fluorescence intensity profiles. We also show how fluroscanning can detect structural variations and heterozygotes in a dataset collected from a cancer genome.

%\vspace*{0.5em}
%\noindent       % Don't indent this paragraph.

