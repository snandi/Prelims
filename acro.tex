% acrobat.tex
%
% This file explains how to generate Adobe Acrobat files
%
% Eric Benedict, July 2000
%
% It is provided without warranty on an AS IS basis.

\newcommand{\pdf}{\mbox{\tt *.pdf}}

\chapter{Adobe Acrobat (\pdf ) Files}
The Adobe Acrobat file format has pretty much become the {\em de facto}
standard for document sharing.  As such, some faculty members and/or
departments may be requiring a final copy of the thesis in Acrobat format
(\pdf ).

There are several different methods of obtaining a \pdf\ file from a \LaTeX{}
thesis; however, they are all very site specific.  A couple of different
methods which have been found to work are mentioned as suggested ideas to try
as a starting point.  Depending on what is installed at your site/location some
of these may be applicable.

\section{Converting from {\tt *.ps} to \pdf}
One option to obtain the \pdf\ file would be to generate the thesis in a normal
manner and then use the Acrobat\ {\tt Distiller}\ to convert the postscript file into a
\pdf\ file.

If the\ {\tt Distiller}\ program is available and convenient to use, then this is quite easy to do.

Depending on the choice of document fonts, the results may not be satisfactory since
some of the fonts may end up as bit-mapped fonts and will display poorly at any resolution
other than what they were sampled on.  Also, since the\ {\tt Distiller}\ program is an expensive
program to obtain, it is not always available.

An alternative to the Adobe\ {\tt Distiller}\ program is the Alladin\ {\tt Ghostscript}\ program.  This is
available for free from

{\tt \verb|   http://www.cs.wisc.edu/~ghost/index.html|}

This program is available for most common operating systems as a compiled binary, but the source code
is available for other systems.  One drawback is that this conversion must be performed as a
command line invocation and isn't very user friendly.  This may be addressed in a future version of\
{\tt Ghostview}, the program which provides a nice user interface to\ {\tt Ghostscript}.

\section{Converting from {\tt *.dvi} to \pdf}
There are two programs available which will convert from {\tt *.dvi} to \pdf,\ {\tt dvipdf}\ and\
{\tt dvipdfm}.  The\ {\tt dvipdfm}\ program  will be discussed here.  In version 0.12, it can generate
bookmarks, thumbnails (with assistance from\ {\tt Ghostscript}), scaling and rotation, JPEG and
PNG bitmaps and font encoding and re-encoding (to support fonts which aren't fully supported by
the Acrobat suite).  When\ {\tt Ghostscript}\ is properly installed,\ {\tt dvipdfm}\ will automatically
convert any encapsulated PostScript figures into the required \pdf{} format.  This program behaved in a
similar manner to the {\tt dvips} program and was used to produce the \pdf{} format of this document.



\section{Generating \pdf{} Initially}
There are now some programs which are similar to \TeX{} but instead of producing a\ {\tt .dvi}\ output,
they produce \pdf as a native output.  One such program, {\sc pdf}\TeX{} / {\sc pdf}\LaTeX{},
is available from

{\tt \verb|   http://www.tug.org/applications/pdftex|}

Note that as of this date, July 2000, {\sc pdf}\TeX{} / {\sc pdf}\LaTeX{} while currently quite usable, it
is still in a beta version.  Look at the web site for more current information.

The present version was able to produce a \pdf{} file of this document without any required
changes, except for the Postscript figure inclusion  (Figure~\ref{vwcontr}).  To properly include
this figure, requires the conversion of the postscript figure into a \pdf{} figure.  The procedure
is described in the manual for {\sc pdf}\TeX{} / {\sc pdf}\LaTeX{}.  Note that the figure conversion will
require either\ {\tt Distiller}\ or\ {\tt Ghostscript}.
