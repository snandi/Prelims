\chapter{Statistical Methodology}

\section{Functional Data Analysis}
\noindent
{\bf{Definition 3.1}} {\emph{A random variable $\Y$ is called a functional variable if it takes values in an infinite-dimensional space. An observation $Y$ of $\Y$ is called a functional data. \cite{Ferraty_Vieu_2006_Nonparametric}}}\\
\noindent
{\bf{Definition 3.2}} {\emph{A functional dataset $Y_1, \dots, Y_n$ is an observation of $n$ functional variables $\Y_1, \dots, \Y_n$ distributed as $\Y$.}}\\

Functional data (FD) are discrete observations of a continuous underlying process. They arise in many different scenarios. For example, children growing taller can be considered a continuous process, but recording their heights every few months is a set of discrete observation of that process. Pioneered by Ramsay and Silvermann \cite{Ramsay_2006_Functional}, there has been substantial progress in statistical approaches to analyzing FD in recent past. Following are some important aspects of functional data analysis (FDA) that are relevant to our dataset, pertaining to the goal of our research. 

\subsection{Smoothing}

\subsection{Amplitude and Phase variability}

\subsection{Registration}

\section{Curve Registration}
\subsection{Ramsay-Silverman}
\subsection{Sangalli, et al}
\subsection{Srivastava, et al}

\section{Forumlation of iterated registration}

\section{Simulation Results}


