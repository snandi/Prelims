%=====================================================================
% Document Style
%=====================================================================
% Choose only one of the following document classes:
%
% for a 12 Point UW PhD Thesis without Margin Check
\documentclass[12pt]{article}
%
%%%%%%%%%%%%%%%%%%%%%%%%%%%%%%%%%%%%%%%%%%%%%%%%%%%%%%%%%%%%%%%%%%%%%%
%% Usepackages
%%%%%%%%%%%%%%%%%%%%%%%%%%%%%%%%%%%%%%%%%%%%%%%%%%%%%%%%%%%%%%%%%%%%%%
../../TexScripts/HeaderfileTexDocs.tex

%%%%%%%%%%%%%%%%%%% To change the margins and stuff %%%%%%%%%%%%%%%%%%%
\geometry{left=1.0in, right=1.0in, top=1.0in, bottom=1.0in}
%\setlength{\voffset}{0.5in}
%\setlength{\hoffset}{-0.4in}
%\setlength{\textwidth}{7.6in}
%\setlength{\textheight}{10in}
%%%%%%%%%%%%%%%%%%% To change the margins and stuff %%%%%%%%%%%%%%%%%%%

\begin{document}

\title{Statistical methods for analyzing images of fluorochrome stained DNA molecules}
\author{Subhrangshu Nandi}
\date{Mar 1, 2016}

\maketitle

\begin{abstract}
One of the most important components of precision medicine is analysis and use of an individual's genomic information and development of targeted intervention, tailored to the individual. This will entail building systems that will allow researchers and practitioners to extract important information from a genome, in a parsimonious manner. Next generation sequencing (NGS) systems are enhancing molecular biology in the right direction, but show little promise towards sequencing every human being's genome. In addition, NGS still have difficulty inferring repetitive structures and duplications \cite{Lander_etal_2001_Nature}, further complicated by gaps and errors in the reference genome. Pioneered by laboratory of computational genomics (LMCG) at UW-Madison, {\emph{Fluoroscanning}} is a single-molecule-based whole genome analysis system that can glean extract important information out of a genome in a swift, economical manner. The aims of this project are developing statistical methods for fluoroscanning - analysis of intensity profiles measured from imaged fluorochrome stained DNA molecules. Aim 1 is to establish that fluorescence intensity profiles measured at different regions have discernible differences if the underlying sequence compositions are distinct. Aim 2 is to detect two sets of fluorescence intensity profiles in heterozygous genomic loci. These analyses will be performed on experiments conducted at LMCG, on both ${\emph{M. florum}}$ and a human genome. 
\end{abstract}

% Choose your bibliography style
% plain is the basic style, others include ieeetr, siam, asm, etc
\bibliographystyle{apalike}

\newpage
%% prelude.tex
%   - titlepage
%   - dedication
%   - acknowledgments
%   - table of contents, list of tables and list of figures
%   - nomenclature
%   - abstract
%============================================================================


\clearpage\pagenumbering{roman}  % This makes the page numbers Roman (i, ii, etc)


% TITLE PAGE
%   - define \title{} \author{} \date{}
\title{Statistical framework in the discovery of Fluoroscanning \\ (the next generation precision genomics)}
\author{Subhrangshu Nandi}
\date{February 18, 2016}
%   - The default degree is ``Doctor of Philosophy''
%     (unless the document style msthesis is specified
%      and then the default degree is ``Master of Science'')
%     Degree can be changed using the command \degree{}
\degree{Doctor of Philosophy}
%   - The default is dissertation, unless the document style
%     msthesis was specified in which case it becomes thesis.
%     If msthesis is specified for the MS margins, you can
%     still have a dissertation if you specify \disseration
%\disseration
%   - for a masters project report, specify \project
%\project
%   - for a preliminary report, specify \prelim
\prelim
%   - for a masters thesis, specify \thesis
%\thesis
%   - The default department is ``Electrical Engineering''
%     The department can be changed using the command \department{}
\department{Statistics}
%   - once the above are defined, use \maketitle to generate the titlepage
\maketitle

% COPYRIGHT PAGE
%   - To include a copyright page use \copyrightpage
\copyrightpage

% DEDICATION
\begin{dedication}
TBD
\end{dedication}

% ACKNOWLEDGMENTS
\begin{acknowledgments}
I thank the many people who have done lots of nice things for me.
\end{acknowledgments}

% CONTENTS, TABLES, FIGURES
\tableofcontents
\listoftables
\listoffigures

% NOMENCLATURE
%% \begin{nomenclature}
%% \begin{description}
%% \item{\makebox[0.75in][l]{\TeX}}
%%        \parbox[t]{5in}{a typesetting system by Donald Knuth~\cite{knuth}.  It
%%        also refers to the ``plain'' format.  The proper pronounciation
%%        rhymes with ``heck'' and ``peck'' and does not sound like
%%        ``hex'' or ``Rex.''\\}

%% \item{\makebox[0.75in][l]{\LaTeX}}  
%%         \parbox[t]{5in}{a set of \TeX{} macros originally written by Leslie 
%%         Lamport~\cite{lamport}.  The proper pronunciation is 
%%         {\tt l\={a}$\cdot$tek'} and not {\tt l\={a}'$\cdot$teks} (see above).\\}

%% \item{\makebox[0.75in][l]{{\sc Bib}\TeX}} 
%%          \parbox[t]{5in}{a bibliography generation program by Oren 
%%                 Patashnik~\cite{lamport}
%%                 that can be used with either plain \TeX{} or \LaTeX{}.\\}

%% \item{\makebox[0.75in][l]{$C_1$}} Constant 1

%% \item{\makebox[0.75in][l]{$V$}}    Voltage 

%% \item{\makebox[0.75in][l]{\$}}     US Dollars
%% \end{description}
%% \end{nomenclature}


\advisorname{Michael A. Newton}
\advisortitle{Professor}
% ABSTRACT
\begin{umiabstract}
  Write up from Professor Newton

\end{umiabstract}

\begin{abstract}
  % abstract.tex
%
% This file has the abstract for the withesis style documentation
%
% Eric Benedict, Aug 2000
%
% It is provided without warranty on an AS IS basis.

\noindent       % Don't indent this paragraph.
The Human Genome Project (HGP), completed in 2003, is considered one of the greatest accomplishments of exploration in history of science. Since then thousands of genomes have been sequenced. However, no individual human genome has been annotated to completion. Nanocoding \cite{Jo_etal_2007_PNAS} (PNAS, 2007), developed by Laboratory of Molecular and Computational Genomics (LMCG), UW Madison , is a novel system for physically mapping genomes, using measurements of single DNA molecules to construct a high-resolution genome-wide restriction map, whose representation of genome structure complements genome sequences to yield biological insight. Staining the DNA molecules with cyanine dyes and imaging them is a critical step of nanocoding. It turns out that the quantum yield of the fluorescence intensity of these stained molecules are sequence dependent. In fact, for YO complexed with GC-rich DNA sequences the quantum yield are about twice as large as for YO complexed with AT-rich sequences. Hence, regions with distinct sequence compositions should exhibit unique fluorescence intensity profiles. Establishing the fluorescence intensity profiles of a genome would provide invaluable insights into its sequence compositions without having to sequence it. We name this technique ``Fluoroscanning''. Imaged DNA molecules from the same region on a genome should exhibit similar intensity profiles, unless there has been a modification in the underlying genomic sequence. Fluoroscanning can be used to identify heterozygotes and detect large scale structural variations as a result of cancer or other diseases.  

%\vspace*{0.5em}
%\noindent       % Don't indent this paragraph.


\end{abstract}


\clearpage\pagenumbering{arabic} % This makes the page numbers Arabic (1, 2, etc)
                % Title page, abstract, table of contents, etc
\section{Introduction}

\subsection{Motivation}
The Human Genome Project (HGP), completed in 2003, is considered one of the greatest accomplishments of exploration in history of science. Since then thousands of genomes have been sequenced. However, no individual human genome has been annotated to completion. DNA sequencing-based genomic analysis continue to evolve, but their abilities to detect large scale structural variations, or heterozygosity in diploid genomes, remain limited. Next generation sequencing (NGS) is considerable more cost effective, with longer reads, but still have difficulty inferring repetitive structures and duplications \cite{Lander_etal_2001_Nature}, further complicated by gaps and errors in the reference genome. NGS also has inferior sensitivity of detecting heterozygotes \cite{Wheeler_etal_2008_Nature}. In addition, the sheer size of the diploid human genome (6 gigabases) presents multiple challenges of just using NGS technologies to analyze its complexities. The analysis of cancer genomes is made even more complex by the accumulation of large and small-scale structural variations (SVs), and genotype heterogeneity fostered by on-going mutagenesis processes, especially apparent in in solid tumor. The sequencing technologies currently being used were developed  primarily for characterization of single genes, not entire genomes and, as such, are not ideal to analyze polygenic diseases, complex trait inheritance, and population-based molecular genetics \cite{Samad_etal_1995_GenomeResearch}. Given the current need for comprehensively analyzed human and cancer genomes that are readily created, contemporary sequencing and mapping approaches are insufficient to meet these challenges. In order to achieve our dreams of improving healthcare by precision genomics we need techniques that can overcome the shortcomings of NGS, yet, maintaining economic viability. The answer lies in the latest developments of single molecule genome mapping techniques like optical mapping and nanocoding. 

\subsection{Background}
\subsubsection*{Optical Mapping}
Pioneered by LMCG, single molecule genome mapping techniques like optical mapping \cite{Schwartz_etal_1993_Science}, \cite{Dimalanta_etal_2004_AnalChem}, and nanocoding \cite{Jo_etal_2007_PNAS} have changed the landscape of whole genome analysis. Optical Mapping is a novel platform for analyzing genomes: it uses measurements of single DNA molecules to infer a high-resolution genome-wide restriction map, whose representation of genome structure complements genome sequence to yield biological insight. Briefly, DNA from thousands of cells in solution is randomly sheared to produce pieces that are around 500 Kb long. The solution is then passed through a micro-channel, where the DNA molecules are stretched and then attached to a positively charged glass support. A restriction enzyme is then applied, cleaving the DNA at corresponding restriction sites. The DNA molecules remain attached to the surface. The surface is photographed under a microscope after being stained with a fluorochrome. The cleavage sites show up in the image as tiny gaps in the fluorescent line of the molecule, giving an snapshot of the full restriction map. Even though these molecules are large by many standards, they may still represent only a small fraction of the chromosome they come from. Naturally, the amount of information in an optical map data set is related to the size of the underlying genome. 

Information about genomic variation can thus be obtained from these restriction maps, that do not record the full nucleotide sequence. A physical map is a listing of the locations along the genome where certain markers occur. A restriction map is a physical map induced by restriction enzymes. The ordered sequence of distances in base pairs between successive marker positions summarizes the genome sequence and can be viewed as a sort of bar code of the genome. Genomic differences can affect the presence or absence of markers, the distances between them and their orientation, inducing analogous changes in the bar code. Having a reference copy of the human genome allows us to perform such {\emph{in silico}} experiments. The availability of in silico reference maps can be extremely helpful.

\subsubsection*{Nanocoding}
Nanocoding \cite{Jo_etal_2007_PNAS}, \cite{Jo_etal_2009}, also invented in LMCG, is a more advanced DNA barcoding technique, to obtain genome-wide restriction maps. In nanocoding, the restriction enzyme used is Nb.BbvCI, which cleaves the sequence GC\^\ TGAGG on single strands of double-stranded DNA molecules. The cleaved sites are made detectable by nick translation using fluorochrome-labelled nucleotides. The cleaved sites are called punctates. The biggest advantage over optical mapping is being able to retain long DNA molecules intact and hence improving labeling efficiency. DNA molecules are stained with an intercalating dye YOYO-1. It is a green fluorescent dye which belongs to the family of monomethine cyanine dyes and is a tetracationic homodimer of Oxazole Yellow (abbreviated YO, hence the name YOYO). The labelled molecules are imaged by microscopes equipped with two CCD cameras, that have two filtering optics for green and red colors. The green channel acquires images of DNA backbone stained with YOYO-1, and the red channel acquires images of sequence-specific decorations of Alexa Fluor 547 punctuates via fluorescence resonance energy transfer (7). The restriction maps are obtained by meticulous image processing software that analyze the images from the green and red channels, to measure the lengths between two consecutive punctates.  \\

\subsubsection*{Dye - DNA interaction} 
The dye that is used to stain the DNA molecules is a tetracationic homodimer of Oxazole Yellow (abbreviated YO, hence the name YOYO). It is a green fluorescent dye which belongs to the family of monomethine cyanine dyes. Cyanine dyes have a rich history in the photographic industry. Cationic  cyanine dyes  exhibit very large degrees of fluorescence enhancement on binding to nucleic acids \cite{Rye_etal_1992_NAR}, \cite{Lee_etal_1986_Cytometry}. These  characteristics of fluorescence enhancement and  high binding affinity are crucial for nanocoding. Netzel, et al, 1995 \cite{Netzel_etal_1995_JPC} observe that there are differences in emission quantum yield between dyes with pyridinium and quinolinium structural components, such as YOYO-1, when bound to $(\text{dAdT})_{10}$ and $(\text{dGdC})_6$ duplexes. In fact, they observe a 2-fold quantum yield increase when switching from AT-rich regions to GC-rich regions. Larsson, et al, 1994, (\cite{Larsson_etal_1994_JACS}) also observe that the fluorescence intensity of YO depends on the base sequence. In fact, their observation suggests that the quantum yield and fluorescence lifetime for YO complexed with GC-rich DNA sequences are about twice as large as for YO complexed with AT-rich sequences. Netzel, et al, 1995 (10) also note that one of the reasons that could differentially alter emission enhancements for cyanine dyes bound to DNA  duplexes is excited-state electron transfer quenching by DNA nucleotides. They argue that Guanosine is the easiest nucleotide to oxidize, while thymidine and cytidine are the easiest nucleotides to reduce. Adenosine is difficult to oxidize and is also the nucleotide least easy to reduce. These chemical properties could be responsible for higher fluorescence intensity observed from GC-rich regions compared to AT-rich ones. 

\subsection{Discovery of a new effect - Fluoroscanning}
In nanocoding, the microscope, equipped with CCD cameras, capture the images of fluorescent DNA molecules, dyed with YOYO-1. From the physical and chemical properties of the differential interaction between the dye molecules and the nucleotides mentioned above, it is clear that the fluorescence intensity profiles of the DNA molecules will be strongly dependent on the underlying nucleotide sequence composition. In fact, DNA molecules from the same region on the genome should have similar fluorescence intensity profiles. In addition, different regions on the genome, with discernible differences between their sequence compositions, should exhibit discernible difference in their corresponding fluorescence intensity profiles. This novel technique of inferring about genomic sequence compositions from analyzing fluorescent intensity profiles of imaged DNA molecules is termed {\bf{Fluoroscanning}}. \\

This dissertation addresses the statistical and computational challenges associated with this newly discovered effect, fluoroscanning. 


              % Chapter 1
\section{Data Structure} \label{Ch2}

\subsection{Fluorescence Intensity Profiles}
An integrated image acquisition and machine vision system was developed in LMCG to automatically detect and analyze molecule within the collected image data. Details of the image processing steps can be found in \cite{Dimalanta_etal_2004_AnalChem}, \cite{Jo_etal_2007_PNAS}, \cite{Ravindran_Gupta_2015_GigaScience}. INCA (not published) is machine vision software developed by Dr. Prabu Ravindran that considers two color, Nanocoding image data-sets and extracts restriction maps from individual molecules to create larges files of Nmaps (1 Nmap = one restriction mapped DNA molecule). Fig \ref{fig:Fig2_FrameImage} below shows images of multiple straight DNA molecules which are stained with the YOYO-I dye which fluoresce when excited by the laser-illuminated microscope.

\begin{figure}[H]
\begin{center}
%\includegraphics[width=0.55\textwidth, bb=0 0 700 350]{Images/Image4_FrameImage.png}
\includegraphics[scale = 0.45]{Images/Image4_FrameImage.pdf}
\end{center}
\caption{Image of a surface with stained DNA molecules}
\label{fig:Fig2_FrameImage}
\end{figure}

The restriction maps (named Nmaps, for ``nanocoded'' maps) are obtained by meticulous image processing software that analyze the images, to measure the lengths between two consecutive punctuates. 
\begin{figure}[H]
\begin{center}
\includegraphics[scale=0.8]{Images/Image5_NMap.png}
\end{center}
\caption{5 DNA molecules aligned to reference \cite{Kounovsky_etal_2013_Macromolecules}}
\label{fig:Fig2_NMap}
\end{figure}

These Nmaps are used to align the molecules to an {\emph{in silico}} reference map of a reference genome. Fig \ref{fig:Fig2_NMap} shows how five DNA molecules are nanocoded and aligned to a reference genome. Notice that intervals 5 and 6 (from the left) of the reference genome have a depth of 5, i.e., 5 molecular fragments are aligned to that interval. However, interval 4 has a depth of 4. \\

\subsection{Data description} \label{Ch2_data}
Throughout this thesis, we will use images collected during nanocoding {\emph{Mesoplasma florum (M. florum)}} genome and a multiple myeloma (MM) genome. Below are brief data descriptions of the two data-sets.
\subsubsection*{\emph{Mesoplasma florum (M. florum)}}
{\emph{M. florums}} are members of the class Mollicutes, a large group of bacteria that lack a cell wall and have a characteristically low GC content (\cite{Razin_etal_1998_MMBR}). These diverse organisms are parasites in a wide range of hosts, including humans, animals, insects, plants, and cells grown in tissue culture (\cite{Razin_etal_1998_MMBR}). Aside from their role as potential pathogens, {\emph{M. florums}} are of interest because of their extremely small genome size. The {\emph{M. florum's}} NMap consists of 39 intervals. This implies, the reference genome has 38 restriction sites. The whole genome is approximately 793 kb long. The intervals are between 2.111 kb and 81.621 kb. 

One natural consequence of fluoroscanning is that the fluorescence intensity profiles of molecular fragments aligned to the same reference interval will have similar features, since the underlying sequence composition is the same. This will be tested as part of Aim 1 \ref{Aims}. 

\begin{figure}[H]
\begin{center}
\includegraphics[scale=0.42,page=1]{Plots/MF_Frag15.pdf}
\includegraphics[scale=0.42,page=2]{Plots/MF_Frag15.pdf}
\end{center}
\caption{NMaps aligned to {\emph{M. florum}} Interval 15}
\label{fig:Fig2_MF_Frag15}
\end{figure}

Fig \ref{fig:Fig2_MF_Frag15} shows fluorescence intensity profiles of fragments of 12 DNA molecules that have been aligned to interval 15 of {\emph{M. florum}} genome. The interval is 11.119kb long and each pixel of the captured images correspond to around 209 base pairs on the genome. So, under perfect experimental conditions, each of these fragments should be 53 ($\frac{11119}{209} = 53.2$) pixels long. However, due to several reasons the molecule lengths do not perfectly align with that of the reference. As the dye molecules intercalate between the bases of the DNA molecules, there could be local deformation of the molecules, resulting in more dye molecules to seep in. This could result in slight elongation of the DNA molecule and hence it captures more pixels in the image. This phenomenon of ``stretching'' of molecules is hard to control experimentally, as it involves meticulous measurement of dye and DNA molecule concentrations. Furthermore, when one end of a negatively charged DNA molecule attaches to the positively charged glass surface and the rest of the molecule elongates in the direction of the analyte flow, it also causes differential stretching. 

In the {\emph{M. florum}} data-set, we have molecular fragments aligned to 39 intervals on the reference genome. The lengths of these intervals vary from 2.111 kb - 81.620 kb, with the average lengths of these intervals being 20.34 kb. 
% latex table generated in R 3.2.2 by xtable 1.8-0 package
% Tue Feb 16 15:11:00 2016
\begin{table}[H]
\centering
%\begin{tabular}{l*{7}{c}}
\begin{tabular}{lrrr|rrr}
  \hline
  \hline
  \multicolumn{4}{c}{Reference Interval} & \multicolumn{3}{c}{Molecular Fragment lengths} \\
  \hline
   Int  & molecules & pixels & length(kb) & min (kb) & avg (kb) & max (kb)\\ 
  \hline
  \hline
    0 &  66 & 391 & 81.62 & 65.67 & 81.07 & 92.79 \\ 
    1 & 208 & 89 & 18.68 & 13.27 & 18.64 & 21.55 \\ 
    2 & 467 & 284 & 59.40 & 43.92 & 59.24 & 69.39 \\ 
    3 & 734 & 67 & 13.94 & 9.59 & 13.86 & 17.34 \\ 
    4 & 895 & 43 & 9.03 & 6.47 & 8.99 & 11.48 \\ 
    5 & 849 & 24 & 5.04 & 2.14 & 5.02 & 5.90 \\ 
    6 & 939 & 59 & 12.34 & 6.58 & 12.29 & 15.55 \\ 
    7 & 1200 & 49 & 10.24 & 6.74 & 10.20 & 12.22 \\ 
  \hline
  \hline
\end{tabular}
\caption{Coverage of {\emph{M. florum}} data}
\label{tab:mftable}
\end{table}

Table \ref{tab:mftable} lists the coverage of the {\emph{M. florum}} data from interval 0 - 7. The rest of the intervals are in \ref{tab:App_mftable}. For example, interval 7 has 1200 molecular fragments aligned to it. Ideally, all of them should be of length 10.24 kb, but in reality the lengths of these fragments lie between 6.74 kb and 12.22 kb, the average being 10.20 kb. 

\subsubsection*{Human genome - multiple myeloma}
We have nanocoded data of a human genome from DNA samples were prepared from purified CD138 plasma cells (MM-S and MM-R sample) and paired cultured stromal cells (normal) from a 58-y-old male MM patient with International Staging System (ISS) Stage IIIb disease. Multiple myeloma is the malignancy of B lymphocytes that terminally differentiate into long-lived, antibody-producing plasma cells. Although it is a cancer genome, substantial portions of it are still identical to the reference human genome. This genome has been comprehensively analyzed to characterize its genome structure and variation by integrating findings from optical mapping with those from DNA sequencing-based genomic analysis \cite{Gupta_etal_2015_PNAS}. While the {\emph{M. florum}} genome only had 39 intervals, the table below (table \ref{tab:mm52intervals}) lists the number of intervals each chromosome of the human genome contains.

\begin{table}[H]
\centering
\begin{tabular}{c | r || c | r}
  \hline
  \hline
  Chromosome & Number of intervals & Chromosome & Number of intervals \\ 
  \hline
  1 & 26069  & 13 & 9916 \\
  2 & 26772  & 14 & 9764 \\
  3 & 21334  & 15 & 9701 \\
  4 & 19406  & 16 & 9106 \\
  5 & 19359  & 17 & 9291 \\
  6 & 18504  & 18 & 8306 \\
  7 & 16797  & 19 & 5763 \\
  8 & 15828  & 20 & 7311 \\
  9 & 13553  & 21 & 3900 \\
  10 & 15053  & 22 & 4326 \\
  11 & 15212  & X & 15953 \\
  12 & 14371  & Y & 2582 \\
  \hline
  \hline
\end{tabular}
\caption{Number of intervals in each human chromosome}
\label{tab:mm52intervals}
\end{table}
Although there are many more intervals in the MM data-set, the coverage is not as deep as that of the {\emph{M. florum}} data-set. Below are some plots of the coverage of 4 of the chromosomes
\begin{figure}[H]
\begin{center}
\includegraphics[scale = 0.46, page = 1]{Plots/mm52_Coverage.pdf}
\includegraphics[scale = 0.46, page = 6]{Plots/mm52_Coverage.pdf}
\end{center}
\caption{Coverage of intervals (at least 6 kb long)}
\label{fig:Coverage_mm1}
\end{figure}

\begin{figure}[H]
\begin{center}
\includegraphics[scale = 0.46, page = 10]{Plots/mm52_Coverage.pdf}
\includegraphics[scale = 0.46, page = 13]{Plots/mm52_Coverage.pdf}
\end{center}
\caption{Coverage of intervals (at least 6 kb long)}
\label{fig:Coverage_mm2}
\end{figure}

Notice in fig \ref{fig:Coverage_mm1} that in chromosome 1, out of 26,069 intervals, only 12,701 are at least 6kb long, and on an average have a coverage of 30 molecules (NMaps) per interval.

\subsection{Sources of noise in the data} \label{sec_noise}
Following are some sources of noise that at least need to be controlled for, if eliminating them is not feasible
\begin{itemize}
\item When the molecules are stretched on the glass surfaces, different portions of the molecules experience different amount of stretch. Consequently, the features of the intensity profiles could get warped. 
\item In regions where the molecules are stretched too much, more dye molecules could intercalate, resulting in a brighter image. The quantum yield of the dye molecules is sequence dependent. However, that dependence could reduce where there is a higher density of dye molecules. 
\item The DNA molecules might not be straight. This could result in inaccurate intensity profiles. There could be other DNA fragments overlapping longer molecules, resulting in local brightness in the image. Some examples of images which would result in noisy observations are shown below
\begin{figure}[H]
\begin{center}
\includegraphics[scale = 0.5]{Images/Outlier_1.png}
\includegraphics[scale = 0.3]{Images/Outlier_2.png} \\
\includegraphics[scale = 0.35]{Images/Outlier_3.png}
\includegraphics[scale = 0.45]{Images/Outlier_4.png}
\end{center}
\caption{Image 1: Molecules too close to each other; \\ Image 2: Too much overlap, and curves molecules; \\Image 3: Fragments overlapping DNA molecules; \\Image 4: Molecules not straight}
\label{fig:Fig2_OutlierImages}
\end{figure}

\end{itemize}

\section{Research questions}
\noindent
{\bf{Aim 1}}: We want to establish that usable signals with information about genomic sequences could be extracted from the features of the fluorescence intensity profiles of imaged DNA molecules. \\
\noindent
{\bf{Aim 2}}: A diploid organism (like human being) is ``heterozygous'' at a genomic locus when there are multiple genotypes at that locus. To extract two sets of fluorescence intensity profiles in heterozygous regions we want to test the sensitivity of fluoroscanning, i.e., how little of a difference in sequences is translated to detectable dissimilarity in the fluorescence intensity profiles?

\begin{tcolorbox}[colback=green!5,colframe=green!40!black,title=Statistical challenges] %green!40!black=40%green and 60%black
The primary challenge of is to build statistical tools to enable fluoroscanning data analysis. This involves
\begin{itemize}
\item Preprocessing the data by understanding the different sources of noise, and appropriately reducing or controlling for them
\item Estimating a consensus fluorescence intensity profile for each genomic interval
\item Setting up a hypothesis testing framework where intensity profiles from different genomic intervals could be tested for similarity
\end{itemize}
\end{tcolorbox}


          % Chapter 2
\section{Statistical Methodology}

\subsection{Preprocessing}
\noindent
{\bf{Definition 3.1}} {\emph{A random variable $\Y$ is called a functional variable if it takes values in an infinite-dimensional space. An observation $Y$ of $\Y$ is called a functional data. \cite{Ferraty_Vieu_2006_Nonparametric}}}\\
\noindent
{\bf{Definition 3.2}} {\emph{A functional data set $y_1, \dots, y_n$ is an observation of $n$ functional variables $\Y_1, \dots, \Y_n$ as $\Y$.}} $y_i = {y_{i,1}, \dots, y_{i,p}}$ are $p$ discretely observed points on the function $y_i$\\

The basic philosophy of functional data analysis (FDA) is to think of observed data functions as single entities, rather than merely as a string of individual observations \cite{Ramsay_2006_Functional}. In
practice, functional data are discrete observations as $p$ pairs $(x_j, y_j)$, and $y_j (\text{or } y_{ij})$ is a snapshot of the function at position (or time) $x_j$, possibly blurred by measurement error. Pioneered by Ramsay and Silvermann \cite{Ramsay_2006_Functional}, there has been substantial progress in statistical approaches to FDA in recent past. Following are some important aspects FDA that are relevant to our data set, pertaining to the goals of our research. 

\subsubsection*{Smoothing} \label{ch3_smooth}
In fluoroscanning, consider the image of a DNA molecule, as shown in \ref{fig:Fig3_NMap_Intensity}. The grey level of each pixel point of the image of a DNA molecule corresponds to approximately 200 bp of genomic sequence. 
\begin{figure}[H]
\begin{center}
%\includegraphics[scale=1]{Images/Image6_NMap_Intensity.pdf}
\includegraphics[scale=0.7]{Images/Image6_NMap_Intensity.pdf}
\end{center}
\caption{Image of DNA molecule \cite{Kounovsky_etal_2013_Macromolecules}}
\label{fig:Fig3_NMap_Intensity}
\end{figure}
The pixel intensity value is a result of superimposition of point spread functions of the fluorescent dye molecules intercalated between the nucleotide bases. As demonstrated in fig \ref{fig:Fig3_Bivariate} and \ref{fig:Fig3_Bivariate} below, the intensity value of each pixel is a result of neighboring bases as well. Fig \ref{fig:Fig3_Bivariate} is a superimposition of the point spread functions in fig \ref{fig:Fig3_Bivariate}. \footnote{These images were generated on the computer to illustrate point spread functions and the gray levels of intensity profiles. These do not represent any fluoroscanning data or analysis.}  
\begin{figure}[H]
\begin{center}
\includegraphics[scale = 0.27, page = 2]{Plots/BivariatePlots.pdf}
\includegraphics[scale = 0.27, page = 1]{Plots/BivariatePlots.pdf}
\includegraphics[scale = 0.27, page = 3]{Plots/BivariatePlots.pdf} \\
\includegraphics[scale = 0.5, page = 4]{Plots/BivariatePlots.pdf}
\end{center}
\caption{Image 1-3: Point spread functions of 3 adjacent fluorescent sources \\ Image 4: Superimposition of point spread function} 
\label{fig:Fig3_Bivariate}
\end{figure}

Hence, there is an inherent smoothness in the underlying fluorescence intensity profiles of the images of the DNA molecules. Fig \ref{fig:Fig3_frag1058_orig} is an example of fluorescence intensity profiles of fragments of DNA molecules that have been aligned to interval 1058 of chromosome 13, of the reference human genome. 

Another important aspect of the data set is the different lengths of the molecular fragments, aligned to the same genomic interval. As explained in \ref{sec_noise}, there could be several possible reasons causing this. Part of Aim 1 of this research is to estimate consensus fluorescence intensity profiles of all genomic intervals with large enough sample size. In order to estimate a consensus $\hat{\Y}$ from $n$ sample profiles $y_1, \dots, y_n$, each $y_i$ need to be of the same length. This is achieved by smoothing each profile and evaluating the function at regular intervals, based on the length of the reference interval. 

\begin{figure}[H]
\begin{center}
\includegraphics[scale = 0.5, page = 2]{Plots/chr13_frag1058.pdf}
\end{center}
\caption{Intensity profiles of fragments of molecules aligned to human chr 13, interval 1058}
\label{fig:Fig3_frag1058_orig}
\end{figure}
Below is the smoothing procedure applied to fluorescence intensity data, before any further analysis:
\begin{enumerate}
\item Normalizing:\\
As can be seen in fig \ref{fig:Fig3_frag1058_orig} the intensity values range between 6,000 and 12,000. This is because these fragments are parts of molecules that have been imaged on different glass surfaces. Some surfaces appear brighter than others. In order to eliminate the surface ``effect'', each intensity profile was divided by the median value of that fragment. The normalization is done after truncating 5 pixels from either end of the intensity profiles, as those values could be dampened (or amplified) by the presence of the red punctuates that separate the molecular fragments.

\item Basis choice:\\
We used B-spline \cite{deBoor_1978_Splines} to smooth each of intensity profile individually. For an intensity profile $y_i$, with $p$ observed points $y_{i,1}, \dots, y_{i,p}$, we used $\frac{p}{2}$ breakpoints, with $4^{th}$ order basis functions. Since the number of basis functions and their orders have the following relationship, 
\begin{eqnarray*}
n_{\text{basis}} &=& n_{\text{knots}} + n_{\text{order}} - 2 \\
                 &=& \frac{p}{2} + 4 - 2\\ 
                 &=& \frac{p}{2} + 2
\end{eqnarray*}
\item Smoothing:\\
We use {\emph{generalized cross validation}} (GCV) measure, developed by Craven and Wahba (1979) \cite{Craven_Wahba_1978_NumMath} to estimate the roughness penalty $\lambda_{i}$ for each intensity profile $y_i$.  
\[ \lambda_i = \argmin_{\lambda} \text{GCV}(\lambda), \ \ \text{for\ \ } e^{-5} \leq \lambda \leq e^5 \]

\item Evaluation at regular intervals: \\
For a genomic interval with $Q$ bp, since each pixel in the image accounts for 206 bp, ideally each molecular fragment that aligns to that location should be $\frac{Q}{206}$ pixels long. After smoothing the intensity profiles the smooth functions are evaluated at $\frac{Q}{206} + 1$ equidistant points. 
\end{enumerate}
Below is a demonstration of applying the above mentioned data preparation procedure to the intensity profiles of fragments of DNA molecules that have been aligned to interval 1058 of chromosome 13, of the reference human genome, as in fig \ref{fig:Fig3_frag1058_orig}.
\begin{figure}[H]
\begin{center}
\includegraphics[scale = 0.42, page = 3]{Plots/chr13_frag1058.pdf}
\includegraphics[scale = 0.42, page = 4]{Plots/chr13_frag1058.pdf}
\end{center}
\caption{Normalized and Smoothed Intensity profiles of fragments of molecules aligned to human chr 13, interval 1058, evaluated at 22 equidistant points. $(p = 22)$}
\label{fig:Fig3_frag1058_norm}
\end{figure}

\subsubsection*{Functional outlier detection} \label{ch3_outlier}
In addition to variability in the data introduced by different lengths of molecular fragments (aligned to the same genomic locus), and phase variability, there are several other reasons why some molecule could produce an ``outlier'' intensity profile. Below are some images (fig \ref{fig:Fig3_OutlierImages}) of the surfaces with DNA molecules that are examples of outlier intensity profiles. There could be some DNA fragments near the molecule under consideration. There could be molecules crossing each other. Each of those instances would result in some pixel intensity values that would not be appropriate to use in our analyses. 

Using the intensity grey levels of up to three pixels both above and below a molecule, a ``quality score'' was developed to detect some of these outliers. However, more subtle reasons such has slight curvature in the molecule backbones, existence of surface noise underneath the molecule backbones, etc, could still be responsible for producing outlier observations. To address these cases, we used functional data depth measures, as proposed in \cite{Febrero-Bande_etal_2007_Environmetrics}. We use the Fraiman and Muniz depth, introduced in \cite{Fraiman_Muniz_2001_SEIO}. Let $F_{n,x} (y_i (x))$ be the empirical cumulative distribution function of the values of the curves $y_1(x), \dots , y_n(x)$ at a given position $x \in [a, b]$, given by
\[ F_{n,x} (y_i (x)) = \frac{1}{n}\sum\limits_{k=1}^{n} \Ind\{ y_k(x) \leq y_i(x)\}\]
and, the univariate depth of a point $y_i(x)$ is given by
\[ D_n(y_i(x)) = 1 - \left| \frac{1}{2} - F_{n,x} (y_i (x)) \right| \]
Then, the Fraiman and Muniz functional depth (FMD), of a curve $y_i$ with respect the set $y_1(x), \dots , y_n(x)$ is given by 
\begin{equation}
\text{FMD}_n(y_i) = \int\limits_a^b D_n(y_i(x)) dx
\label{eq:3_depth}
\end{equation}
In order to identify functional outliers we use the functional depth FMD. As mentioned in the introduction, depth and outlyingness are inverse notions, so that if an outlier is in the data set, 
the corresponding curve will have a significant low depth. Following is the procedure for functional outlier detection in a given data set of functional curves like fluorescence intensity profiles
$y_1, \dots, y_n$
\begin{enumerate}
\item Obtain the functional depths $D_n(y_1), \dots, D_n(y_n)$, for FMD
\item Let $y_{i1}, \dots, y_{ik}$ be the $k$ curves such that $D_n(y_{ik}) \leq C$, for a given cutoff $C$. Then, assume that $y_{i1}, \dots, y_{ik}$ are outliers and delete them from the sample.
\item Then, come back to step 1 with the new data set after deleting the outliers found in step 2. Repeat this until no more outliers are found.
\end{enumerate}
To ensure type-I error of detecting outliers is under some small threshold $\alpha$, we need to choose $C$ such that
\[ \Prob(D_n(y_i) \leq C) = \alpha,\ \ i = 1, \dots, n\]
However, since the distribution of the functional depth statistic FMD is unknown, it is estimated here using a bootstrap procedure. For details, please refer to \cite{Febrero-Bande_etal_2007_Environmetrics} and \cite{Febrero-Bande_delaFuente_2012_JSS}. 

In fig \ref{fig:Fig3_frag1058_outlier} we show the detection of outlier in the fluorescence intensity profiles of fragments of molecules aligned to human chr 13, interval 1058. Refer to figures \ref{fig:Fig3_frag1058_orig} and \ref{fig:Fig3_frag1058_norm}. 
\begin{figure}[H]
\begin{center}
\includegraphics[scale = 0.5, page = 5]{Plots/chr13_frag1058.pdf}
\end{center}
\caption{Outlier detection in Intensity profiles of fragments of molecules aligned to human chr 13, interval 1058}
\label{fig:Fig3_frag1058_outlier}
\end{figure}
The red curve in fig \ref{fig:Fig3_frag1058_outlier} is flagged as the outlier in this group of curves.
\begin{tcolorbox}[colback=green!5,colframe=green!40!black,title=Work in progress] %green!40!black=40%green and 60%black
Some recent work was published on outliers in functional data. We are working on comparing and contrasting the different methods to choose the most appropriate one for this data set.
\end{tcolorbox}

\subsection{Estimation of consensus} \label{Ch3_Regist}

We have observed substantial amount of phase variability in the fluorescence intensity data set. Hence, before estimating the consensus fluorescence intensity profiles of genomic regions, it is imperative that we address this variability. Pioneered by Ramsay and Silverman \cite{Ramsay_2006_Functional}, \cite{Ramsay_Li_1998_JRSSB}, \cite{Ramsay_etal_2009_Functional_R}, {\emph{Curve Registration}} is a FDA technique to address the phase variability problem. Section \ref{App_Phase} in Appendix lists some common examples from publicly available data-sets that exhibit phase variability in functional data.

\subsubsection*{Distance metric}
Before discussing the techniques of curve registration, it is important to establish the notion of ``distance'' or ``similarity'' between two smooth functions, or two curves. We use the {\emph{similarity index}} between two curves in $\Real$, introduced in \cite{Sangalli_etal_2009_JASA}. Let $y_i \in L^2(S_i \subset \Real; \Real)$ and $y_j \in L^2(S_j \subset \Real; \Real)$ be differentiable with $y'_i \in L^2(S_i \subset \Real; \Real)$ and $y'_j \in L^2(S_j \subset \Real; \Real)$, and let the domains $S_i \subset T$ and $S_j  \subset T$ be closed intervals in $\Real$ such that $S_{ij} = S_i \intersect S_j$ has a positive Lebesgue measure. $S_{(.)}$ are Sobolev spaces. Assuming that $\|y'_i\|_{L^2(S_{ij})} \ne 0$ and $\|y'_j\|_{L^2(S_{ij})} \ne 0$, the similarity index between $y_i$ and $y_j$ is defined as
\begin{equation}
\rho(y_i, y_j) = \frac{\int _{S_{ij}}y'_i(x)y'_j(x) dx}{\sqrt{\int _{S_{ij}}y'_i(x)^2 ds}\sqrt{ \int _{S_{ij}} y'_j(x)^2 dx}}
\label{eq:3_similarity}
\end{equation}
This is the cosine of the angle $\theta_{ij}$ between first derivatives of the functions $y_i$ and $y_j$, with the inner product $\int _{S_{ij}}y'_i(x)y'_j(x) dx$.  $\rho(y_i, y_j)$ can also be interpreted as a continuous version of Pearson’s uncentered correlation coefficient for first derivatives. Following are some useful properties of $\rho(y_i, y_j)$:
\begin{enumerate}
\item[(i)] From Cauchy-Schwartz inequality it follows that $|\rho(y_i, y_j)| \leq 1$
\item[(ii)] $\rho(y_i, y_j) = 1 \ \Leftrightarrow \ \exists \ a \in \Real^{+}, b \in \Real, \ni y_i = ay_j + b $
\item[(iii)] For all invertible affine transformations of $y_i$ and $y_j$, say $t_1 \circ y_i = a_1y_i + b_2$ and $t_2 \circ y_j = a_2y_j + b_2$, with $a_1, a_2 \ne 0$, 
\[ \rho(y_i, y_j) = \text{sign}(a_1 a_2)\rho(t_1 \circ y_i, t_2 \circ y_j)\]
\item[(iv)] For all invertible affine transformations of the abscissa $x$, say $h_1(x) = a_1 x + b_1$ and $h_2(x) = a_2 x + b_2$, with $a_1, a_2 > 0$, we have
\[ \rho(y_i \circ h_1, y_j \circ h_2) = \rho(y_i \circ h_1 \circ h_2^{-1}, y_j) = \rho(y_i , y_j \circ h_2 \circ h_1^{-1}) \]
\end{enumerate}

\subsubsection*{The registration problem}
It is imperative to eliminate (or at least, reduce) phase variability between replicated curves before further analysis. Pioneered by Silverman in \cite{Silverman_1995_JRSSB}, Ramsay and Li in \cite{Ramsay_Li_1998_JRSSB} and further advanced by Ramsay and Silverman in \cite{Ramsay_2006_Functional}, {\emph{curve registration}} is the technique that address this problem. Let $n$ functions (or curves) $y_1, \dots, y_n$ be defined on a closed real interval $[0, X]$. Let $h_i(x)$ be a transformation of the abcissa $x$ for curve $i$. The warping function is often referred to as ``time warping'' as time is a common abcissa in problems with phase variability. In the context of fluoroscanning, the abcissa is genomic location. The warping function should satisfy the following
\begin{itemize}
\item $h(0) = 0$ and $h(X) = X$, $0$ and $X$ being the endpoints of the interval on which the functions are defined.
\item The timings of events remain in the same order regardless of the timescale entails that $h_i$, the time-warping function, should be strictly increasing, i.e. $h_i(x_1) > h_i(x_2)$ for $x_1 > x_2$. 
\item $h^{-1}[h(x)] = x$
\end{itemize}
The objective of curve registration is that the {\emph{registered}} functions $y_1(h_1(x)), \dots, y_n(h_n(x))$ will have no phase variability. 

\subsubsection*{Existing work on curve registration} \label{ch3_registration}
Marker registration is often used in engineering, biology and other fields. It is the process of aligning curves by identifying the timing of certain salient features in the curves. Curves are then aligned by transforming time (or the abcissa) so that marker events occur at the same values of the transformed times. Comparisons between marker timings can also be made by using corresponding transformed times. Sakoe in Chiba in \cite{Sakoe_Chiba_1978_IEEE} estimated the warping function $h$ at marker timings by minimizing the sum of weighted distances of two speech patterns at the marker timings and imposing monotonicity and continuity on $h$. They solved for the discrete values of $h$ using a dynamic programming algorithm, which widely popularized as {\emph{dynamic time warping}}. Kneip and Gasser in \cite{Kneip_Gasser_1992_AnnStat} studies the statistical aspects, including the asymptotic properties of these estimators \cite{Kneip_etal_2000_CJS}. These methods, however, can be sensitive to errors in feature location, and the  required features may even be missing in some curves. Moreover,substantial phase variation may remain between widely separated markers. In \cite{Kneip_etal_2000_CJS}, they introduce a local nonlinear regression technique, but acknowledge that a lot of tuning parameters are left to user's experimentation and best guess. Following are some of the subsequent curve registration techniques that helped develop an iterated registration algorithm to analyze the fluoroscanning data. \\

\noindent
{\bf{Penalized least square criterion}} \\
Ramsay and Li in \cite{Ramsay_Li_1998_JRSSB} set it up as a penalized least square (PLS) fitting problem. When registering $y(x)$ to a template $y_0(x)$, they minimize the penalized squared error criterion
\[ F_{\lambda}(y_0, y|h) = \displaystyle \int \| y_0(x) - y\{h(x)\} \|^2 dx + \lambda \displaystyle \int w^2(x)dx\]
where, $w(x) = \frac{D^2h}{Dh},\ \ D = \frac{\partial}{\partial x}$. $w(x)$ is the relative curvature of the warping function.\\

\noindent
{\bf{Minimum second eigenvalue method}} \\
Ramsay and Silverman in \cite{Ramsay_2006_Functional} argued the PSL technique addressed more of the amplitude variability and not as much of the phase variability. They suggested the following continuous registration technique. Suppose two curves $y_0(x)$ and $y_1(x)$ differ only in amplitude but not in phase. Then, if we plot the function values against each other, we will see a straight line. Amplitude
differences will then be reflected in the slope of the line, a line at $45^{\circ}$ corresponding to no amplitude differences. To evaluate both the target function $y_0(x)$ and the registered function 
$y*$ at a fine mesh of $n$ values of $x$ to obtain the pairs of values $(y_0(x), y\{h(x) \}$. Let the $n$ x $2$ matrix $\mathbf{X}$ contain these pairs of values. Then, to analyze the principal components, we would analyze the functional analog of the cross product matrix $\mathbf{X'X}$. 
\begin{equation}
C(h) = 
\begin{bmatrix}
\int \{y_0(x) \}^2dx & \int y_0(x) y[h(x)]dx \vspace{0.5cm} \\ 
\int y_0(x) y[h(x)]dx & \int \{y[h(x)]\}^2dx
\end{bmatrix}
\end{equation}
Then, PCA of $C(h)$ should reveal essentially one component (smallest eigen value $\approx 0$). Now, the objective function for curve registration between a template $y_0(x)$ and a sample curve $y(x)$ becomes
\[ \text{MINEIG}(h) = \mu_2[C(h)]\]
where, $\mu_2$ is the size of the second eigenvalue of its argument. When $\text{MINEIG}(h)=0$, registration is achieved, and $h$ is the warping function that does the job. Including the penalty term on the relative curvature of the warping function, the Minimum second eigenvalue (MSE) fitting criterion is
\begin{equation}
\text{MINEIG}_{\lambda}(h) = \text{MINEIG}(h) + \lambda \displaystyle \int \left\{ w^{(m)}(x) \right\}^2 dx
\end{equation}
When registering a sample of replicated curves, they \cite{Ramsay_etal_2009_Functional_R} recommended using the mean of the curves as the template and registering all of the curves to that template. \\

\noindent
{\bf{Iterative registration using similarity index}} \\
Sangalli, et al, in \cite{Sangalli_etal_2009_JASA} introduce an iterative registration (E-M) algorithm. Given the similarity index $\rho(y_i, y_0)$, defined in \ref{eq:3_similarity}, between two curves, registering a curve $y_i$ to a template curve $y_0$ means finding the function $h$ in a class of warping functions $W$ that maximizes 
\[ \rho(y_i \circ h^{-1}, y_0)\]
The consider $W$ as a group of strictly increasing affine transformations of the abcissa:
\begin{equation}
W = \{h:h(s) = as + b,\ \ a \in \Real^{+}, b \in \Real  \}
\label{eq:3_affine}
\end{equation}
This approach is quite intuitive and elegant. This is the first approach that defines a metric to estimate the quality of a registration procedure and use it as a convergence criterion of the iterative E-M algorithm. Similar to local linear regression method in \cite{Kneip_etal_2000_CJS} and the MSE method in \cite{Ramsay_2006_Functional}, they start the registration procedure with the mean of the curves as the template and update the template at each ``Expectation'' step of the E-M algorithm. They successfully extend this algorithm to simultaneous clustering and registration of curves in \cite{Sangalli_etal_2010_CSDA} and \cite{Sangalli_etal_2014_EJS}. \\

\noindent
{\bf{Registration using the Fisher-Rao metric}} \\
Registration using the {\emph{Fisher-Rao}} metric was developed by Srivastava et al., in \cite{Srivastava_etal_2011_v2_arXiv}. The space of functions considered here is
\[ F = \left\{f: [0,1] \rightarrow \Real, f \text{ absolutely continuous}  \right\}\] and the space of warping functions:
\[ \Gamma = \left\{\gamma: [0,1] \rightarrow [0,1], \gamma(0) = 0, \gamma(1) = 1, \gamma \text{ monotone increasing }, \gamma \text{ is a diffeomorphism}  \right\} \]
The Fisher-Rao metric is defined on the space tangent to the variety of $F$ as:
{\bf{Definition 3.3}} For $f \in F$ and $v_1, v_2 \in T_f(F)$, the Fisher-Rao metric is defined as: 
\[ \langle\langle \cdot , \cdot \rangle \rangle_f: T_f(F) \times T_f(F) \rightarrow \Real; \langle\langle v_1 , v_2 \rangle \rangle_f = \frac{1}{4} \displaystyle \int \limits_0^1 \frac{\dot{v}_1(t) \dot{v}_2(t)}{|\dot{f}(t) |}dt \]
To find this distance it requires the geodetic path which connect functions $f$ and $g$ with respect to the Fisher-Rao metric. The minimization of this quantity is therefore quite challenging. This is why a  new representation of the functions was introduced.
{\bf{Definition 3.4}} The square-root velocity function (SRVF) of a function $f \in F$ is defined as
\[ q:\Real \rightarrow \Real, \ q(x) = 
  \begin{cases}
    \frac{x}{\sqrt{\|x \|}}       & \|x\| \ne 0 \\
    0 & \text{otherwise} \\
  \end{cases} 
\] 
A fundamental property of this metric is that under the SRVF representation, the Fisher-Rao metric becomes the $L^2$ norm. In $L^2$ space the geodesic distance is simply the $L^2$ norm. They estimate the warping functions on this SRVF space, performing the following optimization:
\[ \gamma_i^* = \argmin_{\gamma \in Gamma} \|\mu - (q_i \circ \gamma)\sqrt{\dot{\gamma}}  \|_{L^2} \]
In the Fisher-Rao method they were able to include in the space of warping functions almost all kind of transformations, with the only constraint being diffeomorphism. This leads to a very large class of transformations and hence, generally, to a better result in aligning functional data. However, this is also its limitation \cite{Patriarca_2013_PhDThesis}. With this method it is possible to align almost any group of functions, however disparate they might be. In addition, the warping function loses interpretability. In many cases, including fluoroscanning data, the warping function plays an important role in analyzing the causes of phase variability. \\

\noindent
{\bf{Bayesian approach to registering and clustering}} \\
Zhang and Inoue developed a Bayesian hierarchical curve registration in \cite{Telesca_Inoue_2008_JASA}. Their approach provides a natural framework for assessing uncertainty in the estimated time-transformation and shape functions and allows derivation of exact inferences about a richer set of quantities of interest. Zhang and Telesca in \cite{Zhang_Telesca_2014_arXiv} extended it to a Bayesian hierarchical approach to joint clustering and registration of functional data. They combined a Dirichlet process mixture model for clustering of common shapes, with a reproducing kernel representation of phase variability for registration. Earls and Hooker in \cite{Earls_Hooker_2015_arXiv} proposed an adapted variational Bayes algorithm for registering, smoothing and prediction for functional data. They model the registered 
functions as Gaussian processes. They go beyond estimation, and inference and provide the framework of prediction of registered functions. 

%{\emph{Note: Not sure how I can compare and contrast these Bayesian methods with the other approaches. Other than the computational time and resources, I am not yet in a position to comment on their relative performance!}}

\subsubsection*{Why iterated registration?}
Enumerated below are the reasons why iterated registration is necessary for fluoroscanning analysis. 
\begin{itemize}
\item {\bf{Outlier}}: None of the above methods consider the possibility of presence of a functional outlier in their sample. Hence, their estimates of the warping functions might not be robust to possible outliers. In the fluoroscanning data, as a consequence of the experimental setup of presenting DNA molecules on glass surfaces, there are multiple reasons why some intensity profiles could be outliers. It is important to detect, but not necessarily discard, intensity profiles with unusual features.

\item {\bf{Starting template}}: In most of the methods above, the cross-sectional mean of the curve sample is the starting template to which the rest of the curves are registered to. This approach works well if the noise to signal ratio is small, and the extent of abcissa warping is limited. However, in the fluoroscanning data, the signals are warped to a much larger extent and consequently, the cross-sectional mean of intensity profiles is not representative of the true features of that sample of curves. We present an iterated registration scheme, similar to \cite{Sangalli_etal_2009_JASA}, but the estimation of our template at each ``expectation'' step of the E-M algorithm will be more sophisticated and robust.

\item {\bf{Warping functions}}: Estimating the inverse of the warping functions are critical to registering out-of-phase functional data. These functions contain information about the phase variability. In fluoroscanning, it is important that the estimation of these warping functions are consistent even when sub-intervals of longer genomic intervals are analyzed. It is important to investigate if the warping functions have any sequence dependence, or if they result from experimental artifacts. Hence, it is critical that warping functions are smooth, and interpretable. However, at the same time, it is important for warping functions to contain higher order terms to contain information about sub-genomic-interval warping. Hence, warping functions cannot be as simple as affine transformations of the abcissa, as considered in \cite{Sangalli_etal_2009_JASA}. Neither could they be as complex and uninterpretable as in \cite{Srivastava_etal_2011_v2_arXiv}. 

\item {\bf{Quality of registration}}: As introduced in \cite{Sangalli_etal_2009_JASA}, it is important to evaluate the quality of the registration procedure. 

\item {\bf{Simultaneous clustering and registration}}: Our registration methods needs to be robust enough to be extended to simultaneous registration and clustering to address the question of detecting heterozygotes in diploid human genomes.  
\end{itemize}

\subsubsection*{Iterated registration algorithm}
Here, we re-define the registration problem with respect to the fluoroscanning data and introduce an ``Iterated registration with weighted average template'' scheme to register the fluorescent intensity profiles.
For any genomic interval where there are $n$ molecular fragments aligned, let the intensity profiles observed be represented as $z_{i,1}, \dots, z_{i, p_i},\ \ i = 1,\dots, n$, where curve $i$ has $p_i$ points. 
\begin{itemize}
\item {\bf{Step 1}} Normalize and smooth: As explained in section \ref{ch3_smooth}, we normalize all the intensity profiles, smooth them using B-splines, and evaluate at $p + 1$ equidistant points, where $p = \frac{Q}{206}$, where the genomic interval has $Q$ base pairs. Let these smooth functions be $y_{i,1}, \dots, y_{i,p},\ \ i = 1,\dots,n$. 
\item {\bf{Step 2}} Detect outliers: As explained in section \ref{ch3_outlier}, we detect functional outliers in the data set, using Fraiman and Muniz functional depth. 
\item {\bf{Step 3}} Template estimation [{\emph{Expectation step}}]: To estimate the template to register the curves to, we employ a 2-step approach. 
\begin{enumerate}
\item {\emph{Median}}: Estimate $L_1-\text{Median}$ by the algorithm proposed by Vardi and Zhang in \cite{Vardi_Zhang_2000_PNAS}, where $L_1-\text{Median }$ $y_m$ is the minimizer of 
\[ \sum\limits_{i = 1}^n \|y_i - y_m \| \]
where $y_i \in \Real^p, \ i = 1,\dots,n$ and $\|u \| = \sqrt{\sum\limits_{j = 1}^p u_j^2}$. Here we ensure that the median is estimated only from the curves not deemed as ``functional outliers'' in Step 2. 
\item {\emph{Weighted mean}}: Estimate the similarity index between the curves and the median $\rho(y_i, y_m), \ i = 1,\dots,n$ and estimate the template $\phi$ as the weighted average of the curves, with the weights being these similarity indices. 
\[ \phi = \frac{1}{n}\sum\limits_{i = 1}^n \rho(y_i, y_m) y_i \]
\end{enumerate}
This 2-step process ensures that the template to register to has little influence from an outlier curve, at the same time giving higher weights to curves that have retained some similarity between their features, in spite of being warped. 
\item {\bf{Step 4}} Registration with varying roughness penalty [{\emph{Maximization step}}]: We use the minimum second eigenvalue method to register the curves to the template $\phi$. As explained in \ref{ch3_registration}, the penalty parameter $\lambda$ plays an important role in registering nearby features of the curves. For a higher value of $\lambda$, even distant features will get registered, and for lower values of $\lambda$ only the features that are close by will be registered. We start the iterated process with $\lambda = 1$ and gradually lower the values every iteration. This ensures that we gradually increase our confidence in the template that we register to. \\
\begin{tcolorbox}[colback=green!5,colframe=green!40!black,title=Work in progress] %green!40!black=40%green and 60%black
We want to quantify the choice of $\lambda$ to some measure of the phase variability, and it is currently under investigation. We believe this is an interesting contribution to the ``registration'' technique, particularly when applied to a complex data set like fluoroscanning.
\end{tcolorbox}
\item {\bf{Step 5}} Convergence of iteration: The objective of iterated registration is to maximize the average similarity to the consensus, i.e., maximize 
\[ \bar{\rho}_{\phi, n} = \frac{1}{n} \sum \limits_{i = 1}^{n} \rho(\phi, f_i)\]
Hence, we iterate steps 3 and 4, until
\[ |\bar{\rho}_{\phi_{(t+1)}, n}^{\ (t+1)} - \bar{\rho}_{\phi_{(t)}, n}^{ \ (t)} | < \epsilon \]
where $t$ denotes iteration number, for some predetermined $\epsilon$
\end{itemize}

\subsection{Hypothesis Testing}
\subsubsection*{Permutation test for power of registration}
We describe a permutation t-type test for two groups of functional data objects. We will extend this test to verify that the iterated registration method improves power. \\
Let $Y_{1,1}, \dots, Y_{1,n_1}$ and $Y_{2,1}, \dots, Y_{2,n_2}$ be two sets of $n_1$ and $n_2$ curves respectively. Without making any distributional assumptions we want to test if they are from the same distribution.
Assume 
\begin{itemize}
\item $Y_{1,.}\  \stackrel{iid}{\sim} \ \Y_1$
\item $Y_{2,.}\  \stackrel{iid}{\sim} \ \Y_2$
\item $Y_{1,i}\  \indep \ Y_{2,j}, \ \ \forall i, j$
\item Each curve $Y_{1,.}, Y_{2,.}$ have the same number of data points
\end{itemize}
\[ H_0: \Y_1 \ \stackrel{\mathcal{D}}{=} \ \Y_2 \ \ vs \ \ H_a: \Y_1 \ \stackrel{\mathcal{D}}{\ne} \ \Y_2 \]
\begin{enumerate}
\item Register curves of group 1 and obtain the consensus $\bar{Y}_1(x): $
\item Register curves of group 2 and obtain the consensus $\bar{Y}_2(x): $
\item Let $\Var[Y_1(x)]: $ Variance of curves of group 1 at each point $x$ 
\item Let $\Var[Y_2(x)]: $ Variance of curves of group 2 at each point $x$ 
\item $T(x) = \frac{|\bar{Y}_1(x) - \bar{Y}_2(x)|}{\sqrt{\frac{1}{n_1}\Var[Y_1(x)] + 
\frac{1}{n_2}\Var[Y_2(x)]}}$: Absolute value of t-statistic at each point
\item $T_{\text{sup}}:  \sup\limits_{x} T(x)$ is our Test statistic (measure of difference of two sets of curves)
\item To get a null distribution of $T_{\text{sup}}$, permute the curves between the two groups, and repeat steps 1 through 6. \\
\[  \text{p-value} = \frac{1}{N}\sum\limits_{j = 1}^{N}\Ind\{T_{\text{sup, obs}} > T_{\text{sup, permute}} \} \]
where, $N:$ number of permutations
\end{enumerate}

We have implemented the iterated registered algorithm in a simulation and used the above mentioned permutation test to verify improvement of power. 
\subsubsection*{Simulation set up}
This simulation set up is considerably more complex than all the simulation studies used to test different curve registration techniques in \cite{Kneip_Ramsay_2008_JASA}, \cite{Srivastava_etal_2011_v2_arXiv}, etc. The goal was to generate data more representative of the noise structure observed in the fluoroscanning data-set. 
\begin{enumerate}
\item Set up the abcissa $x$ from $1$ to $40$.

\item Randomly choose 3 locations $x_1, x_2, x_3 \in (1, 40)$ where the curves will have distinctive features.

\item Generate a true curve by the following formula:
\[ y = ae^{-\frac{(x - x_1)^2}{16}} - ae^{-\frac{(x - x_2)^2}{16}} + ae^{-\frac{(x - x_3)^2}{16}},\ \ a=0.04  \]
\begin{figure}[H]
\begin{center}
\includegraphics[scale = 0.4, page = 1]{Plots/SimData_Seed67_Plots.pdf}
\end{center}
\caption{Simulation: True Signal, Seed 67}
\label{fig:FigSim_True}
\end{figure}
The vertical line in fig \ref{fig:FigSim_True} are the signal locations.

\item Adding smooth noise at the signal locations and two more randomly chosen locations $x_{12} \subset (x_1, x_2) \ \& \ x_{23} \subset (x_2, x_3) $, the resulting curves are of the form:
\[ y = (a + \epsilon_1)e^{-\frac{(x - x_1)^2}{16}} + \epsilon_2e^{-\frac{(x - x_12)^2}{16}} - (a + \epsilon_3)e^{-\frac{(x - x_2)^2}{16}} + \epsilon_4e^{-\frac{(x - x_23)^2}{16}} + (a + \epsilon_5)e^{-\frac{(x - x_5)^2}{16}}\]
where, $\epsilon_i \sim \mathcal{N}(0, 0.05)$
The 40 noisy curves are plotted below:
\begin{figure}[H]
\begin{center}
\includegraphics[scale = 0.4, page = 2]{Plots/SimData_Seed67_Plots.pdf}
\end{center}
\caption{Simulation: Noisy Curves, Seed 67}
\label{fig:FigSim_Noisy}
\end{figure}
The red broken line in fig \ref{fig:FigSim_Noisy} is the average of the noisy curves, which is not too far off from the true signal. 

\item Warp the noisy curves by the following operation on the abcissa: Let $x_1$ and $x_p$ be the end points of the abcissa. Randomly choose a location $x^*$ at which the warping direction will change from positive to negative or vice-versa. Note that at any $x$ if $h(x) > x$, then the abcissa has been stretched outward and if $h(x) < x$, it has been squeezed inward. 
\begin{eqnarray*}
h_1(x) &=& x_1 + (x^* - x_1) \frac{e^{\frac{u_i(x - x_1)}{x^* - x_1}} - 1}{e^{u_i} - 1} \\
h_2(x) &=& x_* + (x_p - x^*) \frac{e^{\frac{v_i(x - x^*)}{x_p - x^*}} - 1}{e^{v_i} - 1} \\
h(x) &=& \Ind_{\{ x \le x^*\}} h_1(x) + \Ind_{\{ x > x^*\}} h_2(x) \\
  \ u_i, v_i \sim \mathcal{N}(0, 1)  &;& i = 1, \dots, 40
\end{eqnarray*}
Below is a plot of the warping functions and the warped, noisy curves.
\begin{figure}[H]
\begin{center}
\includegraphics[scale = 0.4, page = 3]{Plots/SimData_Seed67_Plots.pdf}
\includegraphics[scale = 0.4, page = 4]{Plots/SimData_Seed67_Plots.pdf}
\end{center}
\caption{Simulation: Warped Noisy Curves, Seed 67}
\label{fig:FigSim_Warped}
\end{figure}
The green line in fig \ref{fig:FigSim_Warped} is the cross-sectional mean of the noisy, warped curves. Notice that it shows substantial deviation from the true signal and illustrates the need for registering them to retrieve the true signal.

\item 2 groups: To test the performance of the registration procedure under the null distribution, we randomly separate the 40 noisy curves into two groups of 20 each. We will register these groups separately and test if they produce different consensus signals. 
\begin{figure}[H]
\begin{center}
\includegraphics[scale = 0.4, page = 5]{Plots/SimData_Seed67_Plots.pdf}
\includegraphics[scale = 0.4, page = 6]{Plots/SimData_Seed67_Plots.pdf}
\end{center}
\caption{Simulation: 2 groups of curves, Seed 67}
\label{fig:FigSim_2groups}
\end{figure}

\end{enumerate}

We apply the permutation test to two groups of simulated data produced from seed 25. There are 20 curves in each group and all of them are noisy, warped version of the same curve. Below are the two groups
\begin{figure}[H]
\begin{center}
\includegraphics[scale = 0.4, page = 5]{Plots/SimData_Seed25_Plots.pdf}
\includegraphics[scale = 0.4, page = 6]{Plots/SimData_Seed25_Plots.pdf}
\end{center}
\caption{Simulation: 2 groups of curves, Seed 25}
\label{fig:FigSim_2groupsSeed25}
\end{figure}
\begin{figure}[H]
\begin{center}
\includegraphics[scale = 0.5]{Plots/PermTest_Seed25.pdf}
\end{center}
\caption{Permutation test between two groups of Seed 25 curves}
\label{fig:Permtest_2groupsSeed25}
\end{figure}
Fig \ref{fig:Permtest_2groupsSeed25} shows the result of the permutation test between two groups of curves generated by seed 25. As expected, there is no significant different between the two sets. The p-value is $0.576$. Now, testing for differences between two sets of curves, one generated by seed 25 and the other by seed 30, we have
\begin{figure}[H]
\begin{center}
\includegraphics[scale = 0.4, page = 4]{Plots/SimData_Seed25_Plots.pdf}
\includegraphics[scale = 0.4, page = 4]{Plots/SimData_Seed30_Plots.pdf}
\end{center}
\caption{Simulated curves, seed 25 and seed 30}
\label{fig:FigSim_Seed25and30}
\end{figure}
\begin{figure}[H]
\begin{center}
\includegraphics[scale = 0.5, page = 1]{Plots/PermTest_Seed25_Seed30.pdf}
\end{center}
\caption{Permutation test between seed 25 and seed 30 curves}
\label{fig:Permtest_Seed25and30}
\end{figure}
The permutation test rejects the $H_0$ of equality of distributions of the two sets of curves, p-value = 0.019. 

\subsubsection*{Simulation results}
To test that iterated registration improves power when detecting differences between two groups of curves, we did the following:
\begin{itemize}
\item Estimated the permutation p-value (as above) between 1200 seed combinations. The distribution of these p-values is under the alternative hypothesis, since the curves from distinct seeds are being compared.
\begin{figure}[H]
\begin{center}
\includegraphics[scale = 0.6, page = 2]{Plots/pValue_Comparison_Run08_6.pdf}
\end{center}
\caption{Distribution of p-values under the alternative hypothesis, before registration}
\label{fig:BeforeRegist}
\end{figure}

\item Took a seed pair, and registered their curves using the iterated registration algorithm, and estimated $T_{\text{sup, obs}}$. Then, permuted the two groups of curves $N = 1000$ times to obtain the permutation distribution from  $T_{\text{sup, permute}}$. Then, we estimated the permutation p-value. Below are the distributions of the p-values after single iteration (using minimum second eigen value method) and after the proposed iterated registration. Fig \ref{fig:AfterRegist} indicates a power improvement as a result of using the iterated registration method from 19\% to 47\%, whereas, only up to 36\% using single iteration registration. 

\vspace{-1cm}
\begin{figure}[H]
\begin{center}
\includegraphics[scale = 0.5, page = 3]{Plots/pValue_Comparison_forPrelims.pdf}
\end{center}
\caption{Distribution of p-values under the alternative hypothesis, after {\underline{single iteration}}}
\label{fig:AfterSingleIter}
\end{figure}

\vspace{-1cm}
\begin{figure}[H]
\begin{center}
\includegraphics[scale = 0.55, page = 3]{Plots/pValue_Comparison_Run08_6.pdf}
\end{center}
\caption{Distribution of p-values under the alternative hypothesis, after {\underline{iterated}} registration}
\label{fig:AfterRegist}
\end{figure}


\end{itemize} 


       % Chapter 3
\section{Application to human genome}

We applied the iterated registration technique to fluoroscanning data from the human genome. From previous experiments conducted on this genome, we know that chromosome 13 of this genome is homozygous and we expect to find only one cluster of intensity of profiles. Below is a step-by-step application of the statistical techniques described in chapter 3, to interval 7491 of chromosome 13.
\begin{itemize}
\item There are 30 DNA molecular fragments that are aligned to this interval
\begin{figure}[H]
\begin{center}
\includegraphics[scale = 0.45, page = 2]{Plots/chr13_frag7491_registered.pdf}
\end{center}
\caption{30 fluorescence intensity profiles of DNA molecular fragments}
\label{fig:Frag7491_Orig}
\end{figure}

\item After normalizing the fluorescence intensity profiles by the median values of each fragment
\begin{figure}[H]
\begin{center}
\includegraphics[scale = 0.45, page = 3]{Plots/chr13_frag7491_registered.pdf}
\end{center}
\caption{30 normalized fluorescence intensity profiles of DNA molecular fragments}
\label{fig:Frag7491_Norm}
\end{figure}

\item After smoothing the normalized the fluorescence intensity profiles by B-spline and evaluating them at equidistant points
\begin{figure}[H]
\begin{center}
\includegraphics[scale = 0.45, page = 4]{Plots/chr13_frag7491_registered.pdf}
\end{center}
\caption{30 smoothed, normalized fluorescence intensity profiles of DNA molecular fragments}
\label{fig:Frag7491_Smooth}
\end{figure}

\item After iterated registration
\begin{figure}[H]
\begin{center}
\includegraphics[scale = 0.45, page = 6]{Plots/chr13_frag7491_registered.pdf}
\end{center}
\caption{Registered values of fluorescence intensity profiles}
\label{fig:Frag7491_Regist}
\end{figure}

\item Comparing final consensus intensity profile estimated by registration, with GCAT composition of the interval
\begin{figure}[H]
\begin{center}
\includegraphics[scale = 0.6, page = 7]{Plots/chr13_frag7491_registered.pdf}
\end{center}
\caption{Comparison of GCAT composition with consensus intensity profile of the interval}
\label{fig:Frag7491_Compare}
\end{figure}
\end{itemize}

\newpage
Below is the application of the methodology to interval 9774 of chromosome 13
\begin{figure}[H]
\begin{center}
\includegraphics[scale = 0.42, page = 2]{Plots/chr13_frag9774_registered.pdf}
\includegraphics[scale = 0.42, page = 3]{Plots/chr13_frag9774_registered.pdf} \\
\includegraphics[scale = 0.42, page = 6]{Plots/chr13_frag9774_registered.pdf} 
\includegraphics[scale = 0.42, page = 7]{Plots/chr13_frag9774_registered.pdf}
\end{center}
\caption{Fragment 9774, of human chromosome 13}
\label{fig:Frag9774_All}
\end{figure}

\subsection{Reproducible Fluorescence Intensity signals}
\begin{tcolorbox}[colback=green!5,colframe=green!40!black,title=Work in progress] %green!40!black=40%green and 60%black
The human genome consists of Ribosomal repeat regions which have the same nucleotide composition. Currently, we are working on acquiring the data to test if intensity profiles aligned to intervals of those repeat regions exhibit similar structure.
\end{tcolorbox}





        % Chapter 4
\section{Future work}

\subsection{Relationship between stretch and signal}

\subsection{Prove that iterated registration increases power more than any other existing algorithm}

\subsection{Quantify the relationship between sequence composition and fluorescence intensity profile}

\subsection{Quantify the relationship between warping function and sequence composition}


            % Chapter 5
\section{Appendix}
\subsection{Phase Variability} \label{App_Phase}
A problem of critical importance in FDA is the presence of both amplitude and phase variability in functional observations. Even simple analysis like averaging of replicated curves will produce erroneous estimates if phase variability is not eliminated (or reduced) first. Fig \ref{fig:Fig3_AmpPhase} illustrates the two types of variability. 
\begin{figure}[H]
\begin{center}
\includegraphics[scale = 0.42, page = 1]{Plots/AmpPhaseVar.pdf}
\includegraphics[scale = 0.42, page = 2]{Plots/AmpPhaseVar.pdf}
\end{center}
\caption{Amplitude and Phase variability in FD}
\label{fig:Fig3_AmpPhase}
\end{figure}
Below are some commonly used publicly available data sets where phase variability needs to be eliminated before analyzing the curve samples. 
\begin{figure}[H]
\begin{center}
\includegraphics[scale = 0.4, page = 1]{Plots/PublicData.pdf}
\end{center}
\caption{Amplitude and Phase variability in FD}
\label{fig:Fig3_growthM}
\end{figure}

%\begin{figure}[H]
%\begin{center}
%\includegraphics[scale = 0.4, page = 2]{Plots/PublicData.pdf}
%\end{center}
%\caption{Amplitude and Phase variability in FD}
%\label{fig:Fig3_growthF}
%\end{figure}

%\begin{figure}[H]
%\begin{center}
%\includegraphics[scale = 0.4, page = 3]{Plots/PublicData.pdf}
%\end{center}
%\caption{Amplitude and Phase variabilities in FD}
%\label{fig:Fig3_handwrit}
%\end{figure}

\begin{figure}[H]
\begin{center}
\includegraphics[scale = 0.4, page = 4]{Plots/PublicData.pdf}
\end{center}
\caption{Amplitude and Phase variability in FD}
\label{fig:Fig3_pinch}
\end{figure}

% latex table generated in R 3.2.2 by xtable 1.8-0 package
% Tue Feb 16 15:11:00 2016
\begin{table}[t]
\centering
%\begin{tabular}{l*{7}{c}}
\begin{tabular}{lrrr|rrr}
  \hline
  \hline
  \multicolumn{4}{c}{Reference Interval} & \multicolumn{3}{c}{Molecular Fragment lengths} \\
  \hline
   Int  & molecules & pixels & length(kb) & min (kb) & avg (kb) & max (kb)\\ 
  \hline
  \hline
    0 &  66 & 391 & 81.62 & 65.67 & 81.07 & 92.79 \\ 
    1 & 208 & 89 & 18.68 & 13.27 & 18.64 & 21.55 \\ 
    2 & 467 & 284 & 59.40 & 43.92 & 59.24 & 69.39 \\ 
    3 & 734 & 67 & 13.94 & 9.59 & 13.86 & 17.34 \\ 
    4 & 895 & 43 & 9.03 & 6.47 & 8.99 & 11.48 \\ 
    5 & 849 & 24 & 5.04 & 2.14 & 5.02 & 5.90 \\ 
    6 & 939 & 59 & 12.34 & 6.58 & 12.29 & 15.55 \\ 
    7 & 1200 & 49 & 10.24 & 6.74 & 10.20 & 12.22 \\ 
    8 & 965 & 72 & 15.02 & 11.13 & 15.00 & 19.48 \\ 
    9 & 751 & 122 & 25.45 & 20.52 & 25.45 & 30.91 \\ 
   10 & 784 & 19 & 3.89 & 2.40 & 3.90 & 4.94 \\ 
   11 & 898 & 100 & 20.89 & 14.35 & 20.83 & 26.42 \\ 
   12 & 883 & 75 & 15.57 & 9.97 & 15.43 & 19.24 \\ 
   13 & 855 & 49 & 10.21 & 6.21 & 9.98 & 13.72 \\ 
   14 & 731 & 45 & 9.47 & 6.84 & 9.19 & 12.79 \\ 
   15 & 631 & 53 & 11.12 & 5.69 & 10.42 & 13.94 \\ 
   16 & 203 & 24 & 4.99 & 1.46 & 4.24 & 7.99 \\ 
   17 & 151 & 66 & 13.73 & 8.29 & 12.97 & 16.76 \\ 
   18 & 377 & 126 & 26.28 & 21.17 & 25.66 & 31.02 \\ 
   19 & 551 & 183 & 38.28 & 29.91 & 38.14 & 43.33 \\ 
   20 & 488 & 10 & 2.11 & 1.46 & 2.14 & 3.18 \\ 
   21 & 572 & 148 & 31.02 & 18.48 & 31.12 & 35.62 \\ 
   22 & 712 & 91 & 19.10 & 14.66 & 19.12 & 24.44 \\ 
   23 & 918 & 17 & 3.62 & 1.04 & 3.61 & 6.37 \\ 
   24 & 947 & 154 & 32.19 & 25.89 & 32.24 & 37.16 \\ 
   25 & 876 & 198 & 41.30 & 30.39 & 41.20 & 48.77 \\ 
   26 & 824 & 47 & 9.76 & 4.62 & 9.74 & 13.15 \\ 
   27 & 835 & 78 & 16.38 & 10.50 & 16.34 & 20.35 \\ 
   28 & 666 & 75 & 15.69 & 11.18 & 15.96 & 18.90 \\ 
   29 & 653 & 30 & 6.28 & 4.07 & 5.86 & 7.36 \\ 
   30 & 881 & 175 & 36.50 & 29.11 & 36.34 & 42.61 \\ 
   31 & 795 & 88 & 18.31 & 12.95 & 18.24 & 21.90 \\ 
   32 & 668 & 153 & 32.07 & 25.75 & 31.81 & 38.11 \\ 
   33 & 431 & 100 & 20.95 & 15.15 & 20.86 & 23.97 \\ 
   34 & 334 & 16 & 3.28 & 1.25 & 3.03 & 4.60 \\ 
   35 & 295 & 68 & 14.26 & 11.32 & 14.16 & 16.37 \\ 
   36 & 191 & 245 & 51.31 & 36.60 & 50.81 & 59.52 \\ 
   37 & 103 & 77 & 15.99 & 12.06 & 15.90 & 18.12 \\ 
   38 &  68 & 86 & 17.88 & 15.04 & 17.68 & 20.14 \\ 
  \hline
  \hline
\end{tabular}
\caption{Coverage of {\emph{M.florum}} data}
\label{tab:App_mftable}
\end{table}

           % Chapter 6

%% Essential LaTeX - Jon Warbrick 02/88
%   - Edited May, July 2000 -E. Benedict


% Copyright (C) Jon Warbrick and Plymouth Polytechnic 1989
% Permission is granted to reproduce the document in any way providing
% that it is distributed for free, except for any reasonable charges for
% printing, distribution, staff time, etc.  Direct commercial
% exploitation is not permitted.  Extracts may be made from this
% document providing an acknowledgment of the original source is
% maintained.

% NOTICE: This document has been edited for use in the UW-Madison
% Example Thesis file.


% counters used for the sample file example
\newcounter{savesection}
\newcounter{savesubsection}


% commands to do 'LaTeX Manual-like' examples

\newlength{\egwidth}\setlength{\egwidth}{0.42\textwidth}

\newenvironment{eg}{\begin{list}{}{\setlength{\leftmargin}%
{0.05\textwidth}\setlength{\rightmargin}{\leftmargin}}%
\item[]\footnotesize}{\end{list}}

\newenvironment{egbox}{\begin{minipage}[t]{\egwidth}}{\end{minipage}}

\newcommand{\egstart}{\begin{eg}\begin{egbox}}
\newcommand{\egmid}{\end{egbox}\hfill\begin{egbox}}
\newcommand{\egend}{\end{egbox}\end{eg}}

% one or two other commands
\newcommand{\fn}[1]{\hbox{\tt #1}}
\newcommand{\llo}[1]{(see line #1)}
\newcommand{\lls}[1]{(see lines #1)}
\newcommand{\bs}{$\backslash$}


\chapter{Essential \LaTeX{}}

This chapter introduces some key ideas behind \LaTeX{} and give you the ``essential''
items of information.  This chapter is an edited form of the paper
``Essential \LaTeX{}'' by Jon Warbrick, Plymouth Polytechnic.

\section{Introduction}
This document is an attempt to give you all the essential
information that you will need in order to use the \LaTeX{} Document
Preparation System.  Only very basic features are covered, and a
vast amount of detail has been omitted.  In a document of this size
it is not possible to include everything that you might need to know,
and if you intend to make extensive use of the program you should
refer to a more complete reference.  Attempting to produce complex
documents using only the information found below will require
much more work than it should, and will probably produce a less
than satisfactory result.

The main reference for \LaTeX{} is {\em The \LaTeX{} User's guide and
Reference Manual\/} by Leslie Lamport.  This contains most of the
information that you will ever need to know about the program, and
you will need access to a copy if you are to use \LaTeX{} seriously.
You should also consider getting a copy of {\em The \LaTeX{}
Companion\/} 

\section{How does \LaTeX{} work?}

In order to use \LaTeX{} you generate a file containing
both the text that you wish to print and instructions to tell \LaTeX{}
how you want it to appear.  You will normally create
this file using your system's text editor.  You can give the file any name you
like, but it should end ``\fn{.TEX}'' to identify the file's contents.
You then get \LaTeX{} to process the file, and it creates a
new file of typesetting commands; this has the same name as your file but
the ``\fn{.TEX}'' ending is replaced by ``\fn{.DVI}''.  This stands for
`{\it D\/}e{\it v\/}ice {\it I\/}ndependent' and, as the name implies, this file
can be used to create output on a range of printing devices.
Your {\em local guide\/} will go into more detail.

Rather than encourage you to dictate exactly how your document
should be laid out, \LaTeX{} instructions allow you describe its
{\em logical structure\/}.  For example, you can think of a quotation
embedded within your text as an element of this logical structure: you would
normally expect a quotation to be displayed in a recognisable style to set it
off from the rest of the text.
A human typesetter would recognise the quotation and handle
it accordingly, but since \LaTeX{} is only a computer program it requires
your help.  There are therefore \LaTeX{} commands that allow you to
identify quotations and as a result allow \LaTeX{} to typeset them correctly.

Fundamental to \LaTeX{} is the idea of a {\em document style\/} that
determines exactly how a document will be formatted.  \LaTeX{} provides
standard document styles that describe how standard logical structures
(such as quotations) should be formatted.  You may have to supplement
these styles by specifying the formatting of logical structures
peculiar to your document, such as mathematical formulae.  You can
also modify the standard document styles or even create an entirely
new one, though you should know the basic principles of typographical
design before creating a radically new style.

There are a number of good reasons for concentrating on the logical
structure rather than on the appearance of a document.  It prevents
you from making elementary typographical errors in the mistaken
idea that they improve the aesthetics of a document---you should
remember that the primary function of document design is to make
documents easier to read, not prettier.  It is more flexible, since
you only need to alter the definition of the quotation style
to change the appearance of all the quotations in a document.  Most
important of all, logical design encourages better writing.
A visual system makes it easier to create visual effects rather than
a coherent structure; logical design encourages you to concentrate on
your writing and makes it harder to use formatting as a substitute
for good writing.

\section{A Sample \LaTeX{} file}


\begin{figure} %---------------------------------------------------------------
{\singlespace\tt\footnotesize\begin{verbatim}
 1: % SMALL.TEX -- Released 5 July 1985
 2: % USE THIS FILE AS A MODEL FOR MAKING YOUR OWN LaTeX INPUT FILE.
 3: % EVERYTHING TO THE RIGHT OF A  %  IS A REMARK TO YOU AND IS IGNORED
 4: % BY LaTeX.
 5: %
 6: % WARNING!  DO NOT TYPE ANY OF THE FOLLOWING 10 CHARACTERS EXCEPT AS
 7: % DIRECTED:        &   $   #   %   _   {   }   ^   ~   \
 8:
 9: \documentclass[11pt,a4]{article}  % YOUR INPUT FILE MUST CONTAIN THESE
10: \begin{document}                  % TWO LINES PLUS THE \end COMMAND AT
11:                                   % THE END
12:
13: \section{Simple Text}          % THIS COMMAND MAKES A SECTION TITLE.
14:
15: Words are separated by one or    more      spaces.  Paragraphs are
16:     separated by one or more blank lines.  The output is not affected
17: by adding extra spaces or extra blank lines to the input file.
18:
19:
20: Double quotes are typed like this: ``quoted text''.
21: Single quotes are typed like this: `single-quoted text'.
22:
23: Long dashes are typed as three dash characters---like this.
24:
25: Italic text is typed like this: {\em this is italic text}.
26: Bold   text is typed like this: {\bf this is  bold  text}.
27:
28: \subsection{A Warning or Two}        % THIS MAKES A SUBSECTION TITLE.
29:
30: If you get too much space after a mid-sentence period---abbreviations
31: like etc.\ are the common culprits)---then type a backslash followed by
32: a space after the period, as in this sentence.
33:
34: Remember, don't type the 10 special characters (such as dollar sign and
35: backslash) except as directed!  The following seven are printed by
36: typing a backslash in front of them:  \$  \&  \#  \%  \_  \{  and  \}.
37: The manual tells how to make other symbols.
38:
39: \end{document}                    % THE INPUT FILE ENDS LIKE THIS
\end{verbatim}  }

\caption{A Sample \LaTeX{} File}\label{fig:sample}

\end{figure} %-----------------------------------------------------------------



Have a look at the example \LaTeX{} file in Figure~\ref{fig:sample}.  It
is a slightly modified copy of the standard \LaTeX{} example file
\fn{SMALL.TEX}.  The line numbers down the left-hand side
are not part of the file, but have been added to make it easier to
identify various portions.

Try entering this file (without the line numbers), save the text as \fn{small.tex},
next run \LaTeX{} on it, and then view the output:

{\tt \singlespace\begin{verbatim}
% latex small
% xdvi small               # displays the output on the screen
% dvips -o small.ps small  # to create a PostScript file, small.ps
% lp -d<printer> small.ps  # to print
\end{verbatim}}

\subsection{Running Text}

Most documents consist almost entirely of running text---words formed
into sentences, which are in turn formed into paragraphs---and the example file
is no exception. Describing running text poses no problems, you just type
it in naturally. In the output that it produces, \LaTeX{} will fill
lines and adjust the
spacing between words to give tidy left and right margins.
The spacing and distribution of the words in your input
file will have no effect at all on the eventual output.
Any number of spaces in your input file
are treated as a single space by \LaTeX{}, it also regards the
end of each line as a space between words \lls{15--17}.
A new paragraph is
indicated by a blank line in your input file, so don't leave
any blank lines unless you really wish to start a paragraph.

\LaTeX{} reserves a number of the less common keyboard characters for its
own use. The ten characters
\begin{quote}\begin{verbatim}
#  $  %  &  ~  _  ^  \  {  }
\end{verbatim}\end{quote}
should not appear as part of your text, because if they do
\LaTeX{} will get confused.

\subsection{\LaTeX{} Commands}

There are a number of words in the file that start `\verb|\|' \lls{9,
10 and 13}.  These are \LaTeX{} {\em commands\/} and they describe
the structure of your document. There are a number of things that you
should realize about these commands:
\begin{itemize}

\item All \LaTeX{} commands consist of a `\verb|\|' followed by one or more
characters.

\item \LaTeX{} commands should be typed using the correct mixture of upper- and
lower-case letters.  \verb|\BEGIN| is {\em not\/} the same as \verb|\begin|.

\item Some commands are placed within your text.  These are used to
switch things, like different typestyles, on and off. The \verb|\em|
command is used like this to emphasize text, normally by changing to
an {\it italic\/} typestyle \llo{25}.  The command and the text are
always enclosed between `\verb|{|' and `\verb|}|'---the `\verb|{\em|'
turns the effect on and and the `\verb|}|' turns it off.

\item There are other commands that look like
\begin{quote}\begin{verbatim}
\command{text}
\end{verbatim}\end{quote}
In this case the text is called the ``argument'' of the command.  The
\verb|\section| command is like this \llo{13}.
Sometimes you have to use curly brackets `\verb|{}|' to enclose the argument,
sometimes square brackets `\verb|[]|', and sometimes both at once.
There is method behind this apparent madness, but for the
time being you should be sure to copy the commands exactly as given.

\item When a command's name is made up entirely of letters, you must make sure
that the end of the command is marked by something that isn't a letter.
This is usually either the opening bracket around the command's argument, or
it's a space.  When it's a space, that space is always ignored by \LaTeX. We
will see later that this can sometimes be a problem.

\end{itemize}

\subsection{Overall structure}

There are some \LaTeX{} commands that must appear in every document.
The actual text of the document always starts with a
\verb|\begin{document}| command and ends with an \verb|\end{document}|
command \lls{10 and 39}.  Anything that comes after the \break
\verb|\end{document}| command is ignored.  Everything that comes
before the \break\verb|\begin{document}| command is called the
{\em preamble\/}. The preamble can only contain \LaTeX{} commands
to describe the document's style.

One command that must appear in the preamble is the
\verb|\documentclass| command \llo{9}.  This command specifies the
overall style for the document.  Our example file is a simple
technical document, and uses the {\tt article\/} class.  The document
you are reading was produced with the {\tt withesis\/} class. There
are other classes that you can use, as you will find out later on in
this document.

\subsection{Other Things to Look At}

\LaTeX{} can print both opening and closing quote characters, and can manage
either of these either single or double.  To do this it uses the two quote
characters from your keyboard: {\tt `} and {\tt '}. You will probably think of
{\tt '} as the ordinary single quote character which probably looks like
{\tt\symbol{'23}} or {\tt\symbol{'15}} on your keyboard,

and {\tt `} as a ``funny'' character that probably appears as
{\tt\symbol{'22}}. You type these characters once for single quote
\llo{21},  and twice for double quotes \llo{20}. The double quote
character {\tt "} itself is almost never used and should not be used
unless you want your text to look "funny" (compare the quote in the
previous sentence).

\LaTeX{} can produce three different kinds of dashes.
A long dash, for use as a punctuation symbol, as is typed as three dash
characters in a row, like this `\verb|---|' \llo{23}.  A shorter dash,
used between numbers as in `10--20', is typed as two dash
characters in a row, while a single dash character is used as a hyphen.

From time to time you will need to include one or more of the \LaTeX{}
special symbols in your text.  Seven of them can be printed by
making them into commands by proceeding them by backslash
\llo{36}.  The remaining three symbols can be produced by more
advanced commands, as can symbols that do not appear on your keyboard
such as \dag, \ddag, \S, \pounds, \copyright, $\sharp$ and $\clubsuit$.

It is sometimes useful to include comments in a \LaTeX{} file, to remind
you of what you have done or why you did it.  Everything to the
right of a \verb|%| sign is ignored by \LaTeX{}, and so it can
be used to introduce a comment.

\section{Document Classes and Class Options}\label{sec:styles}

There are four standard document classes available in \LaTeX:
\nobreak

\begin{description}

\item[{\tt article}]  intended for short documents and articles for publication.
Articles do not have chapters, and when \verb|\maketitle| is used to generate

a title (see Section~\ref{sec:title}) it appears at the top of the first page

rather than on a page of its own.

\item[{\tt report}] intended for longer technical documents.
It is similar to
{\tt article}, except that it contains chapters and the title appears on a page
of its own.

\item[{\tt book}] intended as a basis for book publication.  Page layout is
adjusted assuming that the output will eventually be used to print on
both sides of the paper.

\item[{\tt letter}]  intended for producing personal letters.  This style
will allow you to produce all the elements of a well laid out letter:
addresses, date, signature, etc.
\end{description}

An additional document class, the one used for this document and for
University of Wisconsin--Madison theses, is \fn{withesis}.


These standard classes can be modified by a number of {\em class
options\/}. They appear in square brackets after the
\verb|\documentclass| command. Only one class can ever be used but
you can have more than one class option, in which case their names
should be separated by commas.  The standard style options are:
\begin{description}

\item[{\tt 11pt}]  prints the document using eleven-point type for the running
 text
rather that the ten-point type normally used. Eleven-point type is about
ten percent larger than ten-point.

\item[{\tt 12pt}]  prints the document using twelve-point type for the running
 text
rather than the ten-point type normally used. Twelve-point type is about
twenty percent larger than ten-point.

\item[{\tt twoside}]  causes documents in the article or report styles to be
formatted for printing on both sides of the paper.  This is the default for the
book style.

\item[{\tt twocolumn}] produces two column on each page.

\item[{\tt titlepage}]  causes the \verb|\maketitle| command to generate a
title on a separate page for documents in the \fn{article} style.
A separate page is always used in both the \fn{report} and \fn{book} styles.

\end{description}

The University of Wisconsin--Madison thesis style, \fn{withesis} also
has some class options defined.  These class options are for
ten-point type (\fn{10pt}), tweleve-point type (\fn{12pt}), two-sided
printing (\fn{twoside}), Master Thesis margins (\fn{msthesis}) and an
option to print a small black box on lines which exceed the margins
(\fn{margincheck}).

\section{Environments}

We mentioned earlier the idea of identifying a quotation to \LaTeX{} so that
it could arrange to typeset it correctly. To do this you enclose the
quotation between the commands \verb|\begin{quotation}| and
\verb|\end{quotation}|.
This is an example of a \LaTeX{} construction called an {\em environment\/}.
A number of
special effects are obtained by putting text into particular environments.

\subsection{Quotations}

There are two environments for quotations: \fn{quote} and \fn{quotation}.
\fn{quote} is used either for a short quotation or for a sequence of
short quotations separated by blank lines:
\egstart\singlespace
\begin{verbatim}
US presidents ... remarks:
\begin{quote}
The buck stops here.

I am not a crook.
\end{quote}
\end{verbatim}
\egmid%
US presidents have been known for their pithy remarks:
\begin{quote}
The buck stops here.

I am not a crook.
\end{quote}
\egend

Use the \fn{quotation} environment for quotations that consist of more
than one paragraph.  Paragraphs in the input are separated by blank
lines as usual:
\egstart\singlespace
\begin{verbatim}

Here is some advice to remember:
\begin{quotation}
Environments for making
...other things as well.

Many problems
...environments.
\end{quotation}
\end{verbatim}
\egmid%
Here is some advice to remember:
\begin{quotation}
Environments for making quotations
can be used for other things as well.

Many problems can be solved by
novel applications of existing
environments.
\end{quotation}
\egend

\subsection{Centering and Flushing}

Text can be centered on the page by putting it within the \fn{center}
environment, and it will appear flush against the left or right margins if it
is placed within the \fn{flushleft} or \fn{flushright} environments.

Text within these environments will be formatted in the normal way, in
{\samepage
particular the ends of the lines that you type are just regarded as spaces.  To
indicate a ``newline'' you need to type the \verb|\\| command.  For example:
\egstart\singlespace
\begin{verbatim}
\begin{center}
one
two
three \\
four \\
five
\end{center}

\end{verbatim}
\egmid%
\begin{center}

one
two
three \\
four \\

five
\end{center}
\egend
}

\subsection{Lists}

There are three environments for constructing lists.  In each one each new
item is begun with an \verb|\item| command.  In the \fn{itemize} environment
the start of each item is given a marker, in the \fn{enumerate}
environment each item is marked by a number.  These environments can be nested
within each other in which case the amount of indentation used
is adjusted accordingly:
\egstart\singlespace

\begin{verbatim}
\begin{itemize}
\item Itemized lists are handy.
\item However, don't forget
  \begin{enumerate}
  \item The `item' command.
  \item The `end' command.
  \end{enumerate}
\end{itemize}
\end{verbatim}
\egmid%
\begin{itemize}
\item Itemized lists are handy.
\item However, don't forget
  \begin{enumerate}
  \item The `item' command.
  \item The `end' command.
  \end{enumerate}
\end{itemize}
\egend


The third list making environment is \fn{description}.  In a description you
specify the item labels inside square brackets after the \verb|\item| command.
For example:
\egstart\singlespace
\begin{verbatim}
Three animals that you should
know about are:
\begin{description}
  \item[gnat] A small
            animal...
  \item[gnu] A large
           animal...
  \item[armadillo] A ...
\end{description}
\end{verbatim}
\egmid%
Three animals that you should
know about are:
\begin{description}
  \item[gnat] A small animal that causes no end of trouble.
  \item[gnu] A large animal that causes no end of trouble.
  \item[armadillo] A medium-sized animal.
\end{description}
\egend

\subsection{Tables}

Because \LaTeX{} will almost always convert a sequence of spaces
into a single space, it can be rather difficult to lay out tables.
See what happens in this example
 \nolinebreak
\begin{eg}
\begin{minipage}[t]{0.55\textwidth} \singlespace
\begin{verbatim}
\begin{flushleft}
Income  Expenditure Result   \\
20s 0d  19s 11d     happiness \\
20s 0d  20s 1d      misery  \\
\end{flushleft}
\end{verbatim}
\end{minipage}
\begin{minipage}[t]{0.3\textwidth}
\begin{flushleft}
Income  Expenditure Result   \\
20s 0d  19s 11d     happiness \\
20s 0d  20s 1d      misery  \\
\end{flushleft}
\end{minipage}
\end{eg}

The \fn{tabbing} environment overcomes this problem. Within it you
set tabstops and tab to them much like you do on a typewriter.
Tabstops are set with the \verb|\=| command, and the \verb|\>|
command moves to the next stop.  The \verb|\\| command is used to
separate each line.  A line that ends \verb|\kill| produces no
output, and can be used to set tabstops:
\nolinebreak
\begin{eg}
\begin{minipage}[t]{0.6\textwidth}
\singlespace
\begin{verbatim}
\begin{tabbing}
Income \=Expenditure \=    \kill
Income \>Expenditure \>Result \\
20s 0d \>19s 11d \>Happiness \\
20s 0d \>20s 1d  \>Misery    \\
\end{tabbing}
\end{verbatim}
\end{minipage}
\vspace{1ex}
\begin{minipage}[t]{0.35\textwidth}
\begin{tabbing}
\singlespace
Income \=Expenditure \=    \kill
Income \>Expenditure \>Result \\
20s 0d \>19s 11d \>Happiness \\
20s 0d \>20s 1d  \>Misery    \\
\end{tabbing}
\end{minipage}
\end{eg}

Unlike a typewriter's tab key, the \verb|\>| command always moves to the next
tabstop in sequence, even if this means moving to the left.  This can cause
text to be overwritten if the gap between two tabstops is too small.

\subsection{Verbatim Output}

Sometimes you will want to include text exactly as it appears on a terminal
screen.  For example, you might want to include part of a computer program.
Not only do you want \LaTeX{} to stop playing around with the layout of your
text, you also want to be able to type all the characters on your keyboard
without confusing \LaTeX. The \fn{verbatim} environment has this effect:

\egstart

\begin{flushleft}\singlespace
\verb|The section of program in|  \\
 \verb|question is :|\\
 \verb|\begin{verbatim}|           \\
\verb|{ this finds %a & %b }|     \\[2ex]

\verb|for i := 1 to 27 do|        \\
\ \ \ \verb|begin|                \\
\ \ \ \verb|table[i] := fn(i);|   \\

\ \ \ \verb|process(i)|           \\
\ \ \ \verb|end;|                 \\
\verb|\end{verbatim}|
\end{flushleft}
\egmid%
The section of program in
question is:
\begin{verbatim}
{ this finds %a & %b }

for i := 1 to 27 do
   begin
   table[i] := fn(i);
   process(i)
   end;

\end{verbatim}
\egend

The \fn{withesis} document style also provides the command {\tt \verb|\verbatimfile{foo.fe}|}
which will read in the file {\tt foo.fe} into the document in \fn{verbatim} format with
the font \verb|\tt|.  See Appendix~\ref{matlab} for an example.

\section{Type Styles}

We have already come across the \verb|\em| command for changing
typeface.  Here is a full list of the available typefaces:
\begin{quote}\singlespace\begin{tabbing}
\verb|\sc|~~ \= \sc Small Caps~~~ \= \verb|\sc|~~ \= \sc Small Caps~~~
                                  \= \verb|\sc|~~ \=                   \kill
\verb|\rm|   \> \rm Roman         \> \verb|\it|   \> \it Italic
                                  \> \verb|\sc|   \> \sc Small Caps    \\
\verb|\em|   \> \em Emphatic      \> \verb|\sl| \> \sl Slanted
                                  \> \verb|\tt|   \> \tt Typewriter     \\
\verb|\bf|   \> \bf Boldface      \> \verb|\sf| \> \sf Sans Serif
\end{tabbing}\end{quote}

Remember that these commands are used {\em inside\/} a pair of braces to limit
the amount of text that they effect.  In addition to the eight typeface
commands, there are a set of commands that alter the size of the type.  These
commands are:
\begin{quotation}\singlespace\begin{tabbing}
\verb|\footnotesize|~~ \= \verb|\footnotesize|~~ \= \verb|\footnotesize| \=
 \kill
\verb|\tiny|           \> \verb|\small|          \> \verb|\large|        \>
\verb|\huge|  \\
\verb|\scriptsize|     \> \verb|\normalsize|     \> \verb|\Large|        \>
\verb|\Huge|  \\
\verb|\footnotesize|   \>                        \> \verb|\LARGE|
\end{tabbing}\end{quotation}

\section{Sectioning Commands and Tables of Contents}
\label{ess:sectioning}

Technical documents, like this one, are often divided into sections.
Each section has a heading containing a title and a number for easy
reference.  \LaTeX{} has a series of commands that will allow you to identify
different sorts of sections.  Once you have done this \LaTeX{} takes on the
responsibility of laying out the title and of providing the numbers.

The commands that you can use are:
\begin{quote}\singlespace\begin{tabbing}
\verb|\subsubsection| \= \verb|\subsubsection|~~~~~~~~~~ \=           \kill
\verb|\chapter|       \> \verb|\subsection|    \> \verb|\paragraph|    \\
\verb|\section|       \> \verb|\subsubsection| \> \verb|\subparagraph| \\
\end{tabbing}\end{quote}
The naming of these last two is unfortunate, since they do not really have
anything to do with `paragraphs' in the normal sense of the word; they are just
lower levels of section.  In most document styles, headings made with
\verb|\paragraph| and \verb|\subparagraph| are not numbered.  \verb|\chapter|
is not available in document style \fn{article}.  The commands should be used
in the order given, since sections are numbered within chapters, subsections
within sections, etc.

A seventh sectioning command, \verb|\part|, is also available.  Its use is
always optional, and it is used to divide a large document into series of
parts.  It does not alter the numbering used for any of the other commands.

Including the command \verb|\tableofcontents| in you document will cause a
contents list to be included, containing information collected from the various
sectioning commands.  You will notice that each time your document is run
through \LaTeX{} the table of contents is always made up of the headings from
the previous version of the document.  This is because \LaTeX{} collects
information for the table as it processes the document, and then includes it
the next time it is run.  This can sometimes mean that the document has to be
processed through \LaTeX{} twice to get a correct table of contents.

\section{Producing Special Symbols}

You can include in you \LaTeX{} document a wide range of symbols that do not
appear on you your keyboard. For a start, you can add an accent to any letter:
\begin{quote}\singlespace\begin{tabbing}

\t{oo} \= \verb|\t{oo}|~~~ \=
\t{oo} \= \verb|\t{oo}|~~~ \=
\t{oo} \= \verb|\t{oo}|~~~ \=
\t{oo} \= \verb|\t{oo}|~~~ \=
\t{oo} \= \verb|\t{oo}|~~~ \=
\t{oo} \=                       \kill

\a`{o} \> \verb|\`{o}|  \> \~{o}  \> \verb|\~{o}|  \> \v{o}  \> \verb|\v{o}| \>
\c{o}  \> \verb|\c{o}|  \> \a'{o} \> \verb|\'{o}|  \\
\a={o} \> \verb|\={o}|  \> \H{o}  \> \verb|\H{o}|  \> \d{o}  \> \verb|\d{o}| \>
\^{o}  \> \verb|\^{o}|  \> \.{o}  \> \verb|\.{o}|  \\
\t{oo} \> \verb|\t{oo}| \> \b{o}  \> \verb|\b{o}|  \\  \"{o} \> \verb|\"{o}| \>
\u{o}  \> \verb|\u{o}|  \\
\end{tabbing}\end{quote}

A number of other symbols are available, and can be used by including the
following commands:
\begin{quote}\singlespace\begin{tabbing}

\LaTeX~\= \verb|\copyright|~~~~ \= \LaTeX~\= \verb|\copyright|~~~~ \=
\LaTeX~\=  \kill

\dag       \> \verb|\dag|       \> \S     \> \verb|\S|     \>
\copyright \> \verb|\copyright| \\
\ddag      \> \verb|\ddag|      \> \P     \> \verb|\P|     \>
\pounds    \> \verb|\pounds|    \\
\oe        \> \verb|\oe|        \> \OE    \> \verb|\OE|    \>
\ae        \> \verb|\AE|        \\
\AE        \> \verb|\AE|        \> \aa    \> \verb|\aa|    \>
\AA        \> \verb|\AA|        \\
\o         \> \verb|\o|         \> \O     \> \verb|\O|     \>
\l         \> \verb|\l|         \\
\L         \> \verb|\E|         \> \ss    \> \verb|\ss|    \>
?`         \> \verb|?`|         \\
!`         \> \verb|!`|         \> \ldots \> \verb|\ldots| \>
\LaTeX     \> \verb|\LaTeX|     \\
\end{tabbing}\end{quote}
There is also a \verb|\today| command that prints the current date. When you
use these commands remember that \LaTeX{} will ignore any spaces that
follow them, so that you can type `\verb|\pounds 20|' to get `\pounds 20'.
However, if you type `\verb|LaTeX is wonderful|' you will get `\LaTeX is
wonderful'---notice the lack of space after \LaTeX.
To overcome this problem you can follow any of these commands by a
pair of empty brackets and then any spaces that you wish to include,
and you will see that
\verb|\LaTeX{} really is wonderful!| (\LaTeX{} really is wonderful!).

\section{Titles}\label{sec:title}

Most documents have a title.  To title a \LaTeX{} document, you include the
following commands in your document, usually just after
\verb|begin{document}|.
\begin{quote}\singlespace\begin{verbatim}
\title{required title}
\author{required author}
\date{required date}
\maketitle
\end{verbatim}\end{quote}
If there are several authors, then their names should be separated by
\verb|\and|; they can also be separated by \verb|\\| if you want them to be
centred on different lines.  If the \verb|\date| command is left out, then the
current date will be printed.
\egstart
\singlespace
\begin{verbatim}
\title{Essential \LaTeX}
\author{J Warbrick \and An Other}
\date{14th February 1988}
\maketitle
\end{verbatim}
\egmid
\begin{center}
{\normalsize Essential \LaTeX}\\[4ex]
J Warbrick\hspace{1em}A N Other\\[2ex]
14th February 1988
\end{center}
\egend

The exact appearance of the title varies depending on
the document style.  In styles \fn{report} and \fn{book} the title appears on a
page of its own. In the \fn{article} style it normally appears at the top
of the first page, the style option \fn{titlepage} will alter this (see
Section~\ref{sec:styles}).  In the \fn{withesis} style, the title is created on a
seperate page in the format appropriate to a UW-Madison thesis or dissertation.

\section{Errors}

When you create a new input file for \LaTeX{} you will probably make mistakes.
Everybody does, and it's nothing to be worried about.  As with most computer
programs, there are two sorts of mistake that you can make: those that \LaTeX{}
notices and those that it doesn't.  To take a rather silly example, since
\LaTeX{} doesn't understand what you are saying it isn't going to be worried if
you mis-spell some of the words in your text.  You will just have to accurately
proof-read your printed output.  On the other hand, if you mis-spell one of
the environment names in your file then \LaTeX won't know what you want it
to do.

When this sort of thing happens, \LaTeX{} prints an error message on your
terminal screen and then stops and waits for you to take some action.
Unfortunately, the error messages that it produces are rather user-unfriendly
and a little frightening.  Nevertheless, if you know where to look they
will probably tell you where the error is and went wrong.

Consider what would happen if you mistyped \verb|\begin{itemize}| so that it
became \break\verb|\begin{itemie}|.  When \LaTeX{} processes this instruction, it
displays the following on your terminal:
\begin{quote}\singlespace\begin{verbatim}
LaTeX error.  See LaTeX manual for explanation.
              Type  H <return>  for immediate help.
! Environment itemie undefined.
\@latexerr ...for immediate help.}\errmessage {#1}
                                                  \endgroup
l.140 \begin{itemie}

?
\end{verbatim}\end{quote}
After typing the `?' \LaTeX{} stops and waits for you to tell it what to do.

The first two lines of the message just tell you that the error was detected by
\LaTeX{}. The third line, the one that starts `!' is the {\em error indicator}.
 It
tells you what the problem is, though until you have had some experience of
\LaTeX{} this may not mean a lot to you.  In this case it is just telling you
that it doesn't recognise an environment called \fn{itemie}.
The next two lines tell you what
\LaTeX{} was doing when it found the error, they are irrelevant at the moment
and can be ignored. The final line is called the {\em error locator}, and is
a copy of the line from your file that caused the problem.
It start with a line number to help you to find it in your file, and
if the error was in the middle of a line it will be shown
broken at the point where \LaTeX{} realised that there was an error.  \LaTeX{}
can sometimes pass the point where the real error is before discovering that
something is wrong, but it doesn't usually get very far.

At this point you could do several things.  If you knew enough about \LaTeX{}
you might be able to fix the problem, or you could type `X' and press the
return key to stop \LaTeX{} running while you go and correct the error.  The
best thing to do, however, is just to press the return key.  This will allow
\LaTeX{} to go on running as if nothing had happened.  If you have made one
mistake, then you have probably made several and you may as well try to find
them all in one go.  It's much more efficient to do it this way than to run
\LaTeX{} over and over again fixing one error at a time. Don't worry about
remembering what the errors were---a copy of all the error messages is being
saved in a {\em log\/} file so that you can look at them afterwards.

If you look at the line that caused the error it's normally obvious what the
problem was.  If you can't work out what you problem is look at the hints
below, and if they don't help consult Chapter~6 of the manual~\cite{lamport}.
  It contains a
list of all of the error messages that you are likely to encounter together with
some hints as to what may have caused them.

Some of the most common mistakes that cause errors are
\begin{itemize}
\item A mispelled command or environment name.
\item Improperly matched `\verb|{|' and `\verb|}|'---remember that they should
 always
come in pairs.
\item Trying to use one of the ten special characters \verb|# $ % & _ { } ~ ^|
and \verb|\| as an ordinary printing symbol.
\item A missing \verb|\end| command.
\item A missing command argument (that's the bit enclosed in '\verb|{|' and
`\verb|}|').
\end{itemize}

One error can get \LaTeX{} so confused that it reports a series of spurious
errors as a result.  If you have an error that you understand, followed by a
series that you don't, then try correcting the first error---the rest
may vanish as if by magic.
Sometimes \LaTeX{} may write a {\tt *} and stop without an error message.  This
is normally caused by a missing \verb|\end{document}| command, but other errors
can cause it.  If this happens type \verb|\stop| and press the return key.

Finally, \LaTeX{} will sometimes print {\em warning\/} messages.  They report
problems that were not bad enough to cause \LaTeX{} to stop processing, but
nevertheless may require investigation.  The most common problems are
`overfull' and `underfull' lines of text.  A message like:
\begin{quote}\footnotesize\begin{verbatim}
Overfull \hbox (10.58649pt too wide) in paragraph at lines 172--175
[]\tenrm Mathematical for-mu-las may be dis-played. A dis-played
\end{verbatim}\end{quote}
indicates that \LaTeX{} could not find a good place to break a line when laying
out a paragraph.  As a result, it was forced to let the line stick out into the
right-hand margin, in this case by 10.6 points.  Since a point is about 1/72nd
of an inch this may be rather hard to see, but it will be there none the less.

This particular problem happens because \LaTeX{} is rather fussy about line
breaking, and it would rather generate a line that is too long than generate a
paragraph that doesn't meet its high standards.  The simplest way around the
problem is to enclose the entire offending paragraph between
\verb|\begin{sloppypar}| and \verb|\end{sloppypar}| commands.  This tells
\LaTeX{} that you are happy for it to break its own rules while it is working on
that particular bit of text.

Alternatively, messages about ``Underfull \verb|\hbox'es''| may appear.
These are lines that had to have more space inserted between
words than \LaTeX{} would have liked.  In general there is not much that you
can do about these.  Your output will look fine, even if the line looks a bit
stretched.  About the only thing you could do is to re-write the offending
paragraph!

\section{A Final Reminder}

You now know enough \LaTeX{} to produce a wide range of documents.  But this
document has only scratched the surface of the
things that \LaTeX{} can do.  This entire document was itself produced with
\LaTeX{} (with no sticking things in or clever use of a photocopier) and even
it hasn't used all the features that it could.  From this you may get some
feeling for the power that \LaTeX{} puts at your disposal.

Please remember what was said in the introduction: if you {\bf do} have a
complex document to produce then {\bf go and read the manual}.  You will be
wasting your time if you rely only on what you have read here.
            % Edited ``Essential LaTeX'' by Jon Warbrick
%\chapter{Figures and Tables}\label{quad}
This chapter\footnote{Most of the text in this chapter's introduction is from {\em How to
\TeX{} a Thesis: The Purdue Thesis Styles}} shows some example ways of incorporating tables and figures into \LaTeX{}.
Special environments exist for tables and figures and are special because they are
allowed to {\em float}---that is, \LaTeX{} doesn't always put them in the exact place
that they occur in your input file.  An algorithm is used to place the floating environments,
or floats, at locations which are typographically correct.  This may cause endless frustration
if you want to have a figure or table occur at a specific location.  There are a few
methods for solving this.

You can exert some influence on \LaTeX{}'s float placement algorithm by using
{\em float position specifiers}.  These specifiers, listed below, tell \LaTeX{}
what you prefer.
\begin{tabbing}
{\tt hhhhhh} \= ``bottom'' \=  \kill
{\tt h}\> ``here'' \> do not move this object \\
{\tt p}\> ``page'' \> put this object on a page of floats \\
{\tt b}\> ``bottom'' \> put this object at the bottom of a page\\
{\tt t}\> ``top'' \> put this object at the top of a page\\
\end{tabbing}

Any combination of these can be used:
\begin{quote}\tt\singlespace\begin{verbatim}
\begin{figure}[htbp]
 ...
\caption{A Figure!}
\end{figure}
\end{verbatim}\end{quote}

In this example, we asked \LaTeX{} to ``put the figure `here' if possible.  If it
is not possible (according to the rule encoded in the float algorithm), put it on the
next float page.  A float page is a page which contains nothing but floating objects,
{\em e.g.} a page of nothing but figures or tables.  If this isn't possible, try to put it
at the `top' of a page.  The last thing to try is to put the figure at the `bottom' of
a page.''

The remainder of this chapter deals with some examples of what to put into the figure,
the ellipsis (\ldots ) in the example above.

\section{Tables}
Table~\ref{pde.tab1} is an example table from the UW Math Department.
\begin{table}[htbp]
\centering
\caption{PDE solve times, $15^3+1$
equations.\label{pde.tab1}}
\begin{tabular}{||l|l|l|l|l|l||}\hline
Precond. & Time & Nonlinear & Krylov
& Function & Precond. \\
 & & Iterations & Iterations & calls & solves \\ \hline
None & 1260.9u & 3 & 26 & 30 & 0  \\
 &(21:09) & & & &  \\ \hline
FFT  & 983.4u & 2  & 5  & 8  & 7 \\
&(16:31) & & & & \\ \hline
\end{tabular}
\end{table}
The code to generate it is as follows:
\begin{quote}\tt\singlespace\begin{verbatim}
\begin{table}[htbp]
\centering
\caption{PDE solve times, $15^3+1$
equations.\label{pde.tab1}}
\begin{tabular}{||l|l|l|l|l|l||}\hline
Precond. & Time & Nonlinear & Krylov
& Function & Precond. \\
 & & Iterations & Iterations & calls & solves \\ \hline
None & 1260.9u & 3 & 26 & 30 & 0  \\
 &(21:09) & & & &  \\ \hline
FFT  & 983.4u & 2  & 5  & 8  & 7 \\
&(16:31) & & & & \\ \hline
\end{tabular}
\end{table}
\end{verbatim}\end{quote}

\section{Figures}
There are many different ways to incorporate figures into a \LaTeX{}
document.  \LaTeX{} has an internal {\tt picture} environment and
some programs will generate files which are in this format and can
be simply {\tt include}d.  In addition to \LaTeX{} native {\tt picture}
format, additional packages can be loaded in the {\tt\verb|\documentstyle|}
command (or using the {\tt input} command) to allow \LaTeX{} to process
non-native formats such as PostScript.

\subsection{\tt gnuplot}
The graph of Figure~\ref{gelfand.fig2}
 was created by gnuplot. For simple graphs this is a
 great utility.  For example, if you want a sin curve in your thesis
 try the following:
\begin{quote}\tt\singlespace\begin{verbatim}
 (terminal window): gnuplot
 (in gnuplot):
                 set terminal latex
                 set output "foo.tex"
                 plot sin(x)
                 quit
\end{verbatim}\end{quote}
This will generate a file called {\tt foo.tex} which can be read in
with the following statements.
\begin{figure}[htbp]
\centering
% GNUPLOT: LaTeX picture
\setlength{\unitlength}{0.240900pt}
\ifx\plotpoint\undefined\newsavebox{\plotpoint}\fi
\sbox{\plotpoint}{\rule[-0.175pt]{0.350pt}{0.350pt}}%
\begin{picture}(1500,900)(0,0)
%\tenrm
\sbox{\plotpoint}{\rule[-0.175pt]{0.350pt}{0.350pt}}%
\put(264,158){\rule[-0.175pt]{282.335pt}{0.350pt}}
\put(264,158){\rule[-0.175pt]{0.350pt}{151.526pt}}
\put(264,158){\rule[-0.175pt]{4.818pt}{0.350pt}}
%\put(242,158){\makebox(0,0)[r]{0}}
\put(1416,158){\rule[-0.175pt]{4.818pt}{0.350pt}}
\put(264,284){\rule[-0.175pt]{4.818pt}{0.350pt}}
%\put(242,284){\makebox(0,0)[r]{2}}
\put(1416,284){\rule[-0.175pt]{4.818pt}{0.350pt}}
\put(264,410){\rule[-0.175pt]{4.818pt}{0.350pt}}
%\put(242,410){\makebox(0,0)[r]{4}}
\put(1416,410){\rule[-0.175pt]{4.818pt}{0.350pt}}
\put(264,535){\rule[-0.175pt]{4.818pt}{0.350pt}}
%\put(242,535){\makebox(0,0)[r]{6}}
\put(1416,535){\rule[-0.175pt]{4.818pt}{0.350pt}}
\put(264,661){\rule[-0.175pt]{4.818pt}{0.350pt}}
%\put(242,661){\makebox(0,0)[r]{8}}
\put(1416,661){\rule[-0.175pt]{4.818pt}{0.350pt}}
\put(264,787){\rule[-0.175pt]{4.818pt}{0.350pt}}
%\put(242,787){\makebox(0,0)[r]{10}}
\put(1416,787){\rule[-0.175pt]{4.818pt}{0.350pt}}
\put(264,158){\rule[-0.175pt]{0.350pt}{4.818pt}}
%\put(264,113){\makebox(0,0){0}}
\put(264,767){\rule[-0.175pt]{0.350pt}{4.818pt}}
\put(411,158){\rule[-0.175pt]{0.350pt}{4.818pt}}
%\put(411,113){\makebox(0,0){0.5}}
\put(411,767){\rule[-0.175pt]{0.350pt}{4.818pt}}
\put(557,158){\rule[-0.175pt]{0.350pt}{4.818pt}}
%\put(557,113){\makebox(0,0){1}}
\put(557,767){\rule[-0.175pt]{0.350pt}{4.818pt}}
\put(704,158){\rule[-0.175pt]{0.350pt}{4.818pt}}
%\put(704,113){\makebox(0,0){1.5}}
\put(704,767){\rule[-0.175pt]{0.350pt}{4.818pt}}
\put(850,158){\rule[-0.175pt]{0.350pt}{4.818pt}}
%\put(850,113){\makebox(0,0){2}}
\put(850,767){\rule[-0.175pt]{0.350pt}{4.818pt}}
\put(997,158){\rule[-0.175pt]{0.350pt}{4.818pt}}
%\put(997,113){\makebox(0,0){2.5}}
\put(997,767){\rule[-0.175pt]{0.350pt}{4.818pt}}
\put(1143,158){\rule[-0.175pt]{0.350pt}{4.818pt}}
%\put(1143,113){\makebox(0,0){3}}
\put(1143,767){\rule[-0.175pt]{0.350pt}{4.818pt}}
\put(1290,158){\rule[-0.175pt]{0.350pt}{4.818pt}}
%\put(1290,113){\makebox(0,0){3.5}}
\put(1290,767){\rule[-0.175pt]{0.350pt}{4.818pt}}
\put(1436,158){\rule[-0.175pt]{0.350pt}{4.818pt}}
%\put(1436,113){\makebox(0,0){4}}
\put(1436,767){\rule[-0.175pt]{0.350pt}{4.818pt}}
\put(264,158){\rule[-0.175pt]{282.335pt}{0.350pt}}
\put(1436,158){\rule[-0.175pt]{0.350pt}{151.526pt}}
\put(264,787){\rule[-0.175pt]{282.335pt}{0.350pt}}
\put(100,472){\makebox(0,0)[l]{\shortstack{$\| u\|$}}}
\put(850,68){\makebox(0,0){$\lambda$}}
%\put(850,832){\makebox(0,0){plot}}
\put(264,158){\rule[-0.175pt]{0.350pt}{151.526pt}}
%\put(1306,722){\makebox(0,0)[r]{}}
%\put(1328,722){\rule[-0.175pt]{15.899pt}{0.350pt}}
\put(264,158){\usebox{\plotpoint}}
\put(264,158){\rule[-0.175pt]{6.304pt}{0.350pt}}
\put(290,159){\rule[-0.175pt]{6.304pt}{0.350pt}}
\put(316,160){\rule[-0.175pt]{6.304pt}{0.350pt}}
\put(342,161){\rule[-0.175pt]{6.304pt}{0.350pt}}
\put(368,162){\rule[-0.175pt]{6.304pt}{0.350pt}}
\put(394,163){\rule[-0.175pt]{6.304pt}{0.350pt}}
\put(420,164){\rule[-0.175pt]{5.644pt}{0.350pt}}
\put(444,165){\rule[-0.175pt]{5.644pt}{0.350pt}}
\put(467,166){\rule[-0.175pt]{5.644pt}{0.350pt}}
\put(491,167){\rule[-0.175pt]{5.644pt}{0.350pt}}
\put(514,168){\rule[-0.175pt]{5.644pt}{0.350pt}}
\put(538,169){\rule[-0.175pt]{5.644pt}{0.350pt}}
\put(561,170){\rule[-0.175pt]{5.644pt}{0.350pt}}
\put(585,171){\rule[-0.175pt]{6.384pt}{0.350pt}}
\put(611,172){\rule[-0.175pt]{6.384pt}{0.350pt}}
\put(638,173){\rule[-0.175pt]{6.384pt}{0.350pt}}
\put(664,174){\rule[-0.175pt]{6.384pt}{0.350pt}}
\put(691,175){\rule[-0.175pt]{6.384pt}{0.350pt}}
\put(717,176){\rule[-0.175pt]{6.384pt}{0.350pt}}
\put(744,177){\rule[-0.175pt]{5.862pt}{0.350pt}}
\put(768,178){\rule[-0.175pt]{5.862pt}{0.350pt}}
\put(792,179){\rule[-0.175pt]{5.862pt}{0.350pt}}
\put(816,180){\rule[-0.175pt]{5.862pt}{0.350pt}}
\put(841,181){\rule[-0.175pt]{5.862pt}{0.350pt}}
\put(865,182){\rule[-0.175pt]{5.862pt}{0.350pt}}
\put(889,183){\rule[-0.175pt]{4.371pt}{0.350pt}}
\put(908,184){\rule[-0.175pt]{4.371pt}{0.350pt}}
\put(926,185){\rule[-0.175pt]{4.371pt}{0.350pt}}
\put(944,186){\rule[-0.175pt]{4.371pt}{0.350pt}}
\put(962,187){\rule[-0.175pt]{4.371pt}{0.350pt}}
\put(980,188){\rule[-0.175pt]{4.371pt}{0.350pt}}
\put(998,189){\rule[-0.175pt]{4.371pt}{0.350pt}}
\put(1017,190){\rule[-0.175pt]{4.216pt}{0.350pt}}
\put(1034,191){\rule[-0.175pt]{4.216pt}{0.350pt}}
\put(1052,192){\rule[-0.175pt]{4.216pt}{0.350pt}}
\put(1069,193){\rule[-0.175pt]{4.216pt}{0.350pt}}
\put(1087,194){\rule[-0.175pt]{4.216pt}{0.350pt}}
\put(1104,195){\rule[-0.175pt]{4.216pt}{0.350pt}}
\put(1122,196){\rule[-0.175pt]{3.172pt}{0.350pt}}
\put(1135,197){\rule[-0.175pt]{3.172pt}{0.350pt}}
\put(1148,198){\rule[-0.175pt]{3.172pt}{0.350pt}}
\put(1161,199){\rule[-0.175pt]{3.172pt}{0.350pt}}
\put(1174,200){\rule[-0.175pt]{3.172pt}{0.350pt}}
\put(1187,201){\rule[-0.175pt]{3.172pt}{0.350pt}}
\put(1200,202){\rule[-0.175pt]{1.893pt}{0.350pt}}
\put(1208,203){\rule[-0.175pt]{1.893pt}{0.350pt}}
\put(1216,204){\rule[-0.175pt]{1.893pt}{0.350pt}}
\put(1224,205){\rule[-0.175pt]{1.893pt}{0.350pt}}
\put(1232,206){\rule[-0.175pt]{1.893pt}{0.350pt}}
\put(1240,207){\rule[-0.175pt]{1.893pt}{0.350pt}}
\put(1248,208){\rule[-0.175pt]{1.893pt}{0.350pt}}
\put(1256,209){\rule[-0.175pt]{1.245pt}{0.350pt}}
\put(1261,210){\rule[-0.175pt]{1.245pt}{0.350pt}}
\put(1266,211){\rule[-0.175pt]{1.245pt}{0.350pt}}
\put(1271,212){\rule[-0.175pt]{1.245pt}{0.350pt}}
\put(1276,213){\rule[-0.175pt]{1.245pt}{0.350pt}}
\put(1281,214){\rule[-0.175pt]{1.245pt}{0.350pt}}
\put(1286,215){\usebox{\plotpoint}}
\put(1288,216){\usebox{\plotpoint}}
\put(1289,217){\usebox{\plotpoint}}
\put(1291,218){\usebox{\plotpoint}}
\put(1292,219){\usebox{\plotpoint}}
\put(1294,220){\usebox{\plotpoint}}
\put(1295,221){\usebox{\plotpoint}}
\put(1295,222){\rule[-0.175pt]{0.361pt}{0.350pt}}
\put(1294,223){\rule[-0.175pt]{0.361pt}{0.350pt}}
\put(1292,224){\rule[-0.175pt]{0.361pt}{0.350pt}}
\put(1291,225){\rule[-0.175pt]{0.361pt}{0.350pt}}
\put(1289,226){\rule[-0.175pt]{0.361pt}{0.350pt}}
\put(1288,227){\rule[-0.175pt]{0.361pt}{0.350pt}}
\put(1284,228){\rule[-0.175pt]{0.964pt}{0.350pt}}
\put(1280,229){\rule[-0.175pt]{0.964pt}{0.350pt}}
\put(1276,230){\rule[-0.175pt]{0.964pt}{0.350pt}}
\put(1272,231){\rule[-0.175pt]{0.964pt}{0.350pt}}
\put(1268,232){\rule[-0.175pt]{0.964pt}{0.350pt}}
\put(1264,233){\rule[-0.175pt]{0.964pt}{0.350pt}}
\put(1258,234){\rule[-0.175pt]{1.273pt}{0.350pt}}
\put(1253,235){\rule[-0.175pt]{1.273pt}{0.350pt}}
\put(1248,236){\rule[-0.175pt]{1.273pt}{0.350pt}}
\put(1242,237){\rule[-0.175pt]{1.273pt}{0.350pt}}
\put(1237,238){\rule[-0.175pt]{1.273pt}{0.350pt}}
\put(1232,239){\rule[-0.175pt]{1.273pt}{0.350pt}}
\put(1227,240){\rule[-0.175pt]{1.273pt}{0.350pt}}
\put(1219,241){\rule[-0.175pt]{1.847pt}{0.350pt}}
\put(1211,242){\rule[-0.175pt]{1.847pt}{0.350pt}}
\put(1204,243){\rule[-0.175pt]{1.847pt}{0.350pt}}
\put(1196,244){\rule[-0.175pt]{1.847pt}{0.350pt}}
\put(1188,245){\rule[-0.175pt]{1.847pt}{0.350pt}}
\put(1181,246){\rule[-0.175pt]{1.847pt}{0.350pt}}
\put(1172,247){\rule[-0.175pt]{2.128pt}{0.350pt}}
\put(1163,248){\rule[-0.175pt]{2.128pt}{0.350pt}}
\put(1154,249){\rule[-0.175pt]{2.128pt}{0.350pt}}
\put(1145,250){\rule[-0.175pt]{2.128pt}{0.350pt}}
\put(1136,251){\rule[-0.175pt]{2.128pt}{0.350pt}}
\put(1128,252){\rule[-0.175pt]{2.128pt}{0.350pt}}
\put(1120,253){\rule[-0.175pt]{1.893pt}{0.350pt}}
\put(1112,254){\rule[-0.175pt]{1.893pt}{0.350pt}}
\put(1104,255){\rule[-0.175pt]{1.893pt}{0.350pt}}
\put(1096,256){\rule[-0.175pt]{1.893pt}{0.350pt}}
\put(1088,257){\rule[-0.175pt]{1.893pt}{0.350pt}}
\put(1080,258){\rule[-0.175pt]{1.893pt}{0.350pt}}
\put(1073,259){\rule[-0.175pt]{1.893pt}{0.350pt}}
\put(1063,260){\rule[-0.175pt]{2.208pt}{0.350pt}}
\put(1054,261){\rule[-0.175pt]{2.208pt}{0.350pt}}
\put(1045,262){\rule[-0.175pt]{2.208pt}{0.350pt}}
\put(1036,263){\rule[-0.175pt]{2.208pt}{0.350pt}}
\put(1027,264){\rule[-0.175pt]{2.208pt}{0.350pt}}
\put(1018,265){\rule[-0.175pt]{2.208pt}{0.350pt}}
\put(1009,266){\rule[-0.175pt]{2.168pt}{0.350pt}}
\put(1000,267){\rule[-0.175pt]{2.168pt}{0.350pt}}
\put(991,268){\rule[-0.175pt]{2.168pt}{0.350pt}}
\put(982,269){\rule[-0.175pt]{2.168pt}{0.350pt}}
\put(973,270){\rule[-0.175pt]{2.168pt}{0.350pt}}
\put(964,271){\rule[-0.175pt]{2.168pt}{0.350pt}}
\put(957,272){\rule[-0.175pt]{1.686pt}{0.350pt}}
\put(950,273){\rule[-0.175pt]{1.686pt}{0.350pt}}
\put(943,274){\rule[-0.175pt]{1.686pt}{0.350pt}}
\put(936,275){\rule[-0.175pt]{1.686pt}{0.350pt}}
\put(929,276){\rule[-0.175pt]{1.686pt}{0.350pt}}
\put(922,277){\rule[-0.175pt]{1.686pt}{0.350pt}}
\put(915,278){\rule[-0.175pt]{1.686pt}{0.350pt}}
\put(907,279){\rule[-0.175pt]{1.767pt}{0.350pt}}
\put(900,280){\rule[-0.175pt]{1.767pt}{0.350pt}}
\put(893,281){\rule[-0.175pt]{1.767pt}{0.350pt}}
\put(885,282){\rule[-0.175pt]{1.767pt}{0.350pt}}
\put(878,283){\rule[-0.175pt]{1.767pt}{0.350pt}}
\put(871,284){\rule[-0.175pt]{1.767pt}{0.350pt}}
\put(864,285){\rule[-0.175pt]{1.486pt}{0.350pt}}
\put(858,286){\rule[-0.175pt]{1.486pt}{0.350pt}}
\put(852,287){\rule[-0.175pt]{1.486pt}{0.350pt}}
\put(846,288){\rule[-0.175pt]{1.486pt}{0.350pt}}
\put(840,289){\rule[-0.175pt]{1.486pt}{0.350pt}}
\put(834,290){\rule[-0.175pt]{1.486pt}{0.350pt}}
\put(829,291){\rule[-0.175pt]{0.998pt}{0.350pt}}
\put(825,292){\rule[-0.175pt]{0.998pt}{0.350pt}}
\put(821,293){\rule[-0.175pt]{0.998pt}{0.350pt}}
\put(817,294){\rule[-0.175pt]{0.998pt}{0.350pt}}
\put(813,295){\rule[-0.175pt]{0.998pt}{0.350pt}}
\put(809,296){\rule[-0.175pt]{0.998pt}{0.350pt}}
\put(805,297){\rule[-0.175pt]{0.998pt}{0.350pt}}
\put(801,298){\rule[-0.175pt]{0.883pt}{0.350pt}}
\put(797,299){\rule[-0.175pt]{0.883pt}{0.350pt}}
\put(793,300){\rule[-0.175pt]{0.883pt}{0.350pt}}
\put(790,301){\rule[-0.175pt]{0.883pt}{0.350pt}}
\put(786,302){\rule[-0.175pt]{0.883pt}{0.350pt}}
\put(783,303){\rule[-0.175pt]{0.883pt}{0.350pt}}
\put(780,304){\rule[-0.175pt]{0.522pt}{0.350pt}}
\put(778,305){\rule[-0.175pt]{0.522pt}{0.350pt}}
\put(776,306){\rule[-0.175pt]{0.522pt}{0.350pt}}
\put(774,307){\rule[-0.175pt]{0.522pt}{0.350pt}}
\put(772,308){\rule[-0.175pt]{0.522pt}{0.350pt}}
\put(770,309){\rule[-0.175pt]{0.522pt}{0.350pt}}
\put(770,310){\usebox{\plotpoint}}
\put(769,311){\usebox{\plotpoint}}
\put(768,312){\usebox{\plotpoint}}
\put(767,314){\usebox{\plotpoint}}
\put(766,315){\usebox{\plotpoint}}
\put(765,316){\rule[-0.175pt]{0.350pt}{0.723pt}}
\put(766,320){\rule[-0.175pt]{0.350pt}{0.723pt}}
\put(767,323){\usebox{\plotpoint}}
\put(768,324){\usebox{\plotpoint}}
\put(769,325){\usebox{\plotpoint}}
\put(771,326){\usebox{\plotpoint}}
\put(772,327){\usebox{\plotpoint}}
\put(774,328){\usebox{\plotpoint}}
\put(775,329){\usebox{\plotpoint}}
\put(777,330){\rule[-0.175pt]{0.602pt}{0.350pt}}
\put(779,331){\rule[-0.175pt]{0.602pt}{0.350pt}}
\put(782,332){\rule[-0.175pt]{0.602pt}{0.350pt}}
\put(784,333){\rule[-0.175pt]{0.602pt}{0.350pt}}
\put(787,334){\rule[-0.175pt]{0.602pt}{0.350pt}}
\put(789,335){\rule[-0.175pt]{0.602pt}{0.350pt}}
\put(792,336){\rule[-0.175pt]{0.843pt}{0.350pt}}
\put(795,337){\rule[-0.175pt]{0.843pt}{0.350pt}}
\put(799,338){\rule[-0.175pt]{0.843pt}{0.350pt}}
\put(802,339){\rule[-0.175pt]{0.843pt}{0.350pt}}
\put(806,340){\rule[-0.175pt]{0.843pt}{0.350pt}}
\put(809,341){\rule[-0.175pt]{0.843pt}{0.350pt}}
\put(813,342){\rule[-0.175pt]{0.826pt}{0.350pt}}
\put(816,343){\rule[-0.175pt]{0.826pt}{0.350pt}}
\put(819,344){\rule[-0.175pt]{0.826pt}{0.350pt}}
\put(823,345){\rule[-0.175pt]{0.826pt}{0.350pt}}
\put(826,346){\rule[-0.175pt]{0.826pt}{0.350pt}}
\put(830,347){\rule[-0.175pt]{0.826pt}{0.350pt}}
\put(833,348){\rule[-0.175pt]{0.826pt}{0.350pt}}
\put(837,349){\rule[-0.175pt]{1.084pt}{0.350pt}}
\put(841,350){\rule[-0.175pt]{1.084pt}{0.350pt}}
\put(846,351){\rule[-0.175pt]{1.084pt}{0.350pt}}
\put(850,352){\rule[-0.175pt]{1.084pt}{0.350pt}}
\put(855,353){\rule[-0.175pt]{1.084pt}{0.350pt}}
\put(859,354){\rule[-0.175pt]{1.084pt}{0.350pt}}
\put(864,355){\rule[-0.175pt]{1.164pt}{0.350pt}}
\put(868,356){\rule[-0.175pt]{1.164pt}{0.350pt}}
\put(873,357){\rule[-0.175pt]{1.164pt}{0.350pt}}
\put(878,358){\rule[-0.175pt]{1.164pt}{0.350pt}}
\put(883,359){\rule[-0.175pt]{1.164pt}{0.350pt}}
\put(888,360){\rule[-0.175pt]{1.164pt}{0.350pt}}
\put(892,361){\rule[-0.175pt]{1.032pt}{0.350pt}}
\put(897,362){\rule[-0.175pt]{1.032pt}{0.350pt}}
\put(901,363){\rule[-0.175pt]{1.032pt}{0.350pt}}
\put(905,364){\rule[-0.175pt]{1.032pt}{0.350pt}}
\put(910,365){\rule[-0.175pt]{1.032pt}{0.350pt}}
\put(914,366){\rule[-0.175pt]{1.032pt}{0.350pt}}
\put(918,367){\rule[-0.175pt]{1.032pt}{0.350pt}}
\put(922,368){\rule[-0.175pt]{1.205pt}{0.350pt}}
\put(928,369){\rule[-0.175pt]{1.204pt}{0.350pt}}
\put(933,370){\rule[-0.175pt]{1.204pt}{0.350pt}}
\put(938,371){\rule[-0.175pt]{1.204pt}{0.350pt}}
\put(943,372){\rule[-0.175pt]{1.204pt}{0.350pt}}
\put(948,373){\rule[-0.175pt]{1.204pt}{0.350pt}}
\put(953,374){\rule[-0.175pt]{1.124pt}{0.350pt}}
\put(957,375){\rule[-0.175pt]{1.124pt}{0.350pt}}
\put(962,376){\rule[-0.175pt]{1.124pt}{0.350pt}}
\put(967,377){\rule[-0.175pt]{1.124pt}{0.350pt}}
\put(971,378){\rule[-0.175pt]{1.124pt}{0.350pt}}
\put(976,379){\rule[-0.175pt]{1.124pt}{0.350pt}}
\put(981,380){\rule[-0.175pt]{0.929pt}{0.350pt}}
\put(984,381){\rule[-0.175pt]{0.929pt}{0.350pt}}
\put(988,382){\rule[-0.175pt]{0.929pt}{0.350pt}}
\put(992,383){\rule[-0.175pt]{0.929pt}{0.350pt}}
\put(996,384){\rule[-0.175pt]{0.929pt}{0.350pt}}
\put(1000,385){\rule[-0.175pt]{0.929pt}{0.350pt}}
\put(1004,386){\rule[-0.175pt]{0.929pt}{0.350pt}}
\put(1007,387){\rule[-0.175pt]{0.923pt}{0.350pt}}
\put(1011,388){\rule[-0.175pt]{0.923pt}{0.350pt}}
\put(1015,389){\rule[-0.175pt]{0.923pt}{0.350pt}}
\put(1019,390){\rule[-0.175pt]{0.923pt}{0.350pt}}
\put(1023,391){\rule[-0.175pt]{0.923pt}{0.350pt}}
\put(1027,392){\rule[-0.175pt]{0.923pt}{0.350pt}}
\put(1031,393){\rule[-0.175pt]{0.843pt}{0.350pt}}
\put(1034,394){\rule[-0.175pt]{0.843pt}{0.350pt}}
\put(1038,395){\rule[-0.175pt]{0.843pt}{0.350pt}}
\put(1041,396){\rule[-0.175pt]{0.843pt}{0.350pt}}
\put(1045,397){\rule[-0.175pt]{0.843pt}{0.350pt}}
\put(1048,398){\rule[-0.175pt]{0.843pt}{0.350pt}}
\put(1052,399){\rule[-0.175pt]{0.585pt}{0.350pt}}
\put(1054,400){\rule[-0.175pt]{0.585pt}{0.350pt}}
\put(1056,401){\rule[-0.175pt]{0.585pt}{0.350pt}}
\put(1059,402){\rule[-0.175pt]{0.585pt}{0.350pt}}
\put(1061,403){\rule[-0.175pt]{0.585pt}{0.350pt}}
\put(1064,404){\rule[-0.175pt]{0.585pt}{0.350pt}}
\put(1066,405){\rule[-0.175pt]{0.585pt}{0.350pt}}
\put(1069,406){\rule[-0.175pt]{0.522pt}{0.350pt}}
\put(1071,407){\rule[-0.175pt]{0.522pt}{0.350pt}}
\put(1073,408){\rule[-0.175pt]{0.522pt}{0.350pt}}
\put(1075,409){\rule[-0.175pt]{0.522pt}{0.350pt}}
\put(1077,410){\rule[-0.175pt]{0.522pt}{0.350pt}}
\put(1079,411){\rule[-0.175pt]{0.522pt}{0.350pt}}
\put(1081,412){\rule[-0.175pt]{0.402pt}{0.350pt}}
\put(1083,413){\rule[-0.175pt]{0.401pt}{0.350pt}}
\put(1085,414){\rule[-0.175pt]{0.401pt}{0.350pt}}
\put(1086,415){\rule[-0.175pt]{0.401pt}{0.350pt}}
\put(1088,416){\rule[-0.175pt]{0.401pt}{0.350pt}}
\put(1090,417){\rule[-0.175pt]{0.401pt}{0.350pt}}
\put(1091,418){\usebox{\plotpoint}}
\put(1092,418){\usebox{\plotpoint}}
\put(1093,419){\usebox{\plotpoint}}
\put(1094,420){\usebox{\plotpoint}}
\put(1095,422){\usebox{\plotpoint}}
\put(1096,423){\usebox{\plotpoint}}
\put(1097,424){\rule[-0.175pt]{0.350pt}{0.723pt}}
\put(1098,428){\rule[-0.175pt]{0.350pt}{0.723pt}}
\put(1099,431){\rule[-0.175pt]{0.350pt}{1.686pt}}
\put(1098,438){\usebox{\plotpoint}}
\put(1097,439){\usebox{\plotpoint}}
\put(1096,440){\usebox{\plotpoint}}
\put(1095,441){\usebox{\plotpoint}}
\put(1094,442){\usebox{\plotpoint}}
\put(1091,444){\usebox{\plotpoint}}
\put(1090,445){\usebox{\plotpoint}}
\put(1089,446){\usebox{\plotpoint}}
\put(1088,447){\usebox{\plotpoint}}
\put(1087,448){\usebox{\plotpoint}}
\put(1086,449){\usebox{\plotpoint}}
\put(1084,450){\usebox{\plotpoint}}
\put(1083,451){\usebox{\plotpoint}}
\put(1081,452){\usebox{\plotpoint}}
\put(1080,453){\usebox{\plotpoint}}
\put(1078,454){\usebox{\plotpoint}}
\put(1077,455){\usebox{\plotpoint}}
\put(1076,456){\usebox{\plotpoint}}
\put(1074,457){\rule[-0.175pt]{0.442pt}{0.350pt}}
\put(1072,458){\rule[-0.175pt]{0.442pt}{0.350pt}}
\put(1070,459){\rule[-0.175pt]{0.442pt}{0.350pt}}
\put(1068,460){\rule[-0.175pt]{0.442pt}{0.350pt}}
\put(1066,461){\rule[-0.175pt]{0.442pt}{0.350pt}}
\put(1065,462){\rule[-0.175pt]{0.442pt}{0.350pt}}
\put(1063,463){\rule[-0.175pt]{0.482pt}{0.350pt}}
\put(1061,464){\rule[-0.175pt]{0.482pt}{0.350pt}}
\put(1059,465){\rule[-0.175pt]{0.482pt}{0.350pt}}
\put(1057,466){\rule[-0.175pt]{0.482pt}{0.350pt}}
\put(1055,467){\rule[-0.175pt]{0.482pt}{0.350pt}}
\put(1053,468){\rule[-0.175pt]{0.482pt}{0.350pt}}
\put(1051,469){\rule[-0.175pt]{0.482pt}{0.350pt}}
\put(1049,470){\rule[-0.175pt]{0.482pt}{0.350pt}}
\put(1047,471){\rule[-0.175pt]{0.482pt}{0.350pt}}
\put(1045,472){\rule[-0.175pt]{0.482pt}{0.350pt}}
\put(1043,473){\rule[-0.175pt]{0.482pt}{0.350pt}}
\put(1041,474){\rule[-0.175pt]{0.482pt}{0.350pt}}
\put(1039,475){\rule[-0.175pt]{0.482pt}{0.350pt}}
\put(1036,476){\rule[-0.175pt]{0.522pt}{0.350pt}}
\put(1034,477){\rule[-0.175pt]{0.522pt}{0.350pt}}
\put(1032,478){\rule[-0.175pt]{0.522pt}{0.350pt}}
\put(1030,479){\rule[-0.175pt]{0.522pt}{0.350pt}}
\put(1028,480){\rule[-0.175pt]{0.522pt}{0.350pt}}
\put(1026,481){\rule[-0.175pt]{0.522pt}{0.350pt}}
\put(1023,482){\rule[-0.175pt]{0.522pt}{0.350pt}}
\put(1021,483){\rule[-0.175pt]{0.522pt}{0.350pt}}
\put(1019,484){\rule[-0.175pt]{0.522pt}{0.350pt}}
\put(1017,485){\rule[-0.175pt]{0.522pt}{0.350pt}}
\put(1015,486){\rule[-0.175pt]{0.522pt}{0.350pt}}
\put(1013,487){\rule[-0.175pt]{0.522pt}{0.350pt}}
\put(1011,488){\rule[-0.175pt]{0.447pt}{0.350pt}}
\put(1009,489){\rule[-0.175pt]{0.447pt}{0.350pt}}
\put(1007,490){\rule[-0.175pt]{0.447pt}{0.350pt}}
\put(1005,491){\rule[-0.175pt]{0.447pt}{0.350pt}}
\put(1003,492){\rule[-0.175pt]{0.447pt}{0.350pt}}
\put(1001,493){\rule[-0.175pt]{0.447pt}{0.350pt}}
\put(1000,494){\rule[-0.175pt]{0.447pt}{0.350pt}}
\put(998,495){\rule[-0.175pt]{0.442pt}{0.350pt}}
\put(996,496){\rule[-0.175pt]{0.442pt}{0.350pt}}
\put(994,497){\rule[-0.175pt]{0.442pt}{0.350pt}}
\put(992,498){\rule[-0.175pt]{0.442pt}{0.350pt}}
\put(990,499){\rule[-0.175pt]{0.442pt}{0.350pt}}
\put(989,500){\rule[-0.175pt]{0.442pt}{0.350pt}}
\put(987,501){\rule[-0.175pt]{0.442pt}{0.350pt}}
\put(985,502){\rule[-0.175pt]{0.442pt}{0.350pt}}
\put(983,503){\rule[-0.175pt]{0.442pt}{0.350pt}}
\put(981,504){\rule[-0.175pt]{0.442pt}{0.350pt}}
\put(979,505){\rule[-0.175pt]{0.442pt}{0.350pt}}
\put(978,506){\rule[-0.175pt]{0.442pt}{0.350pt}}
\put(976,507){\usebox{\plotpoint}}
\put(975,508){\usebox{\plotpoint}}
\put(974,509){\usebox{\plotpoint}}
\put(972,510){\usebox{\plotpoint}}
\put(971,511){\usebox{\plotpoint}}
\put(970,512){\usebox{\plotpoint}}
\put(969,513){\usebox{\plotpoint}}
\put(967,514){\usebox{\plotpoint}}
\put(966,515){\usebox{\plotpoint}}
\put(965,516){\usebox{\plotpoint}}
\put(964,517){\usebox{\plotpoint}}
\put(963,518){\usebox{\plotpoint}}
\put(962,519){\usebox{\plotpoint}}
\put(962,520){\usebox{\plotpoint}}
\put(961,521){\usebox{\plotpoint}}
\put(960,522){\usebox{\plotpoint}}
\put(959,524){\usebox{\plotpoint}}
\put(958,525){\usebox{\plotpoint}}
\put(957,527){\rule[-0.175pt]{0.350pt}{0.361pt}}
\put(956,528){\rule[-0.175pt]{0.350pt}{0.361pt}}
\put(955,530){\rule[-0.175pt]{0.350pt}{0.361pt}}
\put(954,531){\rule[-0.175pt]{0.350pt}{0.361pt}}
\put(953,533){\rule[-0.175pt]{0.350pt}{0.723pt}}
\put(952,536){\rule[-0.175pt]{0.350pt}{0.723pt}}
\put(951,539){\rule[-0.175pt]{0.350pt}{1.686pt}}
\put(950,546){\rule[-0.175pt]{0.350pt}{1.445pt}}
\put(951,552){\rule[-0.175pt]{0.350pt}{0.482pt}}
\put(952,554){\rule[-0.175pt]{0.350pt}{0.482pt}}
\put(953,556){\rule[-0.175pt]{0.350pt}{0.482pt}}
\put(954,558){\rule[-0.175pt]{0.350pt}{0.562pt}}
\put(955,560){\rule[-0.175pt]{0.350pt}{0.562pt}}
\put(956,562){\rule[-0.175pt]{0.350pt}{0.562pt}}
\put(957,564){\usebox{\plotpoint}}
\put(958,566){\usebox{\plotpoint}}
\put(959,567){\usebox{\plotpoint}}
\put(960,568){\usebox{\plotpoint}}
\put(961,569){\usebox{\plotpoint}}
\put(962,571){\usebox{\plotpoint}}
\put(963,572){\usebox{\plotpoint}}
\put(964,573){\usebox{\plotpoint}}
\put(965,574){\usebox{\plotpoint}}
\put(966,575){\usebox{\plotpoint}}
\put(967,577){\usebox{\plotpoint}}
\put(968,578){\usebox{\plotpoint}}
\put(969,579){\usebox{\plotpoint}}
\put(970,580){\usebox{\plotpoint}}
\put(971,581){\usebox{\plotpoint}}
\put(972,582){\usebox{\plotpoint}}
\put(973,584){\usebox{\plotpoint}}
\put(974,585){\usebox{\plotpoint}}
\put(975,586){\usebox{\plotpoint}}
\put(976,587){\usebox{\plotpoint}}
\put(977,588){\usebox{\plotpoint}}
\put(978,589){\usebox{\plotpoint}}
\put(979,590){\usebox{\plotpoint}}
\put(980,591){\usebox{\plotpoint}}
\put(981,592){\usebox{\plotpoint}}
\put(982,593){\usebox{\plotpoint}}
\put(983,594){\usebox{\plotpoint}}
\put(984,595){\usebox{\plotpoint}}
\put(985,596){\usebox{\plotpoint}}
\put(986,597){\usebox{\plotpoint}}
\put(987,598){\usebox{\plotpoint}}
\put(988,600){\usebox{\plotpoint}}
\put(989,601){\usebox{\plotpoint}}
\put(990,603){\usebox{\plotpoint}}
\put(991,604){\usebox{\plotpoint}}
\put(992,605){\usebox{\plotpoint}}
\put(993,606){\usebox{\plotpoint}}
\put(994,607){\usebox{\plotpoint}}
\put(995,608){\usebox{\plotpoint}}
\put(996,609){\usebox{\plotpoint}}
\put(997,610){\usebox{\plotpoint}}
\put(998,611){\usebox{\plotpoint}}
\put(999,612){\usebox{\plotpoint}}
\put(1000,613){\usebox{\plotpoint}}
\put(1001,615){\usebox{\plotpoint}}
\put(1002,616){\usebox{\plotpoint}}
\put(1003,617){\usebox{\plotpoint}}
\put(1004,619){\usebox{\plotpoint}}
\put(1005,620){\usebox{\plotpoint}}
\put(1006,622){\rule[-0.175pt]{0.350pt}{0.361pt}}
\put(1007,623){\rule[-0.175pt]{0.350pt}{0.361pt}}
\put(1008,625){\rule[-0.175pt]{0.350pt}{0.361pt}}
\put(1009,626){\rule[-0.175pt]{0.350pt}{0.361pt}}
\put(1010,628){\rule[-0.175pt]{0.350pt}{0.562pt}}
\put(1011,630){\rule[-0.175pt]{0.350pt}{0.562pt}}
\put(1012,632){\rule[-0.175pt]{0.350pt}{0.562pt}}
\put(1013,634){\rule[-0.175pt]{0.350pt}{0.723pt}}
\put(1014,638){\rule[-0.175pt]{0.350pt}{0.723pt}}
\put(1015,641){\rule[-0.175pt]{0.350pt}{1.445pt}}
\put(1016,647){\rule[-0.175pt]{0.350pt}{1.686pt}}
\put(1017,654){\rule[-0.175pt]{0.350pt}{3.734pt}}
\put(1016,669){\rule[-0.175pt]{0.350pt}{0.843pt}}
\put(1015,673){\rule[-0.175pt]{0.350pt}{1.445pt}}
\put(1014,679){\rule[-0.175pt]{0.350pt}{0.723pt}}
\put(1013,682){\rule[-0.175pt]{0.350pt}{0.723pt}}
\put(1012,685){\rule[-0.175pt]{0.350pt}{0.562pt}}
\put(1011,687){\rule[-0.175pt]{0.350pt}{0.562pt}}
\put(1010,689){\rule[-0.175pt]{0.350pt}{0.562pt}}
\put(1009,691){\rule[-0.175pt]{0.350pt}{0.723pt}}
\put(1008,695){\rule[-0.175pt]{0.350pt}{0.723pt}}
\put(1007,698){\rule[-0.175pt]{0.350pt}{0.482pt}}
\put(1006,700){\rule[-0.175pt]{0.350pt}{0.482pt}}
\put(1005,702){\rule[-0.175pt]{0.350pt}{0.482pt}}
\put(1004,704){\rule[-0.175pt]{0.350pt}{0.562pt}}
\put(1003,706){\rule[-0.175pt]{0.350pt}{0.562pt}}
\put(1002,708){\rule[-0.175pt]{0.350pt}{0.562pt}}
\put(1001,710){\rule[-0.175pt]{0.350pt}{0.723pt}}
\put(1000,714){\rule[-0.175pt]{0.350pt}{0.723pt}}
\put(999,717){\rule[-0.175pt]{0.350pt}{0.482pt}}
\put(998,719){\rule[-0.175pt]{0.350pt}{0.482pt}}
\put(997,721){\rule[-0.175pt]{0.350pt}{0.482pt}}
\put(996,723){\rule[-0.175pt]{0.350pt}{0.843pt}}
\put(995,726){\rule[-0.175pt]{0.350pt}{0.843pt}}
\put(994,730){\rule[-0.175pt]{0.350pt}{0.723pt}}
\put(993,733){\rule[-0.175pt]{0.350pt}{0.723pt}}
\put(992,736){\rule[-0.175pt]{0.350pt}{0.843pt}}
\put(991,739){\rule[-0.175pt]{0.350pt}{0.843pt}}
\put(990,743){\rule[-0.175pt]{0.350pt}{1.445pt}}
\put(989,749){\rule[-0.175pt]{0.350pt}{1.445pt}}
\put(988,755){\rule[-0.175pt]{0.350pt}{1.686pt}}
\put(987,762){\rule[-0.175pt]{0.350pt}{4.577pt}}
\put(988,781){\rule[-0.175pt]{0.350pt}{1.445pt}}
\end{picture}

\caption{Gelfand equation on the ball, $3\leq n \leq 9$.
\label{gelfand.fig2}}
\end{figure}
\begin{quote}\tt\singlespace\begin{verbatim}
\begin{figure}[htbp]
\centering
% GNUPLOT: LaTeX picture
\setlength{\unitlength}{0.240900pt}
\ifx\plotpoint\undefined\newsavebox{\plotpoint}\fi
\sbox{\plotpoint}{\rule[-0.175pt]{0.350pt}{0.350pt}}%
\begin{picture}(1500,900)(0,0)
%\tenrm
\sbox{\plotpoint}{\rule[-0.175pt]{0.350pt}{0.350pt}}%
\put(264,158){\rule[-0.175pt]{282.335pt}{0.350pt}}
\put(264,158){\rule[-0.175pt]{0.350pt}{151.526pt}}
\put(264,158){\rule[-0.175pt]{4.818pt}{0.350pt}}
%\put(242,158){\makebox(0,0)[r]{0}}
\put(1416,158){\rule[-0.175pt]{4.818pt}{0.350pt}}
\put(264,284){\rule[-0.175pt]{4.818pt}{0.350pt}}
%\put(242,284){\makebox(0,0)[r]{2}}
\put(1416,284){\rule[-0.175pt]{4.818pt}{0.350pt}}
\put(264,410){\rule[-0.175pt]{4.818pt}{0.350pt}}
%\put(242,410){\makebox(0,0)[r]{4}}
\put(1416,410){\rule[-0.175pt]{4.818pt}{0.350pt}}
\put(264,535){\rule[-0.175pt]{4.818pt}{0.350pt}}
%\put(242,535){\makebox(0,0)[r]{6}}
\put(1416,535){\rule[-0.175pt]{4.818pt}{0.350pt}}
\put(264,661){\rule[-0.175pt]{4.818pt}{0.350pt}}
%\put(242,661){\makebox(0,0)[r]{8}}
\put(1416,661){\rule[-0.175pt]{4.818pt}{0.350pt}}
\put(264,787){\rule[-0.175pt]{4.818pt}{0.350pt}}
%\put(242,787){\makebox(0,0)[r]{10}}
\put(1416,787){\rule[-0.175pt]{4.818pt}{0.350pt}}
\put(264,158){\rule[-0.175pt]{0.350pt}{4.818pt}}
%\put(264,113){\makebox(0,0){0}}
\put(264,767){\rule[-0.175pt]{0.350pt}{4.818pt}}
\put(411,158){\rule[-0.175pt]{0.350pt}{4.818pt}}
%\put(411,113){\makebox(0,0){0.5}}
\put(411,767){\rule[-0.175pt]{0.350pt}{4.818pt}}
\put(557,158){\rule[-0.175pt]{0.350pt}{4.818pt}}
%\put(557,113){\makebox(0,0){1}}
\put(557,767){\rule[-0.175pt]{0.350pt}{4.818pt}}
\put(704,158){\rule[-0.175pt]{0.350pt}{4.818pt}}
%\put(704,113){\makebox(0,0){1.5}}
\put(704,767){\rule[-0.175pt]{0.350pt}{4.818pt}}
\put(850,158){\rule[-0.175pt]{0.350pt}{4.818pt}}
%\put(850,113){\makebox(0,0){2}}
\put(850,767){\rule[-0.175pt]{0.350pt}{4.818pt}}
\put(997,158){\rule[-0.175pt]{0.350pt}{4.818pt}}
%\put(997,113){\makebox(0,0){2.5}}
\put(997,767){\rule[-0.175pt]{0.350pt}{4.818pt}}
\put(1143,158){\rule[-0.175pt]{0.350pt}{4.818pt}}
%\put(1143,113){\makebox(0,0){3}}
\put(1143,767){\rule[-0.175pt]{0.350pt}{4.818pt}}
\put(1290,158){\rule[-0.175pt]{0.350pt}{4.818pt}}
%\put(1290,113){\makebox(0,0){3.5}}
\put(1290,767){\rule[-0.175pt]{0.350pt}{4.818pt}}
\put(1436,158){\rule[-0.175pt]{0.350pt}{4.818pt}}
%\put(1436,113){\makebox(0,0){4}}
\put(1436,767){\rule[-0.175pt]{0.350pt}{4.818pt}}
\put(264,158){\rule[-0.175pt]{282.335pt}{0.350pt}}
\put(1436,158){\rule[-0.175pt]{0.350pt}{151.526pt}}
\put(264,787){\rule[-0.175pt]{282.335pt}{0.350pt}}
\put(100,472){\makebox(0,0)[l]{\shortstack{$\| u\|$}}}
\put(850,68){\makebox(0,0){$\lambda$}}
%\put(850,832){\makebox(0,0){plot}}
\put(264,158){\rule[-0.175pt]{0.350pt}{151.526pt}}
%\put(1306,722){\makebox(0,0)[r]{}}
%\put(1328,722){\rule[-0.175pt]{15.899pt}{0.350pt}}
\put(264,158){\usebox{\plotpoint}}
\put(264,158){\rule[-0.175pt]{6.304pt}{0.350pt}}
\put(290,159){\rule[-0.175pt]{6.304pt}{0.350pt}}
\put(316,160){\rule[-0.175pt]{6.304pt}{0.350pt}}
\put(342,161){\rule[-0.175pt]{6.304pt}{0.350pt}}
\put(368,162){\rule[-0.175pt]{6.304pt}{0.350pt}}
\put(394,163){\rule[-0.175pt]{6.304pt}{0.350pt}}
\put(420,164){\rule[-0.175pt]{5.644pt}{0.350pt}}
\put(444,165){\rule[-0.175pt]{5.644pt}{0.350pt}}
\put(467,166){\rule[-0.175pt]{5.644pt}{0.350pt}}
\put(491,167){\rule[-0.175pt]{5.644pt}{0.350pt}}
\put(514,168){\rule[-0.175pt]{5.644pt}{0.350pt}}
\put(538,169){\rule[-0.175pt]{5.644pt}{0.350pt}}
\put(561,170){\rule[-0.175pt]{5.644pt}{0.350pt}}
\put(585,171){\rule[-0.175pt]{6.384pt}{0.350pt}}
\put(611,172){\rule[-0.175pt]{6.384pt}{0.350pt}}
\put(638,173){\rule[-0.175pt]{6.384pt}{0.350pt}}
\put(664,174){\rule[-0.175pt]{6.384pt}{0.350pt}}
\put(691,175){\rule[-0.175pt]{6.384pt}{0.350pt}}
\put(717,176){\rule[-0.175pt]{6.384pt}{0.350pt}}
\put(744,177){\rule[-0.175pt]{5.862pt}{0.350pt}}
\put(768,178){\rule[-0.175pt]{5.862pt}{0.350pt}}
\put(792,179){\rule[-0.175pt]{5.862pt}{0.350pt}}
\put(816,180){\rule[-0.175pt]{5.862pt}{0.350pt}}
\put(841,181){\rule[-0.175pt]{5.862pt}{0.350pt}}
\put(865,182){\rule[-0.175pt]{5.862pt}{0.350pt}}
\put(889,183){\rule[-0.175pt]{4.371pt}{0.350pt}}
\put(908,184){\rule[-0.175pt]{4.371pt}{0.350pt}}
\put(926,185){\rule[-0.175pt]{4.371pt}{0.350pt}}
\put(944,186){\rule[-0.175pt]{4.371pt}{0.350pt}}
\put(962,187){\rule[-0.175pt]{4.371pt}{0.350pt}}
\put(980,188){\rule[-0.175pt]{4.371pt}{0.350pt}}
\put(998,189){\rule[-0.175pt]{4.371pt}{0.350pt}}
\put(1017,190){\rule[-0.175pt]{4.216pt}{0.350pt}}
\put(1034,191){\rule[-0.175pt]{4.216pt}{0.350pt}}
\put(1052,192){\rule[-0.175pt]{4.216pt}{0.350pt}}
\put(1069,193){\rule[-0.175pt]{4.216pt}{0.350pt}}
\put(1087,194){\rule[-0.175pt]{4.216pt}{0.350pt}}
\put(1104,195){\rule[-0.175pt]{4.216pt}{0.350pt}}
\put(1122,196){\rule[-0.175pt]{3.172pt}{0.350pt}}
\put(1135,197){\rule[-0.175pt]{3.172pt}{0.350pt}}
\put(1148,198){\rule[-0.175pt]{3.172pt}{0.350pt}}
\put(1161,199){\rule[-0.175pt]{3.172pt}{0.350pt}}
\put(1174,200){\rule[-0.175pt]{3.172pt}{0.350pt}}
\put(1187,201){\rule[-0.175pt]{3.172pt}{0.350pt}}
\put(1200,202){\rule[-0.175pt]{1.893pt}{0.350pt}}
\put(1208,203){\rule[-0.175pt]{1.893pt}{0.350pt}}
\put(1216,204){\rule[-0.175pt]{1.893pt}{0.350pt}}
\put(1224,205){\rule[-0.175pt]{1.893pt}{0.350pt}}
\put(1232,206){\rule[-0.175pt]{1.893pt}{0.350pt}}
\put(1240,207){\rule[-0.175pt]{1.893pt}{0.350pt}}
\put(1248,208){\rule[-0.175pt]{1.893pt}{0.350pt}}
\put(1256,209){\rule[-0.175pt]{1.245pt}{0.350pt}}
\put(1261,210){\rule[-0.175pt]{1.245pt}{0.350pt}}
\put(1266,211){\rule[-0.175pt]{1.245pt}{0.350pt}}
\put(1271,212){\rule[-0.175pt]{1.245pt}{0.350pt}}
\put(1276,213){\rule[-0.175pt]{1.245pt}{0.350pt}}
\put(1281,214){\rule[-0.175pt]{1.245pt}{0.350pt}}
\put(1286,215){\usebox{\plotpoint}}
\put(1288,216){\usebox{\plotpoint}}
\put(1289,217){\usebox{\plotpoint}}
\put(1291,218){\usebox{\plotpoint}}
\put(1292,219){\usebox{\plotpoint}}
\put(1294,220){\usebox{\plotpoint}}
\put(1295,221){\usebox{\plotpoint}}
\put(1295,222){\rule[-0.175pt]{0.361pt}{0.350pt}}
\put(1294,223){\rule[-0.175pt]{0.361pt}{0.350pt}}
\put(1292,224){\rule[-0.175pt]{0.361pt}{0.350pt}}
\put(1291,225){\rule[-0.175pt]{0.361pt}{0.350pt}}
\put(1289,226){\rule[-0.175pt]{0.361pt}{0.350pt}}
\put(1288,227){\rule[-0.175pt]{0.361pt}{0.350pt}}
\put(1284,228){\rule[-0.175pt]{0.964pt}{0.350pt}}
\put(1280,229){\rule[-0.175pt]{0.964pt}{0.350pt}}
\put(1276,230){\rule[-0.175pt]{0.964pt}{0.350pt}}
\put(1272,231){\rule[-0.175pt]{0.964pt}{0.350pt}}
\put(1268,232){\rule[-0.175pt]{0.964pt}{0.350pt}}
\put(1264,233){\rule[-0.175pt]{0.964pt}{0.350pt}}
\put(1258,234){\rule[-0.175pt]{1.273pt}{0.350pt}}
\put(1253,235){\rule[-0.175pt]{1.273pt}{0.350pt}}
\put(1248,236){\rule[-0.175pt]{1.273pt}{0.350pt}}
\put(1242,237){\rule[-0.175pt]{1.273pt}{0.350pt}}
\put(1237,238){\rule[-0.175pt]{1.273pt}{0.350pt}}
\put(1232,239){\rule[-0.175pt]{1.273pt}{0.350pt}}
\put(1227,240){\rule[-0.175pt]{1.273pt}{0.350pt}}
\put(1219,241){\rule[-0.175pt]{1.847pt}{0.350pt}}
\put(1211,242){\rule[-0.175pt]{1.847pt}{0.350pt}}
\put(1204,243){\rule[-0.175pt]{1.847pt}{0.350pt}}
\put(1196,244){\rule[-0.175pt]{1.847pt}{0.350pt}}
\put(1188,245){\rule[-0.175pt]{1.847pt}{0.350pt}}
\put(1181,246){\rule[-0.175pt]{1.847pt}{0.350pt}}
\put(1172,247){\rule[-0.175pt]{2.128pt}{0.350pt}}
\put(1163,248){\rule[-0.175pt]{2.128pt}{0.350pt}}
\put(1154,249){\rule[-0.175pt]{2.128pt}{0.350pt}}
\put(1145,250){\rule[-0.175pt]{2.128pt}{0.350pt}}
\put(1136,251){\rule[-0.175pt]{2.128pt}{0.350pt}}
\put(1128,252){\rule[-0.175pt]{2.128pt}{0.350pt}}
\put(1120,253){\rule[-0.175pt]{1.893pt}{0.350pt}}
\put(1112,254){\rule[-0.175pt]{1.893pt}{0.350pt}}
\put(1104,255){\rule[-0.175pt]{1.893pt}{0.350pt}}
\put(1096,256){\rule[-0.175pt]{1.893pt}{0.350pt}}
\put(1088,257){\rule[-0.175pt]{1.893pt}{0.350pt}}
\put(1080,258){\rule[-0.175pt]{1.893pt}{0.350pt}}
\put(1073,259){\rule[-0.175pt]{1.893pt}{0.350pt}}
\put(1063,260){\rule[-0.175pt]{2.208pt}{0.350pt}}
\put(1054,261){\rule[-0.175pt]{2.208pt}{0.350pt}}
\put(1045,262){\rule[-0.175pt]{2.208pt}{0.350pt}}
\put(1036,263){\rule[-0.175pt]{2.208pt}{0.350pt}}
\put(1027,264){\rule[-0.175pt]{2.208pt}{0.350pt}}
\put(1018,265){\rule[-0.175pt]{2.208pt}{0.350pt}}
\put(1009,266){\rule[-0.175pt]{2.168pt}{0.350pt}}
\put(1000,267){\rule[-0.175pt]{2.168pt}{0.350pt}}
\put(991,268){\rule[-0.175pt]{2.168pt}{0.350pt}}
\put(982,269){\rule[-0.175pt]{2.168pt}{0.350pt}}
\put(973,270){\rule[-0.175pt]{2.168pt}{0.350pt}}
\put(964,271){\rule[-0.175pt]{2.168pt}{0.350pt}}
\put(957,272){\rule[-0.175pt]{1.686pt}{0.350pt}}
\put(950,273){\rule[-0.175pt]{1.686pt}{0.350pt}}
\put(943,274){\rule[-0.175pt]{1.686pt}{0.350pt}}
\put(936,275){\rule[-0.175pt]{1.686pt}{0.350pt}}
\put(929,276){\rule[-0.175pt]{1.686pt}{0.350pt}}
\put(922,277){\rule[-0.175pt]{1.686pt}{0.350pt}}
\put(915,278){\rule[-0.175pt]{1.686pt}{0.350pt}}
\put(907,279){\rule[-0.175pt]{1.767pt}{0.350pt}}
\put(900,280){\rule[-0.175pt]{1.767pt}{0.350pt}}
\put(893,281){\rule[-0.175pt]{1.767pt}{0.350pt}}
\put(885,282){\rule[-0.175pt]{1.767pt}{0.350pt}}
\put(878,283){\rule[-0.175pt]{1.767pt}{0.350pt}}
\put(871,284){\rule[-0.175pt]{1.767pt}{0.350pt}}
\put(864,285){\rule[-0.175pt]{1.486pt}{0.350pt}}
\put(858,286){\rule[-0.175pt]{1.486pt}{0.350pt}}
\put(852,287){\rule[-0.175pt]{1.486pt}{0.350pt}}
\put(846,288){\rule[-0.175pt]{1.486pt}{0.350pt}}
\put(840,289){\rule[-0.175pt]{1.486pt}{0.350pt}}
\put(834,290){\rule[-0.175pt]{1.486pt}{0.350pt}}
\put(829,291){\rule[-0.175pt]{0.998pt}{0.350pt}}
\put(825,292){\rule[-0.175pt]{0.998pt}{0.350pt}}
\put(821,293){\rule[-0.175pt]{0.998pt}{0.350pt}}
\put(817,294){\rule[-0.175pt]{0.998pt}{0.350pt}}
\put(813,295){\rule[-0.175pt]{0.998pt}{0.350pt}}
\put(809,296){\rule[-0.175pt]{0.998pt}{0.350pt}}
\put(805,297){\rule[-0.175pt]{0.998pt}{0.350pt}}
\put(801,298){\rule[-0.175pt]{0.883pt}{0.350pt}}
\put(797,299){\rule[-0.175pt]{0.883pt}{0.350pt}}
\put(793,300){\rule[-0.175pt]{0.883pt}{0.350pt}}
\put(790,301){\rule[-0.175pt]{0.883pt}{0.350pt}}
\put(786,302){\rule[-0.175pt]{0.883pt}{0.350pt}}
\put(783,303){\rule[-0.175pt]{0.883pt}{0.350pt}}
\put(780,304){\rule[-0.175pt]{0.522pt}{0.350pt}}
\put(778,305){\rule[-0.175pt]{0.522pt}{0.350pt}}
\put(776,306){\rule[-0.175pt]{0.522pt}{0.350pt}}
\put(774,307){\rule[-0.175pt]{0.522pt}{0.350pt}}
\put(772,308){\rule[-0.175pt]{0.522pt}{0.350pt}}
\put(770,309){\rule[-0.175pt]{0.522pt}{0.350pt}}
\put(770,310){\usebox{\plotpoint}}
\put(769,311){\usebox{\plotpoint}}
\put(768,312){\usebox{\plotpoint}}
\put(767,314){\usebox{\plotpoint}}
\put(766,315){\usebox{\plotpoint}}
\put(765,316){\rule[-0.175pt]{0.350pt}{0.723pt}}
\put(766,320){\rule[-0.175pt]{0.350pt}{0.723pt}}
\put(767,323){\usebox{\plotpoint}}
\put(768,324){\usebox{\plotpoint}}
\put(769,325){\usebox{\plotpoint}}
\put(771,326){\usebox{\plotpoint}}
\put(772,327){\usebox{\plotpoint}}
\put(774,328){\usebox{\plotpoint}}
\put(775,329){\usebox{\plotpoint}}
\put(777,330){\rule[-0.175pt]{0.602pt}{0.350pt}}
\put(779,331){\rule[-0.175pt]{0.602pt}{0.350pt}}
\put(782,332){\rule[-0.175pt]{0.602pt}{0.350pt}}
\put(784,333){\rule[-0.175pt]{0.602pt}{0.350pt}}
\put(787,334){\rule[-0.175pt]{0.602pt}{0.350pt}}
\put(789,335){\rule[-0.175pt]{0.602pt}{0.350pt}}
\put(792,336){\rule[-0.175pt]{0.843pt}{0.350pt}}
\put(795,337){\rule[-0.175pt]{0.843pt}{0.350pt}}
\put(799,338){\rule[-0.175pt]{0.843pt}{0.350pt}}
\put(802,339){\rule[-0.175pt]{0.843pt}{0.350pt}}
\put(806,340){\rule[-0.175pt]{0.843pt}{0.350pt}}
\put(809,341){\rule[-0.175pt]{0.843pt}{0.350pt}}
\put(813,342){\rule[-0.175pt]{0.826pt}{0.350pt}}
\put(816,343){\rule[-0.175pt]{0.826pt}{0.350pt}}
\put(819,344){\rule[-0.175pt]{0.826pt}{0.350pt}}
\put(823,345){\rule[-0.175pt]{0.826pt}{0.350pt}}
\put(826,346){\rule[-0.175pt]{0.826pt}{0.350pt}}
\put(830,347){\rule[-0.175pt]{0.826pt}{0.350pt}}
\put(833,348){\rule[-0.175pt]{0.826pt}{0.350pt}}
\put(837,349){\rule[-0.175pt]{1.084pt}{0.350pt}}
\put(841,350){\rule[-0.175pt]{1.084pt}{0.350pt}}
\put(846,351){\rule[-0.175pt]{1.084pt}{0.350pt}}
\put(850,352){\rule[-0.175pt]{1.084pt}{0.350pt}}
\put(855,353){\rule[-0.175pt]{1.084pt}{0.350pt}}
\put(859,354){\rule[-0.175pt]{1.084pt}{0.350pt}}
\put(864,355){\rule[-0.175pt]{1.164pt}{0.350pt}}
\put(868,356){\rule[-0.175pt]{1.164pt}{0.350pt}}
\put(873,357){\rule[-0.175pt]{1.164pt}{0.350pt}}
\put(878,358){\rule[-0.175pt]{1.164pt}{0.350pt}}
\put(883,359){\rule[-0.175pt]{1.164pt}{0.350pt}}
\put(888,360){\rule[-0.175pt]{1.164pt}{0.350pt}}
\put(892,361){\rule[-0.175pt]{1.032pt}{0.350pt}}
\put(897,362){\rule[-0.175pt]{1.032pt}{0.350pt}}
\put(901,363){\rule[-0.175pt]{1.032pt}{0.350pt}}
\put(905,364){\rule[-0.175pt]{1.032pt}{0.350pt}}
\put(910,365){\rule[-0.175pt]{1.032pt}{0.350pt}}
\put(914,366){\rule[-0.175pt]{1.032pt}{0.350pt}}
\put(918,367){\rule[-0.175pt]{1.032pt}{0.350pt}}
\put(922,368){\rule[-0.175pt]{1.205pt}{0.350pt}}
\put(928,369){\rule[-0.175pt]{1.204pt}{0.350pt}}
\put(933,370){\rule[-0.175pt]{1.204pt}{0.350pt}}
\put(938,371){\rule[-0.175pt]{1.204pt}{0.350pt}}
\put(943,372){\rule[-0.175pt]{1.204pt}{0.350pt}}
\put(948,373){\rule[-0.175pt]{1.204pt}{0.350pt}}
\put(953,374){\rule[-0.175pt]{1.124pt}{0.350pt}}
\put(957,375){\rule[-0.175pt]{1.124pt}{0.350pt}}
\put(962,376){\rule[-0.175pt]{1.124pt}{0.350pt}}
\put(967,377){\rule[-0.175pt]{1.124pt}{0.350pt}}
\put(971,378){\rule[-0.175pt]{1.124pt}{0.350pt}}
\put(976,379){\rule[-0.175pt]{1.124pt}{0.350pt}}
\put(981,380){\rule[-0.175pt]{0.929pt}{0.350pt}}
\put(984,381){\rule[-0.175pt]{0.929pt}{0.350pt}}
\put(988,382){\rule[-0.175pt]{0.929pt}{0.350pt}}
\put(992,383){\rule[-0.175pt]{0.929pt}{0.350pt}}
\put(996,384){\rule[-0.175pt]{0.929pt}{0.350pt}}
\put(1000,385){\rule[-0.175pt]{0.929pt}{0.350pt}}
\put(1004,386){\rule[-0.175pt]{0.929pt}{0.350pt}}
\put(1007,387){\rule[-0.175pt]{0.923pt}{0.350pt}}
\put(1011,388){\rule[-0.175pt]{0.923pt}{0.350pt}}
\put(1015,389){\rule[-0.175pt]{0.923pt}{0.350pt}}
\put(1019,390){\rule[-0.175pt]{0.923pt}{0.350pt}}
\put(1023,391){\rule[-0.175pt]{0.923pt}{0.350pt}}
\put(1027,392){\rule[-0.175pt]{0.923pt}{0.350pt}}
\put(1031,393){\rule[-0.175pt]{0.843pt}{0.350pt}}
\put(1034,394){\rule[-0.175pt]{0.843pt}{0.350pt}}
\put(1038,395){\rule[-0.175pt]{0.843pt}{0.350pt}}
\put(1041,396){\rule[-0.175pt]{0.843pt}{0.350pt}}
\put(1045,397){\rule[-0.175pt]{0.843pt}{0.350pt}}
\put(1048,398){\rule[-0.175pt]{0.843pt}{0.350pt}}
\put(1052,399){\rule[-0.175pt]{0.585pt}{0.350pt}}
\put(1054,400){\rule[-0.175pt]{0.585pt}{0.350pt}}
\put(1056,401){\rule[-0.175pt]{0.585pt}{0.350pt}}
\put(1059,402){\rule[-0.175pt]{0.585pt}{0.350pt}}
\put(1061,403){\rule[-0.175pt]{0.585pt}{0.350pt}}
\put(1064,404){\rule[-0.175pt]{0.585pt}{0.350pt}}
\put(1066,405){\rule[-0.175pt]{0.585pt}{0.350pt}}
\put(1069,406){\rule[-0.175pt]{0.522pt}{0.350pt}}
\put(1071,407){\rule[-0.175pt]{0.522pt}{0.350pt}}
\put(1073,408){\rule[-0.175pt]{0.522pt}{0.350pt}}
\put(1075,409){\rule[-0.175pt]{0.522pt}{0.350pt}}
\put(1077,410){\rule[-0.175pt]{0.522pt}{0.350pt}}
\put(1079,411){\rule[-0.175pt]{0.522pt}{0.350pt}}
\put(1081,412){\rule[-0.175pt]{0.402pt}{0.350pt}}
\put(1083,413){\rule[-0.175pt]{0.401pt}{0.350pt}}
\put(1085,414){\rule[-0.175pt]{0.401pt}{0.350pt}}
\put(1086,415){\rule[-0.175pt]{0.401pt}{0.350pt}}
\put(1088,416){\rule[-0.175pt]{0.401pt}{0.350pt}}
\put(1090,417){\rule[-0.175pt]{0.401pt}{0.350pt}}
\put(1091,418){\usebox{\plotpoint}}
\put(1092,418){\usebox{\plotpoint}}
\put(1093,419){\usebox{\plotpoint}}
\put(1094,420){\usebox{\plotpoint}}
\put(1095,422){\usebox{\plotpoint}}
\put(1096,423){\usebox{\plotpoint}}
\put(1097,424){\rule[-0.175pt]{0.350pt}{0.723pt}}
\put(1098,428){\rule[-0.175pt]{0.350pt}{0.723pt}}
\put(1099,431){\rule[-0.175pt]{0.350pt}{1.686pt}}
\put(1098,438){\usebox{\plotpoint}}
\put(1097,439){\usebox{\plotpoint}}
\put(1096,440){\usebox{\plotpoint}}
\put(1095,441){\usebox{\plotpoint}}
\put(1094,442){\usebox{\plotpoint}}
\put(1091,444){\usebox{\plotpoint}}
\put(1090,445){\usebox{\plotpoint}}
\put(1089,446){\usebox{\plotpoint}}
\put(1088,447){\usebox{\plotpoint}}
\put(1087,448){\usebox{\plotpoint}}
\put(1086,449){\usebox{\plotpoint}}
\put(1084,450){\usebox{\plotpoint}}
\put(1083,451){\usebox{\plotpoint}}
\put(1081,452){\usebox{\plotpoint}}
\put(1080,453){\usebox{\plotpoint}}
\put(1078,454){\usebox{\plotpoint}}
\put(1077,455){\usebox{\plotpoint}}
\put(1076,456){\usebox{\plotpoint}}
\put(1074,457){\rule[-0.175pt]{0.442pt}{0.350pt}}
\put(1072,458){\rule[-0.175pt]{0.442pt}{0.350pt}}
\put(1070,459){\rule[-0.175pt]{0.442pt}{0.350pt}}
\put(1068,460){\rule[-0.175pt]{0.442pt}{0.350pt}}
\put(1066,461){\rule[-0.175pt]{0.442pt}{0.350pt}}
\put(1065,462){\rule[-0.175pt]{0.442pt}{0.350pt}}
\put(1063,463){\rule[-0.175pt]{0.482pt}{0.350pt}}
\put(1061,464){\rule[-0.175pt]{0.482pt}{0.350pt}}
\put(1059,465){\rule[-0.175pt]{0.482pt}{0.350pt}}
\put(1057,466){\rule[-0.175pt]{0.482pt}{0.350pt}}
\put(1055,467){\rule[-0.175pt]{0.482pt}{0.350pt}}
\put(1053,468){\rule[-0.175pt]{0.482pt}{0.350pt}}
\put(1051,469){\rule[-0.175pt]{0.482pt}{0.350pt}}
\put(1049,470){\rule[-0.175pt]{0.482pt}{0.350pt}}
\put(1047,471){\rule[-0.175pt]{0.482pt}{0.350pt}}
\put(1045,472){\rule[-0.175pt]{0.482pt}{0.350pt}}
\put(1043,473){\rule[-0.175pt]{0.482pt}{0.350pt}}
\put(1041,474){\rule[-0.175pt]{0.482pt}{0.350pt}}
\put(1039,475){\rule[-0.175pt]{0.482pt}{0.350pt}}
\put(1036,476){\rule[-0.175pt]{0.522pt}{0.350pt}}
\put(1034,477){\rule[-0.175pt]{0.522pt}{0.350pt}}
\put(1032,478){\rule[-0.175pt]{0.522pt}{0.350pt}}
\put(1030,479){\rule[-0.175pt]{0.522pt}{0.350pt}}
\put(1028,480){\rule[-0.175pt]{0.522pt}{0.350pt}}
\put(1026,481){\rule[-0.175pt]{0.522pt}{0.350pt}}
\put(1023,482){\rule[-0.175pt]{0.522pt}{0.350pt}}
\put(1021,483){\rule[-0.175pt]{0.522pt}{0.350pt}}
\put(1019,484){\rule[-0.175pt]{0.522pt}{0.350pt}}
\put(1017,485){\rule[-0.175pt]{0.522pt}{0.350pt}}
\put(1015,486){\rule[-0.175pt]{0.522pt}{0.350pt}}
\put(1013,487){\rule[-0.175pt]{0.522pt}{0.350pt}}
\put(1011,488){\rule[-0.175pt]{0.447pt}{0.350pt}}
\put(1009,489){\rule[-0.175pt]{0.447pt}{0.350pt}}
\put(1007,490){\rule[-0.175pt]{0.447pt}{0.350pt}}
\put(1005,491){\rule[-0.175pt]{0.447pt}{0.350pt}}
\put(1003,492){\rule[-0.175pt]{0.447pt}{0.350pt}}
\put(1001,493){\rule[-0.175pt]{0.447pt}{0.350pt}}
\put(1000,494){\rule[-0.175pt]{0.447pt}{0.350pt}}
\put(998,495){\rule[-0.175pt]{0.442pt}{0.350pt}}
\put(996,496){\rule[-0.175pt]{0.442pt}{0.350pt}}
\put(994,497){\rule[-0.175pt]{0.442pt}{0.350pt}}
\put(992,498){\rule[-0.175pt]{0.442pt}{0.350pt}}
\put(990,499){\rule[-0.175pt]{0.442pt}{0.350pt}}
\put(989,500){\rule[-0.175pt]{0.442pt}{0.350pt}}
\put(987,501){\rule[-0.175pt]{0.442pt}{0.350pt}}
\put(985,502){\rule[-0.175pt]{0.442pt}{0.350pt}}
\put(983,503){\rule[-0.175pt]{0.442pt}{0.350pt}}
\put(981,504){\rule[-0.175pt]{0.442pt}{0.350pt}}
\put(979,505){\rule[-0.175pt]{0.442pt}{0.350pt}}
\put(978,506){\rule[-0.175pt]{0.442pt}{0.350pt}}
\put(976,507){\usebox{\plotpoint}}
\put(975,508){\usebox{\plotpoint}}
\put(974,509){\usebox{\plotpoint}}
\put(972,510){\usebox{\plotpoint}}
\put(971,511){\usebox{\plotpoint}}
\put(970,512){\usebox{\plotpoint}}
\put(969,513){\usebox{\plotpoint}}
\put(967,514){\usebox{\plotpoint}}
\put(966,515){\usebox{\plotpoint}}
\put(965,516){\usebox{\plotpoint}}
\put(964,517){\usebox{\plotpoint}}
\put(963,518){\usebox{\plotpoint}}
\put(962,519){\usebox{\plotpoint}}
\put(962,520){\usebox{\plotpoint}}
\put(961,521){\usebox{\plotpoint}}
\put(960,522){\usebox{\plotpoint}}
\put(959,524){\usebox{\plotpoint}}
\put(958,525){\usebox{\plotpoint}}
\put(957,527){\rule[-0.175pt]{0.350pt}{0.361pt}}
\put(956,528){\rule[-0.175pt]{0.350pt}{0.361pt}}
\put(955,530){\rule[-0.175pt]{0.350pt}{0.361pt}}
\put(954,531){\rule[-0.175pt]{0.350pt}{0.361pt}}
\put(953,533){\rule[-0.175pt]{0.350pt}{0.723pt}}
\put(952,536){\rule[-0.175pt]{0.350pt}{0.723pt}}
\put(951,539){\rule[-0.175pt]{0.350pt}{1.686pt}}
\put(950,546){\rule[-0.175pt]{0.350pt}{1.445pt}}
\put(951,552){\rule[-0.175pt]{0.350pt}{0.482pt}}
\put(952,554){\rule[-0.175pt]{0.350pt}{0.482pt}}
\put(953,556){\rule[-0.175pt]{0.350pt}{0.482pt}}
\put(954,558){\rule[-0.175pt]{0.350pt}{0.562pt}}
\put(955,560){\rule[-0.175pt]{0.350pt}{0.562pt}}
\put(956,562){\rule[-0.175pt]{0.350pt}{0.562pt}}
\put(957,564){\usebox{\plotpoint}}
\put(958,566){\usebox{\plotpoint}}
\put(959,567){\usebox{\plotpoint}}
\put(960,568){\usebox{\plotpoint}}
\put(961,569){\usebox{\plotpoint}}
\put(962,571){\usebox{\plotpoint}}
\put(963,572){\usebox{\plotpoint}}
\put(964,573){\usebox{\plotpoint}}
\put(965,574){\usebox{\plotpoint}}
\put(966,575){\usebox{\plotpoint}}
\put(967,577){\usebox{\plotpoint}}
\put(968,578){\usebox{\plotpoint}}
\put(969,579){\usebox{\plotpoint}}
\put(970,580){\usebox{\plotpoint}}
\put(971,581){\usebox{\plotpoint}}
\put(972,582){\usebox{\plotpoint}}
\put(973,584){\usebox{\plotpoint}}
\put(974,585){\usebox{\plotpoint}}
\put(975,586){\usebox{\plotpoint}}
\put(976,587){\usebox{\plotpoint}}
\put(977,588){\usebox{\plotpoint}}
\put(978,589){\usebox{\plotpoint}}
\put(979,590){\usebox{\plotpoint}}
\put(980,591){\usebox{\plotpoint}}
\put(981,592){\usebox{\plotpoint}}
\put(982,593){\usebox{\plotpoint}}
\put(983,594){\usebox{\plotpoint}}
\put(984,595){\usebox{\plotpoint}}
\put(985,596){\usebox{\plotpoint}}
\put(986,597){\usebox{\plotpoint}}
\put(987,598){\usebox{\plotpoint}}
\put(988,600){\usebox{\plotpoint}}
\put(989,601){\usebox{\plotpoint}}
\put(990,603){\usebox{\plotpoint}}
\put(991,604){\usebox{\plotpoint}}
\put(992,605){\usebox{\plotpoint}}
\put(993,606){\usebox{\plotpoint}}
\put(994,607){\usebox{\plotpoint}}
\put(995,608){\usebox{\plotpoint}}
\put(996,609){\usebox{\plotpoint}}
\put(997,610){\usebox{\plotpoint}}
\put(998,611){\usebox{\plotpoint}}
\put(999,612){\usebox{\plotpoint}}
\put(1000,613){\usebox{\plotpoint}}
\put(1001,615){\usebox{\plotpoint}}
\put(1002,616){\usebox{\plotpoint}}
\put(1003,617){\usebox{\plotpoint}}
\put(1004,619){\usebox{\plotpoint}}
\put(1005,620){\usebox{\plotpoint}}
\put(1006,622){\rule[-0.175pt]{0.350pt}{0.361pt}}
\put(1007,623){\rule[-0.175pt]{0.350pt}{0.361pt}}
\put(1008,625){\rule[-0.175pt]{0.350pt}{0.361pt}}
\put(1009,626){\rule[-0.175pt]{0.350pt}{0.361pt}}
\put(1010,628){\rule[-0.175pt]{0.350pt}{0.562pt}}
\put(1011,630){\rule[-0.175pt]{0.350pt}{0.562pt}}
\put(1012,632){\rule[-0.175pt]{0.350pt}{0.562pt}}
\put(1013,634){\rule[-0.175pt]{0.350pt}{0.723pt}}
\put(1014,638){\rule[-0.175pt]{0.350pt}{0.723pt}}
\put(1015,641){\rule[-0.175pt]{0.350pt}{1.445pt}}
\put(1016,647){\rule[-0.175pt]{0.350pt}{1.686pt}}
\put(1017,654){\rule[-0.175pt]{0.350pt}{3.734pt}}
\put(1016,669){\rule[-0.175pt]{0.350pt}{0.843pt}}
\put(1015,673){\rule[-0.175pt]{0.350pt}{1.445pt}}
\put(1014,679){\rule[-0.175pt]{0.350pt}{0.723pt}}
\put(1013,682){\rule[-0.175pt]{0.350pt}{0.723pt}}
\put(1012,685){\rule[-0.175pt]{0.350pt}{0.562pt}}
\put(1011,687){\rule[-0.175pt]{0.350pt}{0.562pt}}
\put(1010,689){\rule[-0.175pt]{0.350pt}{0.562pt}}
\put(1009,691){\rule[-0.175pt]{0.350pt}{0.723pt}}
\put(1008,695){\rule[-0.175pt]{0.350pt}{0.723pt}}
\put(1007,698){\rule[-0.175pt]{0.350pt}{0.482pt}}
\put(1006,700){\rule[-0.175pt]{0.350pt}{0.482pt}}
\put(1005,702){\rule[-0.175pt]{0.350pt}{0.482pt}}
\put(1004,704){\rule[-0.175pt]{0.350pt}{0.562pt}}
\put(1003,706){\rule[-0.175pt]{0.350pt}{0.562pt}}
\put(1002,708){\rule[-0.175pt]{0.350pt}{0.562pt}}
\put(1001,710){\rule[-0.175pt]{0.350pt}{0.723pt}}
\put(1000,714){\rule[-0.175pt]{0.350pt}{0.723pt}}
\put(999,717){\rule[-0.175pt]{0.350pt}{0.482pt}}
\put(998,719){\rule[-0.175pt]{0.350pt}{0.482pt}}
\put(997,721){\rule[-0.175pt]{0.350pt}{0.482pt}}
\put(996,723){\rule[-0.175pt]{0.350pt}{0.843pt}}
\put(995,726){\rule[-0.175pt]{0.350pt}{0.843pt}}
\put(994,730){\rule[-0.175pt]{0.350pt}{0.723pt}}
\put(993,733){\rule[-0.175pt]{0.350pt}{0.723pt}}
\put(992,736){\rule[-0.175pt]{0.350pt}{0.843pt}}
\put(991,739){\rule[-0.175pt]{0.350pt}{0.843pt}}
\put(990,743){\rule[-0.175pt]{0.350pt}{1.445pt}}
\put(989,749){\rule[-0.175pt]{0.350pt}{1.445pt}}
\put(988,755){\rule[-0.175pt]{0.350pt}{1.686pt}}
\put(987,762){\rule[-0.175pt]{0.350pt}{4.577pt}}
\put(988,781){\rule[-0.175pt]{0.350pt}{1.445pt}}
\end{picture}

\caption{Gelfand equation on the ball, $3\leq n \leq 9$.
\label{gelfand.fig2}}
\end{figure}
\end{verbatim}\end{quote}
One advantage to using the native \LaTeX{} {\tt picture} environment
is that the fonts will be assured to agree and the pictures can be viewed
in the {\tt .dvi} viewer.

\subsection{PostScript}
Many drawing applications now allow the export of a graphic to the
{\em Encapsulated PostScript} format.  These files have a suffix of
{\tt .EPS} or {\tt .EPSF} and are similar to a regular PostScript
file except that they contain a {\em bounding box} which describes
the dimensions of the figure.

In order to include PostScript figures, the {\tt epsfig} (or {\tt psfig}
depending on the system you are using) style file must be included in either
the {\tt\verb|\documentstyle|} command or the preamble using the {\tt input} command.

Figure~\ref{vwcontr} is a plot from Matlab.
\begin{figure}[htbp]
\centerline{
\psfig{figure=vwcontr.eps,width=5in,angle=0}
           }
\caption{$\sigma$ as a Function of Voltage and Speed, $\alpha = 20$}
\label{vwcontr}
\end{figure}
The commands to include this figure are
\begin{quote}\tt\singlespace\begin{verbatim}
\begin{figure}[htbp]
\centerline{
\psfig{figure=vwcontr.ps,width=5in,angle=0}
           }
\caption{$\sigma$ as a Function of Voltage and Speed, $\alpha = 20$}
\label{vwcontr}
\end{figure}
\end{verbatim}\end{quote}

Observe that the {\tt \verb|\psfig|} command allows the scaling of the figure
by setting either the {\tt width} or {\tt height} of the figure.  If only one
dimension is specified, the other is computed to keep the same aspect ratio.
The figure can also be rotated by setting {\tt angle} to the desired value in
degrees.
                  % Chapter 3 Edited from UW Math Dept's Sample Thesis
%% bibs.tex
%
% This chapter briefly talks about BibTex and is mostly
% copied from a similar chapter from "How to TeX a Thesis:
% The Purdue Thesis Styles" by James Darrell McCauley and
% Scott Hucker
%

\newcommand{\BibTeX}{{\sc Bib}\TeX}

\chapter{Citations and Bibliographies}
This chapter is an edited form of the same chapter from {\em How to 
\TeX{} a Thesis: The Purdue Thesis Styles} by James Darrell McCauley and
Scott Hucker.

The task of compiling and formatting the sources cited in papers can
be quite tedious, especially for large documents like theses.  A program
separate from \LaTeX{}, called ``\BibTeX{},''can be used to automate this task~\cite{lamport}.

\section{The Citation Command}
When referring to the work of someone else, the {\tt \verb|\cite|} command is used.
This generates the citation in the text for you.  In the above paragraph, the command
{\tt \verb|\cite{lamport}|} was used after the word ``task.''  The formatting of your
citation is handled by either the document style or a style option.  The default citation
style uses the number system (a number in square brackets).  Other citation styles
may use the author-date system, (Lamport, 1986) or the superscript$^3$ system.

\section{Bibliography Styles}
The way that a reference is formatted in your bibliography depends on the bibliography
style, which is specified near the beginning of your document with the\break
{\tt \verb|\bibliographstyle{file}|} command.  The file {\tt file.bst} is the name of the 
bibliography style file.  Standard \BibTeX{} bibliography style files include {\tt plain},
{\tt unsrt}, {\tt alpha}, and {\tt abbrev}.  The bibliography style governs whether or not
references are sorted, whether first names or initials are used for authors, whether or 
not last names are listed first, the location of the year in the references (after the
author or at the end of the reference), {\em etc.}.  You may be required by your
department or major professor to follow as style for a particular journal.  If so, then you
will need to find a \BibTeX{} style file to suit your needs.  Most major journals have
style files.  If you cannot locate an appropriate \BibTeX{} style file, then choose the
one which is closest and then edit the {\tt .bbl} file by hand.  See Section~\ref{BBL}
for a brief discussion on the {\tt .bbl} file.  Some common, but non-standard \BibTeX{}
styles include
\begin{tabbing}
{\tt jacs-new.bstxxxx}\= {\em Journal of the American Chemical Society}\kill
{\tt acm.bst}\>The Association for Computing Machinery\\
{\tt ieeetr.bst}\> The {\em IEEE Transactions} style\\
{\tt jacs-new.bst}\> {\em Journal of the American Chemical Society}
\end{tabbing}

\section{The Database}
The  {\tt \verb|\bibliography{file}|} command is placed in your input file at the location
where the ``List of References'' section\footnote{or ``Bibliography'' 
if {\tt \char92 altbibtitle } has been specified in the preamble.} would be.  It specifies the name (or names) of
your bibliographic data base, {\tt file.bib}.  An example entry in a \BibTeX{}
database is:
\begin{quote}\singlespace\tt\begin{verbatim}
@book{ lamport86 ,
     author =    "Leslie Lamport" ,
     title =     "\LaTeX: A Document Preparation System" ,
     publisher = "Addison--Wesley Pub.\ Co." ,
     year =      "1986" ,
     address =   "Reading, MA" 
}
\end{verbatim}\end{quote}

The citation key is the first field in this entry--- citing this book in a \LaTeX{}
file would look like
\begin{quote}\singlespace\tt\begin{verbatim}
According to Lamport~\cite{lamport86} ...
\end{verbatim}\end{quote}
The tilde ({\tt \verb|~|}) is used to tie the word ``Lamport'' to the citation
generated.  The space between these words is then unbreakable---the word ``Lamport''
and the citation \cite{lamport} will not be split across two lines if they happen to occur
near the end of a line.

A listing of all entry types with their required and optional fields is given in 
Appendix~\ref{bibrefs}. There are several tools which exist to help in editing a \BibTeX{}
file, however, their use is beyond the scope of this manual and can be found by searching
the net.  You can simply use a plain text editor like {\tt vi} or {\tt WordPad} to edit
and create the database files.

There are several rules which you must follow when creating your database.  Authors are
always listed by their full names, first name first, and multiple authors are separated
by {\tt and}.  For example
\begin{quote}\singlespace\tt\begin{verbatim}
author = "John Jay Park and Frederick Gene Watson and
          Michelle Catherine Smith",
\end{verbatim}\end{quote}
If you were using {\tt abbrv} as your {\tt bibliographystyle}, a reference for these
authors may look like:
\begin{quote}
J.J. Park, F.G. Watson, and M.C. Smith \ldots
\end{quote}

Some styles only capitalize the first word of the title.  If you use any acronyms or
other words that should always be capitalized in titles, then they should be 
enclosed in {\tt \{\}}'s ({\em e.g.}, {\tt \{Fortran\}}, {\tt \{N\}ewton}).
This protects the case of these characters.

There are several other rules for \BibTeX{} listed in~\cite{lamport} which should be
referred to because they are not discussed here.

\section{Putting It All Together}
\label{BBL}
To aid the reader in understanding how all of this works together, the following 
excerpt was taken from Lamport~\cite{lamport}:
\begin{quotation}\singlespace
When you ran \LaTeX{} with the input file {\tt sample.tex}, you may have
noticed that \LaTeX{} created a file named {\tt sample.aux}.  This file,
called an {\em auxiliary} file, contains cross-referencing information.  Since
{\tt sample.tex} contains no cross-referencing commands, the auxiliary file it
produces has no information.  However, suppose that \LaTeX{} is run with an
input file named {\tt myfile.tex} that has citations and bibliography-making
[or referencing] commands.  The auxiliary file {\tt myfile.aux} that it produces
will contain all of the citation keys and the arguments of the {\tt \verb|\bibliography|}
and {\tt\verb|\bibliographystyle|} commands.  When \BibTeX{} is run, it reads
this information from the auxiliary file and produces a file named {\tt myfile.bbl}
containing \LaTeX{} commands to produce the source list \ldots The next time
\LaTeX{} is run on {\tt myfile.tex}, the {\tt \verb|\bibliography|} command reads
the {\tt bbl} file ({\tt myfile.bbl}), which generates the source list.
\end{quotation}

Thus, the command sequence for a source file called {\tt main.tex} which is going to
use \BibTeX{} would be:
\begin{quote}\singlespace\tt\begin{verbatim}
latex main.tex
bibtex main
latex main
latex main
\end{verbatim}\end{quote}
The first \LaTeX{} is to collect all of the citations for \BibTeX{}.  Then
\BibTeX{} is run to generate the bibliography.  \LaTeX{} is run again to
incorporate the bibliography into the document and the run the last time to
update any references (like pages in the Table of Contents) which changed when
the bibliography was included.
                  % Chapter 4 From PU Thesis styles, by J.D. McCauley
%% usage.tex
%
% This file explains how to use the withesis style
%   it is heavily modelled after a similar chapter by McCauley
%   for the Purdue Thesis style
%
% Eric Benedict, May 2000
%
% It is provided without warranty on an AS IS basis.


\chapter{Using the {\tt withesis} Style}

You can get a copy of the \LaTeX{} style for creating a University
of Wisconsin--Madison thesis or dissertation from:

{\tt http://www.cae.wisc.edu/\verb+~+benedict/LaTeX.html}

After somehow unpacking it, you will have the style files ({\tt withesis.sty}
{\tt withe10.sty}, and {\tt withe12.sty}) as well the files used to create
this document.  The files used for this document can be copied and used as a
template for your own thesis or dissertation.

The final printed form of this document is useful, but the
combination of the source code and final copy form a much more valuable
reference.  Keeping a working copy of the this document can be helpful
when you are later working on your thesis or disseration and want to know
how to do something.  If you find a similar example in this document,
then you can simply look at the corresponding source code and add it to
your document.    Because many parts of this document were written by
different people, the styles and techniques are also different and provide
different ways of achieving the same or similar results.

Because of the typical size of theses, it makes sense to break the document
up into several smaller files.  Usually this is done at the chapter level.
These files can then be {\tt \verb|\include|}d in a {\em root} file.  It is
the {\em root} file that you will run \LaTeX{} on.  For this manual, the
root file is called {\tt main.tex}.

\section{The Root File and the Preamble}
The {\tt \verb|\documentclass|} command is used to tell \LaTeX{} that you will
be using the {\tt withesis} document class and it is the first command in your
root file.  Class options such as {\tt 10pt}, {\tt 12pt}, {\tt msthesis} or
{\tt margincheck} are specified here:

{\tt \verb|\documentclass[12pt,msthesis]{withesis}|}

The class option {\tt msthesis} sets the margins to be appropriate for depositing
with the UW library, namely a 1.25 inch left margin with the remaining margins 1 inch.
The defaults for the title page are also defined for a thesis and for a Master of
Science degree.

The class option {\tt margincheck} will place a small black square at the end of
each line which exceeds the margins.\footnote{In reality, the square is
placed at the end of lines which exceed their {\tt \char92hbox}.  This usually
(but not always) indicates a  margin violation on the right margin.  Left
margin violations aren't indicated and if the margin violation is large enough,
there isn't room for the black box to be visiable.}  This is visible both in the {\tt .dvi} file
as well as in the {\tt .ps} file.

The area immediately following this command is called the {\em preamble} and is
used for things like including different style packages,
defining new macros and declaring the page style.

The style packages can be used to easily change the thesis font.  For example,
this document is set in Times Roman instead of the \LaTeX default of Computer
Modern.  This change was performed by including the {\tt times} package:

{\tt\verb|\usepackage{times}|}\footnote{In this document, the typewriter font
{\tt $\backslash$tt} was redefined to use the Computer Modern font with the command
{\tt $\backslash$renewcommand\{$\backslash$ttdefault\}\{cmtt\}}.  
For more information, see~\cite{goossens}.}

Remember that if you change the fonts from the default Computer Modern to
PostScript ({\em e.g.} Times Roman) then in order to correctly see the
document, you will need to convert the {\tt *.dvi} output into a {\tt *.ps}
file and view the document with a PostScript viewer. This is required since 
most {\tt *.dvi} previewer programs cannot 
display PostScript fonts.  Usually, the previewer will substitute
default fonts so the document may be viewed; however, since the alternate
fonts may not be the same size, the formatting of the document may appear
to be incorrect.

The style package for including Postscript figures, {\tt epsfig}, is included with

{\tt\verb|\usepackage{epsfig}|}

If multiple style packages are required, then they can be combined into one statement
as follows:

{\tt\verb|\usepackage{epsfig,times}|}

Many different style packages are available.  For more information, see~\cite{goossens}.

The page styles are defined using a similar method.
A special style is defined for the {\tt withesis} style:

{\tt\verb|\pagestyle{thesisdraft}|}

This style causes the footer text to become:

{\verb| DRAFT: Do Not Distribute        <time><Date>        <input file name>|}

This appears at the bottom of every page.

In addition to the page style command, the {\tt withesis} has defined several useful
commands which are specified in the preamble.  They include {\tt \verb| \draftmargin|},
{\tt \verb|\draftscreen|}, {\tt \verb|\noappendixtables|}, and
{\tt \verb|\noappendixfigures|}.

The command  {\tt \verb|\draftmargin|} draws a PostScript box with the dimensions of
the margins.  This makes it easy to check that the margins are correct and to see if
any of the text or figures are outside of the required margins.  This box is only visible
in the {\tt .ps} file since it is a PostScript special.


The command  {\tt \verb|\draftscreen|} draws a PostScript screen with the word {\em DRAFT}
in light grey and diagonally across the page.  This screen is only visible
in the {\tt .ps} file since it is a PostScript special.

The commands {\tt \verb|\noappendixtables|} and/or {\tt \verb|\noappendixfigures|} should
be used if the appendix does not have either tables or figures respectively.  These commands
inhibit the Appendix Table or Appendix Figure titles in the List of Tables or List of
Figures.\label{usage:noapp}


If you have specified the {\tt psfig} or {\tt epsfig} document style package, then a useful
command is {\tt \verb|\psdraft|}.  This command will show the bounding box that the figure
would occupy (instead of actually including the figure).  This speeds up the draft copy
printing, reduces toner usage and the drawn box is visible in the {\tt .dvi} file.

The next usual command is {\tt \verb|\begin{document}|}.  The following example is part
of the root file used for this manual.

\begin{quote} \singlespace\footnotesize\tt
\begin{verbatim}
\bibliographystyle{plain}
% prelude.tex
%   - titlepage
%   - dedication
%   - acknowledgments
%   - table of contents, list of tables and list of figures
%   - nomenclature
%   - abstract
%============================================================================


\clearpage\pagenumbering{roman}  % This makes the page numbers Roman (i, ii, etc)


% TITLE PAGE
%   - define \title{} \author{} \date{}
\title{Statistical framework in the discovery of Fluoroscanning \\ (the next generation precision genomics)}
\author{Subhrangshu Nandi}
\date{February 18, 2016}
%   - The default degree is ``Doctor of Philosophy''
%     (unless the document style msthesis is specified
%      and then the default degree is ``Master of Science'')
%     Degree can be changed using the command \degree{}
\degree{Doctor of Philosophy}
%   - The default is dissertation, unless the document style
%     msthesis was specified in which case it becomes thesis.
%     If msthesis is specified for the MS margins, you can
%     still have a dissertation if you specify \disseration
%\disseration
%   - for a masters project report, specify \project
%\project
%   - for a preliminary report, specify \prelim
\prelim
%   - for a masters thesis, specify \thesis
%\thesis
%   - The default department is ``Electrical Engineering''
%     The department can be changed using the command \department{}
\department{Statistics}
%   - once the above are defined, use \maketitle to generate the titlepage
\maketitle

% COPYRIGHT PAGE
%   - To include a copyright page use \copyrightpage
\copyrightpage

% DEDICATION
\begin{dedication}
TBD
\end{dedication}

% ACKNOWLEDGMENTS
\begin{acknowledgments}
I thank the many people who have done lots of nice things for me.
\end{acknowledgments}

% CONTENTS, TABLES, FIGURES
\tableofcontents
\listoftables
\listoffigures

% NOMENCLATURE
%% \begin{nomenclature}
%% \begin{description}
%% \item{\makebox[0.75in][l]{\TeX}}
%%        \parbox[t]{5in}{a typesetting system by Donald Knuth~\cite{knuth}.  It
%%        also refers to the ``plain'' format.  The proper pronounciation
%%        rhymes with ``heck'' and ``peck'' and does not sound like
%%        ``hex'' or ``Rex.''\\}

%% \item{\makebox[0.75in][l]{\LaTeX}}  
%%         \parbox[t]{5in}{a set of \TeX{} macros originally written by Leslie 
%%         Lamport~\cite{lamport}.  The proper pronunciation is 
%%         {\tt l\={a}$\cdot$tek'} and not {\tt l\={a}'$\cdot$teks} (see above).\\}

%% \item{\makebox[0.75in][l]{{\sc Bib}\TeX}} 
%%          \parbox[t]{5in}{a bibliography generation program by Oren 
%%                 Patashnik~\cite{lamport}
%%                 that can be used with either plain \TeX{} or \LaTeX{}.\\}

%% \item{\makebox[0.75in][l]{$C_1$}} Constant 1

%% \item{\makebox[0.75in][l]{$V$}}    Voltage 

%% \item{\makebox[0.75in][l]{\$}}     US Dollars
%% \end{description}
%% \end{nomenclature}


\advisorname{Michael A. Newton}
\advisortitle{Professor}
% ABSTRACT
\begin{umiabstract}
  Write up from Professor Newton

\end{umiabstract}

\begin{abstract}
  % abstract.tex
%
% This file has the abstract for the withesis style documentation
%
% Eric Benedict, Aug 2000
%
% It is provided without warranty on an AS IS basis.

\noindent       % Don't indent this paragraph.
The Human Genome Project (HGP), completed in 2003, is considered one of the greatest accomplishments of exploration in history of science. Since then thousands of genomes have been sequenced. However, no individual human genome has been annotated to completion. Nanocoding \cite{Jo_etal_2007_PNAS} (PNAS, 2007), developed by Laboratory of Molecular and Computational Genomics (LMCG), UW Madison , is a novel system for physically mapping genomes, using measurements of single DNA molecules to construct a high-resolution genome-wide restriction map, whose representation of genome structure complements genome sequences to yield biological insight. Staining the DNA molecules with cyanine dyes and imaging them is a critical step of nanocoding. It turns out that the quantum yield of the fluorescence intensity of these stained molecules are sequence dependent. In fact, for YO complexed with GC-rich DNA sequences the quantum yield are about twice as large as for YO complexed with AT-rich sequences. Hence, regions with distinct sequence compositions should exhibit unique fluorescence intensity profiles. Establishing the fluorescence intensity profiles of a genome would provide invaluable insights into its sequence compositions without having to sequence it. We name this technique ``Fluoroscanning''. Imaged DNA molecules from the same region on a genome should exhibit similar intensity profiles, unless there has been a modification in the underlying genomic sequence. Fluoroscanning can be used to identify heterozygotes and detect large scale structural variations as a result of cancer or other diseases.  

%\vspace*{0.5em}
%\noindent       % Don't indent this paragraph.


\end{abstract}


\clearpage\pagenumbering{arabic} % This makes the page numbers Arabic (1, 2, etc)
        % Title page, abstract, table of contents, etc
% Pre-lim
% by Eric Benedict


\chapter{Introducing the {\tt withesis} \LaTeX{} Style Guide}
This manual is was written to test the {\tt withesis} style
file and to provide documentation for this style file.  

\section{History}
The
idea for this came from a similar manual written by James Darrell
McCauley and Scott Hucker in 1993 for the Purdue University thesis
style file.  Content ideas were liberally borrowed from this document.
The {\tt withesis} style file is based on the Purdue thesis file
written by Dave Kraynie and edited by Darrell McCauley.  This base was
edited to meet the format requirements of the University of 
Wisconsin--Madison and several additional new commands were created.
In addition, environments from the UW Mathematics Department were also
incorporated.

\section{Producing Your Thesis or Dissertation}
The {\tt withesis} style file will take care of most of the formatting
requirements for submitting your thesis or dissertation at the University
of Wisconsin-Madison.  There are some requirements on the printing of your
document.  From the Graduate School's {\em UW-Madison Guide To Preparing 
Your Doctoral Dissertation},
\begin{quote}\singlespace
Print your dissertation on a laser printer. (Some high quality dot-matrix
printers may be acceptable.) The printer must produce output that
meets all format and legibility requirements. A professional copy shop
can produce an acceptable copy to be submitted to the Graduate School.
Some copiers enlarge the original between one and two percent. To avoid
problems with margins, produce the original copy with margins larger than
the required minimum. Look carefully at the copy before paying for the
services and ask for pages to be recopied if necessary. Common flaws are:
smudges, copy lines, specks, missing pages, margin shifts, slanting of
the printed image on the page, and poor paper quality.
\end{quote}

\subsection{Required Paper}
The paper which is used for PhD Dissertations should be:
\begin{itemize}
\item 8-1/2 x 11 inches
\item High-quality, white
\item 20 pound weight, bond
\end{itemize}
 
While for Masters Theses, the paper should be:

\begin{itemize}
\item 8-1/2 x 11 inches
\item White
\item Acid-free or pH neutral
\item 20 pound weight
\item 25\% cotton bond minimum
\end{itemize}

Paper that meets these requirements can be purchased at book and stationery
stores.

\subsection{Copyright Page}
\label{copyright}
If you choose to retain and register copyright of the dissertation, prepare
a copyright page using the {\tt withesis} {\tt \verb|\copyrightpage|} command. 
Center the text in the bottom third of the page within the dissertation
margins. This page is not numbered. There is an additional fee for copyrighting
your dissertation which is payable at the bursars office along with the
microfilming and binding fee.

\subsection{Prechecks}
The Graduate School has reserved 9:00-9:30 each morning to answer specific formatting questions
(for example: use of tables, graphs and charts). You may bring in 8-10
pages to be reviewed. No appointment is necessary.

\subsection{Final Checks}
\sloppypar
For information about the final Graduate School review and about depositing
your dissertation in the library, see {\em The Three D's: Deadlines, Defending, 
Depositing Your Doctoral Dissertation} or look
at the web site 
\begin{quote}
{\tt http://www.wisc.edu/grad/gs/degrees/ddd.html}
\end{quote}

\section{Disclaimer}
This software and documentation is provided ``as is'' without any
express or implied warranty.
While care has been taken by the authors of this style file such that the
final product will probably meet the University of Wisconsin's formatting 
requirements this is not guaranteed. 
          % Chapter 1
% Essential LaTeX - Jon Warbrick 02/88
%   - Edited May, July 2000 -E. Benedict


% Copyright (C) Jon Warbrick and Plymouth Polytechnic 1989
% Permission is granted to reproduce the document in any way providing
% that it is distributed for free, except for any reasonable charges for
% printing, distribution, staff time, etc.  Direct commercial
% exploitation is not permitted.  Extracts may be made from this
% document providing an acknowledgment of the original source is
% maintained.

% NOTICE: This document has been edited for use in the UW-Madison
% Example Thesis file.


% counters used for the sample file example
\newcounter{savesection}
\newcounter{savesubsection}


% commands to do 'LaTeX Manual-like' examples

\newlength{\egwidth}\setlength{\egwidth}{0.42\textwidth}

\newenvironment{eg}{\begin{list}{}{\setlength{\leftmargin}%
{0.05\textwidth}\setlength{\rightmargin}{\leftmargin}}%
\item[]\footnotesize}{\end{list}}

\newenvironment{egbox}{\begin{minipage}[t]{\egwidth}}{\end{minipage}}

\newcommand{\egstart}{\begin{eg}\begin{egbox}}
\newcommand{\egmid}{\end{egbox}\hfill\begin{egbox}}
\newcommand{\egend}{\end{egbox}\end{eg}}

% one or two other commands
\newcommand{\fn}[1]{\hbox{\tt #1}}
\newcommand{\llo}[1]{(see line #1)}
\newcommand{\lls}[1]{(see lines #1)}
\newcommand{\bs}{$\backslash$}


\chapter{Essential \LaTeX{}}

This chapter introduces some key ideas behind \LaTeX{} and give you the ``essential''
items of information.  This chapter is an edited form of the paper
``Essential \LaTeX{}'' by Jon Warbrick, Plymouth Polytechnic.

\section{Introduction}
This document is an attempt to give you all the essential
information that you will need in order to use the \LaTeX{} Document
Preparation System.  Only very basic features are covered, and a
vast amount of detail has been omitted.  In a document of this size
it is not possible to include everything that you might need to know,
and if you intend to make extensive use of the program you should
refer to a more complete reference.  Attempting to produce complex
documents using only the information found below will require
much more work than it should, and will probably produce a less
than satisfactory result.

The main reference for \LaTeX{} is {\em The \LaTeX{} User's guide and
Reference Manual\/} by Leslie Lamport.  This contains most of the
information that you will ever need to know about the program, and
you will need access to a copy if you are to use \LaTeX{} seriously.
You should also consider getting a copy of {\em The \LaTeX{}
Companion\/} 

\section{How does \LaTeX{} work?}

In order to use \LaTeX{} you generate a file containing
both the text that you wish to print and instructions to tell \LaTeX{}
how you want it to appear.  You will normally create
this file using your system's text editor.  You can give the file any name you
like, but it should end ``\fn{.TEX}'' to identify the file's contents.
You then get \LaTeX{} to process the file, and it creates a
new file of typesetting commands; this has the same name as your file but
the ``\fn{.TEX}'' ending is replaced by ``\fn{.DVI}''.  This stands for
`{\it D\/}e{\it v\/}ice {\it I\/}ndependent' and, as the name implies, this file
can be used to create output on a range of printing devices.
Your {\em local guide\/} will go into more detail.

Rather than encourage you to dictate exactly how your document
should be laid out, \LaTeX{} instructions allow you describe its
{\em logical structure\/}.  For example, you can think of a quotation
embedded within your text as an element of this logical structure: you would
normally expect a quotation to be displayed in a recognisable style to set it
off from the rest of the text.
A human typesetter would recognise the quotation and handle
it accordingly, but since \LaTeX{} is only a computer program it requires
your help.  There are therefore \LaTeX{} commands that allow you to
identify quotations and as a result allow \LaTeX{} to typeset them correctly.

Fundamental to \LaTeX{} is the idea of a {\em document style\/} that
determines exactly how a document will be formatted.  \LaTeX{} provides
standard document styles that describe how standard logical structures
(such as quotations) should be formatted.  You may have to supplement
these styles by specifying the formatting of logical structures
peculiar to your document, such as mathematical formulae.  You can
also modify the standard document styles or even create an entirely
new one, though you should know the basic principles of typographical
design before creating a radically new style.

There are a number of good reasons for concentrating on the logical
structure rather than on the appearance of a document.  It prevents
you from making elementary typographical errors in the mistaken
idea that they improve the aesthetics of a document---you should
remember that the primary function of document design is to make
documents easier to read, not prettier.  It is more flexible, since
you only need to alter the definition of the quotation style
to change the appearance of all the quotations in a document.  Most
important of all, logical design encourages better writing.
A visual system makes it easier to create visual effects rather than
a coherent structure; logical design encourages you to concentrate on
your writing and makes it harder to use formatting as a substitute
for good writing.

\section{A Sample \LaTeX{} file}


\begin{figure} %---------------------------------------------------------------
{\singlespace\tt\footnotesize\begin{verbatim}
 1: % SMALL.TEX -- Released 5 July 1985
 2: % USE THIS FILE AS A MODEL FOR MAKING YOUR OWN LaTeX INPUT FILE.
 3: % EVERYTHING TO THE RIGHT OF A  %  IS A REMARK TO YOU AND IS IGNORED
 4: % BY LaTeX.
 5: %
 6: % WARNING!  DO NOT TYPE ANY OF THE FOLLOWING 10 CHARACTERS EXCEPT AS
 7: % DIRECTED:        &   $   #   %   _   {   }   ^   ~   \
 8:
 9: \documentclass[11pt,a4]{article}  % YOUR INPUT FILE MUST CONTAIN THESE
10: \begin{document}                  % TWO LINES PLUS THE \end COMMAND AT
11:                                   % THE END
12:
13: \section{Simple Text}          % THIS COMMAND MAKES A SECTION TITLE.
14:
15: Words are separated by one or    more      spaces.  Paragraphs are
16:     separated by one or more blank lines.  The output is not affected
17: by adding extra spaces or extra blank lines to the input file.
18:
19:
20: Double quotes are typed like this: ``quoted text''.
21: Single quotes are typed like this: `single-quoted text'.
22:
23: Long dashes are typed as three dash characters---like this.
24:
25: Italic text is typed like this: {\em this is italic text}.
26: Bold   text is typed like this: {\bf this is  bold  text}.
27:
28: \subsection{A Warning or Two}        % THIS MAKES A SUBSECTION TITLE.
29:
30: If you get too much space after a mid-sentence period---abbreviations
31: like etc.\ are the common culprits)---then type a backslash followed by
32: a space after the period, as in this sentence.
33:
34: Remember, don't type the 10 special characters (such as dollar sign and
35: backslash) except as directed!  The following seven are printed by
36: typing a backslash in front of them:  \$  \&  \#  \%  \_  \{  and  \}.
37: The manual tells how to make other symbols.
38:
39: \end{document}                    % THE INPUT FILE ENDS LIKE THIS
\end{verbatim}  }

\caption{A Sample \LaTeX{} File}\label{fig:sample}

\end{figure} %-----------------------------------------------------------------



Have a look at the example \LaTeX{} file in Figure~\ref{fig:sample}.  It
is a slightly modified copy of the standard \LaTeX{} example file
\fn{SMALL.TEX}.  The line numbers down the left-hand side
are not part of the file, but have been added to make it easier to
identify various portions.

Try entering this file (without the line numbers), save the text as \fn{small.tex},
next run \LaTeX{} on it, and then view the output:

{\tt \singlespace\begin{verbatim}
% latex small
% xdvi small               # displays the output on the screen
% dvips -o small.ps small  # to create a PostScript file, small.ps
% lp -d<printer> small.ps  # to print
\end{verbatim}}

\subsection{Running Text}

Most documents consist almost entirely of running text---words formed
into sentences, which are in turn formed into paragraphs---and the example file
is no exception. Describing running text poses no problems, you just type
it in naturally. In the output that it produces, \LaTeX{} will fill
lines and adjust the
spacing between words to give tidy left and right margins.
The spacing and distribution of the words in your input
file will have no effect at all on the eventual output.
Any number of spaces in your input file
are treated as a single space by \LaTeX{}, it also regards the
end of each line as a space between words \lls{15--17}.
A new paragraph is
indicated by a blank line in your input file, so don't leave
any blank lines unless you really wish to start a paragraph.

\LaTeX{} reserves a number of the less common keyboard characters for its
own use. The ten characters
\begin{quote}\begin{verbatim}
#  $  %  &  ~  _  ^  \  {  }
\end{verbatim}\end{quote}
should not appear as part of your text, because if they do
\LaTeX{} will get confused.

\subsection{\LaTeX{} Commands}

There are a number of words in the file that start `\verb|\|' \lls{9,
10 and 13}.  These are \LaTeX{} {\em commands\/} and they describe
the structure of your document. There are a number of things that you
should realize about these commands:
\begin{itemize}

\item All \LaTeX{} commands consist of a `\verb|\|' followed by one or more
characters.

\item \LaTeX{} commands should be typed using the correct mixture of upper- and
lower-case letters.  \verb|\BEGIN| is {\em not\/} the same as \verb|\begin|.

\item Some commands are placed within your text.  These are used to
switch things, like different typestyles, on and off. The \verb|\em|
command is used like this to emphasize text, normally by changing to
an {\it italic\/} typestyle \llo{25}.  The command and the text are
always enclosed between `\verb|{|' and `\verb|}|'---the `\verb|{\em|'
turns the effect on and and the `\verb|}|' turns it off.

\item There are other commands that look like
\begin{quote}\begin{verbatim}
\command{text}
\end{verbatim}\end{quote}
In this case the text is called the ``argument'' of the command.  The
\verb|\section| command is like this \llo{13}.
Sometimes you have to use curly brackets `\verb|{}|' to enclose the argument,
sometimes square brackets `\verb|[]|', and sometimes both at once.
There is method behind this apparent madness, but for the
time being you should be sure to copy the commands exactly as given.

\item When a command's name is made up entirely of letters, you must make sure
that the end of the command is marked by something that isn't a letter.
This is usually either the opening bracket around the command's argument, or
it's a space.  When it's a space, that space is always ignored by \LaTeX. We
will see later that this can sometimes be a problem.

\end{itemize}

\subsection{Overall structure}

There are some \LaTeX{} commands that must appear in every document.
The actual text of the document always starts with a
\verb|\begin{document}| command and ends with an \verb|\end{document}|
command \lls{10 and 39}.  Anything that comes after the \break
\verb|\end{document}| command is ignored.  Everything that comes
before the \break\verb|\begin{document}| command is called the
{\em preamble\/}. The preamble can only contain \LaTeX{} commands
to describe the document's style.

One command that must appear in the preamble is the
\verb|\documentclass| command \llo{9}.  This command specifies the
overall style for the document.  Our example file is a simple
technical document, and uses the {\tt article\/} class.  The document
you are reading was produced with the {\tt withesis\/} class. There
are other classes that you can use, as you will find out later on in
this document.

\subsection{Other Things to Look At}

\LaTeX{} can print both opening and closing quote characters, and can manage
either of these either single or double.  To do this it uses the two quote
characters from your keyboard: {\tt `} and {\tt '}. You will probably think of
{\tt '} as the ordinary single quote character which probably looks like
{\tt\symbol{'23}} or {\tt\symbol{'15}} on your keyboard,

and {\tt `} as a ``funny'' character that probably appears as
{\tt\symbol{'22}}. You type these characters once for single quote
\llo{21},  and twice for double quotes \llo{20}. The double quote
character {\tt "} itself is almost never used and should not be used
unless you want your text to look "funny" (compare the quote in the
previous sentence).

\LaTeX{} can produce three different kinds of dashes.
A long dash, for use as a punctuation symbol, as is typed as three dash
characters in a row, like this `\verb|---|' \llo{23}.  A shorter dash,
used between numbers as in `10--20', is typed as two dash
characters in a row, while a single dash character is used as a hyphen.

From time to time you will need to include one or more of the \LaTeX{}
special symbols in your text.  Seven of them can be printed by
making them into commands by proceeding them by backslash
\llo{36}.  The remaining three symbols can be produced by more
advanced commands, as can symbols that do not appear on your keyboard
such as \dag, \ddag, \S, \pounds, \copyright, $\sharp$ and $\clubsuit$.

It is sometimes useful to include comments in a \LaTeX{} file, to remind
you of what you have done or why you did it.  Everything to the
right of a \verb|%| sign is ignored by \LaTeX{}, and so it can
be used to introduce a comment.

\section{Document Classes and Class Options}\label{sec:styles}

There are four standard document classes available in \LaTeX:
\nobreak

\begin{description}

\item[{\tt article}]  intended for short documents and articles for publication.
Articles do not have chapters, and when \verb|\maketitle| is used to generate

a title (see Section~\ref{sec:title}) it appears at the top of the first page

rather than on a page of its own.

\item[{\tt report}] intended for longer technical documents.
It is similar to
{\tt article}, except that it contains chapters and the title appears on a page
of its own.

\item[{\tt book}] intended as a basis for book publication.  Page layout is
adjusted assuming that the output will eventually be used to print on
both sides of the paper.

\item[{\tt letter}]  intended for producing personal letters.  This style
will allow you to produce all the elements of a well laid out letter:
addresses, date, signature, etc.
\end{description}

An additional document class, the one used for this document and for
University of Wisconsin--Madison theses, is \fn{withesis}.


These standard classes can be modified by a number of {\em class
options\/}. They appear in square brackets after the
\verb|\documentclass| command. Only one class can ever be used but
you can have more than one class option, in which case their names
should be separated by commas.  The standard style options are:
\begin{description}

\item[{\tt 11pt}]  prints the document using eleven-point type for the running
 text
rather that the ten-point type normally used. Eleven-point type is about
ten percent larger than ten-point.

\item[{\tt 12pt}]  prints the document using twelve-point type for the running
 text
rather than the ten-point type normally used. Twelve-point type is about
twenty percent larger than ten-point.

\item[{\tt twoside}]  causes documents in the article or report styles to be
formatted for printing on both sides of the paper.  This is the default for the
book style.

\item[{\tt twocolumn}] produces two column on each page.

\item[{\tt titlepage}]  causes the \verb|\maketitle| command to generate a
title on a separate page for documents in the \fn{article} style.
A separate page is always used in both the \fn{report} and \fn{book} styles.

\end{description}

The University of Wisconsin--Madison thesis style, \fn{withesis} also
has some class options defined.  These class options are for
ten-point type (\fn{10pt}), tweleve-point type (\fn{12pt}), two-sided
printing (\fn{twoside}), Master Thesis margins (\fn{msthesis}) and an
option to print a small black box on lines which exceed the margins
(\fn{margincheck}).

\section{Environments}

We mentioned earlier the idea of identifying a quotation to \LaTeX{} so that
it could arrange to typeset it correctly. To do this you enclose the
quotation between the commands \verb|\begin{quotation}| and
\verb|\end{quotation}|.
This is an example of a \LaTeX{} construction called an {\em environment\/}.
A number of
special effects are obtained by putting text into particular environments.

\subsection{Quotations}

There are two environments for quotations: \fn{quote} and \fn{quotation}.
\fn{quote} is used either for a short quotation or for a sequence of
short quotations separated by blank lines:
\egstart\singlespace
\begin{verbatim}
US presidents ... remarks:
\begin{quote}
The buck stops here.

I am not a crook.
\end{quote}
\end{verbatim}
\egmid%
US presidents have been known for their pithy remarks:
\begin{quote}
The buck stops here.

I am not a crook.
\end{quote}
\egend

Use the \fn{quotation} environment for quotations that consist of more
than one paragraph.  Paragraphs in the input are separated by blank
lines as usual:
\egstart\singlespace
\begin{verbatim}

Here is some advice to remember:
\begin{quotation}
Environments for making
...other things as well.

Many problems
...environments.
\end{quotation}
\end{verbatim}
\egmid%
Here is some advice to remember:
\begin{quotation}
Environments for making quotations
can be used for other things as well.

Many problems can be solved by
novel applications of existing
environments.
\end{quotation}
\egend

\subsection{Centering and Flushing}

Text can be centered on the page by putting it within the \fn{center}
environment, and it will appear flush against the left or right margins if it
is placed within the \fn{flushleft} or \fn{flushright} environments.

Text within these environments will be formatted in the normal way, in
{\samepage
particular the ends of the lines that you type are just regarded as spaces.  To
indicate a ``newline'' you need to type the \verb|\\| command.  For example:
\egstart\singlespace
\begin{verbatim}
\begin{center}
one
two
three \\
four \\
five
\end{center}

\end{verbatim}
\egmid%
\begin{center}

one
two
three \\
four \\

five
\end{center}
\egend
}

\subsection{Lists}

There are three environments for constructing lists.  In each one each new
item is begun with an \verb|\item| command.  In the \fn{itemize} environment
the start of each item is given a marker, in the \fn{enumerate}
environment each item is marked by a number.  These environments can be nested
within each other in which case the amount of indentation used
is adjusted accordingly:
\egstart\singlespace

\begin{verbatim}
\begin{itemize}
\item Itemized lists are handy.
\item However, don't forget
  \begin{enumerate}
  \item The `item' command.
  \item The `end' command.
  \end{enumerate}
\end{itemize}
\end{verbatim}
\egmid%
\begin{itemize}
\item Itemized lists are handy.
\item However, don't forget
  \begin{enumerate}
  \item The `item' command.
  \item The `end' command.
  \end{enumerate}
\end{itemize}
\egend


The third list making environment is \fn{description}.  In a description you
specify the item labels inside square brackets after the \verb|\item| command.
For example:
\egstart\singlespace
\begin{verbatim}
Three animals that you should
know about are:
\begin{description}
  \item[gnat] A small
            animal...
  \item[gnu] A large
           animal...
  \item[armadillo] A ...
\end{description}
\end{verbatim}
\egmid%
Three animals that you should
know about are:
\begin{description}
  \item[gnat] A small animal that causes no end of trouble.
  \item[gnu] A large animal that causes no end of trouble.
  \item[armadillo] A medium-sized animal.
\end{description}
\egend

\subsection{Tables}

Because \LaTeX{} will almost always convert a sequence of spaces
into a single space, it can be rather difficult to lay out tables.
See what happens in this example
 \nolinebreak
\begin{eg}
\begin{minipage}[t]{0.55\textwidth} \singlespace
\begin{verbatim}
\begin{flushleft}
Income  Expenditure Result   \\
20s 0d  19s 11d     happiness \\
20s 0d  20s 1d      misery  \\
\end{flushleft}
\end{verbatim}
\end{minipage}
\begin{minipage}[t]{0.3\textwidth}
\begin{flushleft}
Income  Expenditure Result   \\
20s 0d  19s 11d     happiness \\
20s 0d  20s 1d      misery  \\
\end{flushleft}
\end{minipage}
\end{eg}

The \fn{tabbing} environment overcomes this problem. Within it you
set tabstops and tab to them much like you do on a typewriter.
Tabstops are set with the \verb|\=| command, and the \verb|\>|
command moves to the next stop.  The \verb|\\| command is used to
separate each line.  A line that ends \verb|\kill| produces no
output, and can be used to set tabstops:
\nolinebreak
\begin{eg}
\begin{minipage}[t]{0.6\textwidth}
\singlespace
\begin{verbatim}
\begin{tabbing}
Income \=Expenditure \=    \kill
Income \>Expenditure \>Result \\
20s 0d \>19s 11d \>Happiness \\
20s 0d \>20s 1d  \>Misery    \\
\end{tabbing}
\end{verbatim}
\end{minipage}
\vspace{1ex}
\begin{minipage}[t]{0.35\textwidth}
\begin{tabbing}
\singlespace
Income \=Expenditure \=    \kill
Income \>Expenditure \>Result \\
20s 0d \>19s 11d \>Happiness \\
20s 0d \>20s 1d  \>Misery    \\
\end{tabbing}
\end{minipage}
\end{eg}

Unlike a typewriter's tab key, the \verb|\>| command always moves to the next
tabstop in sequence, even if this means moving to the left.  This can cause
text to be overwritten if the gap between two tabstops is too small.

\subsection{Verbatim Output}

Sometimes you will want to include text exactly as it appears on a terminal
screen.  For example, you might want to include part of a computer program.
Not only do you want \LaTeX{} to stop playing around with the layout of your
text, you also want to be able to type all the characters on your keyboard
without confusing \LaTeX. The \fn{verbatim} environment has this effect:

\egstart

\begin{flushleft}\singlespace
\verb|The section of program in|  \\
 \verb|question is :|\\
 \verb|\begin{verbatim}|           \\
\verb|{ this finds %a & %b }|     \\[2ex]

\verb|for i := 1 to 27 do|        \\
\ \ \ \verb|begin|                \\
\ \ \ \verb|table[i] := fn(i);|   \\

\ \ \ \verb|process(i)|           \\
\ \ \ \verb|end;|                 \\
\verb|\end{verbatim}|
\end{flushleft}
\egmid%
The section of program in
question is:
\begin{verbatim}
{ this finds %a & %b }

for i := 1 to 27 do
   begin
   table[i] := fn(i);
   process(i)
   end;

\end{verbatim}
\egend

The \fn{withesis} document style also provides the command {\tt \verb|\verbatimfile{foo.fe}|}
which will read in the file {\tt foo.fe} into the document in \fn{verbatim} format with
the font \verb|\tt|.  See Appendix~\ref{matlab} for an example.

\section{Type Styles}

We have already come across the \verb|\em| command for changing
typeface.  Here is a full list of the available typefaces:
\begin{quote}\singlespace\begin{tabbing}
\verb|\sc|~~ \= \sc Small Caps~~~ \= \verb|\sc|~~ \= \sc Small Caps~~~
                                  \= \verb|\sc|~~ \=                   \kill
\verb|\rm|   \> \rm Roman         \> \verb|\it|   \> \it Italic
                                  \> \verb|\sc|   \> \sc Small Caps    \\
\verb|\em|   \> \em Emphatic      \> \verb|\sl| \> \sl Slanted
                                  \> \verb|\tt|   \> \tt Typewriter     \\
\verb|\bf|   \> \bf Boldface      \> \verb|\sf| \> \sf Sans Serif
\end{tabbing}\end{quote}

Remember that these commands are used {\em inside\/} a pair of braces to limit
the amount of text that they effect.  In addition to the eight typeface
commands, there are a set of commands that alter the size of the type.  These
commands are:
\begin{quotation}\singlespace\begin{tabbing}
\verb|\footnotesize|~~ \= \verb|\footnotesize|~~ \= \verb|\footnotesize| \=
 \kill
\verb|\tiny|           \> \verb|\small|          \> \verb|\large|        \>
\verb|\huge|  \\
\verb|\scriptsize|     \> \verb|\normalsize|     \> \verb|\Large|        \>
\verb|\Huge|  \\
\verb|\footnotesize|   \>                        \> \verb|\LARGE|
\end{tabbing}\end{quotation}

\section{Sectioning Commands and Tables of Contents}
\label{ess:sectioning}

Technical documents, like this one, are often divided into sections.
Each section has a heading containing a title and a number for easy
reference.  \LaTeX{} has a series of commands that will allow you to identify
different sorts of sections.  Once you have done this \LaTeX{} takes on the
responsibility of laying out the title and of providing the numbers.

The commands that you can use are:
\begin{quote}\singlespace\begin{tabbing}
\verb|\subsubsection| \= \verb|\subsubsection|~~~~~~~~~~ \=           \kill
\verb|\chapter|       \> \verb|\subsection|    \> \verb|\paragraph|    \\
\verb|\section|       \> \verb|\subsubsection| \> \verb|\subparagraph| \\
\end{tabbing}\end{quote}
The naming of these last two is unfortunate, since they do not really have
anything to do with `paragraphs' in the normal sense of the word; they are just
lower levels of section.  In most document styles, headings made with
\verb|\paragraph| and \verb|\subparagraph| are not numbered.  \verb|\chapter|
is not available in document style \fn{article}.  The commands should be used
in the order given, since sections are numbered within chapters, subsections
within sections, etc.

A seventh sectioning command, \verb|\part|, is also available.  Its use is
always optional, and it is used to divide a large document into series of
parts.  It does not alter the numbering used for any of the other commands.

Including the command \verb|\tableofcontents| in you document will cause a
contents list to be included, containing information collected from the various
sectioning commands.  You will notice that each time your document is run
through \LaTeX{} the table of contents is always made up of the headings from
the previous version of the document.  This is because \LaTeX{} collects
information for the table as it processes the document, and then includes it
the next time it is run.  This can sometimes mean that the document has to be
processed through \LaTeX{} twice to get a correct table of contents.

\section{Producing Special Symbols}

You can include in you \LaTeX{} document a wide range of symbols that do not
appear on you your keyboard. For a start, you can add an accent to any letter:
\begin{quote}\singlespace\begin{tabbing}

\t{oo} \= \verb|\t{oo}|~~~ \=
\t{oo} \= \verb|\t{oo}|~~~ \=
\t{oo} \= \verb|\t{oo}|~~~ \=
\t{oo} \= \verb|\t{oo}|~~~ \=
\t{oo} \= \verb|\t{oo}|~~~ \=
\t{oo} \=                       \kill

\a`{o} \> \verb|\`{o}|  \> \~{o}  \> \verb|\~{o}|  \> \v{o}  \> \verb|\v{o}| \>
\c{o}  \> \verb|\c{o}|  \> \a'{o} \> \verb|\'{o}|  \\
\a={o} \> \verb|\={o}|  \> \H{o}  \> \verb|\H{o}|  \> \d{o}  \> \verb|\d{o}| \>
\^{o}  \> \verb|\^{o}|  \> \.{o}  \> \verb|\.{o}|  \\
\t{oo} \> \verb|\t{oo}| \> \b{o}  \> \verb|\b{o}|  \\  \"{o} \> \verb|\"{o}| \>
\u{o}  \> \verb|\u{o}|  \\
\end{tabbing}\end{quote}

A number of other symbols are available, and can be used by including the
following commands:
\begin{quote}\singlespace\begin{tabbing}

\LaTeX~\= \verb|\copyright|~~~~ \= \LaTeX~\= \verb|\copyright|~~~~ \=
\LaTeX~\=  \kill

\dag       \> \verb|\dag|       \> \S     \> \verb|\S|     \>
\copyright \> \verb|\copyright| \\
\ddag      \> \verb|\ddag|      \> \P     \> \verb|\P|     \>
\pounds    \> \verb|\pounds|    \\
\oe        \> \verb|\oe|        \> \OE    \> \verb|\OE|    \>
\ae        \> \verb|\AE|        \\
\AE        \> \verb|\AE|        \> \aa    \> \verb|\aa|    \>
\AA        \> \verb|\AA|        \\
\o         \> \verb|\o|         \> \O     \> \verb|\O|     \>
\l         \> \verb|\l|         \\
\L         \> \verb|\E|         \> \ss    \> \verb|\ss|    \>
?`         \> \verb|?`|         \\
!`         \> \verb|!`|         \> \ldots \> \verb|\ldots| \>
\LaTeX     \> \verb|\LaTeX|     \\
\end{tabbing}\end{quote}
There is also a \verb|\today| command that prints the current date. When you
use these commands remember that \LaTeX{} will ignore any spaces that
follow them, so that you can type `\verb|\pounds 20|' to get `\pounds 20'.
However, if you type `\verb|LaTeX is wonderful|' you will get `\LaTeX is
wonderful'---notice the lack of space after \LaTeX.
To overcome this problem you can follow any of these commands by a
pair of empty brackets and then any spaces that you wish to include,
and you will see that
\verb|\LaTeX{} really is wonderful!| (\LaTeX{} really is wonderful!).

\section{Titles}\label{sec:title}

Most documents have a title.  To title a \LaTeX{} document, you include the
following commands in your document, usually just after
\verb|begin{document}|.
\begin{quote}\singlespace\begin{verbatim}
\title{required title}
\author{required author}
\date{required date}
\maketitle
\end{verbatim}\end{quote}
If there are several authors, then their names should be separated by
\verb|\and|; they can also be separated by \verb|\\| if you want them to be
centred on different lines.  If the \verb|\date| command is left out, then the
current date will be printed.
\egstart
\singlespace
\begin{verbatim}
\title{Essential \LaTeX}
\author{J Warbrick \and An Other}
\date{14th February 1988}
\maketitle
\end{verbatim}
\egmid
\begin{center}
{\normalsize Essential \LaTeX}\\[4ex]
J Warbrick\hspace{1em}A N Other\\[2ex]
14th February 1988
\end{center}
\egend

The exact appearance of the title varies depending on
the document style.  In styles \fn{report} and \fn{book} the title appears on a
page of its own. In the \fn{article} style it normally appears at the top
of the first page, the style option \fn{titlepage} will alter this (see
Section~\ref{sec:styles}).  In the \fn{withesis} style, the title is created on a
seperate page in the format appropriate to a UW-Madison thesis or dissertation.

\section{Errors}

When you create a new input file for \LaTeX{} you will probably make mistakes.
Everybody does, and it's nothing to be worried about.  As with most computer
programs, there are two sorts of mistake that you can make: those that \LaTeX{}
notices and those that it doesn't.  To take a rather silly example, since
\LaTeX{} doesn't understand what you are saying it isn't going to be worried if
you mis-spell some of the words in your text.  You will just have to accurately
proof-read your printed output.  On the other hand, if you mis-spell one of
the environment names in your file then \LaTeX won't know what you want it
to do.

When this sort of thing happens, \LaTeX{} prints an error message on your
terminal screen and then stops and waits for you to take some action.
Unfortunately, the error messages that it produces are rather user-unfriendly
and a little frightening.  Nevertheless, if you know where to look they
will probably tell you where the error is and went wrong.

Consider what would happen if you mistyped \verb|\begin{itemize}| so that it
became \break\verb|\begin{itemie}|.  When \LaTeX{} processes this instruction, it
displays the following on your terminal:
\begin{quote}\singlespace\begin{verbatim}
LaTeX error.  See LaTeX manual for explanation.
              Type  H <return>  for immediate help.
! Environment itemie undefined.
\@latexerr ...for immediate help.}\errmessage {#1}
                                                  \endgroup
l.140 \begin{itemie}

?
\end{verbatim}\end{quote}
After typing the `?' \LaTeX{} stops and waits for you to tell it what to do.

The first two lines of the message just tell you that the error was detected by
\LaTeX{}. The third line, the one that starts `!' is the {\em error indicator}.
 It
tells you what the problem is, though until you have had some experience of
\LaTeX{} this may not mean a lot to you.  In this case it is just telling you
that it doesn't recognise an environment called \fn{itemie}.
The next two lines tell you what
\LaTeX{} was doing when it found the error, they are irrelevant at the moment
and can be ignored. The final line is called the {\em error locator}, and is
a copy of the line from your file that caused the problem.
It start with a line number to help you to find it in your file, and
if the error was in the middle of a line it will be shown
broken at the point where \LaTeX{} realised that there was an error.  \LaTeX{}
can sometimes pass the point where the real error is before discovering that
something is wrong, but it doesn't usually get very far.

At this point you could do several things.  If you knew enough about \LaTeX{}
you might be able to fix the problem, or you could type `X' and press the
return key to stop \LaTeX{} running while you go and correct the error.  The
best thing to do, however, is just to press the return key.  This will allow
\LaTeX{} to go on running as if nothing had happened.  If you have made one
mistake, then you have probably made several and you may as well try to find
them all in one go.  It's much more efficient to do it this way than to run
\LaTeX{} over and over again fixing one error at a time. Don't worry about
remembering what the errors were---a copy of all the error messages is being
saved in a {\em log\/} file so that you can look at them afterwards.

If you look at the line that caused the error it's normally obvious what the
problem was.  If you can't work out what you problem is look at the hints
below, and if they don't help consult Chapter~6 of the manual~\cite{lamport}.
  It contains a
list of all of the error messages that you are likely to encounter together with
some hints as to what may have caused them.

Some of the most common mistakes that cause errors are
\begin{itemize}
\item A mispelled command or environment name.
\item Improperly matched `\verb|{|' and `\verb|}|'---remember that they should
 always
come in pairs.
\item Trying to use one of the ten special characters \verb|# $ % & _ { } ~ ^|
and \verb|\| as an ordinary printing symbol.
\item A missing \verb|\end| command.
\item A missing command argument (that's the bit enclosed in '\verb|{|' and
`\verb|}|').
\end{itemize}

One error can get \LaTeX{} so confused that it reports a series of spurious
errors as a result.  If you have an error that you understand, followed by a
series that you don't, then try correcting the first error---the rest
may vanish as if by magic.
Sometimes \LaTeX{} may write a {\tt *} and stop without an error message.  This
is normally caused by a missing \verb|\end{document}| command, but other errors
can cause it.  If this happens type \verb|\stop| and press the return key.

Finally, \LaTeX{} will sometimes print {\em warning\/} messages.  They report
problems that were not bad enough to cause \LaTeX{} to stop processing, but
nevertheless may require investigation.  The most common problems are
`overfull' and `underfull' lines of text.  A message like:
\begin{quote}\footnotesize\begin{verbatim}
Overfull \hbox (10.58649pt too wide) in paragraph at lines 172--175
[]\tenrm Mathematical for-mu-las may be dis-played. A dis-played
\end{verbatim}\end{quote}
indicates that \LaTeX{} could not find a good place to break a line when laying
out a paragraph.  As a result, it was forced to let the line stick out into the
right-hand margin, in this case by 10.6 points.  Since a point is about 1/72nd
of an inch this may be rather hard to see, but it will be there none the less.

This particular problem happens because \LaTeX{} is rather fussy about line
breaking, and it would rather generate a line that is too long than generate a
paragraph that doesn't meet its high standards.  The simplest way around the
problem is to enclose the entire offending paragraph between
\verb|\begin{sloppypar}| and \verb|\end{sloppypar}| commands.  This tells
\LaTeX{} that you are happy for it to break its own rules while it is working on
that particular bit of text.

Alternatively, messages about ``Underfull \verb|\hbox'es''| may appear.
These are lines that had to have more space inserted between
words than \LaTeX{} would have liked.  In general there is not much that you
can do about these.  Your output will look fine, even if the line looks a bit
stretched.  About the only thing you could do is to re-write the offending
paragraph!

\section{A Final Reminder}

You now know enough \LaTeX{} to produce a wide range of documents.  But this
document has only scratched the surface of the
things that \LaTeX{} can do.  This entire document was itself produced with
\LaTeX{} (with no sticking things in or clever use of a photocopier) and even
it hasn't used all the features that it could.  From this you may get some
feeling for the power that \LaTeX{} puts at your disposal.

Please remember what was said in the introduction: if you {\bf do} have a
complex document to produce then {\bf go and read the manual}.  You will be
wasting your time if you rely only on what you have read here.
     % Edited ``Essential LaTeX'' by Jon Warbrick
\chapter{Figures and Tables}\label{quad}
This chapter\footnote{Most of the text in this chapter's introduction is from {\em How to
\TeX{} a Thesis: The Purdue Thesis Styles}} shows some example ways of incorporating tables and figures into \LaTeX{}.
Special environments exist for tables and figures and are special because they are
allowed to {\em float}---that is, \LaTeX{} doesn't always put them in the exact place
that they occur in your input file.  An algorithm is used to place the floating environments,
or floats, at locations which are typographically correct.  This may cause endless frustration
if you want to have a figure or table occur at a specific location.  There are a few
methods for solving this.

You can exert some influence on \LaTeX{}'s float placement algorithm by using
{\em float position specifiers}.  These specifiers, listed below, tell \LaTeX{}
what you prefer.
\begin{tabbing}
{\tt hhhhhh} \= ``bottom'' \=  \kill
{\tt h}\> ``here'' \> do not move this object \\
{\tt p}\> ``page'' \> put this object on a page of floats \\
{\tt b}\> ``bottom'' \> put this object at the bottom of a page\\
{\tt t}\> ``top'' \> put this object at the top of a page\\
\end{tabbing}

Any combination of these can be used:
\begin{quote}\tt\singlespace\begin{verbatim}
\begin{figure}[htbp]
 ...
\caption{A Figure!}
\end{figure}
\end{verbatim}\end{quote}

In this example, we asked \LaTeX{} to ``put the figure `here' if possible.  If it
is not possible (according to the rule encoded in the float algorithm), put it on the
next float page.  A float page is a page which contains nothing but floating objects,
{\em e.g.} a page of nothing but figures or tables.  If this isn't possible, try to put it
at the `top' of a page.  The last thing to try is to put the figure at the `bottom' of
a page.''

The remainder of this chapter deals with some examples of what to put into the figure,
the ellipsis (\ldots ) in the example above.

\section{Tables}
Table~\ref{pde.tab1} is an example table from the UW Math Department.
\begin{table}[htbp]
\centering
\caption{PDE solve times, $15^3+1$
equations.\label{pde.tab1}}
\begin{tabular}{||l|l|l|l|l|l||}\hline
Precond. & Time & Nonlinear & Krylov
& Function & Precond. \\
 & & Iterations & Iterations & calls & solves \\ \hline
None & 1260.9u & 3 & 26 & 30 & 0  \\
 &(21:09) & & & &  \\ \hline
FFT  & 983.4u & 2  & 5  & 8  & 7 \\
&(16:31) & & & & \\ \hline
\end{tabular}
\end{table}
The code to generate it is as follows:
\begin{quote}\tt\singlespace\begin{verbatim}
\begin{table}[htbp]
\centering
\caption{PDE solve times, $15^3+1$
equations.\label{pde.tab1}}
\begin{tabular}{||l|l|l|l|l|l||}\hline
Precond. & Time & Nonlinear & Krylov
& Function & Precond. \\
 & & Iterations & Iterations & calls & solves \\ \hline
None & 1260.9u & 3 & 26 & 30 & 0  \\
 &(21:09) & & & &  \\ \hline
FFT  & 983.4u & 2  & 5  & 8  & 7 \\
&(16:31) & & & & \\ \hline
\end{tabular}
\end{table}
\end{verbatim}\end{quote}

\section{Figures}
There are many different ways to incorporate figures into a \LaTeX{}
document.  \LaTeX{} has an internal {\tt picture} environment and
some programs will generate files which are in this format and can
be simply {\tt include}d.  In addition to \LaTeX{} native {\tt picture}
format, additional packages can be loaded in the {\tt\verb|\documentstyle|}
command (or using the {\tt input} command) to allow \LaTeX{} to process
non-native formats such as PostScript.

\subsection{\tt gnuplot}
The graph of Figure~\ref{gelfand.fig2}
 was created by gnuplot. For simple graphs this is a
 great utility.  For example, if you want a sin curve in your thesis
 try the following:
\begin{quote}\tt\singlespace\begin{verbatim}
 (terminal window): gnuplot
 (in gnuplot):
                 set terminal latex
                 set output "foo.tex"
                 plot sin(x)
                 quit
\end{verbatim}\end{quote}
This will generate a file called {\tt foo.tex} which can be read in
with the following statements.
\begin{figure}[htbp]
\centering
% GNUPLOT: LaTeX picture
\setlength{\unitlength}{0.240900pt}
\ifx\plotpoint\undefined\newsavebox{\plotpoint}\fi
\sbox{\plotpoint}{\rule[-0.175pt]{0.350pt}{0.350pt}}%
\begin{picture}(1500,900)(0,0)
%\tenrm
\sbox{\plotpoint}{\rule[-0.175pt]{0.350pt}{0.350pt}}%
\put(264,158){\rule[-0.175pt]{282.335pt}{0.350pt}}
\put(264,158){\rule[-0.175pt]{0.350pt}{151.526pt}}
\put(264,158){\rule[-0.175pt]{4.818pt}{0.350pt}}
%\put(242,158){\makebox(0,0)[r]{0}}
\put(1416,158){\rule[-0.175pt]{4.818pt}{0.350pt}}
\put(264,284){\rule[-0.175pt]{4.818pt}{0.350pt}}
%\put(242,284){\makebox(0,0)[r]{2}}
\put(1416,284){\rule[-0.175pt]{4.818pt}{0.350pt}}
\put(264,410){\rule[-0.175pt]{4.818pt}{0.350pt}}
%\put(242,410){\makebox(0,0)[r]{4}}
\put(1416,410){\rule[-0.175pt]{4.818pt}{0.350pt}}
\put(264,535){\rule[-0.175pt]{4.818pt}{0.350pt}}
%\put(242,535){\makebox(0,0)[r]{6}}
\put(1416,535){\rule[-0.175pt]{4.818pt}{0.350pt}}
\put(264,661){\rule[-0.175pt]{4.818pt}{0.350pt}}
%\put(242,661){\makebox(0,0)[r]{8}}
\put(1416,661){\rule[-0.175pt]{4.818pt}{0.350pt}}
\put(264,787){\rule[-0.175pt]{4.818pt}{0.350pt}}
%\put(242,787){\makebox(0,0)[r]{10}}
\put(1416,787){\rule[-0.175pt]{4.818pt}{0.350pt}}
\put(264,158){\rule[-0.175pt]{0.350pt}{4.818pt}}
%\put(264,113){\makebox(0,0){0}}
\put(264,767){\rule[-0.175pt]{0.350pt}{4.818pt}}
\put(411,158){\rule[-0.175pt]{0.350pt}{4.818pt}}
%\put(411,113){\makebox(0,0){0.5}}
\put(411,767){\rule[-0.175pt]{0.350pt}{4.818pt}}
\put(557,158){\rule[-0.175pt]{0.350pt}{4.818pt}}
%\put(557,113){\makebox(0,0){1}}
\put(557,767){\rule[-0.175pt]{0.350pt}{4.818pt}}
\put(704,158){\rule[-0.175pt]{0.350pt}{4.818pt}}
%\put(704,113){\makebox(0,0){1.5}}
\put(704,767){\rule[-0.175pt]{0.350pt}{4.818pt}}
\put(850,158){\rule[-0.175pt]{0.350pt}{4.818pt}}
%\put(850,113){\makebox(0,0){2}}
\put(850,767){\rule[-0.175pt]{0.350pt}{4.818pt}}
\put(997,158){\rule[-0.175pt]{0.350pt}{4.818pt}}
%\put(997,113){\makebox(0,0){2.5}}
\put(997,767){\rule[-0.175pt]{0.350pt}{4.818pt}}
\put(1143,158){\rule[-0.175pt]{0.350pt}{4.818pt}}
%\put(1143,113){\makebox(0,0){3}}
\put(1143,767){\rule[-0.175pt]{0.350pt}{4.818pt}}
\put(1290,158){\rule[-0.175pt]{0.350pt}{4.818pt}}
%\put(1290,113){\makebox(0,0){3.5}}
\put(1290,767){\rule[-0.175pt]{0.350pt}{4.818pt}}
\put(1436,158){\rule[-0.175pt]{0.350pt}{4.818pt}}
%\put(1436,113){\makebox(0,0){4}}
\put(1436,767){\rule[-0.175pt]{0.350pt}{4.818pt}}
\put(264,158){\rule[-0.175pt]{282.335pt}{0.350pt}}
\put(1436,158){\rule[-0.175pt]{0.350pt}{151.526pt}}
\put(264,787){\rule[-0.175pt]{282.335pt}{0.350pt}}
\put(100,472){\makebox(0,0)[l]{\shortstack{$\| u\|$}}}
\put(850,68){\makebox(0,0){$\lambda$}}
%\put(850,832){\makebox(0,0){plot}}
\put(264,158){\rule[-0.175pt]{0.350pt}{151.526pt}}
%\put(1306,722){\makebox(0,0)[r]{}}
%\put(1328,722){\rule[-0.175pt]{15.899pt}{0.350pt}}
\put(264,158){\usebox{\plotpoint}}
\put(264,158){\rule[-0.175pt]{6.304pt}{0.350pt}}
\put(290,159){\rule[-0.175pt]{6.304pt}{0.350pt}}
\put(316,160){\rule[-0.175pt]{6.304pt}{0.350pt}}
\put(342,161){\rule[-0.175pt]{6.304pt}{0.350pt}}
\put(368,162){\rule[-0.175pt]{6.304pt}{0.350pt}}
\put(394,163){\rule[-0.175pt]{6.304pt}{0.350pt}}
\put(420,164){\rule[-0.175pt]{5.644pt}{0.350pt}}
\put(444,165){\rule[-0.175pt]{5.644pt}{0.350pt}}
\put(467,166){\rule[-0.175pt]{5.644pt}{0.350pt}}
\put(491,167){\rule[-0.175pt]{5.644pt}{0.350pt}}
\put(514,168){\rule[-0.175pt]{5.644pt}{0.350pt}}
\put(538,169){\rule[-0.175pt]{5.644pt}{0.350pt}}
\put(561,170){\rule[-0.175pt]{5.644pt}{0.350pt}}
\put(585,171){\rule[-0.175pt]{6.384pt}{0.350pt}}
\put(611,172){\rule[-0.175pt]{6.384pt}{0.350pt}}
\put(638,173){\rule[-0.175pt]{6.384pt}{0.350pt}}
\put(664,174){\rule[-0.175pt]{6.384pt}{0.350pt}}
\put(691,175){\rule[-0.175pt]{6.384pt}{0.350pt}}
\put(717,176){\rule[-0.175pt]{6.384pt}{0.350pt}}
\put(744,177){\rule[-0.175pt]{5.862pt}{0.350pt}}
\put(768,178){\rule[-0.175pt]{5.862pt}{0.350pt}}
\put(792,179){\rule[-0.175pt]{5.862pt}{0.350pt}}
\put(816,180){\rule[-0.175pt]{5.862pt}{0.350pt}}
\put(841,181){\rule[-0.175pt]{5.862pt}{0.350pt}}
\put(865,182){\rule[-0.175pt]{5.862pt}{0.350pt}}
\put(889,183){\rule[-0.175pt]{4.371pt}{0.350pt}}
\put(908,184){\rule[-0.175pt]{4.371pt}{0.350pt}}
\put(926,185){\rule[-0.175pt]{4.371pt}{0.350pt}}
\put(944,186){\rule[-0.175pt]{4.371pt}{0.350pt}}
\put(962,187){\rule[-0.175pt]{4.371pt}{0.350pt}}
\put(980,188){\rule[-0.175pt]{4.371pt}{0.350pt}}
\put(998,189){\rule[-0.175pt]{4.371pt}{0.350pt}}
\put(1017,190){\rule[-0.175pt]{4.216pt}{0.350pt}}
\put(1034,191){\rule[-0.175pt]{4.216pt}{0.350pt}}
\put(1052,192){\rule[-0.175pt]{4.216pt}{0.350pt}}
\put(1069,193){\rule[-0.175pt]{4.216pt}{0.350pt}}
\put(1087,194){\rule[-0.175pt]{4.216pt}{0.350pt}}
\put(1104,195){\rule[-0.175pt]{4.216pt}{0.350pt}}
\put(1122,196){\rule[-0.175pt]{3.172pt}{0.350pt}}
\put(1135,197){\rule[-0.175pt]{3.172pt}{0.350pt}}
\put(1148,198){\rule[-0.175pt]{3.172pt}{0.350pt}}
\put(1161,199){\rule[-0.175pt]{3.172pt}{0.350pt}}
\put(1174,200){\rule[-0.175pt]{3.172pt}{0.350pt}}
\put(1187,201){\rule[-0.175pt]{3.172pt}{0.350pt}}
\put(1200,202){\rule[-0.175pt]{1.893pt}{0.350pt}}
\put(1208,203){\rule[-0.175pt]{1.893pt}{0.350pt}}
\put(1216,204){\rule[-0.175pt]{1.893pt}{0.350pt}}
\put(1224,205){\rule[-0.175pt]{1.893pt}{0.350pt}}
\put(1232,206){\rule[-0.175pt]{1.893pt}{0.350pt}}
\put(1240,207){\rule[-0.175pt]{1.893pt}{0.350pt}}
\put(1248,208){\rule[-0.175pt]{1.893pt}{0.350pt}}
\put(1256,209){\rule[-0.175pt]{1.245pt}{0.350pt}}
\put(1261,210){\rule[-0.175pt]{1.245pt}{0.350pt}}
\put(1266,211){\rule[-0.175pt]{1.245pt}{0.350pt}}
\put(1271,212){\rule[-0.175pt]{1.245pt}{0.350pt}}
\put(1276,213){\rule[-0.175pt]{1.245pt}{0.350pt}}
\put(1281,214){\rule[-0.175pt]{1.245pt}{0.350pt}}
\put(1286,215){\usebox{\plotpoint}}
\put(1288,216){\usebox{\plotpoint}}
\put(1289,217){\usebox{\plotpoint}}
\put(1291,218){\usebox{\plotpoint}}
\put(1292,219){\usebox{\plotpoint}}
\put(1294,220){\usebox{\plotpoint}}
\put(1295,221){\usebox{\plotpoint}}
\put(1295,222){\rule[-0.175pt]{0.361pt}{0.350pt}}
\put(1294,223){\rule[-0.175pt]{0.361pt}{0.350pt}}
\put(1292,224){\rule[-0.175pt]{0.361pt}{0.350pt}}
\put(1291,225){\rule[-0.175pt]{0.361pt}{0.350pt}}
\put(1289,226){\rule[-0.175pt]{0.361pt}{0.350pt}}
\put(1288,227){\rule[-0.175pt]{0.361pt}{0.350pt}}
\put(1284,228){\rule[-0.175pt]{0.964pt}{0.350pt}}
\put(1280,229){\rule[-0.175pt]{0.964pt}{0.350pt}}
\put(1276,230){\rule[-0.175pt]{0.964pt}{0.350pt}}
\put(1272,231){\rule[-0.175pt]{0.964pt}{0.350pt}}
\put(1268,232){\rule[-0.175pt]{0.964pt}{0.350pt}}
\put(1264,233){\rule[-0.175pt]{0.964pt}{0.350pt}}
\put(1258,234){\rule[-0.175pt]{1.273pt}{0.350pt}}
\put(1253,235){\rule[-0.175pt]{1.273pt}{0.350pt}}
\put(1248,236){\rule[-0.175pt]{1.273pt}{0.350pt}}
\put(1242,237){\rule[-0.175pt]{1.273pt}{0.350pt}}
\put(1237,238){\rule[-0.175pt]{1.273pt}{0.350pt}}
\put(1232,239){\rule[-0.175pt]{1.273pt}{0.350pt}}
\put(1227,240){\rule[-0.175pt]{1.273pt}{0.350pt}}
\put(1219,241){\rule[-0.175pt]{1.847pt}{0.350pt}}
\put(1211,242){\rule[-0.175pt]{1.847pt}{0.350pt}}
\put(1204,243){\rule[-0.175pt]{1.847pt}{0.350pt}}
\put(1196,244){\rule[-0.175pt]{1.847pt}{0.350pt}}
\put(1188,245){\rule[-0.175pt]{1.847pt}{0.350pt}}
\put(1181,246){\rule[-0.175pt]{1.847pt}{0.350pt}}
\put(1172,247){\rule[-0.175pt]{2.128pt}{0.350pt}}
\put(1163,248){\rule[-0.175pt]{2.128pt}{0.350pt}}
\put(1154,249){\rule[-0.175pt]{2.128pt}{0.350pt}}
\put(1145,250){\rule[-0.175pt]{2.128pt}{0.350pt}}
\put(1136,251){\rule[-0.175pt]{2.128pt}{0.350pt}}
\put(1128,252){\rule[-0.175pt]{2.128pt}{0.350pt}}
\put(1120,253){\rule[-0.175pt]{1.893pt}{0.350pt}}
\put(1112,254){\rule[-0.175pt]{1.893pt}{0.350pt}}
\put(1104,255){\rule[-0.175pt]{1.893pt}{0.350pt}}
\put(1096,256){\rule[-0.175pt]{1.893pt}{0.350pt}}
\put(1088,257){\rule[-0.175pt]{1.893pt}{0.350pt}}
\put(1080,258){\rule[-0.175pt]{1.893pt}{0.350pt}}
\put(1073,259){\rule[-0.175pt]{1.893pt}{0.350pt}}
\put(1063,260){\rule[-0.175pt]{2.208pt}{0.350pt}}
\put(1054,261){\rule[-0.175pt]{2.208pt}{0.350pt}}
\put(1045,262){\rule[-0.175pt]{2.208pt}{0.350pt}}
\put(1036,263){\rule[-0.175pt]{2.208pt}{0.350pt}}
\put(1027,264){\rule[-0.175pt]{2.208pt}{0.350pt}}
\put(1018,265){\rule[-0.175pt]{2.208pt}{0.350pt}}
\put(1009,266){\rule[-0.175pt]{2.168pt}{0.350pt}}
\put(1000,267){\rule[-0.175pt]{2.168pt}{0.350pt}}
\put(991,268){\rule[-0.175pt]{2.168pt}{0.350pt}}
\put(982,269){\rule[-0.175pt]{2.168pt}{0.350pt}}
\put(973,270){\rule[-0.175pt]{2.168pt}{0.350pt}}
\put(964,271){\rule[-0.175pt]{2.168pt}{0.350pt}}
\put(957,272){\rule[-0.175pt]{1.686pt}{0.350pt}}
\put(950,273){\rule[-0.175pt]{1.686pt}{0.350pt}}
\put(943,274){\rule[-0.175pt]{1.686pt}{0.350pt}}
\put(936,275){\rule[-0.175pt]{1.686pt}{0.350pt}}
\put(929,276){\rule[-0.175pt]{1.686pt}{0.350pt}}
\put(922,277){\rule[-0.175pt]{1.686pt}{0.350pt}}
\put(915,278){\rule[-0.175pt]{1.686pt}{0.350pt}}
\put(907,279){\rule[-0.175pt]{1.767pt}{0.350pt}}
\put(900,280){\rule[-0.175pt]{1.767pt}{0.350pt}}
\put(893,281){\rule[-0.175pt]{1.767pt}{0.350pt}}
\put(885,282){\rule[-0.175pt]{1.767pt}{0.350pt}}
\put(878,283){\rule[-0.175pt]{1.767pt}{0.350pt}}
\put(871,284){\rule[-0.175pt]{1.767pt}{0.350pt}}
\put(864,285){\rule[-0.175pt]{1.486pt}{0.350pt}}
\put(858,286){\rule[-0.175pt]{1.486pt}{0.350pt}}
\put(852,287){\rule[-0.175pt]{1.486pt}{0.350pt}}
\put(846,288){\rule[-0.175pt]{1.486pt}{0.350pt}}
\put(840,289){\rule[-0.175pt]{1.486pt}{0.350pt}}
\put(834,290){\rule[-0.175pt]{1.486pt}{0.350pt}}
\put(829,291){\rule[-0.175pt]{0.998pt}{0.350pt}}
\put(825,292){\rule[-0.175pt]{0.998pt}{0.350pt}}
\put(821,293){\rule[-0.175pt]{0.998pt}{0.350pt}}
\put(817,294){\rule[-0.175pt]{0.998pt}{0.350pt}}
\put(813,295){\rule[-0.175pt]{0.998pt}{0.350pt}}
\put(809,296){\rule[-0.175pt]{0.998pt}{0.350pt}}
\put(805,297){\rule[-0.175pt]{0.998pt}{0.350pt}}
\put(801,298){\rule[-0.175pt]{0.883pt}{0.350pt}}
\put(797,299){\rule[-0.175pt]{0.883pt}{0.350pt}}
\put(793,300){\rule[-0.175pt]{0.883pt}{0.350pt}}
\put(790,301){\rule[-0.175pt]{0.883pt}{0.350pt}}
\put(786,302){\rule[-0.175pt]{0.883pt}{0.350pt}}
\put(783,303){\rule[-0.175pt]{0.883pt}{0.350pt}}
\put(780,304){\rule[-0.175pt]{0.522pt}{0.350pt}}
\put(778,305){\rule[-0.175pt]{0.522pt}{0.350pt}}
\put(776,306){\rule[-0.175pt]{0.522pt}{0.350pt}}
\put(774,307){\rule[-0.175pt]{0.522pt}{0.350pt}}
\put(772,308){\rule[-0.175pt]{0.522pt}{0.350pt}}
\put(770,309){\rule[-0.175pt]{0.522pt}{0.350pt}}
\put(770,310){\usebox{\plotpoint}}
\put(769,311){\usebox{\plotpoint}}
\put(768,312){\usebox{\plotpoint}}
\put(767,314){\usebox{\plotpoint}}
\put(766,315){\usebox{\plotpoint}}
\put(765,316){\rule[-0.175pt]{0.350pt}{0.723pt}}
\put(766,320){\rule[-0.175pt]{0.350pt}{0.723pt}}
\put(767,323){\usebox{\plotpoint}}
\put(768,324){\usebox{\plotpoint}}
\put(769,325){\usebox{\plotpoint}}
\put(771,326){\usebox{\plotpoint}}
\put(772,327){\usebox{\plotpoint}}
\put(774,328){\usebox{\plotpoint}}
\put(775,329){\usebox{\plotpoint}}
\put(777,330){\rule[-0.175pt]{0.602pt}{0.350pt}}
\put(779,331){\rule[-0.175pt]{0.602pt}{0.350pt}}
\put(782,332){\rule[-0.175pt]{0.602pt}{0.350pt}}
\put(784,333){\rule[-0.175pt]{0.602pt}{0.350pt}}
\put(787,334){\rule[-0.175pt]{0.602pt}{0.350pt}}
\put(789,335){\rule[-0.175pt]{0.602pt}{0.350pt}}
\put(792,336){\rule[-0.175pt]{0.843pt}{0.350pt}}
\put(795,337){\rule[-0.175pt]{0.843pt}{0.350pt}}
\put(799,338){\rule[-0.175pt]{0.843pt}{0.350pt}}
\put(802,339){\rule[-0.175pt]{0.843pt}{0.350pt}}
\put(806,340){\rule[-0.175pt]{0.843pt}{0.350pt}}
\put(809,341){\rule[-0.175pt]{0.843pt}{0.350pt}}
\put(813,342){\rule[-0.175pt]{0.826pt}{0.350pt}}
\put(816,343){\rule[-0.175pt]{0.826pt}{0.350pt}}
\put(819,344){\rule[-0.175pt]{0.826pt}{0.350pt}}
\put(823,345){\rule[-0.175pt]{0.826pt}{0.350pt}}
\put(826,346){\rule[-0.175pt]{0.826pt}{0.350pt}}
\put(830,347){\rule[-0.175pt]{0.826pt}{0.350pt}}
\put(833,348){\rule[-0.175pt]{0.826pt}{0.350pt}}
\put(837,349){\rule[-0.175pt]{1.084pt}{0.350pt}}
\put(841,350){\rule[-0.175pt]{1.084pt}{0.350pt}}
\put(846,351){\rule[-0.175pt]{1.084pt}{0.350pt}}
\put(850,352){\rule[-0.175pt]{1.084pt}{0.350pt}}
\put(855,353){\rule[-0.175pt]{1.084pt}{0.350pt}}
\put(859,354){\rule[-0.175pt]{1.084pt}{0.350pt}}
\put(864,355){\rule[-0.175pt]{1.164pt}{0.350pt}}
\put(868,356){\rule[-0.175pt]{1.164pt}{0.350pt}}
\put(873,357){\rule[-0.175pt]{1.164pt}{0.350pt}}
\put(878,358){\rule[-0.175pt]{1.164pt}{0.350pt}}
\put(883,359){\rule[-0.175pt]{1.164pt}{0.350pt}}
\put(888,360){\rule[-0.175pt]{1.164pt}{0.350pt}}
\put(892,361){\rule[-0.175pt]{1.032pt}{0.350pt}}
\put(897,362){\rule[-0.175pt]{1.032pt}{0.350pt}}
\put(901,363){\rule[-0.175pt]{1.032pt}{0.350pt}}
\put(905,364){\rule[-0.175pt]{1.032pt}{0.350pt}}
\put(910,365){\rule[-0.175pt]{1.032pt}{0.350pt}}
\put(914,366){\rule[-0.175pt]{1.032pt}{0.350pt}}
\put(918,367){\rule[-0.175pt]{1.032pt}{0.350pt}}
\put(922,368){\rule[-0.175pt]{1.205pt}{0.350pt}}
\put(928,369){\rule[-0.175pt]{1.204pt}{0.350pt}}
\put(933,370){\rule[-0.175pt]{1.204pt}{0.350pt}}
\put(938,371){\rule[-0.175pt]{1.204pt}{0.350pt}}
\put(943,372){\rule[-0.175pt]{1.204pt}{0.350pt}}
\put(948,373){\rule[-0.175pt]{1.204pt}{0.350pt}}
\put(953,374){\rule[-0.175pt]{1.124pt}{0.350pt}}
\put(957,375){\rule[-0.175pt]{1.124pt}{0.350pt}}
\put(962,376){\rule[-0.175pt]{1.124pt}{0.350pt}}
\put(967,377){\rule[-0.175pt]{1.124pt}{0.350pt}}
\put(971,378){\rule[-0.175pt]{1.124pt}{0.350pt}}
\put(976,379){\rule[-0.175pt]{1.124pt}{0.350pt}}
\put(981,380){\rule[-0.175pt]{0.929pt}{0.350pt}}
\put(984,381){\rule[-0.175pt]{0.929pt}{0.350pt}}
\put(988,382){\rule[-0.175pt]{0.929pt}{0.350pt}}
\put(992,383){\rule[-0.175pt]{0.929pt}{0.350pt}}
\put(996,384){\rule[-0.175pt]{0.929pt}{0.350pt}}
\put(1000,385){\rule[-0.175pt]{0.929pt}{0.350pt}}
\put(1004,386){\rule[-0.175pt]{0.929pt}{0.350pt}}
\put(1007,387){\rule[-0.175pt]{0.923pt}{0.350pt}}
\put(1011,388){\rule[-0.175pt]{0.923pt}{0.350pt}}
\put(1015,389){\rule[-0.175pt]{0.923pt}{0.350pt}}
\put(1019,390){\rule[-0.175pt]{0.923pt}{0.350pt}}
\put(1023,391){\rule[-0.175pt]{0.923pt}{0.350pt}}
\put(1027,392){\rule[-0.175pt]{0.923pt}{0.350pt}}
\put(1031,393){\rule[-0.175pt]{0.843pt}{0.350pt}}
\put(1034,394){\rule[-0.175pt]{0.843pt}{0.350pt}}
\put(1038,395){\rule[-0.175pt]{0.843pt}{0.350pt}}
\put(1041,396){\rule[-0.175pt]{0.843pt}{0.350pt}}
\put(1045,397){\rule[-0.175pt]{0.843pt}{0.350pt}}
\put(1048,398){\rule[-0.175pt]{0.843pt}{0.350pt}}
\put(1052,399){\rule[-0.175pt]{0.585pt}{0.350pt}}
\put(1054,400){\rule[-0.175pt]{0.585pt}{0.350pt}}
\put(1056,401){\rule[-0.175pt]{0.585pt}{0.350pt}}
\put(1059,402){\rule[-0.175pt]{0.585pt}{0.350pt}}
\put(1061,403){\rule[-0.175pt]{0.585pt}{0.350pt}}
\put(1064,404){\rule[-0.175pt]{0.585pt}{0.350pt}}
\put(1066,405){\rule[-0.175pt]{0.585pt}{0.350pt}}
\put(1069,406){\rule[-0.175pt]{0.522pt}{0.350pt}}
\put(1071,407){\rule[-0.175pt]{0.522pt}{0.350pt}}
\put(1073,408){\rule[-0.175pt]{0.522pt}{0.350pt}}
\put(1075,409){\rule[-0.175pt]{0.522pt}{0.350pt}}
\put(1077,410){\rule[-0.175pt]{0.522pt}{0.350pt}}
\put(1079,411){\rule[-0.175pt]{0.522pt}{0.350pt}}
\put(1081,412){\rule[-0.175pt]{0.402pt}{0.350pt}}
\put(1083,413){\rule[-0.175pt]{0.401pt}{0.350pt}}
\put(1085,414){\rule[-0.175pt]{0.401pt}{0.350pt}}
\put(1086,415){\rule[-0.175pt]{0.401pt}{0.350pt}}
\put(1088,416){\rule[-0.175pt]{0.401pt}{0.350pt}}
\put(1090,417){\rule[-0.175pt]{0.401pt}{0.350pt}}
\put(1091,418){\usebox{\plotpoint}}
\put(1092,418){\usebox{\plotpoint}}
\put(1093,419){\usebox{\plotpoint}}
\put(1094,420){\usebox{\plotpoint}}
\put(1095,422){\usebox{\plotpoint}}
\put(1096,423){\usebox{\plotpoint}}
\put(1097,424){\rule[-0.175pt]{0.350pt}{0.723pt}}
\put(1098,428){\rule[-0.175pt]{0.350pt}{0.723pt}}
\put(1099,431){\rule[-0.175pt]{0.350pt}{1.686pt}}
\put(1098,438){\usebox{\plotpoint}}
\put(1097,439){\usebox{\plotpoint}}
\put(1096,440){\usebox{\plotpoint}}
\put(1095,441){\usebox{\plotpoint}}
\put(1094,442){\usebox{\plotpoint}}
\put(1091,444){\usebox{\plotpoint}}
\put(1090,445){\usebox{\plotpoint}}
\put(1089,446){\usebox{\plotpoint}}
\put(1088,447){\usebox{\plotpoint}}
\put(1087,448){\usebox{\plotpoint}}
\put(1086,449){\usebox{\plotpoint}}
\put(1084,450){\usebox{\plotpoint}}
\put(1083,451){\usebox{\plotpoint}}
\put(1081,452){\usebox{\plotpoint}}
\put(1080,453){\usebox{\plotpoint}}
\put(1078,454){\usebox{\plotpoint}}
\put(1077,455){\usebox{\plotpoint}}
\put(1076,456){\usebox{\plotpoint}}
\put(1074,457){\rule[-0.175pt]{0.442pt}{0.350pt}}
\put(1072,458){\rule[-0.175pt]{0.442pt}{0.350pt}}
\put(1070,459){\rule[-0.175pt]{0.442pt}{0.350pt}}
\put(1068,460){\rule[-0.175pt]{0.442pt}{0.350pt}}
\put(1066,461){\rule[-0.175pt]{0.442pt}{0.350pt}}
\put(1065,462){\rule[-0.175pt]{0.442pt}{0.350pt}}
\put(1063,463){\rule[-0.175pt]{0.482pt}{0.350pt}}
\put(1061,464){\rule[-0.175pt]{0.482pt}{0.350pt}}
\put(1059,465){\rule[-0.175pt]{0.482pt}{0.350pt}}
\put(1057,466){\rule[-0.175pt]{0.482pt}{0.350pt}}
\put(1055,467){\rule[-0.175pt]{0.482pt}{0.350pt}}
\put(1053,468){\rule[-0.175pt]{0.482pt}{0.350pt}}
\put(1051,469){\rule[-0.175pt]{0.482pt}{0.350pt}}
\put(1049,470){\rule[-0.175pt]{0.482pt}{0.350pt}}
\put(1047,471){\rule[-0.175pt]{0.482pt}{0.350pt}}
\put(1045,472){\rule[-0.175pt]{0.482pt}{0.350pt}}
\put(1043,473){\rule[-0.175pt]{0.482pt}{0.350pt}}
\put(1041,474){\rule[-0.175pt]{0.482pt}{0.350pt}}
\put(1039,475){\rule[-0.175pt]{0.482pt}{0.350pt}}
\put(1036,476){\rule[-0.175pt]{0.522pt}{0.350pt}}
\put(1034,477){\rule[-0.175pt]{0.522pt}{0.350pt}}
\put(1032,478){\rule[-0.175pt]{0.522pt}{0.350pt}}
\put(1030,479){\rule[-0.175pt]{0.522pt}{0.350pt}}
\put(1028,480){\rule[-0.175pt]{0.522pt}{0.350pt}}
\put(1026,481){\rule[-0.175pt]{0.522pt}{0.350pt}}
\put(1023,482){\rule[-0.175pt]{0.522pt}{0.350pt}}
\put(1021,483){\rule[-0.175pt]{0.522pt}{0.350pt}}
\put(1019,484){\rule[-0.175pt]{0.522pt}{0.350pt}}
\put(1017,485){\rule[-0.175pt]{0.522pt}{0.350pt}}
\put(1015,486){\rule[-0.175pt]{0.522pt}{0.350pt}}
\put(1013,487){\rule[-0.175pt]{0.522pt}{0.350pt}}
\put(1011,488){\rule[-0.175pt]{0.447pt}{0.350pt}}
\put(1009,489){\rule[-0.175pt]{0.447pt}{0.350pt}}
\put(1007,490){\rule[-0.175pt]{0.447pt}{0.350pt}}
\put(1005,491){\rule[-0.175pt]{0.447pt}{0.350pt}}
\put(1003,492){\rule[-0.175pt]{0.447pt}{0.350pt}}
\put(1001,493){\rule[-0.175pt]{0.447pt}{0.350pt}}
\put(1000,494){\rule[-0.175pt]{0.447pt}{0.350pt}}
\put(998,495){\rule[-0.175pt]{0.442pt}{0.350pt}}
\put(996,496){\rule[-0.175pt]{0.442pt}{0.350pt}}
\put(994,497){\rule[-0.175pt]{0.442pt}{0.350pt}}
\put(992,498){\rule[-0.175pt]{0.442pt}{0.350pt}}
\put(990,499){\rule[-0.175pt]{0.442pt}{0.350pt}}
\put(989,500){\rule[-0.175pt]{0.442pt}{0.350pt}}
\put(987,501){\rule[-0.175pt]{0.442pt}{0.350pt}}
\put(985,502){\rule[-0.175pt]{0.442pt}{0.350pt}}
\put(983,503){\rule[-0.175pt]{0.442pt}{0.350pt}}
\put(981,504){\rule[-0.175pt]{0.442pt}{0.350pt}}
\put(979,505){\rule[-0.175pt]{0.442pt}{0.350pt}}
\put(978,506){\rule[-0.175pt]{0.442pt}{0.350pt}}
\put(976,507){\usebox{\plotpoint}}
\put(975,508){\usebox{\plotpoint}}
\put(974,509){\usebox{\plotpoint}}
\put(972,510){\usebox{\plotpoint}}
\put(971,511){\usebox{\plotpoint}}
\put(970,512){\usebox{\plotpoint}}
\put(969,513){\usebox{\plotpoint}}
\put(967,514){\usebox{\plotpoint}}
\put(966,515){\usebox{\plotpoint}}
\put(965,516){\usebox{\plotpoint}}
\put(964,517){\usebox{\plotpoint}}
\put(963,518){\usebox{\plotpoint}}
\put(962,519){\usebox{\plotpoint}}
\put(962,520){\usebox{\plotpoint}}
\put(961,521){\usebox{\plotpoint}}
\put(960,522){\usebox{\plotpoint}}
\put(959,524){\usebox{\plotpoint}}
\put(958,525){\usebox{\plotpoint}}
\put(957,527){\rule[-0.175pt]{0.350pt}{0.361pt}}
\put(956,528){\rule[-0.175pt]{0.350pt}{0.361pt}}
\put(955,530){\rule[-0.175pt]{0.350pt}{0.361pt}}
\put(954,531){\rule[-0.175pt]{0.350pt}{0.361pt}}
\put(953,533){\rule[-0.175pt]{0.350pt}{0.723pt}}
\put(952,536){\rule[-0.175pt]{0.350pt}{0.723pt}}
\put(951,539){\rule[-0.175pt]{0.350pt}{1.686pt}}
\put(950,546){\rule[-0.175pt]{0.350pt}{1.445pt}}
\put(951,552){\rule[-0.175pt]{0.350pt}{0.482pt}}
\put(952,554){\rule[-0.175pt]{0.350pt}{0.482pt}}
\put(953,556){\rule[-0.175pt]{0.350pt}{0.482pt}}
\put(954,558){\rule[-0.175pt]{0.350pt}{0.562pt}}
\put(955,560){\rule[-0.175pt]{0.350pt}{0.562pt}}
\put(956,562){\rule[-0.175pt]{0.350pt}{0.562pt}}
\put(957,564){\usebox{\plotpoint}}
\put(958,566){\usebox{\plotpoint}}
\put(959,567){\usebox{\plotpoint}}
\put(960,568){\usebox{\plotpoint}}
\put(961,569){\usebox{\plotpoint}}
\put(962,571){\usebox{\plotpoint}}
\put(963,572){\usebox{\plotpoint}}
\put(964,573){\usebox{\plotpoint}}
\put(965,574){\usebox{\plotpoint}}
\put(966,575){\usebox{\plotpoint}}
\put(967,577){\usebox{\plotpoint}}
\put(968,578){\usebox{\plotpoint}}
\put(969,579){\usebox{\plotpoint}}
\put(970,580){\usebox{\plotpoint}}
\put(971,581){\usebox{\plotpoint}}
\put(972,582){\usebox{\plotpoint}}
\put(973,584){\usebox{\plotpoint}}
\put(974,585){\usebox{\plotpoint}}
\put(975,586){\usebox{\plotpoint}}
\put(976,587){\usebox{\plotpoint}}
\put(977,588){\usebox{\plotpoint}}
\put(978,589){\usebox{\plotpoint}}
\put(979,590){\usebox{\plotpoint}}
\put(980,591){\usebox{\plotpoint}}
\put(981,592){\usebox{\plotpoint}}
\put(982,593){\usebox{\plotpoint}}
\put(983,594){\usebox{\plotpoint}}
\put(984,595){\usebox{\plotpoint}}
\put(985,596){\usebox{\plotpoint}}
\put(986,597){\usebox{\plotpoint}}
\put(987,598){\usebox{\plotpoint}}
\put(988,600){\usebox{\plotpoint}}
\put(989,601){\usebox{\plotpoint}}
\put(990,603){\usebox{\plotpoint}}
\put(991,604){\usebox{\plotpoint}}
\put(992,605){\usebox{\plotpoint}}
\put(993,606){\usebox{\plotpoint}}
\put(994,607){\usebox{\plotpoint}}
\put(995,608){\usebox{\plotpoint}}
\put(996,609){\usebox{\plotpoint}}
\put(997,610){\usebox{\plotpoint}}
\put(998,611){\usebox{\plotpoint}}
\put(999,612){\usebox{\plotpoint}}
\put(1000,613){\usebox{\plotpoint}}
\put(1001,615){\usebox{\plotpoint}}
\put(1002,616){\usebox{\plotpoint}}
\put(1003,617){\usebox{\plotpoint}}
\put(1004,619){\usebox{\plotpoint}}
\put(1005,620){\usebox{\plotpoint}}
\put(1006,622){\rule[-0.175pt]{0.350pt}{0.361pt}}
\put(1007,623){\rule[-0.175pt]{0.350pt}{0.361pt}}
\put(1008,625){\rule[-0.175pt]{0.350pt}{0.361pt}}
\put(1009,626){\rule[-0.175pt]{0.350pt}{0.361pt}}
\put(1010,628){\rule[-0.175pt]{0.350pt}{0.562pt}}
\put(1011,630){\rule[-0.175pt]{0.350pt}{0.562pt}}
\put(1012,632){\rule[-0.175pt]{0.350pt}{0.562pt}}
\put(1013,634){\rule[-0.175pt]{0.350pt}{0.723pt}}
\put(1014,638){\rule[-0.175pt]{0.350pt}{0.723pt}}
\put(1015,641){\rule[-0.175pt]{0.350pt}{1.445pt}}
\put(1016,647){\rule[-0.175pt]{0.350pt}{1.686pt}}
\put(1017,654){\rule[-0.175pt]{0.350pt}{3.734pt}}
\put(1016,669){\rule[-0.175pt]{0.350pt}{0.843pt}}
\put(1015,673){\rule[-0.175pt]{0.350pt}{1.445pt}}
\put(1014,679){\rule[-0.175pt]{0.350pt}{0.723pt}}
\put(1013,682){\rule[-0.175pt]{0.350pt}{0.723pt}}
\put(1012,685){\rule[-0.175pt]{0.350pt}{0.562pt}}
\put(1011,687){\rule[-0.175pt]{0.350pt}{0.562pt}}
\put(1010,689){\rule[-0.175pt]{0.350pt}{0.562pt}}
\put(1009,691){\rule[-0.175pt]{0.350pt}{0.723pt}}
\put(1008,695){\rule[-0.175pt]{0.350pt}{0.723pt}}
\put(1007,698){\rule[-0.175pt]{0.350pt}{0.482pt}}
\put(1006,700){\rule[-0.175pt]{0.350pt}{0.482pt}}
\put(1005,702){\rule[-0.175pt]{0.350pt}{0.482pt}}
\put(1004,704){\rule[-0.175pt]{0.350pt}{0.562pt}}
\put(1003,706){\rule[-0.175pt]{0.350pt}{0.562pt}}
\put(1002,708){\rule[-0.175pt]{0.350pt}{0.562pt}}
\put(1001,710){\rule[-0.175pt]{0.350pt}{0.723pt}}
\put(1000,714){\rule[-0.175pt]{0.350pt}{0.723pt}}
\put(999,717){\rule[-0.175pt]{0.350pt}{0.482pt}}
\put(998,719){\rule[-0.175pt]{0.350pt}{0.482pt}}
\put(997,721){\rule[-0.175pt]{0.350pt}{0.482pt}}
\put(996,723){\rule[-0.175pt]{0.350pt}{0.843pt}}
\put(995,726){\rule[-0.175pt]{0.350pt}{0.843pt}}
\put(994,730){\rule[-0.175pt]{0.350pt}{0.723pt}}
\put(993,733){\rule[-0.175pt]{0.350pt}{0.723pt}}
\put(992,736){\rule[-0.175pt]{0.350pt}{0.843pt}}
\put(991,739){\rule[-0.175pt]{0.350pt}{0.843pt}}
\put(990,743){\rule[-0.175pt]{0.350pt}{1.445pt}}
\put(989,749){\rule[-0.175pt]{0.350pt}{1.445pt}}
\put(988,755){\rule[-0.175pt]{0.350pt}{1.686pt}}
\put(987,762){\rule[-0.175pt]{0.350pt}{4.577pt}}
\put(988,781){\rule[-0.175pt]{0.350pt}{1.445pt}}
\end{picture}

\caption{Gelfand equation on the ball, $3\leq n \leq 9$.
\label{gelfand.fig2}}
\end{figure}
\begin{quote}\tt\singlespace\begin{verbatim}
\begin{figure}[htbp]
\centering
% GNUPLOT: LaTeX picture
\setlength{\unitlength}{0.240900pt}
\ifx\plotpoint\undefined\newsavebox{\plotpoint}\fi
\sbox{\plotpoint}{\rule[-0.175pt]{0.350pt}{0.350pt}}%
\begin{picture}(1500,900)(0,0)
%\tenrm
\sbox{\plotpoint}{\rule[-0.175pt]{0.350pt}{0.350pt}}%
\put(264,158){\rule[-0.175pt]{282.335pt}{0.350pt}}
\put(264,158){\rule[-0.175pt]{0.350pt}{151.526pt}}
\put(264,158){\rule[-0.175pt]{4.818pt}{0.350pt}}
%\put(242,158){\makebox(0,0)[r]{0}}
\put(1416,158){\rule[-0.175pt]{4.818pt}{0.350pt}}
\put(264,284){\rule[-0.175pt]{4.818pt}{0.350pt}}
%\put(242,284){\makebox(0,0)[r]{2}}
\put(1416,284){\rule[-0.175pt]{4.818pt}{0.350pt}}
\put(264,410){\rule[-0.175pt]{4.818pt}{0.350pt}}
%\put(242,410){\makebox(0,0)[r]{4}}
\put(1416,410){\rule[-0.175pt]{4.818pt}{0.350pt}}
\put(264,535){\rule[-0.175pt]{4.818pt}{0.350pt}}
%\put(242,535){\makebox(0,0)[r]{6}}
\put(1416,535){\rule[-0.175pt]{4.818pt}{0.350pt}}
\put(264,661){\rule[-0.175pt]{4.818pt}{0.350pt}}
%\put(242,661){\makebox(0,0)[r]{8}}
\put(1416,661){\rule[-0.175pt]{4.818pt}{0.350pt}}
\put(264,787){\rule[-0.175pt]{4.818pt}{0.350pt}}
%\put(242,787){\makebox(0,0)[r]{10}}
\put(1416,787){\rule[-0.175pt]{4.818pt}{0.350pt}}
\put(264,158){\rule[-0.175pt]{0.350pt}{4.818pt}}
%\put(264,113){\makebox(0,0){0}}
\put(264,767){\rule[-0.175pt]{0.350pt}{4.818pt}}
\put(411,158){\rule[-0.175pt]{0.350pt}{4.818pt}}
%\put(411,113){\makebox(0,0){0.5}}
\put(411,767){\rule[-0.175pt]{0.350pt}{4.818pt}}
\put(557,158){\rule[-0.175pt]{0.350pt}{4.818pt}}
%\put(557,113){\makebox(0,0){1}}
\put(557,767){\rule[-0.175pt]{0.350pt}{4.818pt}}
\put(704,158){\rule[-0.175pt]{0.350pt}{4.818pt}}
%\put(704,113){\makebox(0,0){1.5}}
\put(704,767){\rule[-0.175pt]{0.350pt}{4.818pt}}
\put(850,158){\rule[-0.175pt]{0.350pt}{4.818pt}}
%\put(850,113){\makebox(0,0){2}}
\put(850,767){\rule[-0.175pt]{0.350pt}{4.818pt}}
\put(997,158){\rule[-0.175pt]{0.350pt}{4.818pt}}
%\put(997,113){\makebox(0,0){2.5}}
\put(997,767){\rule[-0.175pt]{0.350pt}{4.818pt}}
\put(1143,158){\rule[-0.175pt]{0.350pt}{4.818pt}}
%\put(1143,113){\makebox(0,0){3}}
\put(1143,767){\rule[-0.175pt]{0.350pt}{4.818pt}}
\put(1290,158){\rule[-0.175pt]{0.350pt}{4.818pt}}
%\put(1290,113){\makebox(0,0){3.5}}
\put(1290,767){\rule[-0.175pt]{0.350pt}{4.818pt}}
\put(1436,158){\rule[-0.175pt]{0.350pt}{4.818pt}}
%\put(1436,113){\makebox(0,0){4}}
\put(1436,767){\rule[-0.175pt]{0.350pt}{4.818pt}}
\put(264,158){\rule[-0.175pt]{282.335pt}{0.350pt}}
\put(1436,158){\rule[-0.175pt]{0.350pt}{151.526pt}}
\put(264,787){\rule[-0.175pt]{282.335pt}{0.350pt}}
\put(100,472){\makebox(0,0)[l]{\shortstack{$\| u\|$}}}
\put(850,68){\makebox(0,0){$\lambda$}}
%\put(850,832){\makebox(0,0){plot}}
\put(264,158){\rule[-0.175pt]{0.350pt}{151.526pt}}
%\put(1306,722){\makebox(0,0)[r]{}}
%\put(1328,722){\rule[-0.175pt]{15.899pt}{0.350pt}}
\put(264,158){\usebox{\plotpoint}}
\put(264,158){\rule[-0.175pt]{6.304pt}{0.350pt}}
\put(290,159){\rule[-0.175pt]{6.304pt}{0.350pt}}
\put(316,160){\rule[-0.175pt]{6.304pt}{0.350pt}}
\put(342,161){\rule[-0.175pt]{6.304pt}{0.350pt}}
\put(368,162){\rule[-0.175pt]{6.304pt}{0.350pt}}
\put(394,163){\rule[-0.175pt]{6.304pt}{0.350pt}}
\put(420,164){\rule[-0.175pt]{5.644pt}{0.350pt}}
\put(444,165){\rule[-0.175pt]{5.644pt}{0.350pt}}
\put(467,166){\rule[-0.175pt]{5.644pt}{0.350pt}}
\put(491,167){\rule[-0.175pt]{5.644pt}{0.350pt}}
\put(514,168){\rule[-0.175pt]{5.644pt}{0.350pt}}
\put(538,169){\rule[-0.175pt]{5.644pt}{0.350pt}}
\put(561,170){\rule[-0.175pt]{5.644pt}{0.350pt}}
\put(585,171){\rule[-0.175pt]{6.384pt}{0.350pt}}
\put(611,172){\rule[-0.175pt]{6.384pt}{0.350pt}}
\put(638,173){\rule[-0.175pt]{6.384pt}{0.350pt}}
\put(664,174){\rule[-0.175pt]{6.384pt}{0.350pt}}
\put(691,175){\rule[-0.175pt]{6.384pt}{0.350pt}}
\put(717,176){\rule[-0.175pt]{6.384pt}{0.350pt}}
\put(744,177){\rule[-0.175pt]{5.862pt}{0.350pt}}
\put(768,178){\rule[-0.175pt]{5.862pt}{0.350pt}}
\put(792,179){\rule[-0.175pt]{5.862pt}{0.350pt}}
\put(816,180){\rule[-0.175pt]{5.862pt}{0.350pt}}
\put(841,181){\rule[-0.175pt]{5.862pt}{0.350pt}}
\put(865,182){\rule[-0.175pt]{5.862pt}{0.350pt}}
\put(889,183){\rule[-0.175pt]{4.371pt}{0.350pt}}
\put(908,184){\rule[-0.175pt]{4.371pt}{0.350pt}}
\put(926,185){\rule[-0.175pt]{4.371pt}{0.350pt}}
\put(944,186){\rule[-0.175pt]{4.371pt}{0.350pt}}
\put(962,187){\rule[-0.175pt]{4.371pt}{0.350pt}}
\put(980,188){\rule[-0.175pt]{4.371pt}{0.350pt}}
\put(998,189){\rule[-0.175pt]{4.371pt}{0.350pt}}
\put(1017,190){\rule[-0.175pt]{4.216pt}{0.350pt}}
\put(1034,191){\rule[-0.175pt]{4.216pt}{0.350pt}}
\put(1052,192){\rule[-0.175pt]{4.216pt}{0.350pt}}
\put(1069,193){\rule[-0.175pt]{4.216pt}{0.350pt}}
\put(1087,194){\rule[-0.175pt]{4.216pt}{0.350pt}}
\put(1104,195){\rule[-0.175pt]{4.216pt}{0.350pt}}
\put(1122,196){\rule[-0.175pt]{3.172pt}{0.350pt}}
\put(1135,197){\rule[-0.175pt]{3.172pt}{0.350pt}}
\put(1148,198){\rule[-0.175pt]{3.172pt}{0.350pt}}
\put(1161,199){\rule[-0.175pt]{3.172pt}{0.350pt}}
\put(1174,200){\rule[-0.175pt]{3.172pt}{0.350pt}}
\put(1187,201){\rule[-0.175pt]{3.172pt}{0.350pt}}
\put(1200,202){\rule[-0.175pt]{1.893pt}{0.350pt}}
\put(1208,203){\rule[-0.175pt]{1.893pt}{0.350pt}}
\put(1216,204){\rule[-0.175pt]{1.893pt}{0.350pt}}
\put(1224,205){\rule[-0.175pt]{1.893pt}{0.350pt}}
\put(1232,206){\rule[-0.175pt]{1.893pt}{0.350pt}}
\put(1240,207){\rule[-0.175pt]{1.893pt}{0.350pt}}
\put(1248,208){\rule[-0.175pt]{1.893pt}{0.350pt}}
\put(1256,209){\rule[-0.175pt]{1.245pt}{0.350pt}}
\put(1261,210){\rule[-0.175pt]{1.245pt}{0.350pt}}
\put(1266,211){\rule[-0.175pt]{1.245pt}{0.350pt}}
\put(1271,212){\rule[-0.175pt]{1.245pt}{0.350pt}}
\put(1276,213){\rule[-0.175pt]{1.245pt}{0.350pt}}
\put(1281,214){\rule[-0.175pt]{1.245pt}{0.350pt}}
\put(1286,215){\usebox{\plotpoint}}
\put(1288,216){\usebox{\plotpoint}}
\put(1289,217){\usebox{\plotpoint}}
\put(1291,218){\usebox{\plotpoint}}
\put(1292,219){\usebox{\plotpoint}}
\put(1294,220){\usebox{\plotpoint}}
\put(1295,221){\usebox{\plotpoint}}
\put(1295,222){\rule[-0.175pt]{0.361pt}{0.350pt}}
\put(1294,223){\rule[-0.175pt]{0.361pt}{0.350pt}}
\put(1292,224){\rule[-0.175pt]{0.361pt}{0.350pt}}
\put(1291,225){\rule[-0.175pt]{0.361pt}{0.350pt}}
\put(1289,226){\rule[-0.175pt]{0.361pt}{0.350pt}}
\put(1288,227){\rule[-0.175pt]{0.361pt}{0.350pt}}
\put(1284,228){\rule[-0.175pt]{0.964pt}{0.350pt}}
\put(1280,229){\rule[-0.175pt]{0.964pt}{0.350pt}}
\put(1276,230){\rule[-0.175pt]{0.964pt}{0.350pt}}
\put(1272,231){\rule[-0.175pt]{0.964pt}{0.350pt}}
\put(1268,232){\rule[-0.175pt]{0.964pt}{0.350pt}}
\put(1264,233){\rule[-0.175pt]{0.964pt}{0.350pt}}
\put(1258,234){\rule[-0.175pt]{1.273pt}{0.350pt}}
\put(1253,235){\rule[-0.175pt]{1.273pt}{0.350pt}}
\put(1248,236){\rule[-0.175pt]{1.273pt}{0.350pt}}
\put(1242,237){\rule[-0.175pt]{1.273pt}{0.350pt}}
\put(1237,238){\rule[-0.175pt]{1.273pt}{0.350pt}}
\put(1232,239){\rule[-0.175pt]{1.273pt}{0.350pt}}
\put(1227,240){\rule[-0.175pt]{1.273pt}{0.350pt}}
\put(1219,241){\rule[-0.175pt]{1.847pt}{0.350pt}}
\put(1211,242){\rule[-0.175pt]{1.847pt}{0.350pt}}
\put(1204,243){\rule[-0.175pt]{1.847pt}{0.350pt}}
\put(1196,244){\rule[-0.175pt]{1.847pt}{0.350pt}}
\put(1188,245){\rule[-0.175pt]{1.847pt}{0.350pt}}
\put(1181,246){\rule[-0.175pt]{1.847pt}{0.350pt}}
\put(1172,247){\rule[-0.175pt]{2.128pt}{0.350pt}}
\put(1163,248){\rule[-0.175pt]{2.128pt}{0.350pt}}
\put(1154,249){\rule[-0.175pt]{2.128pt}{0.350pt}}
\put(1145,250){\rule[-0.175pt]{2.128pt}{0.350pt}}
\put(1136,251){\rule[-0.175pt]{2.128pt}{0.350pt}}
\put(1128,252){\rule[-0.175pt]{2.128pt}{0.350pt}}
\put(1120,253){\rule[-0.175pt]{1.893pt}{0.350pt}}
\put(1112,254){\rule[-0.175pt]{1.893pt}{0.350pt}}
\put(1104,255){\rule[-0.175pt]{1.893pt}{0.350pt}}
\put(1096,256){\rule[-0.175pt]{1.893pt}{0.350pt}}
\put(1088,257){\rule[-0.175pt]{1.893pt}{0.350pt}}
\put(1080,258){\rule[-0.175pt]{1.893pt}{0.350pt}}
\put(1073,259){\rule[-0.175pt]{1.893pt}{0.350pt}}
\put(1063,260){\rule[-0.175pt]{2.208pt}{0.350pt}}
\put(1054,261){\rule[-0.175pt]{2.208pt}{0.350pt}}
\put(1045,262){\rule[-0.175pt]{2.208pt}{0.350pt}}
\put(1036,263){\rule[-0.175pt]{2.208pt}{0.350pt}}
\put(1027,264){\rule[-0.175pt]{2.208pt}{0.350pt}}
\put(1018,265){\rule[-0.175pt]{2.208pt}{0.350pt}}
\put(1009,266){\rule[-0.175pt]{2.168pt}{0.350pt}}
\put(1000,267){\rule[-0.175pt]{2.168pt}{0.350pt}}
\put(991,268){\rule[-0.175pt]{2.168pt}{0.350pt}}
\put(982,269){\rule[-0.175pt]{2.168pt}{0.350pt}}
\put(973,270){\rule[-0.175pt]{2.168pt}{0.350pt}}
\put(964,271){\rule[-0.175pt]{2.168pt}{0.350pt}}
\put(957,272){\rule[-0.175pt]{1.686pt}{0.350pt}}
\put(950,273){\rule[-0.175pt]{1.686pt}{0.350pt}}
\put(943,274){\rule[-0.175pt]{1.686pt}{0.350pt}}
\put(936,275){\rule[-0.175pt]{1.686pt}{0.350pt}}
\put(929,276){\rule[-0.175pt]{1.686pt}{0.350pt}}
\put(922,277){\rule[-0.175pt]{1.686pt}{0.350pt}}
\put(915,278){\rule[-0.175pt]{1.686pt}{0.350pt}}
\put(907,279){\rule[-0.175pt]{1.767pt}{0.350pt}}
\put(900,280){\rule[-0.175pt]{1.767pt}{0.350pt}}
\put(893,281){\rule[-0.175pt]{1.767pt}{0.350pt}}
\put(885,282){\rule[-0.175pt]{1.767pt}{0.350pt}}
\put(878,283){\rule[-0.175pt]{1.767pt}{0.350pt}}
\put(871,284){\rule[-0.175pt]{1.767pt}{0.350pt}}
\put(864,285){\rule[-0.175pt]{1.486pt}{0.350pt}}
\put(858,286){\rule[-0.175pt]{1.486pt}{0.350pt}}
\put(852,287){\rule[-0.175pt]{1.486pt}{0.350pt}}
\put(846,288){\rule[-0.175pt]{1.486pt}{0.350pt}}
\put(840,289){\rule[-0.175pt]{1.486pt}{0.350pt}}
\put(834,290){\rule[-0.175pt]{1.486pt}{0.350pt}}
\put(829,291){\rule[-0.175pt]{0.998pt}{0.350pt}}
\put(825,292){\rule[-0.175pt]{0.998pt}{0.350pt}}
\put(821,293){\rule[-0.175pt]{0.998pt}{0.350pt}}
\put(817,294){\rule[-0.175pt]{0.998pt}{0.350pt}}
\put(813,295){\rule[-0.175pt]{0.998pt}{0.350pt}}
\put(809,296){\rule[-0.175pt]{0.998pt}{0.350pt}}
\put(805,297){\rule[-0.175pt]{0.998pt}{0.350pt}}
\put(801,298){\rule[-0.175pt]{0.883pt}{0.350pt}}
\put(797,299){\rule[-0.175pt]{0.883pt}{0.350pt}}
\put(793,300){\rule[-0.175pt]{0.883pt}{0.350pt}}
\put(790,301){\rule[-0.175pt]{0.883pt}{0.350pt}}
\put(786,302){\rule[-0.175pt]{0.883pt}{0.350pt}}
\put(783,303){\rule[-0.175pt]{0.883pt}{0.350pt}}
\put(780,304){\rule[-0.175pt]{0.522pt}{0.350pt}}
\put(778,305){\rule[-0.175pt]{0.522pt}{0.350pt}}
\put(776,306){\rule[-0.175pt]{0.522pt}{0.350pt}}
\put(774,307){\rule[-0.175pt]{0.522pt}{0.350pt}}
\put(772,308){\rule[-0.175pt]{0.522pt}{0.350pt}}
\put(770,309){\rule[-0.175pt]{0.522pt}{0.350pt}}
\put(770,310){\usebox{\plotpoint}}
\put(769,311){\usebox{\plotpoint}}
\put(768,312){\usebox{\plotpoint}}
\put(767,314){\usebox{\plotpoint}}
\put(766,315){\usebox{\plotpoint}}
\put(765,316){\rule[-0.175pt]{0.350pt}{0.723pt}}
\put(766,320){\rule[-0.175pt]{0.350pt}{0.723pt}}
\put(767,323){\usebox{\plotpoint}}
\put(768,324){\usebox{\plotpoint}}
\put(769,325){\usebox{\plotpoint}}
\put(771,326){\usebox{\plotpoint}}
\put(772,327){\usebox{\plotpoint}}
\put(774,328){\usebox{\plotpoint}}
\put(775,329){\usebox{\plotpoint}}
\put(777,330){\rule[-0.175pt]{0.602pt}{0.350pt}}
\put(779,331){\rule[-0.175pt]{0.602pt}{0.350pt}}
\put(782,332){\rule[-0.175pt]{0.602pt}{0.350pt}}
\put(784,333){\rule[-0.175pt]{0.602pt}{0.350pt}}
\put(787,334){\rule[-0.175pt]{0.602pt}{0.350pt}}
\put(789,335){\rule[-0.175pt]{0.602pt}{0.350pt}}
\put(792,336){\rule[-0.175pt]{0.843pt}{0.350pt}}
\put(795,337){\rule[-0.175pt]{0.843pt}{0.350pt}}
\put(799,338){\rule[-0.175pt]{0.843pt}{0.350pt}}
\put(802,339){\rule[-0.175pt]{0.843pt}{0.350pt}}
\put(806,340){\rule[-0.175pt]{0.843pt}{0.350pt}}
\put(809,341){\rule[-0.175pt]{0.843pt}{0.350pt}}
\put(813,342){\rule[-0.175pt]{0.826pt}{0.350pt}}
\put(816,343){\rule[-0.175pt]{0.826pt}{0.350pt}}
\put(819,344){\rule[-0.175pt]{0.826pt}{0.350pt}}
\put(823,345){\rule[-0.175pt]{0.826pt}{0.350pt}}
\put(826,346){\rule[-0.175pt]{0.826pt}{0.350pt}}
\put(830,347){\rule[-0.175pt]{0.826pt}{0.350pt}}
\put(833,348){\rule[-0.175pt]{0.826pt}{0.350pt}}
\put(837,349){\rule[-0.175pt]{1.084pt}{0.350pt}}
\put(841,350){\rule[-0.175pt]{1.084pt}{0.350pt}}
\put(846,351){\rule[-0.175pt]{1.084pt}{0.350pt}}
\put(850,352){\rule[-0.175pt]{1.084pt}{0.350pt}}
\put(855,353){\rule[-0.175pt]{1.084pt}{0.350pt}}
\put(859,354){\rule[-0.175pt]{1.084pt}{0.350pt}}
\put(864,355){\rule[-0.175pt]{1.164pt}{0.350pt}}
\put(868,356){\rule[-0.175pt]{1.164pt}{0.350pt}}
\put(873,357){\rule[-0.175pt]{1.164pt}{0.350pt}}
\put(878,358){\rule[-0.175pt]{1.164pt}{0.350pt}}
\put(883,359){\rule[-0.175pt]{1.164pt}{0.350pt}}
\put(888,360){\rule[-0.175pt]{1.164pt}{0.350pt}}
\put(892,361){\rule[-0.175pt]{1.032pt}{0.350pt}}
\put(897,362){\rule[-0.175pt]{1.032pt}{0.350pt}}
\put(901,363){\rule[-0.175pt]{1.032pt}{0.350pt}}
\put(905,364){\rule[-0.175pt]{1.032pt}{0.350pt}}
\put(910,365){\rule[-0.175pt]{1.032pt}{0.350pt}}
\put(914,366){\rule[-0.175pt]{1.032pt}{0.350pt}}
\put(918,367){\rule[-0.175pt]{1.032pt}{0.350pt}}
\put(922,368){\rule[-0.175pt]{1.205pt}{0.350pt}}
\put(928,369){\rule[-0.175pt]{1.204pt}{0.350pt}}
\put(933,370){\rule[-0.175pt]{1.204pt}{0.350pt}}
\put(938,371){\rule[-0.175pt]{1.204pt}{0.350pt}}
\put(943,372){\rule[-0.175pt]{1.204pt}{0.350pt}}
\put(948,373){\rule[-0.175pt]{1.204pt}{0.350pt}}
\put(953,374){\rule[-0.175pt]{1.124pt}{0.350pt}}
\put(957,375){\rule[-0.175pt]{1.124pt}{0.350pt}}
\put(962,376){\rule[-0.175pt]{1.124pt}{0.350pt}}
\put(967,377){\rule[-0.175pt]{1.124pt}{0.350pt}}
\put(971,378){\rule[-0.175pt]{1.124pt}{0.350pt}}
\put(976,379){\rule[-0.175pt]{1.124pt}{0.350pt}}
\put(981,380){\rule[-0.175pt]{0.929pt}{0.350pt}}
\put(984,381){\rule[-0.175pt]{0.929pt}{0.350pt}}
\put(988,382){\rule[-0.175pt]{0.929pt}{0.350pt}}
\put(992,383){\rule[-0.175pt]{0.929pt}{0.350pt}}
\put(996,384){\rule[-0.175pt]{0.929pt}{0.350pt}}
\put(1000,385){\rule[-0.175pt]{0.929pt}{0.350pt}}
\put(1004,386){\rule[-0.175pt]{0.929pt}{0.350pt}}
\put(1007,387){\rule[-0.175pt]{0.923pt}{0.350pt}}
\put(1011,388){\rule[-0.175pt]{0.923pt}{0.350pt}}
\put(1015,389){\rule[-0.175pt]{0.923pt}{0.350pt}}
\put(1019,390){\rule[-0.175pt]{0.923pt}{0.350pt}}
\put(1023,391){\rule[-0.175pt]{0.923pt}{0.350pt}}
\put(1027,392){\rule[-0.175pt]{0.923pt}{0.350pt}}
\put(1031,393){\rule[-0.175pt]{0.843pt}{0.350pt}}
\put(1034,394){\rule[-0.175pt]{0.843pt}{0.350pt}}
\put(1038,395){\rule[-0.175pt]{0.843pt}{0.350pt}}
\put(1041,396){\rule[-0.175pt]{0.843pt}{0.350pt}}
\put(1045,397){\rule[-0.175pt]{0.843pt}{0.350pt}}
\put(1048,398){\rule[-0.175pt]{0.843pt}{0.350pt}}
\put(1052,399){\rule[-0.175pt]{0.585pt}{0.350pt}}
\put(1054,400){\rule[-0.175pt]{0.585pt}{0.350pt}}
\put(1056,401){\rule[-0.175pt]{0.585pt}{0.350pt}}
\put(1059,402){\rule[-0.175pt]{0.585pt}{0.350pt}}
\put(1061,403){\rule[-0.175pt]{0.585pt}{0.350pt}}
\put(1064,404){\rule[-0.175pt]{0.585pt}{0.350pt}}
\put(1066,405){\rule[-0.175pt]{0.585pt}{0.350pt}}
\put(1069,406){\rule[-0.175pt]{0.522pt}{0.350pt}}
\put(1071,407){\rule[-0.175pt]{0.522pt}{0.350pt}}
\put(1073,408){\rule[-0.175pt]{0.522pt}{0.350pt}}
\put(1075,409){\rule[-0.175pt]{0.522pt}{0.350pt}}
\put(1077,410){\rule[-0.175pt]{0.522pt}{0.350pt}}
\put(1079,411){\rule[-0.175pt]{0.522pt}{0.350pt}}
\put(1081,412){\rule[-0.175pt]{0.402pt}{0.350pt}}
\put(1083,413){\rule[-0.175pt]{0.401pt}{0.350pt}}
\put(1085,414){\rule[-0.175pt]{0.401pt}{0.350pt}}
\put(1086,415){\rule[-0.175pt]{0.401pt}{0.350pt}}
\put(1088,416){\rule[-0.175pt]{0.401pt}{0.350pt}}
\put(1090,417){\rule[-0.175pt]{0.401pt}{0.350pt}}
\put(1091,418){\usebox{\plotpoint}}
\put(1092,418){\usebox{\plotpoint}}
\put(1093,419){\usebox{\plotpoint}}
\put(1094,420){\usebox{\plotpoint}}
\put(1095,422){\usebox{\plotpoint}}
\put(1096,423){\usebox{\plotpoint}}
\put(1097,424){\rule[-0.175pt]{0.350pt}{0.723pt}}
\put(1098,428){\rule[-0.175pt]{0.350pt}{0.723pt}}
\put(1099,431){\rule[-0.175pt]{0.350pt}{1.686pt}}
\put(1098,438){\usebox{\plotpoint}}
\put(1097,439){\usebox{\plotpoint}}
\put(1096,440){\usebox{\plotpoint}}
\put(1095,441){\usebox{\plotpoint}}
\put(1094,442){\usebox{\plotpoint}}
\put(1091,444){\usebox{\plotpoint}}
\put(1090,445){\usebox{\plotpoint}}
\put(1089,446){\usebox{\plotpoint}}
\put(1088,447){\usebox{\plotpoint}}
\put(1087,448){\usebox{\plotpoint}}
\put(1086,449){\usebox{\plotpoint}}
\put(1084,450){\usebox{\plotpoint}}
\put(1083,451){\usebox{\plotpoint}}
\put(1081,452){\usebox{\plotpoint}}
\put(1080,453){\usebox{\plotpoint}}
\put(1078,454){\usebox{\plotpoint}}
\put(1077,455){\usebox{\plotpoint}}
\put(1076,456){\usebox{\plotpoint}}
\put(1074,457){\rule[-0.175pt]{0.442pt}{0.350pt}}
\put(1072,458){\rule[-0.175pt]{0.442pt}{0.350pt}}
\put(1070,459){\rule[-0.175pt]{0.442pt}{0.350pt}}
\put(1068,460){\rule[-0.175pt]{0.442pt}{0.350pt}}
\put(1066,461){\rule[-0.175pt]{0.442pt}{0.350pt}}
\put(1065,462){\rule[-0.175pt]{0.442pt}{0.350pt}}
\put(1063,463){\rule[-0.175pt]{0.482pt}{0.350pt}}
\put(1061,464){\rule[-0.175pt]{0.482pt}{0.350pt}}
\put(1059,465){\rule[-0.175pt]{0.482pt}{0.350pt}}
\put(1057,466){\rule[-0.175pt]{0.482pt}{0.350pt}}
\put(1055,467){\rule[-0.175pt]{0.482pt}{0.350pt}}
\put(1053,468){\rule[-0.175pt]{0.482pt}{0.350pt}}
\put(1051,469){\rule[-0.175pt]{0.482pt}{0.350pt}}
\put(1049,470){\rule[-0.175pt]{0.482pt}{0.350pt}}
\put(1047,471){\rule[-0.175pt]{0.482pt}{0.350pt}}
\put(1045,472){\rule[-0.175pt]{0.482pt}{0.350pt}}
\put(1043,473){\rule[-0.175pt]{0.482pt}{0.350pt}}
\put(1041,474){\rule[-0.175pt]{0.482pt}{0.350pt}}
\put(1039,475){\rule[-0.175pt]{0.482pt}{0.350pt}}
\put(1036,476){\rule[-0.175pt]{0.522pt}{0.350pt}}
\put(1034,477){\rule[-0.175pt]{0.522pt}{0.350pt}}
\put(1032,478){\rule[-0.175pt]{0.522pt}{0.350pt}}
\put(1030,479){\rule[-0.175pt]{0.522pt}{0.350pt}}
\put(1028,480){\rule[-0.175pt]{0.522pt}{0.350pt}}
\put(1026,481){\rule[-0.175pt]{0.522pt}{0.350pt}}
\put(1023,482){\rule[-0.175pt]{0.522pt}{0.350pt}}
\put(1021,483){\rule[-0.175pt]{0.522pt}{0.350pt}}
\put(1019,484){\rule[-0.175pt]{0.522pt}{0.350pt}}
\put(1017,485){\rule[-0.175pt]{0.522pt}{0.350pt}}
\put(1015,486){\rule[-0.175pt]{0.522pt}{0.350pt}}
\put(1013,487){\rule[-0.175pt]{0.522pt}{0.350pt}}
\put(1011,488){\rule[-0.175pt]{0.447pt}{0.350pt}}
\put(1009,489){\rule[-0.175pt]{0.447pt}{0.350pt}}
\put(1007,490){\rule[-0.175pt]{0.447pt}{0.350pt}}
\put(1005,491){\rule[-0.175pt]{0.447pt}{0.350pt}}
\put(1003,492){\rule[-0.175pt]{0.447pt}{0.350pt}}
\put(1001,493){\rule[-0.175pt]{0.447pt}{0.350pt}}
\put(1000,494){\rule[-0.175pt]{0.447pt}{0.350pt}}
\put(998,495){\rule[-0.175pt]{0.442pt}{0.350pt}}
\put(996,496){\rule[-0.175pt]{0.442pt}{0.350pt}}
\put(994,497){\rule[-0.175pt]{0.442pt}{0.350pt}}
\put(992,498){\rule[-0.175pt]{0.442pt}{0.350pt}}
\put(990,499){\rule[-0.175pt]{0.442pt}{0.350pt}}
\put(989,500){\rule[-0.175pt]{0.442pt}{0.350pt}}
\put(987,501){\rule[-0.175pt]{0.442pt}{0.350pt}}
\put(985,502){\rule[-0.175pt]{0.442pt}{0.350pt}}
\put(983,503){\rule[-0.175pt]{0.442pt}{0.350pt}}
\put(981,504){\rule[-0.175pt]{0.442pt}{0.350pt}}
\put(979,505){\rule[-0.175pt]{0.442pt}{0.350pt}}
\put(978,506){\rule[-0.175pt]{0.442pt}{0.350pt}}
\put(976,507){\usebox{\plotpoint}}
\put(975,508){\usebox{\plotpoint}}
\put(974,509){\usebox{\plotpoint}}
\put(972,510){\usebox{\plotpoint}}
\put(971,511){\usebox{\plotpoint}}
\put(970,512){\usebox{\plotpoint}}
\put(969,513){\usebox{\plotpoint}}
\put(967,514){\usebox{\plotpoint}}
\put(966,515){\usebox{\plotpoint}}
\put(965,516){\usebox{\plotpoint}}
\put(964,517){\usebox{\plotpoint}}
\put(963,518){\usebox{\plotpoint}}
\put(962,519){\usebox{\plotpoint}}
\put(962,520){\usebox{\plotpoint}}
\put(961,521){\usebox{\plotpoint}}
\put(960,522){\usebox{\plotpoint}}
\put(959,524){\usebox{\plotpoint}}
\put(958,525){\usebox{\plotpoint}}
\put(957,527){\rule[-0.175pt]{0.350pt}{0.361pt}}
\put(956,528){\rule[-0.175pt]{0.350pt}{0.361pt}}
\put(955,530){\rule[-0.175pt]{0.350pt}{0.361pt}}
\put(954,531){\rule[-0.175pt]{0.350pt}{0.361pt}}
\put(953,533){\rule[-0.175pt]{0.350pt}{0.723pt}}
\put(952,536){\rule[-0.175pt]{0.350pt}{0.723pt}}
\put(951,539){\rule[-0.175pt]{0.350pt}{1.686pt}}
\put(950,546){\rule[-0.175pt]{0.350pt}{1.445pt}}
\put(951,552){\rule[-0.175pt]{0.350pt}{0.482pt}}
\put(952,554){\rule[-0.175pt]{0.350pt}{0.482pt}}
\put(953,556){\rule[-0.175pt]{0.350pt}{0.482pt}}
\put(954,558){\rule[-0.175pt]{0.350pt}{0.562pt}}
\put(955,560){\rule[-0.175pt]{0.350pt}{0.562pt}}
\put(956,562){\rule[-0.175pt]{0.350pt}{0.562pt}}
\put(957,564){\usebox{\plotpoint}}
\put(958,566){\usebox{\plotpoint}}
\put(959,567){\usebox{\plotpoint}}
\put(960,568){\usebox{\plotpoint}}
\put(961,569){\usebox{\plotpoint}}
\put(962,571){\usebox{\plotpoint}}
\put(963,572){\usebox{\plotpoint}}
\put(964,573){\usebox{\plotpoint}}
\put(965,574){\usebox{\plotpoint}}
\put(966,575){\usebox{\plotpoint}}
\put(967,577){\usebox{\plotpoint}}
\put(968,578){\usebox{\plotpoint}}
\put(969,579){\usebox{\plotpoint}}
\put(970,580){\usebox{\plotpoint}}
\put(971,581){\usebox{\plotpoint}}
\put(972,582){\usebox{\plotpoint}}
\put(973,584){\usebox{\plotpoint}}
\put(974,585){\usebox{\plotpoint}}
\put(975,586){\usebox{\plotpoint}}
\put(976,587){\usebox{\plotpoint}}
\put(977,588){\usebox{\plotpoint}}
\put(978,589){\usebox{\plotpoint}}
\put(979,590){\usebox{\plotpoint}}
\put(980,591){\usebox{\plotpoint}}
\put(981,592){\usebox{\plotpoint}}
\put(982,593){\usebox{\plotpoint}}
\put(983,594){\usebox{\plotpoint}}
\put(984,595){\usebox{\plotpoint}}
\put(985,596){\usebox{\plotpoint}}
\put(986,597){\usebox{\plotpoint}}
\put(987,598){\usebox{\plotpoint}}
\put(988,600){\usebox{\plotpoint}}
\put(989,601){\usebox{\plotpoint}}
\put(990,603){\usebox{\plotpoint}}
\put(991,604){\usebox{\plotpoint}}
\put(992,605){\usebox{\plotpoint}}
\put(993,606){\usebox{\plotpoint}}
\put(994,607){\usebox{\plotpoint}}
\put(995,608){\usebox{\plotpoint}}
\put(996,609){\usebox{\plotpoint}}
\put(997,610){\usebox{\plotpoint}}
\put(998,611){\usebox{\plotpoint}}
\put(999,612){\usebox{\plotpoint}}
\put(1000,613){\usebox{\plotpoint}}
\put(1001,615){\usebox{\plotpoint}}
\put(1002,616){\usebox{\plotpoint}}
\put(1003,617){\usebox{\plotpoint}}
\put(1004,619){\usebox{\plotpoint}}
\put(1005,620){\usebox{\plotpoint}}
\put(1006,622){\rule[-0.175pt]{0.350pt}{0.361pt}}
\put(1007,623){\rule[-0.175pt]{0.350pt}{0.361pt}}
\put(1008,625){\rule[-0.175pt]{0.350pt}{0.361pt}}
\put(1009,626){\rule[-0.175pt]{0.350pt}{0.361pt}}
\put(1010,628){\rule[-0.175pt]{0.350pt}{0.562pt}}
\put(1011,630){\rule[-0.175pt]{0.350pt}{0.562pt}}
\put(1012,632){\rule[-0.175pt]{0.350pt}{0.562pt}}
\put(1013,634){\rule[-0.175pt]{0.350pt}{0.723pt}}
\put(1014,638){\rule[-0.175pt]{0.350pt}{0.723pt}}
\put(1015,641){\rule[-0.175pt]{0.350pt}{1.445pt}}
\put(1016,647){\rule[-0.175pt]{0.350pt}{1.686pt}}
\put(1017,654){\rule[-0.175pt]{0.350pt}{3.734pt}}
\put(1016,669){\rule[-0.175pt]{0.350pt}{0.843pt}}
\put(1015,673){\rule[-0.175pt]{0.350pt}{1.445pt}}
\put(1014,679){\rule[-0.175pt]{0.350pt}{0.723pt}}
\put(1013,682){\rule[-0.175pt]{0.350pt}{0.723pt}}
\put(1012,685){\rule[-0.175pt]{0.350pt}{0.562pt}}
\put(1011,687){\rule[-0.175pt]{0.350pt}{0.562pt}}
\put(1010,689){\rule[-0.175pt]{0.350pt}{0.562pt}}
\put(1009,691){\rule[-0.175pt]{0.350pt}{0.723pt}}
\put(1008,695){\rule[-0.175pt]{0.350pt}{0.723pt}}
\put(1007,698){\rule[-0.175pt]{0.350pt}{0.482pt}}
\put(1006,700){\rule[-0.175pt]{0.350pt}{0.482pt}}
\put(1005,702){\rule[-0.175pt]{0.350pt}{0.482pt}}
\put(1004,704){\rule[-0.175pt]{0.350pt}{0.562pt}}
\put(1003,706){\rule[-0.175pt]{0.350pt}{0.562pt}}
\put(1002,708){\rule[-0.175pt]{0.350pt}{0.562pt}}
\put(1001,710){\rule[-0.175pt]{0.350pt}{0.723pt}}
\put(1000,714){\rule[-0.175pt]{0.350pt}{0.723pt}}
\put(999,717){\rule[-0.175pt]{0.350pt}{0.482pt}}
\put(998,719){\rule[-0.175pt]{0.350pt}{0.482pt}}
\put(997,721){\rule[-0.175pt]{0.350pt}{0.482pt}}
\put(996,723){\rule[-0.175pt]{0.350pt}{0.843pt}}
\put(995,726){\rule[-0.175pt]{0.350pt}{0.843pt}}
\put(994,730){\rule[-0.175pt]{0.350pt}{0.723pt}}
\put(993,733){\rule[-0.175pt]{0.350pt}{0.723pt}}
\put(992,736){\rule[-0.175pt]{0.350pt}{0.843pt}}
\put(991,739){\rule[-0.175pt]{0.350pt}{0.843pt}}
\put(990,743){\rule[-0.175pt]{0.350pt}{1.445pt}}
\put(989,749){\rule[-0.175pt]{0.350pt}{1.445pt}}
\put(988,755){\rule[-0.175pt]{0.350pt}{1.686pt}}
\put(987,762){\rule[-0.175pt]{0.350pt}{4.577pt}}
\put(988,781){\rule[-0.175pt]{0.350pt}{1.445pt}}
\end{picture}

\caption{Gelfand equation on the ball, $3\leq n \leq 9$.
\label{gelfand.fig2}}
\end{figure}
\end{verbatim}\end{quote}
One advantage to using the native \LaTeX{} {\tt picture} environment
is that the fonts will be assured to agree and the pictures can be viewed
in the {\tt .dvi} viewer.

\subsection{PostScript}
Many drawing applications now allow the export of a graphic to the
{\em Encapsulated PostScript} format.  These files have a suffix of
{\tt .EPS} or {\tt .EPSF} and are similar to a regular PostScript
file except that they contain a {\em bounding box} which describes
the dimensions of the figure.

In order to include PostScript figures, the {\tt epsfig} (or {\tt psfig}
depending on the system you are using) style file must be included in either
the {\tt\verb|\documentstyle|} command or the preamble using the {\tt input} command.

Figure~\ref{vwcontr} is a plot from Matlab.
\begin{figure}[htbp]
\centerline{
\psfig{figure=vwcontr.eps,width=5in,angle=0}
           }
\caption{$\sigma$ as a Function of Voltage and Speed, $\alpha = 20$}
\label{vwcontr}
\end{figure}
The commands to include this figure are
\begin{quote}\tt\singlespace\begin{verbatim}
\begin{figure}[htbp]
\centerline{
\psfig{figure=vwcontr.ps,width=5in,angle=0}
           }
\caption{$\sigma$ as a Function of Voltage and Speed, $\alpha = 20$}
\label{vwcontr}
\end{figure}
\end{verbatim}\end{quote}

Observe that the {\tt \verb|\psfig|} command allows the scaling of the figure
by setting either the {\tt width} or {\tt height} of the figure.  If only one
dimension is specified, the other is computed to keep the same aspect ratio.
The figure can also be rotated by setting {\tt angle} to the desired value in
degrees.
           % Chapter 3 Edited from UW Math Dept's Sample Thesis
% bibs.tex
%
% This chapter briefly talks about BibTex and is mostly
% copied from a similar chapter from "How to TeX a Thesis:
% The Purdue Thesis Styles" by James Darrell McCauley and
% Scott Hucker
%

\newcommand{\BibTeX}{{\sc Bib}\TeX}

\chapter{Citations and Bibliographies}
This chapter is an edited form of the same chapter from {\em How to 
\TeX{} a Thesis: The Purdue Thesis Styles} by James Darrell McCauley and
Scott Hucker.

The task of compiling and formatting the sources cited in papers can
be quite tedious, especially for large documents like theses.  A program
separate from \LaTeX{}, called ``\BibTeX{},''can be used to automate this task~\cite{lamport}.

\section{The Citation Command}
When referring to the work of someone else, the {\tt \verb|\cite|} command is used.
This generates the citation in the text for you.  In the above paragraph, the command
{\tt \verb|\cite{lamport}|} was used after the word ``task.''  The formatting of your
citation is handled by either the document style or a style option.  The default citation
style uses the number system (a number in square brackets).  Other citation styles
may use the author-date system, (Lamport, 1986) or the superscript$^3$ system.

\section{Bibliography Styles}
The way that a reference is formatted in your bibliography depends on the bibliography
style, which is specified near the beginning of your document with the\break
{\tt \verb|\bibliographstyle{file}|} command.  The file {\tt file.bst} is the name of the 
bibliography style file.  Standard \BibTeX{} bibliography style files include {\tt plain},
{\tt unsrt}, {\tt alpha}, and {\tt abbrev}.  The bibliography style governs whether or not
references are sorted, whether first names or initials are used for authors, whether or 
not last names are listed first, the location of the year in the references (after the
author or at the end of the reference), {\em etc.}.  You may be required by your
department or major professor to follow as style for a particular journal.  If so, then you
will need to find a \BibTeX{} style file to suit your needs.  Most major journals have
style files.  If you cannot locate an appropriate \BibTeX{} style file, then choose the
one which is closest and then edit the {\tt .bbl} file by hand.  See Section~\ref{BBL}
for a brief discussion on the {\tt .bbl} file.  Some common, but non-standard \BibTeX{}
styles include
\begin{tabbing}
{\tt jacs-new.bstxxxx}\= {\em Journal of the American Chemical Society}\kill
{\tt acm.bst}\>The Association for Computing Machinery\\
{\tt ieeetr.bst}\> The {\em IEEE Transactions} style\\
{\tt jacs-new.bst}\> {\em Journal of the American Chemical Society}
\end{tabbing}

\section{The Database}
The  {\tt \verb|\bibliography{file}|} command is placed in your input file at the location
where the ``List of References'' section\footnote{or ``Bibliography'' 
if {\tt \char92 altbibtitle } has been specified in the preamble.} would be.  It specifies the name (or names) of
your bibliographic data base, {\tt file.bib}.  An example entry in a \BibTeX{}
database is:
\begin{quote}\singlespace\tt\begin{verbatim}
@book{ lamport86 ,
     author =    "Leslie Lamport" ,
     title =     "\LaTeX: A Document Preparation System" ,
     publisher = "Addison--Wesley Pub.\ Co." ,
     year =      "1986" ,
     address =   "Reading, MA" 
}
\end{verbatim}\end{quote}

The citation key is the first field in this entry--- citing this book in a \LaTeX{}
file would look like
\begin{quote}\singlespace\tt\begin{verbatim}
According to Lamport~\cite{lamport86} ...
\end{verbatim}\end{quote}
The tilde ({\tt \verb|~|}) is used to tie the word ``Lamport'' to the citation
generated.  The space between these words is then unbreakable---the word ``Lamport''
and the citation \cite{lamport} will not be split across two lines if they happen to occur
near the end of a line.

A listing of all entry types with their required and optional fields is given in 
Appendix~\ref{bibrefs}. There are several tools which exist to help in editing a \BibTeX{}
file, however, their use is beyond the scope of this manual and can be found by searching
the net.  You can simply use a plain text editor like {\tt vi} or {\tt WordPad} to edit
and create the database files.

There are several rules which you must follow when creating your database.  Authors are
always listed by their full names, first name first, and multiple authors are separated
by {\tt and}.  For example
\begin{quote}\singlespace\tt\begin{verbatim}
author = "John Jay Park and Frederick Gene Watson and
          Michelle Catherine Smith",
\end{verbatim}\end{quote}
If you were using {\tt abbrv} as your {\tt bibliographystyle}, a reference for these
authors may look like:
\begin{quote}
J.J. Park, F.G. Watson, and M.C. Smith \ldots
\end{quote}

Some styles only capitalize the first word of the title.  If you use any acronyms or
other words that should always be capitalized in titles, then they should be 
enclosed in {\tt \{\}}'s ({\em e.g.}, {\tt \{Fortran\}}, {\tt \{N\}ewton}).
This protects the case of these characters.

There are several other rules for \BibTeX{} listed in~\cite{lamport} which should be
referred to because they are not discussed here.

\section{Putting It All Together}
\label{BBL}
To aid the reader in understanding how all of this works together, the following 
excerpt was taken from Lamport~\cite{lamport}:
\begin{quotation}\singlespace
When you ran \LaTeX{} with the input file {\tt sample.tex}, you may have
noticed that \LaTeX{} created a file named {\tt sample.aux}.  This file,
called an {\em auxiliary} file, contains cross-referencing information.  Since
{\tt sample.tex} contains no cross-referencing commands, the auxiliary file it
produces has no information.  However, suppose that \LaTeX{} is run with an
input file named {\tt myfile.tex} that has citations and bibliography-making
[or referencing] commands.  The auxiliary file {\tt myfile.aux} that it produces
will contain all of the citation keys and the arguments of the {\tt \verb|\bibliography|}
and {\tt\verb|\bibliographystyle|} commands.  When \BibTeX{} is run, it reads
this information from the auxiliary file and produces a file named {\tt myfile.bbl}
containing \LaTeX{} commands to produce the source list \ldots The next time
\LaTeX{} is run on {\tt myfile.tex}, the {\tt \verb|\bibliography|} command reads
the {\tt bbl} file ({\tt myfile.bbl}), which generates the source list.
\end{quotation}

Thus, the command sequence for a source file called {\tt main.tex} which is going to
use \BibTeX{} would be:
\begin{quote}\singlespace\tt\begin{verbatim}
latex main.tex
bibtex main
latex main
latex main
\end{verbatim}\end{quote}
The first \LaTeX{} is to collect all of the citations for \BibTeX{}.  Then
\BibTeX{} is run to generate the bibliography.  \LaTeX{} is run again to
incorporate the bibliography into the document and the run the last time to
update any references (like pages in the Table of Contents) which changed when
the bibliography was included.
           % Chapter 4 From PU Thesis styles, by J.D. McCauley
% usage.tex
%
% This file explains how to use the withesis style
%   it is heavily modelled after a similar chapter by McCauley
%   for the Purdue Thesis style
%
% Eric Benedict, May 2000
%
% It is provided without warranty on an AS IS basis.


\chapter{Using the {\tt withesis} Style}

You can get a copy of the \LaTeX{} style for creating a University
of Wisconsin--Madison thesis or dissertation from:

{\tt http://www.cae.wisc.edu/\verb+~+benedict/LaTeX.html}

After somehow unpacking it, you will have the style files ({\tt withesis.sty}
{\tt withe10.sty}, and {\tt withe12.sty}) as well the files used to create
this document.  The files used for this document can be copied and used as a
template for your own thesis or dissertation.

The final printed form of this document is useful, but the
combination of the source code and final copy form a much more valuable
reference.  Keeping a working copy of the this document can be helpful
when you are later working on your thesis or disseration and want to know
how to do something.  If you find a similar example in this document,
then you can simply look at the corresponding source code and add it to
your document.    Because many parts of this document were written by
different people, the styles and techniques are also different and provide
different ways of achieving the same or similar results.

Because of the typical size of theses, it makes sense to break the document
up into several smaller files.  Usually this is done at the chapter level.
These files can then be {\tt \verb|\include|}d in a {\em root} file.  It is
the {\em root} file that you will run \LaTeX{} on.  For this manual, the
root file is called {\tt main.tex}.

\section{The Root File and the Preamble}
The {\tt \verb|\documentclass|} command is used to tell \LaTeX{} that you will
be using the {\tt withesis} document class and it is the first command in your
root file.  Class options such as {\tt 10pt}, {\tt 12pt}, {\tt msthesis} or
{\tt margincheck} are specified here:

{\tt \verb|\documentclass[12pt,msthesis]{withesis}|}

The class option {\tt msthesis} sets the margins to be appropriate for depositing
with the UW library, namely a 1.25 inch left margin with the remaining margins 1 inch.
The defaults for the title page are also defined for a thesis and for a Master of
Science degree.

The class option {\tt margincheck} will place a small black square at the end of
each line which exceeds the margins.\footnote{In reality, the square is
placed at the end of lines which exceed their {\tt \char92hbox}.  This usually
(but not always) indicates a  margin violation on the right margin.  Left
margin violations aren't indicated and if the margin violation is large enough,
there isn't room for the black box to be visiable.}  This is visible both in the {\tt .dvi} file
as well as in the {\tt .ps} file.

The area immediately following this command is called the {\em preamble} and is
used for things like including different style packages,
defining new macros and declaring the page style.

The style packages can be used to easily change the thesis font.  For example,
this document is set in Times Roman instead of the \LaTeX default of Computer
Modern.  This change was performed by including the {\tt times} package:

{\tt\verb|\usepackage{times}|}\footnote{In this document, the typewriter font
{\tt $\backslash$tt} was redefined to use the Computer Modern font with the command
{\tt $\backslash$renewcommand\{$\backslash$ttdefault\}\{cmtt\}}.  
For more information, see~\cite{goossens}.}

Remember that if you change the fonts from the default Computer Modern to
PostScript ({\em e.g.} Times Roman) then in order to correctly see the
document, you will need to convert the {\tt *.dvi} output into a {\tt *.ps}
file and view the document with a PostScript viewer. This is required since 
most {\tt *.dvi} previewer programs cannot 
display PostScript fonts.  Usually, the previewer will substitute
default fonts so the document may be viewed; however, since the alternate
fonts may not be the same size, the formatting of the document may appear
to be incorrect.

The style package for including Postscript figures, {\tt epsfig}, is included with

{\tt\verb|\usepackage{epsfig}|}

If multiple style packages are required, then they can be combined into one statement
as follows:

{\tt\verb|\usepackage{epsfig,times}|}

Many different style packages are available.  For more information, see~\cite{goossens}.

The page styles are defined using a similar method.
A special style is defined for the {\tt withesis} style:

{\tt\verb|\pagestyle{thesisdraft}|}

This style causes the footer text to become:

{\verb| DRAFT: Do Not Distribute        <time><Date>        <input file name>|}

This appears at the bottom of every page.

In addition to the page style command, the {\tt withesis} has defined several useful
commands which are specified in the preamble.  They include {\tt \verb| \draftmargin|},
{\tt \verb|\draftscreen|}, {\tt \verb|\noappendixtables|}, and
{\tt \verb|\noappendixfigures|}.

The command  {\tt \verb|\draftmargin|} draws a PostScript box with the dimensions of
the margins.  This makes it easy to check that the margins are correct and to see if
any of the text or figures are outside of the required margins.  This box is only visible
in the {\tt .ps} file since it is a PostScript special.


The command  {\tt \verb|\draftscreen|} draws a PostScript screen with the word {\em DRAFT}
in light grey and diagonally across the page.  This screen is only visible
in the {\tt .ps} file since it is a PostScript special.

The commands {\tt \verb|\noappendixtables|} and/or {\tt \verb|\noappendixfigures|} should
be used if the appendix does not have either tables or figures respectively.  These commands
inhibit the Appendix Table or Appendix Figure titles in the List of Tables or List of
Figures.\label{usage:noapp}


If you have specified the {\tt psfig} or {\tt epsfig} document style package, then a useful
command is {\tt \verb|\psdraft|}.  This command will show the bounding box that the figure
would occupy (instead of actually including the figure).  This speeds up the draft copy
printing, reduces toner usage and the drawn box is visible in the {\tt .dvi} file.

The next usual command is {\tt \verb|\begin{document}|}.  The following example is part
of the root file used for this manual.

\begin{quote} \singlespace\footnotesize\tt
\begin{verbatim}
\bibliographystyle{plain}
% prelude.tex
%   - titlepage
%   - dedication
%   - acknowledgments
%   - table of contents, list of tables and list of figures
%   - nomenclature
%   - abstract
%============================================================================


\clearpage\pagenumbering{roman}  % This makes the page numbers Roman (i, ii, etc)


% TITLE PAGE
%   - define \title{} \author{} \date{}
\title{Statistical framework in the discovery of Fluoroscanning \\ (the next generation precision genomics)}
\author{Subhrangshu Nandi}
\date{February 18, 2016}
%   - The default degree is ``Doctor of Philosophy''
%     (unless the document style msthesis is specified
%      and then the default degree is ``Master of Science'')
%     Degree can be changed using the command \degree{}
\degree{Doctor of Philosophy}
%   - The default is dissertation, unless the document style
%     msthesis was specified in which case it becomes thesis.
%     If msthesis is specified for the MS margins, you can
%     still have a dissertation if you specify \disseration
%\disseration
%   - for a masters project report, specify \project
%\project
%   - for a preliminary report, specify \prelim
\prelim
%   - for a masters thesis, specify \thesis
%\thesis
%   - The default department is ``Electrical Engineering''
%     The department can be changed using the command \department{}
\department{Statistics}
%   - once the above are defined, use \maketitle to generate the titlepage
\maketitle

% COPYRIGHT PAGE
%   - To include a copyright page use \copyrightpage
\copyrightpage

% DEDICATION
\begin{dedication}
TBD
\end{dedication}

% ACKNOWLEDGMENTS
\begin{acknowledgments}
I thank the many people who have done lots of nice things for me.
\end{acknowledgments}

% CONTENTS, TABLES, FIGURES
\tableofcontents
\listoftables
\listoffigures

% NOMENCLATURE
%% \begin{nomenclature}
%% \begin{description}
%% \item{\makebox[0.75in][l]{\TeX}}
%%        \parbox[t]{5in}{a typesetting system by Donald Knuth~\cite{knuth}.  It
%%        also refers to the ``plain'' format.  The proper pronounciation
%%        rhymes with ``heck'' and ``peck'' and does not sound like
%%        ``hex'' or ``Rex.''\\}

%% \item{\makebox[0.75in][l]{\LaTeX}}  
%%         \parbox[t]{5in}{a set of \TeX{} macros originally written by Leslie 
%%         Lamport~\cite{lamport}.  The proper pronunciation is 
%%         {\tt l\={a}$\cdot$tek'} and not {\tt l\={a}'$\cdot$teks} (see above).\\}

%% \item{\makebox[0.75in][l]{{\sc Bib}\TeX}} 
%%          \parbox[t]{5in}{a bibliography generation program by Oren 
%%                 Patashnik~\cite{lamport}
%%                 that can be used with either plain \TeX{} or \LaTeX{}.\\}

%% \item{\makebox[0.75in][l]{$C_1$}} Constant 1

%% \item{\makebox[0.75in][l]{$V$}}    Voltage 

%% \item{\makebox[0.75in][l]{\$}}     US Dollars
%% \end{description}
%% \end{nomenclature}


\advisorname{Michael A. Newton}
\advisortitle{Professor}
% ABSTRACT
\begin{umiabstract}
  \input{Writeup_Newton}
\end{umiabstract}

\begin{abstract}
  \input{abstract}
\end{abstract}


\clearpage\pagenumbering{arabic} % This makes the page numbers Arabic (1, 2, etc)
        % Title page, abstract, table of contents, etc
% Pre-lim
% by Eric Benedict


\chapter{Introducing the {\tt withesis} \LaTeX{} Style Guide}
This manual is was written to test the {\tt withesis} style
file and to provide documentation for this style file.  

\section{History}
The
idea for this came from a similar manual written by James Darrell
McCauley and Scott Hucker in 1993 for the Purdue University thesis
style file.  Content ideas were liberally borrowed from this document.
The {\tt withesis} style file is based on the Purdue thesis file
written by Dave Kraynie and edited by Darrell McCauley.  This base was
edited to meet the format requirements of the University of 
Wisconsin--Madison and several additional new commands were created.
In addition, environments from the UW Mathematics Department were also
incorporated.

\section{Producing Your Thesis or Dissertation}
The {\tt withesis} style file will take care of most of the formatting
requirements for submitting your thesis or dissertation at the University
of Wisconsin-Madison.  There are some requirements on the printing of your
document.  From the Graduate School's {\em UW-Madison Guide To Preparing 
Your Doctoral Dissertation},
\begin{quote}\singlespace
Print your dissertation on a laser printer. (Some high quality dot-matrix
printers may be acceptable.) The printer must produce output that
meets all format and legibility requirements. A professional copy shop
can produce an acceptable copy to be submitted to the Graduate School.
Some copiers enlarge the original between one and two percent. To avoid
problems with margins, produce the original copy with margins larger than
the required minimum. Look carefully at the copy before paying for the
services and ask for pages to be recopied if necessary. Common flaws are:
smudges, copy lines, specks, missing pages, margin shifts, slanting of
the printed image on the page, and poor paper quality.
\end{quote}

\subsection{Required Paper}
The paper which is used for PhD Dissertations should be:
\begin{itemize}
\item 8-1/2 x 11 inches
\item High-quality, white
\item 20 pound weight, bond
\end{itemize}
 
While for Masters Theses, the paper should be:

\begin{itemize}
\item 8-1/2 x 11 inches
\item White
\item Acid-free or pH neutral
\item 20 pound weight
\item 25\% cotton bond minimum
\end{itemize}

Paper that meets these requirements can be purchased at book and stationery
stores.

\subsection{Copyright Page}
\label{copyright}
If you choose to retain and register copyright of the dissertation, prepare
a copyright page using the {\tt withesis} {\tt \verb|\copyrightpage|} command. 
Center the text in the bottom third of the page within the dissertation
margins. This page is not numbered. There is an additional fee for copyrighting
your dissertation which is payable at the bursars office along with the
microfilming and binding fee.

\subsection{Prechecks}
The Graduate School has reserved 9:00-9:30 each morning to answer specific formatting questions
(for example: use of tables, graphs and charts). You may bring in 8-10
pages to be reviewed. No appointment is necessary.

\subsection{Final Checks}
\sloppypar
For information about the final Graduate School review and about depositing
your dissertation in the library, see {\em The Three D's: Deadlines, Defending, 
Depositing Your Doctoral Dissertation} or look
at the web site 
\begin{quote}
{\tt http://www.wisc.edu/grad/gs/degrees/ddd.html}
\end{quote}

\section{Disclaimer}
This software and documentation is provided ``as is'' without any
express or implied warranty.
While care has been taken by the authors of this style file such that the
final product will probably meet the University of Wisconsin's formatting 
requirements this is not guaranteed. 
          % Chapter 1
% Essential LaTeX - Jon Warbrick 02/88
%   - Edited May, July 2000 -E. Benedict


% Copyright (C) Jon Warbrick and Plymouth Polytechnic 1989
% Permission is granted to reproduce the document in any way providing
% that it is distributed for free, except for any reasonable charges for
% printing, distribution, staff time, etc.  Direct commercial
% exploitation is not permitted.  Extracts may be made from this
% document providing an acknowledgment of the original source is
% maintained.

% NOTICE: This document has been edited for use in the UW-Madison
% Example Thesis file.


% counters used for the sample file example
\newcounter{savesection}
\newcounter{savesubsection}


% commands to do 'LaTeX Manual-like' examples

\newlength{\egwidth}\setlength{\egwidth}{0.42\textwidth}

\newenvironment{eg}{\begin{list}{}{\setlength{\leftmargin}%
{0.05\textwidth}\setlength{\rightmargin}{\leftmargin}}%
\item[]\footnotesize}{\end{list}}

\newenvironment{egbox}{\begin{minipage}[t]{\egwidth}}{\end{minipage}}

\newcommand{\egstart}{\begin{eg}\begin{egbox}}
\newcommand{\egmid}{\end{egbox}\hfill\begin{egbox}}
\newcommand{\egend}{\end{egbox}\end{eg}}

% one or two other commands
\newcommand{\fn}[1]{\hbox{\tt #1}}
\newcommand{\llo}[1]{(see line #1)}
\newcommand{\lls}[1]{(see lines #1)}
\newcommand{\bs}{$\backslash$}


\chapter{Essential \LaTeX{}}

This chapter introduces some key ideas behind \LaTeX{} and give you the ``essential''
items of information.  This chapter is an edited form of the paper
``Essential \LaTeX{}'' by Jon Warbrick, Plymouth Polytechnic.

\section{Introduction}
This document is an attempt to give you all the essential
information that you will need in order to use the \LaTeX{} Document
Preparation System.  Only very basic features are covered, and a
vast amount of detail has been omitted.  In a document of this size
it is not possible to include everything that you might need to know,
and if you intend to make extensive use of the program you should
refer to a more complete reference.  Attempting to produce complex
documents using only the information found below will require
much more work than it should, and will probably produce a less
than satisfactory result.

The main reference for \LaTeX{} is {\em The \LaTeX{} User's guide and
Reference Manual\/} by Leslie Lamport.  This contains most of the
information that you will ever need to know about the program, and
you will need access to a copy if you are to use \LaTeX{} seriously.
You should also consider getting a copy of {\em The \LaTeX{}
Companion\/} 

\section{How does \LaTeX{} work?}

In order to use \LaTeX{} you generate a file containing
both the text that you wish to print and instructions to tell \LaTeX{}
how you want it to appear.  You will normally create
this file using your system's text editor.  You can give the file any name you
like, but it should end ``\fn{.TEX}'' to identify the file's contents.
You then get \LaTeX{} to process the file, and it creates a
new file of typesetting commands; this has the same name as your file but
the ``\fn{.TEX}'' ending is replaced by ``\fn{.DVI}''.  This stands for
`{\it D\/}e{\it v\/}ice {\it I\/}ndependent' and, as the name implies, this file
can be used to create output on a range of printing devices.
Your {\em local guide\/} will go into more detail.

Rather than encourage you to dictate exactly how your document
should be laid out, \LaTeX{} instructions allow you describe its
{\em logical structure\/}.  For example, you can think of a quotation
embedded within your text as an element of this logical structure: you would
normally expect a quotation to be displayed in a recognisable style to set it
off from the rest of the text.
A human typesetter would recognise the quotation and handle
it accordingly, but since \LaTeX{} is only a computer program it requires
your help.  There are therefore \LaTeX{} commands that allow you to
identify quotations and as a result allow \LaTeX{} to typeset them correctly.

Fundamental to \LaTeX{} is the idea of a {\em document style\/} that
determines exactly how a document will be formatted.  \LaTeX{} provides
standard document styles that describe how standard logical structures
(such as quotations) should be formatted.  You may have to supplement
these styles by specifying the formatting of logical structures
peculiar to your document, such as mathematical formulae.  You can
also modify the standard document styles or even create an entirely
new one, though you should know the basic principles of typographical
design before creating a radically new style.

There are a number of good reasons for concentrating on the logical
structure rather than on the appearance of a document.  It prevents
you from making elementary typographical errors in the mistaken
idea that they improve the aesthetics of a document---you should
remember that the primary function of document design is to make
documents easier to read, not prettier.  It is more flexible, since
you only need to alter the definition of the quotation style
to change the appearance of all the quotations in a document.  Most
important of all, logical design encourages better writing.
A visual system makes it easier to create visual effects rather than
a coherent structure; logical design encourages you to concentrate on
your writing and makes it harder to use formatting as a substitute
for good writing.

\section{A Sample \LaTeX{} file}


\begin{figure} %---------------------------------------------------------------
{\singlespace\tt\footnotesize\begin{verbatim}
 1: % SMALL.TEX -- Released 5 July 1985
 2: % USE THIS FILE AS A MODEL FOR MAKING YOUR OWN LaTeX INPUT FILE.
 3: % EVERYTHING TO THE RIGHT OF A  %  IS A REMARK TO YOU AND IS IGNORED
 4: % BY LaTeX.
 5: %
 6: % WARNING!  DO NOT TYPE ANY OF THE FOLLOWING 10 CHARACTERS EXCEPT AS
 7: % DIRECTED:        &   $   #   %   _   {   }   ^   ~   \
 8:
 9: \documentclass[11pt,a4]{article}  % YOUR INPUT FILE MUST CONTAIN THESE
10: \begin{document}                  % TWO LINES PLUS THE \end COMMAND AT
11:                                   % THE END
12:
13: \section{Simple Text}          % THIS COMMAND MAKES A SECTION TITLE.
14:
15: Words are separated by one or    more      spaces.  Paragraphs are
16:     separated by one or more blank lines.  The output is not affected
17: by adding extra spaces or extra blank lines to the input file.
18:
19:
20: Double quotes are typed like this: ``quoted text''.
21: Single quotes are typed like this: `single-quoted text'.
22:
23: Long dashes are typed as three dash characters---like this.
24:
25: Italic text is typed like this: {\em this is italic text}.
26: Bold   text is typed like this: {\bf this is  bold  text}.
27:
28: \subsection{A Warning or Two}        % THIS MAKES A SUBSECTION TITLE.
29:
30: If you get too much space after a mid-sentence period---abbreviations
31: like etc.\ are the common culprits)---then type a backslash followed by
32: a space after the period, as in this sentence.
33:
34: Remember, don't type the 10 special characters (such as dollar sign and
35: backslash) except as directed!  The following seven are printed by
36: typing a backslash in front of them:  \$  \&  \#  \%  \_  \{  and  \}.
37: The manual tells how to make other symbols.
38:
39: \end{document}                    % THE INPUT FILE ENDS LIKE THIS
\end{verbatim}  }

\caption{A Sample \LaTeX{} File}\label{fig:sample}

\end{figure} %-----------------------------------------------------------------



Have a look at the example \LaTeX{} file in Figure~\ref{fig:sample}.  It
is a slightly modified copy of the standard \LaTeX{} example file
\fn{SMALL.TEX}.  The line numbers down the left-hand side
are not part of the file, but have been added to make it easier to
identify various portions.

Try entering this file (without the line numbers), save the text as \fn{small.tex},
next run \LaTeX{} on it, and then view the output:

{\tt \singlespace\begin{verbatim}
% latex small
% xdvi small               # displays the output on the screen
% dvips -o small.ps small  # to create a PostScript file, small.ps
% lp -d<printer> small.ps  # to print
\end{verbatim}}

\subsection{Running Text}

Most documents consist almost entirely of running text---words formed
into sentences, which are in turn formed into paragraphs---and the example file
is no exception. Describing running text poses no problems, you just type
it in naturally. In the output that it produces, \LaTeX{} will fill
lines and adjust the
spacing between words to give tidy left and right margins.
The spacing and distribution of the words in your input
file will have no effect at all on the eventual output.
Any number of spaces in your input file
are treated as a single space by \LaTeX{}, it also regards the
end of each line as a space between words \lls{15--17}.
A new paragraph is
indicated by a blank line in your input file, so don't leave
any blank lines unless you really wish to start a paragraph.

\LaTeX{} reserves a number of the less common keyboard characters for its
own use. The ten characters
\begin{quote}\begin{verbatim}
#  $  %  &  ~  _  ^  \  {  }
\end{verbatim}\end{quote}
should not appear as part of your text, because if they do
\LaTeX{} will get confused.

\subsection{\LaTeX{} Commands}

There are a number of words in the file that start `\verb|\|' \lls{9,
10 and 13}.  These are \LaTeX{} {\em commands\/} and they describe
the structure of your document. There are a number of things that you
should realize about these commands:
\begin{itemize}

\item All \LaTeX{} commands consist of a `\verb|\|' followed by one or more
characters.

\item \LaTeX{} commands should be typed using the correct mixture of upper- and
lower-case letters.  \verb|\BEGIN| is {\em not\/} the same as \verb|\begin|.

\item Some commands are placed within your text.  These are used to
switch things, like different typestyles, on and off. The \verb|\em|
command is used like this to emphasize text, normally by changing to
an {\it italic\/} typestyle \llo{25}.  The command and the text are
always enclosed between `\verb|{|' and `\verb|}|'---the `\verb|{\em|'
turns the effect on and and the `\verb|}|' turns it off.

\item There are other commands that look like
\begin{quote}\begin{verbatim}
\command{text}
\end{verbatim}\end{quote}
In this case the text is called the ``argument'' of the command.  The
\verb|\section| command is like this \llo{13}.
Sometimes you have to use curly brackets `\verb|{}|' to enclose the argument,
sometimes square brackets `\verb|[]|', and sometimes both at once.
There is method behind this apparent madness, but for the
time being you should be sure to copy the commands exactly as given.

\item When a command's name is made up entirely of letters, you must make sure
that the end of the command is marked by something that isn't a letter.
This is usually either the opening bracket around the command's argument, or
it's a space.  When it's a space, that space is always ignored by \LaTeX. We
will see later that this can sometimes be a problem.

\end{itemize}

\subsection{Overall structure}

There are some \LaTeX{} commands that must appear in every document.
The actual text of the document always starts with a
\verb|\begin{document}| command and ends with an \verb|\end{document}|
command \lls{10 and 39}.  Anything that comes after the \break
\verb|\end{document}| command is ignored.  Everything that comes
before the \break\verb|\begin{document}| command is called the
{\em preamble\/}. The preamble can only contain \LaTeX{} commands
to describe the document's style.

One command that must appear in the preamble is the
\verb|\documentclass| command \llo{9}.  This command specifies the
overall style for the document.  Our example file is a simple
technical document, and uses the {\tt article\/} class.  The document
you are reading was produced with the {\tt withesis\/} class. There
are other classes that you can use, as you will find out later on in
this document.

\subsection{Other Things to Look At}

\LaTeX{} can print both opening and closing quote characters, and can manage
either of these either single or double.  To do this it uses the two quote
characters from your keyboard: {\tt `} and {\tt '}. You will probably think of
{\tt '} as the ordinary single quote character which probably looks like
{\tt\symbol{'23}} or {\tt\symbol{'15}} on your keyboard,

and {\tt `} as a ``funny'' character that probably appears as
{\tt\symbol{'22}}. You type these characters once for single quote
\llo{21},  and twice for double quotes \llo{20}. The double quote
character {\tt "} itself is almost never used and should not be used
unless you want your text to look "funny" (compare the quote in the
previous sentence).

\LaTeX{} can produce three different kinds of dashes.
A long dash, for use as a punctuation symbol, as is typed as three dash
characters in a row, like this `\verb|---|' \llo{23}.  A shorter dash,
used between numbers as in `10--20', is typed as two dash
characters in a row, while a single dash character is used as a hyphen.

From time to time you will need to include one or more of the \LaTeX{}
special symbols in your text.  Seven of them can be printed by
making them into commands by proceeding them by backslash
\llo{36}.  The remaining three symbols can be produced by more
advanced commands, as can symbols that do not appear on your keyboard
such as \dag, \ddag, \S, \pounds, \copyright, $\sharp$ and $\clubsuit$.

It is sometimes useful to include comments in a \LaTeX{} file, to remind
you of what you have done or why you did it.  Everything to the
right of a \verb|%| sign is ignored by \LaTeX{}, and so it can
be used to introduce a comment.

\section{Document Classes and Class Options}\label{sec:styles}

There are four standard document classes available in \LaTeX:
\nobreak

\begin{description}

\item[{\tt article}]  intended for short documents and articles for publication.
Articles do not have chapters, and when \verb|\maketitle| is used to generate

a title (see Section~\ref{sec:title}) it appears at the top of the first page

rather than on a page of its own.

\item[{\tt report}] intended for longer technical documents.
It is similar to
{\tt article}, except that it contains chapters and the title appears on a page
of its own.

\item[{\tt book}] intended as a basis for book publication.  Page layout is
adjusted assuming that the output will eventually be used to print on
both sides of the paper.

\item[{\tt letter}]  intended for producing personal letters.  This style
will allow you to produce all the elements of a well laid out letter:
addresses, date, signature, etc.
\end{description}

An additional document class, the one used for this document and for
University of Wisconsin--Madison theses, is \fn{withesis}.


These standard classes can be modified by a number of {\em class
options\/}. They appear in square brackets after the
\verb|\documentclass| command. Only one class can ever be used but
you can have more than one class option, in which case their names
should be separated by commas.  The standard style options are:
\begin{description}

\item[{\tt 11pt}]  prints the document using eleven-point type for the running
 text
rather that the ten-point type normally used. Eleven-point type is about
ten percent larger than ten-point.

\item[{\tt 12pt}]  prints the document using twelve-point type for the running
 text
rather than the ten-point type normally used. Twelve-point type is about
twenty percent larger than ten-point.

\item[{\tt twoside}]  causes documents in the article or report styles to be
formatted for printing on both sides of the paper.  This is the default for the
book style.

\item[{\tt twocolumn}] produces two column on each page.

\item[{\tt titlepage}]  causes the \verb|\maketitle| command to generate a
title on a separate page for documents in the \fn{article} style.
A separate page is always used in both the \fn{report} and \fn{book} styles.

\end{description}

The University of Wisconsin--Madison thesis style, \fn{withesis} also
has some class options defined.  These class options are for
ten-point type (\fn{10pt}), tweleve-point type (\fn{12pt}), two-sided
printing (\fn{twoside}), Master Thesis margins (\fn{msthesis}) and an
option to print a small black box on lines which exceed the margins
(\fn{margincheck}).

\section{Environments}

We mentioned earlier the idea of identifying a quotation to \LaTeX{} so that
it could arrange to typeset it correctly. To do this you enclose the
quotation between the commands \verb|\begin{quotation}| and
\verb|\end{quotation}|.
This is an example of a \LaTeX{} construction called an {\em environment\/}.
A number of
special effects are obtained by putting text into particular environments.

\subsection{Quotations}

There are two environments for quotations: \fn{quote} and \fn{quotation}.
\fn{quote} is used either for a short quotation or for a sequence of
short quotations separated by blank lines:
\egstart\singlespace
\begin{verbatim}
US presidents ... remarks:
\begin{quote}
The buck stops here.

I am not a crook.
\end{quote}
\end{verbatim}
\egmid%
US presidents have been known for their pithy remarks:
\begin{quote}
The buck stops here.

I am not a crook.
\end{quote}
\egend

Use the \fn{quotation} environment for quotations that consist of more
than one paragraph.  Paragraphs in the input are separated by blank
lines as usual:
\egstart\singlespace
\begin{verbatim}

Here is some advice to remember:
\begin{quotation}
Environments for making
...other things as well.

Many problems
...environments.
\end{quotation}
\end{verbatim}
\egmid%
Here is some advice to remember:
\begin{quotation}
Environments for making quotations
can be used for other things as well.

Many problems can be solved by
novel applications of existing
environments.
\end{quotation}
\egend

\subsection{Centering and Flushing}

Text can be centered on the page by putting it within the \fn{center}
environment, and it will appear flush against the left or right margins if it
is placed within the \fn{flushleft} or \fn{flushright} environments.

Text within these environments will be formatted in the normal way, in
{\samepage
particular the ends of the lines that you type are just regarded as spaces.  To
indicate a ``newline'' you need to type the \verb|\\| command.  For example:
\egstart\singlespace
\begin{verbatim}
\begin{center}
one
two
three \\
four \\
five
\end{center}

\end{verbatim}
\egmid%
\begin{center}

one
two
three \\
four \\

five
\end{center}
\egend
}

\subsection{Lists}

There are three environments for constructing lists.  In each one each new
item is begun with an \verb|\item| command.  In the \fn{itemize} environment
the start of each item is given a marker, in the \fn{enumerate}
environment each item is marked by a number.  These environments can be nested
within each other in which case the amount of indentation used
is adjusted accordingly:
\egstart\singlespace

\begin{verbatim}
\begin{itemize}
\item Itemized lists are handy.
\item However, don't forget
  \begin{enumerate}
  \item The `item' command.
  \item The `end' command.
  \end{enumerate}
\end{itemize}
\end{verbatim}
\egmid%
\begin{itemize}
\item Itemized lists are handy.
\item However, don't forget
  \begin{enumerate}
  \item The `item' command.
  \item The `end' command.
  \end{enumerate}
\end{itemize}
\egend


The third list making environment is \fn{description}.  In a description you
specify the item labels inside square brackets after the \verb|\item| command.
For example:
\egstart\singlespace
\begin{verbatim}
Three animals that you should
know about are:
\begin{description}
  \item[gnat] A small
            animal...
  \item[gnu] A large
           animal...
  \item[armadillo] A ...
\end{description}
\end{verbatim}
\egmid%
Three animals that you should
know about are:
\begin{description}
  \item[gnat] A small animal that causes no end of trouble.
  \item[gnu] A large animal that causes no end of trouble.
  \item[armadillo] A medium-sized animal.
\end{description}
\egend

\subsection{Tables}

Because \LaTeX{} will almost always convert a sequence of spaces
into a single space, it can be rather difficult to lay out tables.
See what happens in this example
 \nolinebreak
\begin{eg}
\begin{minipage}[t]{0.55\textwidth} \singlespace
\begin{verbatim}
\begin{flushleft}
Income  Expenditure Result   \\
20s 0d  19s 11d     happiness \\
20s 0d  20s 1d      misery  \\
\end{flushleft}
\end{verbatim}
\end{minipage}
\begin{minipage}[t]{0.3\textwidth}
\begin{flushleft}
Income  Expenditure Result   \\
20s 0d  19s 11d     happiness \\
20s 0d  20s 1d      misery  \\
\end{flushleft}
\end{minipage}
\end{eg}

The \fn{tabbing} environment overcomes this problem. Within it you
set tabstops and tab to them much like you do on a typewriter.
Tabstops are set with the \verb|\=| command, and the \verb|\>|
command moves to the next stop.  The \verb|\\| command is used to
separate each line.  A line that ends \verb|\kill| produces no
output, and can be used to set tabstops:
\nolinebreak
\begin{eg}
\begin{minipage}[t]{0.6\textwidth}
\singlespace
\begin{verbatim}
\begin{tabbing}
Income \=Expenditure \=    \kill
Income \>Expenditure \>Result \\
20s 0d \>19s 11d \>Happiness \\
20s 0d \>20s 1d  \>Misery    \\
\end{tabbing}
\end{verbatim}
\end{minipage}
\vspace{1ex}
\begin{minipage}[t]{0.35\textwidth}
\begin{tabbing}
\singlespace
Income \=Expenditure \=    \kill
Income \>Expenditure \>Result \\
20s 0d \>19s 11d \>Happiness \\
20s 0d \>20s 1d  \>Misery    \\
\end{tabbing}
\end{minipage}
\end{eg}

Unlike a typewriter's tab key, the \verb|\>| command always moves to the next
tabstop in sequence, even if this means moving to the left.  This can cause
text to be overwritten if the gap between two tabstops is too small.

\subsection{Verbatim Output}

Sometimes you will want to include text exactly as it appears on a terminal
screen.  For example, you might want to include part of a computer program.
Not only do you want \LaTeX{} to stop playing around with the layout of your
text, you also want to be able to type all the characters on your keyboard
without confusing \LaTeX. The \fn{verbatim} environment has this effect:

\egstart

\begin{flushleft}\singlespace
\verb|The section of program in|  \\
 \verb|question is :|\\
 \verb|\begin{verbatim}|           \\
\verb|{ this finds %a & %b }|     \\[2ex]

\verb|for i := 1 to 27 do|        \\
\ \ \ \verb|begin|                \\
\ \ \ \verb|table[i] := fn(i);|   \\

\ \ \ \verb|process(i)|           \\
\ \ \ \verb|end;|                 \\
\verb|\end{verbatim}|
\end{flushleft}
\egmid%
The section of program in
question is:
\begin{verbatim}
{ this finds %a & %b }

for i := 1 to 27 do
   begin
   table[i] := fn(i);
   process(i)
   end;

\end{verbatim}
\egend

The \fn{withesis} document style also provides the command {\tt \verb|\verbatimfile{foo.fe}|}
which will read in the file {\tt foo.fe} into the document in \fn{verbatim} format with
the font \verb|\tt|.  See Appendix~\ref{matlab} for an example.

\section{Type Styles}

We have already come across the \verb|\em| command for changing
typeface.  Here is a full list of the available typefaces:
\begin{quote}\singlespace\begin{tabbing}
\verb|\sc|~~ \= \sc Small Caps~~~ \= \verb|\sc|~~ \= \sc Small Caps~~~
                                  \= \verb|\sc|~~ \=                   \kill
\verb|\rm|   \> \rm Roman         \> \verb|\it|   \> \it Italic
                                  \> \verb|\sc|   \> \sc Small Caps    \\
\verb|\em|   \> \em Emphatic      \> \verb|\sl| \> \sl Slanted
                                  \> \verb|\tt|   \> \tt Typewriter     \\
\verb|\bf|   \> \bf Boldface      \> \verb|\sf| \> \sf Sans Serif
\end{tabbing}\end{quote}

Remember that these commands are used {\em inside\/} a pair of braces to limit
the amount of text that they effect.  In addition to the eight typeface
commands, there are a set of commands that alter the size of the type.  These
commands are:
\begin{quotation}\singlespace\begin{tabbing}
\verb|\footnotesize|~~ \= \verb|\footnotesize|~~ \= \verb|\footnotesize| \=
 \kill
\verb|\tiny|           \> \verb|\small|          \> \verb|\large|        \>
\verb|\huge|  \\
\verb|\scriptsize|     \> \verb|\normalsize|     \> \verb|\Large|        \>
\verb|\Huge|  \\
\verb|\footnotesize|   \>                        \> \verb|\LARGE|
\end{tabbing}\end{quotation}

\section{Sectioning Commands and Tables of Contents}
\label{ess:sectioning}

Technical documents, like this one, are often divided into sections.
Each section has a heading containing a title and a number for easy
reference.  \LaTeX{} has a series of commands that will allow you to identify
different sorts of sections.  Once you have done this \LaTeX{} takes on the
responsibility of laying out the title and of providing the numbers.

The commands that you can use are:
\begin{quote}\singlespace\begin{tabbing}
\verb|\subsubsection| \= \verb|\subsubsection|~~~~~~~~~~ \=           \kill
\verb|\chapter|       \> \verb|\subsection|    \> \verb|\paragraph|    \\
\verb|\section|       \> \verb|\subsubsection| \> \verb|\subparagraph| \\
\end{tabbing}\end{quote}
The naming of these last two is unfortunate, since they do not really have
anything to do with `paragraphs' in the normal sense of the word; they are just
lower levels of section.  In most document styles, headings made with
\verb|\paragraph| and \verb|\subparagraph| are not numbered.  \verb|\chapter|
is not available in document style \fn{article}.  The commands should be used
in the order given, since sections are numbered within chapters, subsections
within sections, etc.

A seventh sectioning command, \verb|\part|, is also available.  Its use is
always optional, and it is used to divide a large document into series of
parts.  It does not alter the numbering used for any of the other commands.

Including the command \verb|\tableofcontents| in you document will cause a
contents list to be included, containing information collected from the various
sectioning commands.  You will notice that each time your document is run
through \LaTeX{} the table of contents is always made up of the headings from
the previous version of the document.  This is because \LaTeX{} collects
information for the table as it processes the document, and then includes it
the next time it is run.  This can sometimes mean that the document has to be
processed through \LaTeX{} twice to get a correct table of contents.

\section{Producing Special Symbols}

You can include in you \LaTeX{} document a wide range of symbols that do not
appear on you your keyboard. For a start, you can add an accent to any letter:
\begin{quote}\singlespace\begin{tabbing}

\t{oo} \= \verb|\t{oo}|~~~ \=
\t{oo} \= \verb|\t{oo}|~~~ \=
\t{oo} \= \verb|\t{oo}|~~~ \=
\t{oo} \= \verb|\t{oo}|~~~ \=
\t{oo} \= \verb|\t{oo}|~~~ \=
\t{oo} \=                       \kill

\a`{o} \> \verb|\`{o}|  \> \~{o}  \> \verb|\~{o}|  \> \v{o}  \> \verb|\v{o}| \>
\c{o}  \> \verb|\c{o}|  \> \a'{o} \> \verb|\'{o}|  \\
\a={o} \> \verb|\={o}|  \> \H{o}  \> \verb|\H{o}|  \> \d{o}  \> \verb|\d{o}| \>
\^{o}  \> \verb|\^{o}|  \> \.{o}  \> \verb|\.{o}|  \\
\t{oo} \> \verb|\t{oo}| \> \b{o}  \> \verb|\b{o}|  \\  \"{o} \> \verb|\"{o}| \>
\u{o}  \> \verb|\u{o}|  \\
\end{tabbing}\end{quote}

A number of other symbols are available, and can be used by including the
following commands:
\begin{quote}\singlespace\begin{tabbing}

\LaTeX~\= \verb|\copyright|~~~~ \= \LaTeX~\= \verb|\copyright|~~~~ \=
\LaTeX~\=  \kill

\dag       \> \verb|\dag|       \> \S     \> \verb|\S|     \>
\copyright \> \verb|\copyright| \\
\ddag      \> \verb|\ddag|      \> \P     \> \verb|\P|     \>
\pounds    \> \verb|\pounds|    \\
\oe        \> \verb|\oe|        \> \OE    \> \verb|\OE|    \>
\ae        \> \verb|\AE|        \\
\AE        \> \verb|\AE|        \> \aa    \> \verb|\aa|    \>
\AA        \> \verb|\AA|        \\
\o         \> \verb|\o|         \> \O     \> \verb|\O|     \>
\l         \> \verb|\l|         \\
\L         \> \verb|\E|         \> \ss    \> \verb|\ss|    \>
?`         \> \verb|?`|         \\
!`         \> \verb|!`|         \> \ldots \> \verb|\ldots| \>
\LaTeX     \> \verb|\LaTeX|     \\
\end{tabbing}\end{quote}
There is also a \verb|\today| command that prints the current date. When you
use these commands remember that \LaTeX{} will ignore any spaces that
follow them, so that you can type `\verb|\pounds 20|' to get `\pounds 20'.
However, if you type `\verb|LaTeX is wonderful|' you will get `\LaTeX is
wonderful'---notice the lack of space after \LaTeX.
To overcome this problem you can follow any of these commands by a
pair of empty brackets and then any spaces that you wish to include,
and you will see that
\verb|\LaTeX{} really is wonderful!| (\LaTeX{} really is wonderful!).

\section{Titles}\label{sec:title}

Most documents have a title.  To title a \LaTeX{} document, you include the
following commands in your document, usually just after
\verb|begin{document}|.
\begin{quote}\singlespace\begin{verbatim}
\title{required title}
\author{required author}
\date{required date}
\maketitle
\end{verbatim}\end{quote}
If there are several authors, then their names should be separated by
\verb|\and|; they can also be separated by \verb|\\| if you want them to be
centred on different lines.  If the \verb|\date| command is left out, then the
current date will be printed.
\egstart
\singlespace
\begin{verbatim}
\title{Essential \LaTeX}
\author{J Warbrick \and An Other}
\date{14th February 1988}
\maketitle
\end{verbatim}
\egmid
\begin{center}
{\normalsize Essential \LaTeX}\\[4ex]
J Warbrick\hspace{1em}A N Other\\[2ex]
14th February 1988
\end{center}
\egend

The exact appearance of the title varies depending on
the document style.  In styles \fn{report} and \fn{book} the title appears on a
page of its own. In the \fn{article} style it normally appears at the top
of the first page, the style option \fn{titlepage} will alter this (see
Section~\ref{sec:styles}).  In the \fn{withesis} style, the title is created on a
seperate page in the format appropriate to a UW-Madison thesis or dissertation.

\section{Errors}

When you create a new input file for \LaTeX{} you will probably make mistakes.
Everybody does, and it's nothing to be worried about.  As with most computer
programs, there are two sorts of mistake that you can make: those that \LaTeX{}
notices and those that it doesn't.  To take a rather silly example, since
\LaTeX{} doesn't understand what you are saying it isn't going to be worried if
you mis-spell some of the words in your text.  You will just have to accurately
proof-read your printed output.  On the other hand, if you mis-spell one of
the environment names in your file then \LaTeX won't know what you want it
to do.

When this sort of thing happens, \LaTeX{} prints an error message on your
terminal screen and then stops and waits for you to take some action.
Unfortunately, the error messages that it produces are rather user-unfriendly
and a little frightening.  Nevertheless, if you know where to look they
will probably tell you where the error is and went wrong.

Consider what would happen if you mistyped \verb|\begin{itemize}| so that it
became \break\verb|\begin{itemie}|.  When \LaTeX{} processes this instruction, it
displays the following on your terminal:
\begin{quote}\singlespace\begin{verbatim}
LaTeX error.  See LaTeX manual for explanation.
              Type  H <return>  for immediate help.
! Environment itemie undefined.
\@latexerr ...for immediate help.}\errmessage {#1}
                                                  \endgroup
l.140 \begin{itemie}

?
\end{verbatim}\end{quote}
After typing the `?' \LaTeX{} stops and waits for you to tell it what to do.

The first two lines of the message just tell you that the error was detected by
\LaTeX{}. The third line, the one that starts `!' is the {\em error indicator}.
 It
tells you what the problem is, though until you have had some experience of
\LaTeX{} this may not mean a lot to you.  In this case it is just telling you
that it doesn't recognise an environment called \fn{itemie}.
The next two lines tell you what
\LaTeX{} was doing when it found the error, they are irrelevant at the moment
and can be ignored. The final line is called the {\em error locator}, and is
a copy of the line from your file that caused the problem.
It start with a line number to help you to find it in your file, and
if the error was in the middle of a line it will be shown
broken at the point where \LaTeX{} realised that there was an error.  \LaTeX{}
can sometimes pass the point where the real error is before discovering that
something is wrong, but it doesn't usually get very far.

At this point you could do several things.  If you knew enough about \LaTeX{}
you might be able to fix the problem, or you could type `X' and press the
return key to stop \LaTeX{} running while you go and correct the error.  The
best thing to do, however, is just to press the return key.  This will allow
\LaTeX{} to go on running as if nothing had happened.  If you have made one
mistake, then you have probably made several and you may as well try to find
them all in one go.  It's much more efficient to do it this way than to run
\LaTeX{} over and over again fixing one error at a time. Don't worry about
remembering what the errors were---a copy of all the error messages is being
saved in a {\em log\/} file so that you can look at them afterwards.

If you look at the line that caused the error it's normally obvious what the
problem was.  If you can't work out what you problem is look at the hints
below, and if they don't help consult Chapter~6 of the manual~\cite{lamport}.
  It contains a
list of all of the error messages that you are likely to encounter together with
some hints as to what may have caused them.

Some of the most common mistakes that cause errors are
\begin{itemize}
\item A mispelled command or environment name.
\item Improperly matched `\verb|{|' and `\verb|}|'---remember that they should
 always
come in pairs.
\item Trying to use one of the ten special characters \verb|# $ % & _ { } ~ ^|
and \verb|\| as an ordinary printing symbol.
\item A missing \verb|\end| command.
\item A missing command argument (that's the bit enclosed in '\verb|{|' and
`\verb|}|').
\end{itemize}

One error can get \LaTeX{} so confused that it reports a series of spurious
errors as a result.  If you have an error that you understand, followed by a
series that you don't, then try correcting the first error---the rest
may vanish as if by magic.
Sometimes \LaTeX{} may write a {\tt *} and stop without an error message.  This
is normally caused by a missing \verb|\end{document}| command, but other errors
can cause it.  If this happens type \verb|\stop| and press the return key.

Finally, \LaTeX{} will sometimes print {\em warning\/} messages.  They report
problems that were not bad enough to cause \LaTeX{} to stop processing, but
nevertheless may require investigation.  The most common problems are
`overfull' and `underfull' lines of text.  A message like:
\begin{quote}\footnotesize\begin{verbatim}
Overfull \hbox (10.58649pt too wide) in paragraph at lines 172--175
[]\tenrm Mathematical for-mu-las may be dis-played. A dis-played
\end{verbatim}\end{quote}
indicates that \LaTeX{} could not find a good place to break a line when laying
out a paragraph.  As a result, it was forced to let the line stick out into the
right-hand margin, in this case by 10.6 points.  Since a point is about 1/72nd
of an inch this may be rather hard to see, but it will be there none the less.

This particular problem happens because \LaTeX{} is rather fussy about line
breaking, and it would rather generate a line that is too long than generate a
paragraph that doesn't meet its high standards.  The simplest way around the
problem is to enclose the entire offending paragraph between
\verb|\begin{sloppypar}| and \verb|\end{sloppypar}| commands.  This tells
\LaTeX{} that you are happy for it to break its own rules while it is working on
that particular bit of text.

Alternatively, messages about ``Underfull \verb|\hbox'es''| may appear.
These are lines that had to have more space inserted between
words than \LaTeX{} would have liked.  In general there is not much that you
can do about these.  Your output will look fine, even if the line looks a bit
stretched.  About the only thing you could do is to re-write the offending
paragraph!

\section{A Final Reminder}

You now know enough \LaTeX{} to produce a wide range of documents.  But this
document has only scratched the surface of the
things that \LaTeX{} can do.  This entire document was itself produced with
\LaTeX{} (with no sticking things in or clever use of a photocopier) and even
it hasn't used all the features that it could.  From this you may get some
feeling for the power that \LaTeX{} puts at your disposal.

Please remember what was said in the introduction: if you {\bf do} have a
complex document to produce then {\bf go and read the manual}.  You will be
wasting your time if you rely only on what you have read here.
     % Edited ``Essential LaTeX'' by Jon Warbrick
\chapter{Figures and Tables}\label{quad}
This chapter\footnote{Most of the text in this chapter's introduction is from {\em How to
\TeX{} a Thesis: The Purdue Thesis Styles}} shows some example ways of incorporating tables and figures into \LaTeX{}.
Special environments exist for tables and figures and are special because they are
allowed to {\em float}---that is, \LaTeX{} doesn't always put them in the exact place
that they occur in your input file.  An algorithm is used to place the floating environments,
or floats, at locations which are typographically correct.  This may cause endless frustration
if you want to have a figure or table occur at a specific location.  There are a few
methods for solving this.

You can exert some influence on \LaTeX{}'s float placement algorithm by using
{\em float position specifiers}.  These specifiers, listed below, tell \LaTeX{}
what you prefer.
\begin{tabbing}
{\tt hhhhhh} \= ``bottom'' \=  \kill
{\tt h}\> ``here'' \> do not move this object \\
{\tt p}\> ``page'' \> put this object on a page of floats \\
{\tt b}\> ``bottom'' \> put this object at the bottom of a page\\
{\tt t}\> ``top'' \> put this object at the top of a page\\
\end{tabbing}

Any combination of these can be used:
\begin{quote}\tt\singlespace\begin{verbatim}
\begin{figure}[htbp]
 ...
\caption{A Figure!}
\end{figure}
\end{verbatim}\end{quote}

In this example, we asked \LaTeX{} to ``put the figure `here' if possible.  If it
is not possible (according to the rule encoded in the float algorithm), put it on the
next float page.  A float page is a page which contains nothing but floating objects,
{\em e.g.} a page of nothing but figures or tables.  If this isn't possible, try to put it
at the `top' of a page.  The last thing to try is to put the figure at the `bottom' of
a page.''

The remainder of this chapter deals with some examples of what to put into the figure,
the ellipsis (\ldots ) in the example above.

\section{Tables}
Table~\ref{pde.tab1} is an example table from the UW Math Department.
\begin{table}[htbp]
\centering
\caption{PDE solve times, $15^3+1$
equations.\label{pde.tab1}}
\begin{tabular}{||l|l|l|l|l|l||}\hline
Precond. & Time & Nonlinear & Krylov
& Function & Precond. \\
 & & Iterations & Iterations & calls & solves \\ \hline
None & 1260.9u & 3 & 26 & 30 & 0  \\
 &(21:09) & & & &  \\ \hline
FFT  & 983.4u & 2  & 5  & 8  & 7 \\
&(16:31) & & & & \\ \hline
\end{tabular}
\end{table}
The code to generate it is as follows:
\begin{quote}\tt\singlespace\begin{verbatim}
\begin{table}[htbp]
\centering
\caption{PDE solve times, $15^3+1$
equations.\label{pde.tab1}}
\begin{tabular}{||l|l|l|l|l|l||}\hline
Precond. & Time & Nonlinear & Krylov
& Function & Precond. \\
 & & Iterations & Iterations & calls & solves \\ \hline
None & 1260.9u & 3 & 26 & 30 & 0  \\
 &(21:09) & & & &  \\ \hline
FFT  & 983.4u & 2  & 5  & 8  & 7 \\
&(16:31) & & & & \\ \hline
\end{tabular}
\end{table}
\end{verbatim}\end{quote}

\section{Figures}
There are many different ways to incorporate figures into a \LaTeX{}
document.  \LaTeX{} has an internal {\tt picture} environment and
some programs will generate files which are in this format and can
be simply {\tt include}d.  In addition to \LaTeX{} native {\tt picture}
format, additional packages can be loaded in the {\tt\verb|\documentstyle|}
command (or using the {\tt input} command) to allow \LaTeX{} to process
non-native formats such as PostScript.

\subsection{\tt gnuplot}
The graph of Figure~\ref{gelfand.fig2}
 was created by gnuplot. For simple graphs this is a
 great utility.  For example, if you want a sin curve in your thesis
 try the following:
\begin{quote}\tt\singlespace\begin{verbatim}
 (terminal window): gnuplot
 (in gnuplot):
                 set terminal latex
                 set output "foo.tex"
                 plot sin(x)
                 quit
\end{verbatim}\end{quote}
This will generate a file called {\tt foo.tex} which can be read in
with the following statements.
\begin{figure}[htbp]
\centering
\input{fig2.tex}
\caption{Gelfand equation on the ball, $3\leq n \leq 9$.
\label{gelfand.fig2}}
\end{figure}
\begin{quote}\tt\singlespace\begin{verbatim}
\begin{figure}[htbp]
\centering
\input{fig2.tex}
\caption{Gelfand equation on the ball, $3\leq n \leq 9$.
\label{gelfand.fig2}}
\end{figure}
\end{verbatim}\end{quote}
One advantage to using the native \LaTeX{} {\tt picture} environment
is that the fonts will be assured to agree and the pictures can be viewed
in the {\tt .dvi} viewer.

\subsection{PostScript}
Many drawing applications now allow the export of a graphic to the
{\em Encapsulated PostScript} format.  These files have a suffix of
{\tt .EPS} or {\tt .EPSF} and are similar to a regular PostScript
file except that they contain a {\em bounding box} which describes
the dimensions of the figure.

In order to include PostScript figures, the {\tt epsfig} (or {\tt psfig}
depending on the system you are using) style file must be included in either
the {\tt\verb|\documentstyle|} command or the preamble using the {\tt input} command.

Figure~\ref{vwcontr} is a plot from Matlab.
\begin{figure}[htbp]
\centerline{
\psfig{figure=vwcontr.eps,width=5in,angle=0}
           }
\caption{$\sigma$ as a Function of Voltage and Speed, $\alpha = 20$}
\label{vwcontr}
\end{figure}
The commands to include this figure are
\begin{quote}\tt\singlespace\begin{verbatim}
\begin{figure}[htbp]
\centerline{
\psfig{figure=vwcontr.ps,width=5in,angle=0}
           }
\caption{$\sigma$ as a Function of Voltage and Speed, $\alpha = 20$}
\label{vwcontr}
\end{figure}
\end{verbatim}\end{quote}

Observe that the {\tt \verb|\psfig|} command allows the scaling of the figure
by setting either the {\tt width} or {\tt height} of the figure.  If only one
dimension is specified, the other is computed to keep the same aspect ratio.
The figure can also be rotated by setting {\tt angle} to the desired value in
degrees.
           % Chapter 3 Edited from UW Math Dept's Sample Thesis
% bibs.tex
%
% This chapter briefly talks about BibTex and is mostly
% copied from a similar chapter from "How to TeX a Thesis:
% The Purdue Thesis Styles" by James Darrell McCauley and
% Scott Hucker
%

\newcommand{\BibTeX}{{\sc Bib}\TeX}

\chapter{Citations and Bibliographies}
This chapter is an edited form of the same chapter from {\em How to 
\TeX{} a Thesis: The Purdue Thesis Styles} by James Darrell McCauley and
Scott Hucker.

The task of compiling and formatting the sources cited in papers can
be quite tedious, especially for large documents like theses.  A program
separate from \LaTeX{}, called ``\BibTeX{},''can be used to automate this task~\cite{lamport}.

\section{The Citation Command}
When referring to the work of someone else, the {\tt \verb|\cite|} command is used.
This generates the citation in the text for you.  In the above paragraph, the command
{\tt \verb|\cite{lamport}|} was used after the word ``task.''  The formatting of your
citation is handled by either the document style or a style option.  The default citation
style uses the number system (a number in square brackets).  Other citation styles
may use the author-date system, (Lamport, 1986) or the superscript$^3$ system.

\section{Bibliography Styles}
The way that a reference is formatted in your bibliography depends on the bibliography
style, which is specified near the beginning of your document with the\break
{\tt \verb|\bibliographstyle{file}|} command.  The file {\tt file.bst} is the name of the 
bibliography style file.  Standard \BibTeX{} bibliography style files include {\tt plain},
{\tt unsrt}, {\tt alpha}, and {\tt abbrev}.  The bibliography style governs whether or not
references are sorted, whether first names or initials are used for authors, whether or 
not last names are listed first, the location of the year in the references (after the
author or at the end of the reference), {\em etc.}.  You may be required by your
department or major professor to follow as style for a particular journal.  If so, then you
will need to find a \BibTeX{} style file to suit your needs.  Most major journals have
style files.  If you cannot locate an appropriate \BibTeX{} style file, then choose the
one which is closest and then edit the {\tt .bbl} file by hand.  See Section~\ref{BBL}
for a brief discussion on the {\tt .bbl} file.  Some common, but non-standard \BibTeX{}
styles include
\begin{tabbing}
{\tt jacs-new.bstxxxx}\= {\em Journal of the American Chemical Society}\kill
{\tt acm.bst}\>The Association for Computing Machinery\\
{\tt ieeetr.bst}\> The {\em IEEE Transactions} style\\
{\tt jacs-new.bst}\> {\em Journal of the American Chemical Society}
\end{tabbing}

\section{The Database}
The  {\tt \verb|\bibliography{file}|} command is placed in your input file at the location
where the ``List of References'' section\footnote{or ``Bibliography'' 
if {\tt \char92 altbibtitle } has been specified in the preamble.} would be.  It specifies the name (or names) of
your bibliographic data base, {\tt file.bib}.  An example entry in a \BibTeX{}
database is:
\begin{quote}\singlespace\tt\begin{verbatim}
@book{ lamport86 ,
     author =    "Leslie Lamport" ,
     title =     "\LaTeX: A Document Preparation System" ,
     publisher = "Addison--Wesley Pub.\ Co." ,
     year =      "1986" ,
     address =   "Reading, MA" 
}
\end{verbatim}\end{quote}

The citation key is the first field in this entry--- citing this book in a \LaTeX{}
file would look like
\begin{quote}\singlespace\tt\begin{verbatim}
According to Lamport~\cite{lamport86} ...
\end{verbatim}\end{quote}
The tilde ({\tt \verb|~|}) is used to tie the word ``Lamport'' to the citation
generated.  The space between these words is then unbreakable---the word ``Lamport''
and the citation \cite{lamport} will not be split across two lines if they happen to occur
near the end of a line.

A listing of all entry types with their required and optional fields is given in 
Appendix~\ref{bibrefs}. There are several tools which exist to help in editing a \BibTeX{}
file, however, their use is beyond the scope of this manual and can be found by searching
the net.  You can simply use a plain text editor like {\tt vi} or {\tt WordPad} to edit
and create the database files.

There are several rules which you must follow when creating your database.  Authors are
always listed by their full names, first name first, and multiple authors are separated
by {\tt and}.  For example
\begin{quote}\singlespace\tt\begin{verbatim}
author = "John Jay Park and Frederick Gene Watson and
          Michelle Catherine Smith",
\end{verbatim}\end{quote}
If you were using {\tt abbrv} as your {\tt bibliographystyle}, a reference for these
authors may look like:
\begin{quote}
J.J. Park, F.G. Watson, and M.C. Smith \ldots
\end{quote}

Some styles only capitalize the first word of the title.  If you use any acronyms or
other words that should always be capitalized in titles, then they should be 
enclosed in {\tt \{\}}'s ({\em e.g.}, {\tt \{Fortran\}}, {\tt \{N\}ewton}).
This protects the case of these characters.

There are several other rules for \BibTeX{} listed in~\cite{lamport} which should be
referred to because they are not discussed here.

\section{Putting It All Together}
\label{BBL}
To aid the reader in understanding how all of this works together, the following 
excerpt was taken from Lamport~\cite{lamport}:
\begin{quotation}\singlespace
When you ran \LaTeX{} with the input file {\tt sample.tex}, you may have
noticed that \LaTeX{} created a file named {\tt sample.aux}.  This file,
called an {\em auxiliary} file, contains cross-referencing information.  Since
{\tt sample.tex} contains no cross-referencing commands, the auxiliary file it
produces has no information.  However, suppose that \LaTeX{} is run with an
input file named {\tt myfile.tex} that has citations and bibliography-making
[or referencing] commands.  The auxiliary file {\tt myfile.aux} that it produces
will contain all of the citation keys and the arguments of the {\tt \verb|\bibliography|}
and {\tt\verb|\bibliographystyle|} commands.  When \BibTeX{} is run, it reads
this information from the auxiliary file and produces a file named {\tt myfile.bbl}
containing \LaTeX{} commands to produce the source list \ldots The next time
\LaTeX{} is run on {\tt myfile.tex}, the {\tt \verb|\bibliography|} command reads
the {\tt bbl} file ({\tt myfile.bbl}), which generates the source list.
\end{quotation}

Thus, the command sequence for a source file called {\tt main.tex} which is going to
use \BibTeX{} would be:
\begin{quote}\singlespace\tt\begin{verbatim}
latex main.tex
bibtex main
latex main
latex main
\end{verbatim}\end{quote}
The first \LaTeX{} is to collect all of the citations for \BibTeX{}.  Then
\BibTeX{} is run to generate the bibliography.  \LaTeX{} is run again to
incorporate the bibliography into the document and the run the last time to
update any references (like pages in the Table of Contents) which changed when
the bibliography was included.
           % Chapter 4 From PU Thesis styles, by J.D. McCauley
% usage.tex
%
% This file explains how to use the withesis style
%   it is heavily modelled after a similar chapter by McCauley
%   for the Purdue Thesis style
%
% Eric Benedict, May 2000
%
% It is provided without warranty on an AS IS basis.


\chapter{Using the {\tt withesis} Style}

You can get a copy of the \LaTeX{} style for creating a University
of Wisconsin--Madison thesis or dissertation from:

{\tt http://www.cae.wisc.edu/\verb+~+benedict/LaTeX.html}

After somehow unpacking it, you will have the style files ({\tt withesis.sty}
{\tt withe10.sty}, and {\tt withe12.sty}) as well the files used to create
this document.  The files used for this document can be copied and used as a
template for your own thesis or dissertation.

The final printed form of this document is useful, but the
combination of the source code and final copy form a much more valuable
reference.  Keeping a working copy of the this document can be helpful
when you are later working on your thesis or disseration and want to know
how to do something.  If you find a similar example in this document,
then you can simply look at the corresponding source code and add it to
your document.    Because many parts of this document were written by
different people, the styles and techniques are also different and provide
different ways of achieving the same or similar results.

Because of the typical size of theses, it makes sense to break the document
up into several smaller files.  Usually this is done at the chapter level.
These files can then be {\tt \verb|\include|}d in a {\em root} file.  It is
the {\em root} file that you will run \LaTeX{} on.  For this manual, the
root file is called {\tt main.tex}.

\section{The Root File and the Preamble}
The {\tt \verb|\documentclass|} command is used to tell \LaTeX{} that you will
be using the {\tt withesis} document class and it is the first command in your
root file.  Class options such as {\tt 10pt}, {\tt 12pt}, {\tt msthesis} or
{\tt margincheck} are specified here:

{\tt \verb|\documentclass[12pt,msthesis]{withesis}|}

The class option {\tt msthesis} sets the margins to be appropriate for depositing
with the UW library, namely a 1.25 inch left margin with the remaining margins 1 inch.
The defaults for the title page are also defined for a thesis and for a Master of
Science degree.

The class option {\tt margincheck} will place a small black square at the end of
each line which exceeds the margins.\footnote{In reality, the square is
placed at the end of lines which exceed their {\tt \char92hbox}.  This usually
(but not always) indicates a  margin violation on the right margin.  Left
margin violations aren't indicated and if the margin violation is large enough,
there isn't room for the black box to be visiable.}  This is visible both in the {\tt .dvi} file
as well as in the {\tt .ps} file.

The area immediately following this command is called the {\em preamble} and is
used for things like including different style packages,
defining new macros and declaring the page style.

The style packages can be used to easily change the thesis font.  For example,
this document is set in Times Roman instead of the \LaTeX default of Computer
Modern.  This change was performed by including the {\tt times} package:

{\tt\verb|\usepackage{times}|}\footnote{In this document, the typewriter font
{\tt $\backslash$tt} was redefined to use the Computer Modern font with the command
{\tt $\backslash$renewcommand\{$\backslash$ttdefault\}\{cmtt\}}.  
For more information, see~\cite{goossens}.}

Remember that if you change the fonts from the default Computer Modern to
PostScript ({\em e.g.} Times Roman) then in order to correctly see the
document, you will need to convert the {\tt *.dvi} output into a {\tt *.ps}
file and view the document with a PostScript viewer. This is required since 
most {\tt *.dvi} previewer programs cannot 
display PostScript fonts.  Usually, the previewer will substitute
default fonts so the document may be viewed; however, since the alternate
fonts may not be the same size, the formatting of the document may appear
to be incorrect.

The style package for including Postscript figures, {\tt epsfig}, is included with

{\tt\verb|\usepackage{epsfig}|}

If multiple style packages are required, then they can be combined into one statement
as follows:

{\tt\verb|\usepackage{epsfig,times}|}

Many different style packages are available.  For more information, see~\cite{goossens}.

The page styles are defined using a similar method.
A special style is defined for the {\tt withesis} style:

{\tt\verb|\pagestyle{thesisdraft}|}

This style causes the footer text to become:

{\verb| DRAFT: Do Not Distribute        <time><Date>        <input file name>|}

This appears at the bottom of every page.

In addition to the page style command, the {\tt withesis} has defined several useful
commands which are specified in the preamble.  They include {\tt \verb| \draftmargin|},
{\tt \verb|\draftscreen|}, {\tt \verb|\noappendixtables|}, and
{\tt \verb|\noappendixfigures|}.

The command  {\tt \verb|\draftmargin|} draws a PostScript box with the dimensions of
the margins.  This makes it easy to check that the margins are correct and to see if
any of the text or figures are outside of the required margins.  This box is only visible
in the {\tt .ps} file since it is a PostScript special.


The command  {\tt \verb|\draftscreen|} draws a PostScript screen with the word {\em DRAFT}
in light grey and diagonally across the page.  This screen is only visible
in the {\tt .ps} file since it is a PostScript special.

The commands {\tt \verb|\noappendixtables|} and/or {\tt \verb|\noappendixfigures|} should
be used if the appendix does not have either tables or figures respectively.  These commands
inhibit the Appendix Table or Appendix Figure titles in the List of Tables or List of
Figures.\label{usage:noapp}


If you have specified the {\tt psfig} or {\tt epsfig} document style package, then a useful
command is {\tt \verb|\psdraft|}.  This command will show the bounding box that the figure
would occupy (instead of actually including the figure).  This speeds up the draft copy
printing, reduces toner usage and the drawn box is visible in the {\tt .dvi} file.

The next usual command is {\tt \verb|\begin{document}|}.  The following example is part
of the root file used for this manual.

\begin{quote} \singlespace\footnotesize\tt
\begin{verbatim}
\bibliographystyle{plain}
\include{prelude}        % Title page, abstract, table of contents, etc
\include{intro}          % Chapter 1
\include{essentials}     % Edited ``Essential LaTeX'' by Jon Warbrick
\include{figs}           % Chapter 3 Edited from UW Math Dept's Sample Thesis
\include{bibs}           % Chapter 4 From PU Thesis styles, by J.D. McCauley
\include{usage}          % Chapter 5 Strongly based on similar by J.D. McCauley
\bibliography{refs}      % Make the bibliography
\begin{appendices}       % Start of the Appendix Chapters.  If there is only
                         % one Appendix Chapter, then use \begin{appendix}
\include{code}         % Including computer code listings
\include{bibref}         % a BibTeX reference
\include{math}           % Complex Equations from the UW Math Department
\include{acro}           % A discussion on generating PDF files.
\end{appendices}         % End of the Appendix Chapters.  ibid on \end{appendix}
%\include{vita}          % Optional Vita, use \begin{vita} vita text \end{vita}
\end{document}
\end{verbatim}
\end{quote}

\section{Prelude}
After the {\tt \verb|\begin{document}|} comes the preliminary information found in
theses.  In this manual, the information is kept in the file {\tt prelude.tex} (see
above).  These pages will need to be numbered with roman numerals, so use
\begin{quote}\tt\singlespace\begin{verbatim}
\clearpage\pagenumbering{roman}
\end{verbatim}\end{quote}

Next, comes your thesis or dissertation title, your name, date of graduation, department
and degree.
\begin{quote}\tt\singlespace\begin{verbatim}
\title{How to \LaTeX\ a Thesis}
\author{Eric R. Benedict}
\date{2000}
%   - The default degree is ``Doctor of Philosophy''
%     Degree can be changed using the command \degree{}
%\degree{New Degree}
%   - for a PhD dissertation (default), specify \dissertation
%\dissertation
%   - for a masters project report, specify \project
%\project
%   - for a preliminary report, specify \prelim
%\prelim
%   - for a masters thesis, specify \thesis
%\thesis
%   - The default department is ``Electrical Engineering''
%     The department can be changed using the command \department{}
%\department{New Department}
\end{verbatim}\end{quote}

If you specified the class option {\tt msthesis}, then the degree is changed to
{\em Master \break of Science} and the {\tt \verb|\thesis|} option is specified.  If you
want to have the masters margins with another document, then the {\tt \verb|\degree|}
and {\tt \verb|\dissertation|},  {\tt \verb|\project|}, {\em etc.\/} can be specified
as needed.

Once the
above are all defined, use  {\tt \verb|\maketitle|} to generate the title page.
\begin{quote}\tt\singlespace\begin{verbatim}
\maketitle
\end{verbatim}\end{quote}

If you wish to include a copyright page (see Section~\ref{copyright} for
information on registering the copyright.), then add the command
\begin{quote}\tt\singlespace\begin{verbatim}
\copyrightpage
\end{verbatim}\end{quote}
This will generate the proper copyright page and will use the name and date specified
in {\tt \verb|\author{}|} and {\tt \verb|\date{}|}.

Next are the dedications and acknowledgements:
\begin{quote}\tt\singlespace\begin{verbatim}
\begin{dedication}
To my pet rock, Skippy.
\end{dedication}

\begin{acknowledgments}
I thank the many people who have done lots of nice things for me.
\end{acknowledgments}
\end{verbatim}\end{quote}

You must tell \LaTeX{} to generate a table of contents, a list of tables and a list of
figures:
\begin{quote}\tt\singlespace\begin{verbatim}
\tableofcontents
\listoftables
\listoffigures
\end{verbatim}\end{quote}

If you wish to have a nomenclature, list of symbols or glossary it can go here.
\begin{quote}\tt\singlespace\begin{verbatim}
\begin{nomenclature}
%\begin{listofsymbols}
%\begin{glossary}
\begin{tabular}{ll}
$C_1$ & Constant 1\\
\ldots
\end{tabular}
%\end{glossary}
%\end{listofsymbols}
\end{nomenclature}
\end{verbatim}\end{quote}

If your abstract will be microfilmed by Bell and Howell (formerly UMI), then you
will need to generate an abstract of less than 350 words.  This abstract can be created
using the {\tt umiabstract} environment.  This environment requires that you define your
advisor and your advisor's title using {\tt \verb|\advisorname{}|} and
{\tt \verb|\advisortitle{}|}.
\begin{quote}\tt\singlespace\begin{verbatim}
\advisorname{Bucky J. Badger}
\advisortitle{Assistant Professor}
% ABSTRACT
\begin{umiabstract}
\noindent       % Don't indent first paragraph.
This explains the basics for using \LaTeX\ to typeset a
dissertation, thesis or project report for the University of
Wisconsin-Madison.

...

\end{umiabstract}
\end{verbatim}\end{quote}
This will place your name, title and required text at the top of the page and follow the
abstract text with your advisor's name at the bottom for your advisor's signature.  This
page is not numbered and would be submitted separately.

If you will have an abstract as part of your document, then the {\tt abstract} environment
should be used.
\begin{quote}\tt\singlespace\begin{verbatim}
\begin{abstract}
\noindent       % Don't indent first paragraph.
This explains the basics for using \LaTeX\ to typeset a
dissertation, thesis or project report for the University of
Wisconsin-Madison.

...

\end{abstract}
\end{verbatim}\end{quote}
This will generate a page number and it will be included in the Table
of Contents.  

If you will have both the UMI and regular abstracts like this document, then
you will probably want to write the abstract once and save it in a seperate
file such as {\tt abstract.tex}.  Then, you can use the same abstract for
both purposes.

\begin{quote}\begin{verbatim}
\begin{umiabstract}
  \input{abstract}
\end{umiabstract}

\begin{abstract}
  \input{abstract}
\end{abstract}
\end{verbatim}\end{quote}

Finally, the page numbers must be changed to arabic numbers to conclude the preliminary
portion of the document.
\begin{quote}\tt\singlespace\begin{verbatim}
\clearpage\pagenumbering{arabic}
\end{verbatim}\end{quote}

\section{The Body}
At the beginning of {\tt intro.tex} there is the following command:
\begin{quote}\tt\singlespace\begin{verbatim}
\chapter{Introducing the {\tt withesis} \LaTeX{} Style Guide}
\end{verbatim}\end{quote}
Following that is the text of the chapter.  The body of your thesis is separated by
sectioning commands like {\tt \verb|\chapter{}|}.  For more information on the sectioning
commands, see Section~\ref{ess:sectioning}.

Remember the basic rule of outlining you learned in grammar school:
\begin{quote}
You cannot have an `A' if you do not have a `B'
\end{quote}
Take care to have at least two {\tt \verb|\section|}s if you use the command; have
two {\tt \verb|\subsection|}s, {\em etc}.



\section{Additional Theorem Like Environments}
The {\tt withesis} style adds numerous additional theorem like environments.  These
environments were included to allow compatibility with the University of Wisconsin's
Math Department's style file.  These environments are
{\tt theorem}, {\tt assertion}, {\tt claim}, {\tt conjecture}, {\tt corollary},
{\tt definition}, {\tt example}, {\tt figger}, {\tt lemma}, {\tt prop} and {\tt remark}.

As an example, consider the following.
\begin{lemma}
Assuming that $\partial\Omega_2 = \emptyset$ and that $h(t) = 1$, we
have $$
\begin{array}{lr}
\Delta u = f, &  x\in\Omega ,\\[2pt]
u =  g_1, &  x\in\partial\Omega .
\end{array}
$$
\end{lemma}
which was produced with the following:
\begin{quote}\tt\singlespace\begin{verbatim}
\begin{lemma}
Assuming that $\partial\Omega_2 = \emptyset$ and that $h(t) = 1$, we
have $$
\begin{array}{lr}
\Delta u = f, &  x\in\Omega ,\\[2pt]
 u =  g_1, &  x\in\partial\Omega .
\end{array}
$$
\end{lemma}
\end{verbatim}\end{quote}

\section{Bibliography or References}
As a final note, the default title for the references chapter is ``LIST OF REFERENCES.''
Since some people may prefer ``BIBLIOGRAPHY'', the command
\break{\tt \verb|\altbibtitle|}
has been added to change the chapter title.

\section{Appendices}
There are two commands which are available to suppress the writing of the auxiliary information
(to the {\tt .lot} and {\tt .lof} files).  They are:
\begin{quote}\tt\singlespace\begin{verbatim}
\noappendixtables                % Don't have appendix tables
\noappendixfigures               % Don't have appendix figures
\end{verbatim}\end{quote}
These commands should be in the preamble.  See Section~\ref{usage:noapp}.

There are two environments for doing the appendix chapter: {\tt appendix} and
\break {\tt appendices}.  If you have only one chapter in the appendix, use the {\tt appendix}
environment.  If you have more than one chapter, like this manual, use the
{\tt appendices} environment.
\begin{quote}\tt\singlespace\footnotesize\begin{verbatim}
\begin{appendices}  % Start of the Appendix Chapters.  If there is only
                    % one Appendix Chapter, then use \begin{appendix}
\include{code}      % Including computer code listings
\include{bibref}    % a BibTeX reference
\include{math}      % Complex Equations from the UW Math Department

\end{appendices}    % End of the Appendix Chapters. ibid on \end{appendix}
\end{verbatim}\end{quote}
The difference between these two environments is the way that the chapter header is
created and how this is listed in the table of contents.
          % Chapter 5 Strongly based on similar by J.D. McCauley
\bibliography{refs}      % Make the bibliography
\begin{appendices}       % Start of the Appendix Chapters.  If there is only
                         % one Appendix Chapter, then use \begin{appendix}
% code.tex
% this file is part of the example UW-Madison Thesis document
% It demonstrates one method for incorporating program listings
% into a document.

\chapter{Matlab Code} \label{matlab}
This is an example of a Matlab m-file.
\verbatimfile{derivs.m}
         % Including computer code listings
\chapter{Bib\TeX\ Entries}
\label{bibrefs}
The following shows the fields required in all types of Bib\TeX\ entries.
Fields with {\tt OPT} prefixed are optional (the three letters {\tt OPT} should 
not be used).  If an optional field is not used, then the entire field can be deleted.

{\tt
\singlespace
\begin{verbatim}

@Unpublished{,                            @Manual{,
  author =      "",                         title =           "",
  title =       "",                         OPTauthor =       "",
  note =        "",                         OPTorganization = "",
  OPTyear =     "",                         OPTaddress =      "",
  OPTmonth =    ""                          OPTedition =      "",
}                                           OPTyear =         "",
                                            OPTmonth =        "",
@TechReport{,                               OPTnote =         "" 
  author =      "",                       }
  title =       "",
  institution = "",                       @InProceedings{,
  year =        "",                         author =          "",
  OPTtype =     "",                         title =           "",
  OPTnumber =   "",                         booktitle =       "",
  OPTaddress =  "",                         year =            "",
  OPTmonth =    "",                         OPTeditor =       "",
  OPTnote =     ""                          OPTpages =        "",
}                                           OPTorganization = "",
                                            OPTpublisher =    "",
@Proceedings{,                              OPTaddress =      "",
  title =           "",                     OPTmonth =        "",
  year =            "",                     OPTnote =         "" 
  OPTeditor =       "",                   }
  OPTpublisher =    "",
  OPTorganization = "",
  OPTaddress =      "",
  OPTmonth =        "",
  OPTnote =         "" 
}



@PhDThesis{,                              @InCollection{,
  author =      "",                         author =          "",
  title =       "",                         title =           "",
  school =      "",                         booktitle =       "",
  year =        "",                         publisher =       "",
  OPTaddress =  "",                         year =            "",
  OPTmonth =    "",                         OPTeditor =       "",
  OPTnote =     ""                          OPTchapter =      "",
}                                           OPTpages =        "",
                                            OPTaddress =      "",
                                            OPTmonth =        "",
                                            OPTnote =         ""
                                          }

 
@Misc{,                                   @InCollection{,
  OPTauthor =       "",                     author =          "",
  OPTtitle =        "",                     title =           "",
  OPThowpublished = "",                     chapter =         "",
  OPTyear =         "",                     publisher =       "",
  OPTmonth =        "",                     year =            "",
  OPTnote =         ""                      OPTeditor =       "",
}                                           OPTpages =        "",
}                                           OPTvolume =       "",
                                            OPTseries =       "",
                                            OPTaddress =      "",
                                            OPTedition =      "",
                                            OPTmonth =        "",
                                            OPTnote =         ""
                                          }

@MastersThesis{,                          @Article{,
  author =      "",                         author =          "",
  title =       "",                         title =           "",
  school =      "",                         journal =         "",
  year =        "",                         year =            "",
  OPTaddress =  "",                         OPTvolume =       "",
  OPTmonth =    "",                         OPTnumber =       "",
  OPTnote =     ""                          OPTpages =        "",
}                                           OPTmonth =        "",
                                            OPTnote =         ""
                                           }\end{verbatim} }
         % a BibTeX reference
\chapter{Mathematics Examples}
This appendix provides an example of \LaTeX's typesetting
capabilities.  Most of text was obtained from the University of
Wisconsin-Madison Math Department's example thesis file.

\section{Matrices}
The equations for the {\em dq}-model of an induction machine in the
synchronous reference frame are
\begin{eqnarray}
 \left[\begin{array}{c} v_{qs}^e\\v_{ds}^e\\v_{qr}^e\\v_{dr}^e  \end{array}\right]                                                                                                                                                                                                                                                                                                                                                                                                                                                                                                              
 &=& \left[ \begin{array}{cccc}
 r_s + x_s\frac{\rho}{\omega_b} & \frac{\omega_e}{\omega_b}x_s & x_m\frac{\rho}{\omega_b} & \frac{\omega_e}{\omega_b}x_m \\
 -\frac{\omega_e}{\omega_b}x_s & r_s + x_s\frac{\rho}{\omega_b} & -\frac{\omega_e}{\omega_b}x_m & x_m\frac{\rho}{\omega_b} \\
 x_m\frac{\rho}{\omega_b} & \frac{\omega_e -\omega_r}{\omega_b}x_m & r_r'+x_r'\frac{\rho}{\omega_b} & \frac{\omega_e - \omega_r}{\omega_b}x_r' \\
 -\frac{\omega_e -\omega_r}{\omega_b}x_m & x_m\frac{\rho}{\omega_b} & -\frac{\omega_e - \omega_r}{\omega_b}x_r' & r_r' + x_r'\frac{\rho}{\omega_b}
 \end{array} \right]
 \left[\begin{array}{c} i_{qs}^e\\i_{ds}^e\\i_{qr}^e\\i_{dr}^e\end{array} \right] \label{volteq}\\
 T_e&=&\frac{3}{2}\frac{P}{2}\frac{x_m}{\omega_b}\left(i_{qs}^ei_{dr}^e - i_{ds}^ei_{qr}^e\right) \label{torqueeq}\\
 T_e-T_l&=&\frac{2J\omega_b}{P}\frac{d}{dt}\left(\frac{\omega_r}{\omega_b}\right) \label{mecheq}.
\end{eqnarray}

\section{Multi-line Equations}

\LaTeX{} has a built-in equation array feature, however the
equation numbers must be on the same line as an equation.  For example:
\begin{eqnarray}
\Delta u + \lambda e^u &= 0&u\in \Omega,  \nonumber \\
u&=0&u\in\partial\Omega.
\end{eqnarray}

Alternatively, the number can be centered in the equation using the
following method.
%
% The equation-array feature in LaTeX is a bad idea.  For centered
% numbers you should set your own equations and arrays as follows:
%
\def\dd{\displaystyle}
\begin{equation}\label{gelfand}
\begin{array}{rl}
\dd \Delta u + \lambda e^u = 0, &
\dd u\in \Omega,\\[8pt] % add 8pt extra vertical space. 1 line=23pt
\dd u=0, & \dd u\in\partial\Omega.
\end{array}
\end{equation}
The previous equation had a label.  It may be referenced as
equation~(\ref{gelfand}).


\section{More Complicated Equations}
\section*{Rellich's identity}\label{rellich.section}
\setcounter{theorem}{0}
%
%

Standard developments of Pohozaev's identity used an identity by
Rellich~\cite{rellich:der40}, reproduced here.

\begin{lemma}[Rellich]
Given $L$ in divergence form and $a,d$ defined above, $u\in C^2
(\Omega )$, we have
\begin{equation}\label{rellich}
\int_{\Omega}(-Lu)\nabla u\cdot (x-\overline{x})\, dx
= (1-\frac{n}{2}) \int_{\Omega} a(\nabla u,\nabla u) \, dx
-
\frac{1}{2} \int_{\Omega}
d(\nabla u, \nabla u) \, dx
\end{equation}
$$
+
\frac{1}{2} \int_{\partial\Omega} a(\nabla u,\nabla u)(x-\overline{x})
\cdot \nu  \, dS
-
\int_{\partial\Omega}
a(\nabla u,\nu )\nabla u\cdot (x-\overline{x}) \, dS.
$$
\end{lemma}
{\bf Proof:}\\
There is no loss in generality to take $\overline{x} = 0$. First
rewrite $L$:
$$Lu = \frac{1}{2}\left[ \sum_{i}\sum_{j}
\frac{\partial}{\partial x_i}
\left( a_{ij} \frac{\partial u}{\partial x_j} \right) +
\sum_{i}\sum_{j}
\frac{\partial}{\partial x_i}
\left( a_{ij} \frac{\partial u}{\partial x_j} \right)
\right]$$
Switching the order of summation on the second term and relabeling
subscripts, $j \rightarrow i$ and $i \rightarrow j$, then using the fact
that $a_{ij}(x)$ is a symmetric matrix,
gives the symmetric form needed to derive Rellich's identity.
\begin{equation}
Lu = \frac{1}{2} \sum_{i,j}\left[
\frac{\partial}{\partial x_i}
\left( a_{ij} \frac{\partial u}{\partial x_j} \right) +
\frac{\partial}{\partial x_j}
\left( a_{ij} \frac{\partial u}{\partial x_i} \right)
\right].
\end{equation}

Multiplying $-Lu$ by $\frac{\partial u}{\partial x_k} x_k$ and integrating
over $\Omega$, yields
$$\int_{\Omega}(-Lu)\frac{\partial u}{\partial x_k} x_k \, dx=
-\frac{1}{2} \int_{\Omega}
\sum_{i,j}\left[
\frac{\partial}{\partial x_i}
\left( a_{ij} \frac{\partial u}{\partial x_j} \right) +
\frac{\partial}{\partial x_j}
\left( a_{ij} \frac{\partial u}{\partial x_i} \right)
\right]
\frac{\partial u}{\partial x_k} x_k \, dx$$
Integrating by parts (for integral theorems see~\cite[p. 20]
{zeidler:nfa88IIa})
gives
$$= \frac{1}{2} \int_{\Omega}
\sum_{i,j} a_{ij} \left[
\frac{\partial u}{\partial x_j}
\frac{\partial^2 u}{\partial x_k\partial x_i} +
\frac{\partial u}{\partial x_i}
\frac{\partial^2 u}{\partial x_k\partial x_j}
\right] x_k \, dx
$$
$$
+
\frac{1}{2} \int_{\Omega}
\sum_{i,j} a_{ij} \left[
\frac{\partial u}{\partial x_j} \delta_{ik} +
\frac{\partial u}{\partial x_i} \delta_{jk}
\right] \frac{\partial u}{\partial x_k} \, dx
$$
$$- \frac{1}{2} \int_{\partial\Omega}
\sum_{i,j} a_{ij} \left[
\frac{\partial u}{\partial x_j} \nu_{i} +
\frac{\partial u}{\partial x_i} \nu_{j}
\right] \frac{\partial u}{\partial x_k} x_k \, dx
$$
= $I_1 + I_2 + I_3$, where the unit normal vector is $\nu$.
One may rewrite $I_1$ as
$$I_1 = \frac{1}{2} \int_{\Omega}
\sum_{i,j} a_{ij} \frac{\partial}{\partial x_k}\left(
\frac{\partial u}{\partial x_i}
\frac{\partial u}{\partial x_j}
\right) x_k \, dx
$$
Integrating the first term by parts again yields
$$I_1 = -\frac{1}{2} \int_{\Omega}
\sum_{i,j} a_{ij} \left(
\frac{\partial u}{\partial x_i}
\frac{\partial u}{\partial x_j}
\right) \, dx
+
\frac{1}{2} \int_{\partial\Omega}
\sum_{i,j} a_{ij} \left(
\frac{\partial u}{\partial x_i}
\frac{\partial u}{\partial x_j}
\right) x_k \nu_k \, dS
$$
$$
-
\frac{1}{2} \int_{\Omega}
\sum_{i,j} \left(
\frac{\partial u}{\partial x_i}
\frac{\partial u}{\partial x_j}
\right) x_k \frac{\partial a_{ij}}{\partial x_k}\, dx.
$$
Summing over $k$ gives
$$\int_{\Omega}(-Lu)(\nabla u\cdot x)\, dx =
-\frac{n}{2} \int_{\Omega}
\sum_{i,j} a_{ij} \left(
\frac{\partial u}{\partial x_i}
\frac{\partial u}{\partial x_j}
\right) \, dx
$$
$$
+
\frac{1}{2} \int_{\partial\Omega}
\sum_{i,j} a_{ij} \left(
\frac{\partial u}{\partial x_i}
\frac{\partial u}{\partial x_j}
\right) (x\cdot \nu ) \, dS
-
\frac{1}{2} \int_{\Omega}
\sum_{i,j} \left(
\frac{\partial u}{\partial x_i}
\frac{\partial u}{\partial x_j}
\right) (x\cdot  \nabla a_{ij}) \, dx
$$
$$
+
\frac{1}{2} \int_{\Omega}
\sum_{i,j,k} a_{ij} \left[
\frac{\partial u}{\partial x_j}
\frac{\partial u}{\partial x_k} \delta_{ik} +
\frac{\partial u}{\partial x_i}
\frac{\partial u}{\partial x_k} \delta_{jk}
\right] \, dx
$$
$$- \frac{1}{2} \int_{\partial\Omega}
\sum_{i,j} a_{ij} \left[
\frac{\partial u}{\partial x_j} \nu_{i} +
\frac{\partial u}{\partial x_i} \nu_{j}
\right] (\nabla u\cdot x) \, dS.
$$
Combining the first and fourth term on the right-hand side
simplifies the expression
$$\int_{\Omega}(-Lu)(\nabla u\cdot x)\, dx
=
(1-\frac{n}{2}) \int_{\Omega}
\sum_{i,j} a_{ij} \left(
\frac{\partial u}{\partial x_i}
\frac{\partial u}{\partial x_j}
\right) \, dx
$$
$$
+
\frac{1}{2} \int_{\partial\Omega}
\sum_{i,j} a_{ij} \left(
\frac{\partial u}{\partial x_i}
\frac{\partial u}{\partial x_j}
\right) (x\cdot \nu ) \, dS
-
\frac{1}{2} \int_{\Omega}
\sum_{i,j} \left(
\frac{\partial u}{\partial x_i}
\frac{\partial u}{\partial x_j}
\right) (x\cdot  \nabla a_{ij}) \, dx
$$
$$
-
\frac{1}{2} \int_{\partial\Omega}
\sum_{i,j} a_{ij} \left[
\frac{\partial u}{\partial x_j} \nu_{i} +
\frac{\partial u}{\partial x_i} \nu_{j}
\right] (\nabla u\cdot x) \, dS.
$$
Using the notation defined above, the result follows.


           % Complex Equations from the UW Math Department
% acrobat.tex
%
% This file explains how to generate Adobe Acrobat files
%
% Eric Benedict, July 2000
%
% It is provided without warranty on an AS IS basis.

\newcommand{\pdf}{\mbox{\tt *.pdf}}

\chapter{Adobe Acrobat (\pdf ) Files}
The Adobe Acrobat file format has pretty much become the {\em de facto}
standard for document sharing.  As such, some faculty members and/or
departments may be requiring a final copy of the thesis in Acrobat format
(\pdf ).

There are several different methods of obtaining a \pdf\ file from a \LaTeX{}
thesis; however, they are all very site specific.  A couple of different
methods which have been found to work are mentioned as suggested ideas to try
as a starting point.  Depending on what is installed at your site/location some
of these may be applicable.

\section{Converting from {\tt *.ps} to \pdf}
One option to obtain the \pdf\ file would be to generate the thesis in a normal
manner and then use the Acrobat\ {\tt Distiller}\ to convert the postscript file into a
\pdf\ file.

If the\ {\tt Distiller}\ program is available and convenient to use, then this is quite easy to do.

Depending on the choice of document fonts, the results may not be satisfactory since
some of the fonts may end up as bit-mapped fonts and will display poorly at any resolution
other than what they were sampled on.  Also, since the\ {\tt Distiller}\ program is an expensive
program to obtain, it is not always available.

An alternative to the Adobe\ {\tt Distiller}\ program is the Alladin\ {\tt Ghostscript}\ program.  This is
available for free from

{\tt \verb|   http://www.cs.wisc.edu/~ghost/index.html|}

This program is available for most common operating systems as a compiled binary, but the source code
is available for other systems.  One drawback is that this conversion must be performed as a
command line invocation and isn't very user friendly.  This may be addressed in a future version of\
{\tt Ghostview}, the program which provides a nice user interface to\ {\tt Ghostscript}.

\section{Converting from {\tt *.dvi} to \pdf}
There are two programs available which will convert from {\tt *.dvi} to \pdf,\ {\tt dvipdf}\ and\
{\tt dvipdfm}.  The\ {\tt dvipdfm}\ program  will be discussed here.  In version 0.12, it can generate
bookmarks, thumbnails (with assistance from\ {\tt Ghostscript}), scaling and rotation, JPEG and
PNG bitmaps and font encoding and re-encoding (to support fonts which aren't fully supported by
the Acrobat suite).  When\ {\tt Ghostscript}\ is properly installed,\ {\tt dvipdfm}\ will automatically
convert any encapsulated PostScript figures into the required \pdf{} format.  This program behaved in a
similar manner to the {\tt dvips} program and was used to produce the \pdf{} format of this document.



\section{Generating \pdf{} Initially}
There are now some programs which are similar to \TeX{} but instead of producing a\ {\tt .dvi}\ output,
they produce \pdf as a native output.  One such program, {\sc pdf}\TeX{} / {\sc pdf}\LaTeX{},
is available from

{\tt \verb|   http://www.tug.org/applications/pdftex|}

Note that as of this date, July 2000, {\sc pdf}\TeX{} / {\sc pdf}\LaTeX{} while currently quite usable, it
is still in a beta version.  Look at the web site for more current information.

The present version was able to produce a \pdf{} file of this document without any required
changes, except for the Postscript figure inclusion  (Figure~\ref{vwcontr}).  To properly include
this figure, requires the conversion of the postscript figure into a \pdf{} figure.  The procedure
is described in the manual for {\sc pdf}\TeX{} / {\sc pdf}\LaTeX{}.  Note that the figure conversion will
require either\ {\tt Distiller}\ or\ {\tt Ghostscript}.
           % A discussion on generating PDF files.
\end{appendices}         % End of the Appendix Chapters.  ibid on \end{appendix}
%\include{vita}          % Optional Vita, use \begin{vita} vita text \end{vita}
\end{document}
\end{verbatim}
\end{quote}

\section{Prelude}
After the {\tt \verb|\begin{document}|} comes the preliminary information found in
theses.  In this manual, the information is kept in the file {\tt prelude.tex} (see
above).  These pages will need to be numbered with roman numerals, so use
\begin{quote}\tt\singlespace\begin{verbatim}
\clearpage\pagenumbering{roman}
\end{verbatim}\end{quote}

Next, comes your thesis or dissertation title, your name, date of graduation, department
and degree.
\begin{quote}\tt\singlespace\begin{verbatim}
\title{How to \LaTeX\ a Thesis}
\author{Eric R. Benedict}
\date{2000}
%   - The default degree is ``Doctor of Philosophy''
%     Degree can be changed using the command \degree{}
%\degree{New Degree}
%   - for a PhD dissertation (default), specify \dissertation
%\dissertation
%   - for a masters project report, specify \project
%\project
%   - for a preliminary report, specify \prelim
%\prelim
%   - for a masters thesis, specify \thesis
%\thesis
%   - The default department is ``Electrical Engineering''
%     The department can be changed using the command \department{}
%\department{New Department}
\end{verbatim}\end{quote}

If you specified the class option {\tt msthesis}, then the degree is changed to
{\em Master \break of Science} and the {\tt \verb|\thesis|} option is specified.  If you
want to have the masters margins with another document, then the {\tt \verb|\degree|}
and {\tt \verb|\dissertation|},  {\tt \verb|\project|}, {\em etc.\/} can be specified
as needed.

Once the
above are all defined, use  {\tt \verb|\maketitle|} to generate the title page.
\begin{quote}\tt\singlespace\begin{verbatim}
\maketitle
\end{verbatim}\end{quote}

If you wish to include a copyright page (see Section~\ref{copyright} for
information on registering the copyright.), then add the command
\begin{quote}\tt\singlespace\begin{verbatim}
\copyrightpage
\end{verbatim}\end{quote}
This will generate the proper copyright page and will use the name and date specified
in {\tt \verb|\author{}|} and {\tt \verb|\date{}|}.

Next are the dedications and acknowledgements:
\begin{quote}\tt\singlespace\begin{verbatim}
\begin{dedication}
To my pet rock, Skippy.
\end{dedication}

\begin{acknowledgments}
I thank the many people who have done lots of nice things for me.
\end{acknowledgments}
\end{verbatim}\end{quote}

You must tell \LaTeX{} to generate a table of contents, a list of tables and a list of
figures:
\begin{quote}\tt\singlespace\begin{verbatim}
\tableofcontents
\listoftables
\listoffigures
\end{verbatim}\end{quote}

If you wish to have a nomenclature, list of symbols or glossary it can go here.
\begin{quote}\tt\singlespace\begin{verbatim}
\begin{nomenclature}
%\begin{listofsymbols}
%\begin{glossary}
\begin{tabular}{ll}
$C_1$ & Constant 1\\
\ldots
\end{tabular}
%\end{glossary}
%\end{listofsymbols}
\end{nomenclature}
\end{verbatim}\end{quote}

If your abstract will be microfilmed by Bell and Howell (formerly UMI), then you
will need to generate an abstract of less than 350 words.  This abstract can be created
using the {\tt umiabstract} environment.  This environment requires that you define your
advisor and your advisor's title using {\tt \verb|\advisorname{}|} and
{\tt \verb|\advisortitle{}|}.
\begin{quote}\tt\singlespace\begin{verbatim}
\advisorname{Bucky J. Badger}
\advisortitle{Assistant Professor}
% ABSTRACT
\begin{umiabstract}
\noindent       % Don't indent first paragraph.
This explains the basics for using \LaTeX\ to typeset a
dissertation, thesis or project report for the University of
Wisconsin-Madison.

...

\end{umiabstract}
\end{verbatim}\end{quote}
This will place your name, title and required text at the top of the page and follow the
abstract text with your advisor's name at the bottom for your advisor's signature.  This
page is not numbered and would be submitted separately.

If you will have an abstract as part of your document, then the {\tt abstract} environment
should be used.
\begin{quote}\tt\singlespace\begin{verbatim}
\begin{abstract}
\noindent       % Don't indent first paragraph.
This explains the basics for using \LaTeX\ to typeset a
dissertation, thesis or project report for the University of
Wisconsin-Madison.

...

\end{abstract}
\end{verbatim}\end{quote}
This will generate a page number and it will be included in the Table
of Contents.  

If you will have both the UMI and regular abstracts like this document, then
you will probably want to write the abstract once and save it in a seperate
file such as {\tt abstract.tex}.  Then, you can use the same abstract for
both purposes.

\begin{quote}\begin{verbatim}
\begin{umiabstract}
  % abstract.tex
%
% This file has the abstract for the withesis style documentation
%
% Eric Benedict, Aug 2000
%
% It is provided without warranty on an AS IS basis.

\noindent       % Don't indent this paragraph.
The Human Genome Project (HGP), completed in 2003, is considered one of the greatest accomplishments of exploration in history of science. Since then thousands of genomes have been sequenced. However, no individual human genome has been annotated to completion. Nanocoding \cite{Jo_etal_2007_PNAS} (PNAS, 2007), developed by Laboratory of Molecular and Computational Genomics (LMCG), UW Madison , is a novel system for physically mapping genomes, using measurements of single DNA molecules to construct a high-resolution genome-wide restriction map, whose representation of genome structure complements genome sequences to yield biological insight. Staining the DNA molecules with cyanine dyes and imaging them is a critical step of nanocoding. It turns out that the quantum yield of the fluorescence intensity of these stained molecules are sequence dependent. In fact, for YO complexed with GC-rich DNA sequences the quantum yield are about twice as large as for YO complexed with AT-rich sequences. Hence, regions with distinct sequence compositions should exhibit unique fluorescence intensity profiles. Establishing the fluorescence intensity profiles of a genome would provide invaluable insights into its sequence compositions without having to sequence it. We name this technique ``Fluoroscanning''. Imaged DNA molecules from the same region on a genome should exhibit similar intensity profiles, unless there has been a modification in the underlying genomic sequence. Fluoroscanning can be used to identify heterozygotes and detect large scale structural variations as a result of cancer or other diseases.  

%\vspace*{0.5em}
%\noindent       % Don't indent this paragraph.


\end{umiabstract}

\begin{abstract}
  % abstract.tex
%
% This file has the abstract for the withesis style documentation
%
% Eric Benedict, Aug 2000
%
% It is provided without warranty on an AS IS basis.

\noindent       % Don't indent this paragraph.
The Human Genome Project (HGP), completed in 2003, is considered one of the greatest accomplishments of exploration in history of science. Since then thousands of genomes have been sequenced. However, no individual human genome has been annotated to completion. Nanocoding \cite{Jo_etal_2007_PNAS} (PNAS, 2007), developed by Laboratory of Molecular and Computational Genomics (LMCG), UW Madison , is a novel system for physically mapping genomes, using measurements of single DNA molecules to construct a high-resolution genome-wide restriction map, whose representation of genome structure complements genome sequences to yield biological insight. Staining the DNA molecules with cyanine dyes and imaging them is a critical step of nanocoding. It turns out that the quantum yield of the fluorescence intensity of these stained molecules are sequence dependent. In fact, for YO complexed with GC-rich DNA sequences the quantum yield are about twice as large as for YO complexed with AT-rich sequences. Hence, regions with distinct sequence compositions should exhibit unique fluorescence intensity profiles. Establishing the fluorescence intensity profiles of a genome would provide invaluable insights into its sequence compositions without having to sequence it. We name this technique ``Fluoroscanning''. Imaged DNA molecules from the same region on a genome should exhibit similar intensity profiles, unless there has been a modification in the underlying genomic sequence. Fluoroscanning can be used to identify heterozygotes and detect large scale structural variations as a result of cancer or other diseases.  

%\vspace*{0.5em}
%\noindent       % Don't indent this paragraph.


\end{abstract}
\end{verbatim}\end{quote}

Finally, the page numbers must be changed to arabic numbers to conclude the preliminary
portion of the document.
\begin{quote}\tt\singlespace\begin{verbatim}
\clearpage\pagenumbering{arabic}
\end{verbatim}\end{quote}

\section{The Body}
At the beginning of {\tt intro.tex} there is the following command:
\begin{quote}\tt\singlespace\begin{verbatim}
\chapter{Introducing the {\tt withesis} \LaTeX{} Style Guide}
\end{verbatim}\end{quote}
Following that is the text of the chapter.  The body of your thesis is separated by
sectioning commands like {\tt \verb|\chapter{}|}.  For more information on the sectioning
commands, see Section~\ref{ess:sectioning}.

Remember the basic rule of outlining you learned in grammar school:
\begin{quote}
You cannot have an `A' if you do not have a `B'
\end{quote}
Take care to have at least two {\tt \verb|\section|}s if you use the command; have
two {\tt \verb|\subsection|}s, {\em etc}.



\section{Additional Theorem Like Environments}
The {\tt withesis} style adds numerous additional theorem like environments.  These
environments were included to allow compatibility with the University of Wisconsin's
Math Department's style file.  These environments are
{\tt theorem}, {\tt assertion}, {\tt claim}, {\tt conjecture}, {\tt corollary},
{\tt definition}, {\tt example}, {\tt figger}, {\tt lemma}, {\tt prop} and {\tt remark}.

As an example, consider the following.
\begin{lemma}
Assuming that $\partial\Omega_2 = \emptyset$ and that $h(t) = 1$, we
have $$
\begin{array}{lr}
\Delta u = f, &  x\in\Omega ,\\[2pt]
u =  g_1, &  x\in\partial\Omega .
\end{array}
$$
\end{lemma}
which was produced with the following:
\begin{quote}\tt\singlespace\begin{verbatim}
\begin{lemma}
Assuming that $\partial\Omega_2 = \emptyset$ and that $h(t) = 1$, we
have $$
\begin{array}{lr}
\Delta u = f, &  x\in\Omega ,\\[2pt]
 u =  g_1, &  x\in\partial\Omega .
\end{array}
$$
\end{lemma}
\end{verbatim}\end{quote}

\section{Bibliography or References}
As a final note, the default title for the references chapter is ``LIST OF REFERENCES.''
Since some people may prefer ``BIBLIOGRAPHY'', the command
\break{\tt \verb|\altbibtitle|}
has been added to change the chapter title.

\section{Appendices}
There are two commands which are available to suppress the writing of the auxiliary information
(to the {\tt .lot} and {\tt .lof} files).  They are:
\begin{quote}\tt\singlespace\begin{verbatim}
\noappendixtables                % Don't have appendix tables
\noappendixfigures               % Don't have appendix figures
\end{verbatim}\end{quote}
These commands should be in the preamble.  See Section~\ref{usage:noapp}.

There are two environments for doing the appendix chapter: {\tt appendix} and
\break {\tt appendices}.  If you have only one chapter in the appendix, use the {\tt appendix}
environment.  If you have more than one chapter, like this manual, use the
{\tt appendices} environment.
\begin{quote}\tt\singlespace\footnotesize\begin{verbatim}
\begin{appendices}  % Start of the Appendix Chapters.  If there is only
                    % one Appendix Chapter, then use \begin{appendix}
% code.tex
% this file is part of the example UW-Madison Thesis document
% It demonstrates one method for incorporating program listings
% into a document.

\chapter{Matlab Code} \label{matlab}
This is an example of a Matlab m-file.
\verbatimfile{derivs.m}
      % Including computer code listings
\chapter{Bib\TeX\ Entries}
\label{bibrefs}
The following shows the fields required in all types of Bib\TeX\ entries.
Fields with {\tt OPT} prefixed are optional (the three letters {\tt OPT} should 
not be used).  If an optional field is not used, then the entire field can be deleted.

{\tt
\singlespace
\begin{verbatim}

@Unpublished{,                            @Manual{,
  author =      "",                         title =           "",
  title =       "",                         OPTauthor =       "",
  note =        "",                         OPTorganization = "",
  OPTyear =     "",                         OPTaddress =      "",
  OPTmonth =    ""                          OPTedition =      "",
}                                           OPTyear =         "",
                                            OPTmonth =        "",
@TechReport{,                               OPTnote =         "" 
  author =      "",                       }
  title =       "",
  institution = "",                       @InProceedings{,
  year =        "",                         author =          "",
  OPTtype =     "",                         title =           "",
  OPTnumber =   "",                         booktitle =       "",
  OPTaddress =  "",                         year =            "",
  OPTmonth =    "",                         OPTeditor =       "",
  OPTnote =     ""                          OPTpages =        "",
}                                           OPTorganization = "",
                                            OPTpublisher =    "",
@Proceedings{,                              OPTaddress =      "",
  title =           "",                     OPTmonth =        "",
  year =            "",                     OPTnote =         "" 
  OPTeditor =       "",                   }
  OPTpublisher =    "",
  OPTorganization = "",
  OPTaddress =      "",
  OPTmonth =        "",
  OPTnote =         "" 
}



@PhDThesis{,                              @InCollection{,
  author =      "",                         author =          "",
  title =       "",                         title =           "",
  school =      "",                         booktitle =       "",
  year =        "",                         publisher =       "",
  OPTaddress =  "",                         year =            "",
  OPTmonth =    "",                         OPTeditor =       "",
  OPTnote =     ""                          OPTchapter =      "",
}                                           OPTpages =        "",
                                            OPTaddress =      "",
                                            OPTmonth =        "",
                                            OPTnote =         ""
                                          }

 
@Misc{,                                   @InCollection{,
  OPTauthor =       "",                     author =          "",
  OPTtitle =        "",                     title =           "",
  OPThowpublished = "",                     chapter =         "",
  OPTyear =         "",                     publisher =       "",
  OPTmonth =        "",                     year =            "",
  OPTnote =         ""                      OPTeditor =       "",
}                                           OPTpages =        "",
}                                           OPTvolume =       "",
                                            OPTseries =       "",
                                            OPTaddress =      "",
                                            OPTedition =      "",
                                            OPTmonth =        "",
                                            OPTnote =         ""
                                          }

@MastersThesis{,                          @Article{,
  author =      "",                         author =          "",
  title =       "",                         title =           "",
  school =      "",                         journal =         "",
  year =        "",                         year =            "",
  OPTaddress =  "",                         OPTvolume =       "",
  OPTmonth =    "",                         OPTnumber =       "",
  OPTnote =     ""                          OPTpages =        "",
}                                           OPTmonth =        "",
                                            OPTnote =         ""
                                           }\end{verbatim} }
    % a BibTeX reference
\chapter{Mathematics Examples}
This appendix provides an example of \LaTeX's typesetting
capabilities.  Most of text was obtained from the University of
Wisconsin-Madison Math Department's example thesis file.

\section{Matrices}
The equations for the {\em dq}-model of an induction machine in the
synchronous reference frame are
\begin{eqnarray}
 \left[\begin{array}{c} v_{qs}^e\\v_{ds}^e\\v_{qr}^e\\v_{dr}^e  \end{array}\right]                                                                                                                                                                                                                                                                                                                                                                                                                                                                                                              
 &=& \left[ \begin{array}{cccc}
 r_s + x_s\frac{\rho}{\omega_b} & \frac{\omega_e}{\omega_b}x_s & x_m\frac{\rho}{\omega_b} & \frac{\omega_e}{\omega_b}x_m \\
 -\frac{\omega_e}{\omega_b}x_s & r_s + x_s\frac{\rho}{\omega_b} & -\frac{\omega_e}{\omega_b}x_m & x_m\frac{\rho}{\omega_b} \\
 x_m\frac{\rho}{\omega_b} & \frac{\omega_e -\omega_r}{\omega_b}x_m & r_r'+x_r'\frac{\rho}{\omega_b} & \frac{\omega_e - \omega_r}{\omega_b}x_r' \\
 -\frac{\omega_e -\omega_r}{\omega_b}x_m & x_m\frac{\rho}{\omega_b} & -\frac{\omega_e - \omega_r}{\omega_b}x_r' & r_r' + x_r'\frac{\rho}{\omega_b}
 \end{array} \right]
 \left[\begin{array}{c} i_{qs}^e\\i_{ds}^e\\i_{qr}^e\\i_{dr}^e\end{array} \right] \label{volteq}\\
 T_e&=&\frac{3}{2}\frac{P}{2}\frac{x_m}{\omega_b}\left(i_{qs}^ei_{dr}^e - i_{ds}^ei_{qr}^e\right) \label{torqueeq}\\
 T_e-T_l&=&\frac{2J\omega_b}{P}\frac{d}{dt}\left(\frac{\omega_r}{\omega_b}\right) \label{mecheq}.
\end{eqnarray}

\section{Multi-line Equations}

\LaTeX{} has a built-in equation array feature, however the
equation numbers must be on the same line as an equation.  For example:
\begin{eqnarray}
\Delta u + \lambda e^u &= 0&u\in \Omega,  \nonumber \\
u&=0&u\in\partial\Omega.
\end{eqnarray}

Alternatively, the number can be centered in the equation using the
following method.
%
% The equation-array feature in LaTeX is a bad idea.  For centered
% numbers you should set your own equations and arrays as follows:
%
\def\dd{\displaystyle}
\begin{equation}\label{gelfand}
\begin{array}{rl}
\dd \Delta u + \lambda e^u = 0, &
\dd u\in \Omega,\\[8pt] % add 8pt extra vertical space. 1 line=23pt
\dd u=0, & \dd u\in\partial\Omega.
\end{array}
\end{equation}
The previous equation had a label.  It may be referenced as
equation~(\ref{gelfand}).


\section{More Complicated Equations}
\section*{Rellich's identity}\label{rellich.section}
\setcounter{theorem}{0}
%
%

Standard developments of Pohozaev's identity used an identity by
Rellich~\cite{rellich:der40}, reproduced here.

\begin{lemma}[Rellich]
Given $L$ in divergence form and $a,d$ defined above, $u\in C^2
(\Omega )$, we have
\begin{equation}\label{rellich}
\int_{\Omega}(-Lu)\nabla u\cdot (x-\overline{x})\, dx
= (1-\frac{n}{2}) \int_{\Omega} a(\nabla u,\nabla u) \, dx
-
\frac{1}{2} \int_{\Omega}
d(\nabla u, \nabla u) \, dx
\end{equation}
$$
+
\frac{1}{2} \int_{\partial\Omega} a(\nabla u,\nabla u)(x-\overline{x})
\cdot \nu  \, dS
-
\int_{\partial\Omega}
a(\nabla u,\nu )\nabla u\cdot (x-\overline{x}) \, dS.
$$
\end{lemma}
{\bf Proof:}\\
There is no loss in generality to take $\overline{x} = 0$. First
rewrite $L$:
$$Lu = \frac{1}{2}\left[ \sum_{i}\sum_{j}
\frac{\partial}{\partial x_i}
\left( a_{ij} \frac{\partial u}{\partial x_j} \right) +
\sum_{i}\sum_{j}
\frac{\partial}{\partial x_i}
\left( a_{ij} \frac{\partial u}{\partial x_j} \right)
\right]$$
Switching the order of summation on the second term and relabeling
subscripts, $j \rightarrow i$ and $i \rightarrow j$, then using the fact
that $a_{ij}(x)$ is a symmetric matrix,
gives the symmetric form needed to derive Rellich's identity.
\begin{equation}
Lu = \frac{1}{2} \sum_{i,j}\left[
\frac{\partial}{\partial x_i}
\left( a_{ij} \frac{\partial u}{\partial x_j} \right) +
\frac{\partial}{\partial x_j}
\left( a_{ij} \frac{\partial u}{\partial x_i} \right)
\right].
\end{equation}

Multiplying $-Lu$ by $\frac{\partial u}{\partial x_k} x_k$ and integrating
over $\Omega$, yields
$$\int_{\Omega}(-Lu)\frac{\partial u}{\partial x_k} x_k \, dx=
-\frac{1}{2} \int_{\Omega}
\sum_{i,j}\left[
\frac{\partial}{\partial x_i}
\left( a_{ij} \frac{\partial u}{\partial x_j} \right) +
\frac{\partial}{\partial x_j}
\left( a_{ij} \frac{\partial u}{\partial x_i} \right)
\right]
\frac{\partial u}{\partial x_k} x_k \, dx$$
Integrating by parts (for integral theorems see~\cite[p. 20]
{zeidler:nfa88IIa})
gives
$$= \frac{1}{2} \int_{\Omega}
\sum_{i,j} a_{ij} \left[
\frac{\partial u}{\partial x_j}
\frac{\partial^2 u}{\partial x_k\partial x_i} +
\frac{\partial u}{\partial x_i}
\frac{\partial^2 u}{\partial x_k\partial x_j}
\right] x_k \, dx
$$
$$
+
\frac{1}{2} \int_{\Omega}
\sum_{i,j} a_{ij} \left[
\frac{\partial u}{\partial x_j} \delta_{ik} +
\frac{\partial u}{\partial x_i} \delta_{jk}
\right] \frac{\partial u}{\partial x_k} \, dx
$$
$$- \frac{1}{2} \int_{\partial\Omega}
\sum_{i,j} a_{ij} \left[
\frac{\partial u}{\partial x_j} \nu_{i} +
\frac{\partial u}{\partial x_i} \nu_{j}
\right] \frac{\partial u}{\partial x_k} x_k \, dx
$$
= $I_1 + I_2 + I_3$, where the unit normal vector is $\nu$.
One may rewrite $I_1$ as
$$I_1 = \frac{1}{2} \int_{\Omega}
\sum_{i,j} a_{ij} \frac{\partial}{\partial x_k}\left(
\frac{\partial u}{\partial x_i}
\frac{\partial u}{\partial x_j}
\right) x_k \, dx
$$
Integrating the first term by parts again yields
$$I_1 = -\frac{1}{2} \int_{\Omega}
\sum_{i,j} a_{ij} \left(
\frac{\partial u}{\partial x_i}
\frac{\partial u}{\partial x_j}
\right) \, dx
+
\frac{1}{2} \int_{\partial\Omega}
\sum_{i,j} a_{ij} \left(
\frac{\partial u}{\partial x_i}
\frac{\partial u}{\partial x_j}
\right) x_k \nu_k \, dS
$$
$$
-
\frac{1}{2} \int_{\Omega}
\sum_{i,j} \left(
\frac{\partial u}{\partial x_i}
\frac{\partial u}{\partial x_j}
\right) x_k \frac{\partial a_{ij}}{\partial x_k}\, dx.
$$
Summing over $k$ gives
$$\int_{\Omega}(-Lu)(\nabla u\cdot x)\, dx =
-\frac{n}{2} \int_{\Omega}
\sum_{i,j} a_{ij} \left(
\frac{\partial u}{\partial x_i}
\frac{\partial u}{\partial x_j}
\right) \, dx
$$
$$
+
\frac{1}{2} \int_{\partial\Omega}
\sum_{i,j} a_{ij} \left(
\frac{\partial u}{\partial x_i}
\frac{\partial u}{\partial x_j}
\right) (x\cdot \nu ) \, dS
-
\frac{1}{2} \int_{\Omega}
\sum_{i,j} \left(
\frac{\partial u}{\partial x_i}
\frac{\partial u}{\partial x_j}
\right) (x\cdot  \nabla a_{ij}) \, dx
$$
$$
+
\frac{1}{2} \int_{\Omega}
\sum_{i,j,k} a_{ij} \left[
\frac{\partial u}{\partial x_j}
\frac{\partial u}{\partial x_k} \delta_{ik} +
\frac{\partial u}{\partial x_i}
\frac{\partial u}{\partial x_k} \delta_{jk}
\right] \, dx
$$
$$- \frac{1}{2} \int_{\partial\Omega}
\sum_{i,j} a_{ij} \left[
\frac{\partial u}{\partial x_j} \nu_{i} +
\frac{\partial u}{\partial x_i} \nu_{j}
\right] (\nabla u\cdot x) \, dS.
$$
Combining the first and fourth term on the right-hand side
simplifies the expression
$$\int_{\Omega}(-Lu)(\nabla u\cdot x)\, dx
=
(1-\frac{n}{2}) \int_{\Omega}
\sum_{i,j} a_{ij} \left(
\frac{\partial u}{\partial x_i}
\frac{\partial u}{\partial x_j}
\right) \, dx
$$
$$
+
\frac{1}{2} \int_{\partial\Omega}
\sum_{i,j} a_{ij} \left(
\frac{\partial u}{\partial x_i}
\frac{\partial u}{\partial x_j}
\right) (x\cdot \nu ) \, dS
-
\frac{1}{2} \int_{\Omega}
\sum_{i,j} \left(
\frac{\partial u}{\partial x_i}
\frac{\partial u}{\partial x_j}
\right) (x\cdot  \nabla a_{ij}) \, dx
$$
$$
-
\frac{1}{2} \int_{\partial\Omega}
\sum_{i,j} a_{ij} \left[
\frac{\partial u}{\partial x_j} \nu_{i} +
\frac{\partial u}{\partial x_i} \nu_{j}
\right] (\nabla u\cdot x) \, dS.
$$
Using the notation defined above, the result follows.


      % Complex Equations from the UW Math Department

\end{appendices}    % End of the Appendix Chapters. ibid on \end{appendix}
\end{verbatim}\end{quote}
The difference between these two environments is the way that the chapter header is
created and how this is listed in the table of contents.
          % Chapter 5 Strongly based on similar by J.D. McCauley
\bibliography{refs}      % Make the bibliography
\begin{appendices}       % Start of the Appendix Chapters.  If there is only
                         % one Appendix Chapter, then use \begin{appendix}
% code.tex
% this file is part of the example UW-Madison Thesis document
% It demonstrates one method for incorporating program listings
% into a document.

\chapter{Matlab Code} \label{matlab}
This is an example of a Matlab m-file.
\verbatimfile{derivs.m}
         % Including computer code listings
\chapter{Bib\TeX\ Entries}
\label{bibrefs}
The following shows the fields required in all types of Bib\TeX\ entries.
Fields with {\tt OPT} prefixed are optional (the three letters {\tt OPT} should 
not be used).  If an optional field is not used, then the entire field can be deleted.

{\tt
\singlespace
\begin{verbatim}

@Unpublished{,                            @Manual{,
  author =      "",                         title =           "",
  title =       "",                         OPTauthor =       "",
  note =        "",                         OPTorganization = "",
  OPTyear =     "",                         OPTaddress =      "",
  OPTmonth =    ""                          OPTedition =      "",
}                                           OPTyear =         "",
                                            OPTmonth =        "",
@TechReport{,                               OPTnote =         "" 
  author =      "",                       }
  title =       "",
  institution = "",                       @InProceedings{,
  year =        "",                         author =          "",
  OPTtype =     "",                         title =           "",
  OPTnumber =   "",                         booktitle =       "",
  OPTaddress =  "",                         year =            "",
  OPTmonth =    "",                         OPTeditor =       "",
  OPTnote =     ""                          OPTpages =        "",
}                                           OPTorganization = "",
                                            OPTpublisher =    "",
@Proceedings{,                              OPTaddress =      "",
  title =           "",                     OPTmonth =        "",
  year =            "",                     OPTnote =         "" 
  OPTeditor =       "",                   }
  OPTpublisher =    "",
  OPTorganization = "",
  OPTaddress =      "",
  OPTmonth =        "",
  OPTnote =         "" 
}



@PhDThesis{,                              @InCollection{,
  author =      "",                         author =          "",
  title =       "",                         title =           "",
  school =      "",                         booktitle =       "",
  year =        "",                         publisher =       "",
  OPTaddress =  "",                         year =            "",
  OPTmonth =    "",                         OPTeditor =       "",
  OPTnote =     ""                          OPTchapter =      "",
}                                           OPTpages =        "",
                                            OPTaddress =      "",
                                            OPTmonth =        "",
                                            OPTnote =         ""
                                          }

 
@Misc{,                                   @InCollection{,
  OPTauthor =       "",                     author =          "",
  OPTtitle =        "",                     title =           "",
  OPThowpublished = "",                     chapter =         "",
  OPTyear =         "",                     publisher =       "",
  OPTmonth =        "",                     year =            "",
  OPTnote =         ""                      OPTeditor =       "",
}                                           OPTpages =        "",
}                                           OPTvolume =       "",
                                            OPTseries =       "",
                                            OPTaddress =      "",
                                            OPTedition =      "",
                                            OPTmonth =        "",
                                            OPTnote =         ""
                                          }

@MastersThesis{,                          @Article{,
  author =      "",                         author =          "",
  title =       "",                         title =           "",
  school =      "",                         journal =         "",
  year =        "",                         year =            "",
  OPTaddress =  "",                         OPTvolume =       "",
  OPTmonth =    "",                         OPTnumber =       "",
  OPTnote =     ""                          OPTpages =        "",
}                                           OPTmonth =        "",
                                            OPTnote =         ""
                                           }\end{verbatim} }
         % a BibTeX reference
\chapter{Mathematics Examples}
This appendix provides an example of \LaTeX's typesetting
capabilities.  Most of text was obtained from the University of
Wisconsin-Madison Math Department's example thesis file.

\section{Matrices}
The equations for the {\em dq}-model of an induction machine in the
synchronous reference frame are
\begin{eqnarray}
 \left[\begin{array}{c} v_{qs}^e\\v_{ds}^e\\v_{qr}^e\\v_{dr}^e  \end{array}\right]                                                                                                                                                                                                                                                                                                                                                                                                                                                                                                              
 &=& \left[ \begin{array}{cccc}
 r_s + x_s\frac{\rho}{\omega_b} & \frac{\omega_e}{\omega_b}x_s & x_m\frac{\rho}{\omega_b} & \frac{\omega_e}{\omega_b}x_m \\
 -\frac{\omega_e}{\omega_b}x_s & r_s + x_s\frac{\rho}{\omega_b} & -\frac{\omega_e}{\omega_b}x_m & x_m\frac{\rho}{\omega_b} \\
 x_m\frac{\rho}{\omega_b} & \frac{\omega_e -\omega_r}{\omega_b}x_m & r_r'+x_r'\frac{\rho}{\omega_b} & \frac{\omega_e - \omega_r}{\omega_b}x_r' \\
 -\frac{\omega_e -\omega_r}{\omega_b}x_m & x_m\frac{\rho}{\omega_b} & -\frac{\omega_e - \omega_r}{\omega_b}x_r' & r_r' + x_r'\frac{\rho}{\omega_b}
 \end{array} \right]
 \left[\begin{array}{c} i_{qs}^e\\i_{ds}^e\\i_{qr}^e\\i_{dr}^e\end{array} \right] \label{volteq}\\
 T_e&=&\frac{3}{2}\frac{P}{2}\frac{x_m}{\omega_b}\left(i_{qs}^ei_{dr}^e - i_{ds}^ei_{qr}^e\right) \label{torqueeq}\\
 T_e-T_l&=&\frac{2J\omega_b}{P}\frac{d}{dt}\left(\frac{\omega_r}{\omega_b}\right) \label{mecheq}.
\end{eqnarray}

\section{Multi-line Equations}

\LaTeX{} has a built-in equation array feature, however the
equation numbers must be on the same line as an equation.  For example:
\begin{eqnarray}
\Delta u + \lambda e^u &= 0&u\in \Omega,  \nonumber \\
u&=0&u\in\partial\Omega.
\end{eqnarray}

Alternatively, the number can be centered in the equation using the
following method.
%
% The equation-array feature in LaTeX is a bad idea.  For centered
% numbers you should set your own equations and arrays as follows:
%
\def\dd{\displaystyle}
\begin{equation}\label{gelfand}
\begin{array}{rl}
\dd \Delta u + \lambda e^u = 0, &
\dd u\in \Omega,\\[8pt] % add 8pt extra vertical space. 1 line=23pt
\dd u=0, & \dd u\in\partial\Omega.
\end{array}
\end{equation}
The previous equation had a label.  It may be referenced as
equation~(\ref{gelfand}).


\section{More Complicated Equations}
\section*{Rellich's identity}\label{rellich.section}
\setcounter{theorem}{0}
%
%

Standard developments of Pohozaev's identity used an identity by
Rellich~\cite{rellich:der40}, reproduced here.

\begin{lemma}[Rellich]
Given $L$ in divergence form and $a,d$ defined above, $u\in C^2
(\Omega )$, we have
\begin{equation}\label{rellich}
\int_{\Omega}(-Lu)\nabla u\cdot (x-\overline{x})\, dx
= (1-\frac{n}{2}) \int_{\Omega} a(\nabla u,\nabla u) \, dx
-
\frac{1}{2} \int_{\Omega}
d(\nabla u, \nabla u) \, dx
\end{equation}
$$
+
\frac{1}{2} \int_{\partial\Omega} a(\nabla u,\nabla u)(x-\overline{x})
\cdot \nu  \, dS
-
\int_{\partial\Omega}
a(\nabla u,\nu )\nabla u\cdot (x-\overline{x}) \, dS.
$$
\end{lemma}
{\bf Proof:}\\
There is no loss in generality to take $\overline{x} = 0$. First
rewrite $L$:
$$Lu = \frac{1}{2}\left[ \sum_{i}\sum_{j}
\frac{\partial}{\partial x_i}
\left( a_{ij} \frac{\partial u}{\partial x_j} \right) +
\sum_{i}\sum_{j}
\frac{\partial}{\partial x_i}
\left( a_{ij} \frac{\partial u}{\partial x_j} \right)
\right]$$
Switching the order of summation on the second term and relabeling
subscripts, $j \rightarrow i$ and $i \rightarrow j$, then using the fact
that $a_{ij}(x)$ is a symmetric matrix,
gives the symmetric form needed to derive Rellich's identity.
\begin{equation}
Lu = \frac{1}{2} \sum_{i,j}\left[
\frac{\partial}{\partial x_i}
\left( a_{ij} \frac{\partial u}{\partial x_j} \right) +
\frac{\partial}{\partial x_j}
\left( a_{ij} \frac{\partial u}{\partial x_i} \right)
\right].
\end{equation}

Multiplying $-Lu$ by $\frac{\partial u}{\partial x_k} x_k$ and integrating
over $\Omega$, yields
$$\int_{\Omega}(-Lu)\frac{\partial u}{\partial x_k} x_k \, dx=
-\frac{1}{2} \int_{\Omega}
\sum_{i,j}\left[
\frac{\partial}{\partial x_i}
\left( a_{ij} \frac{\partial u}{\partial x_j} \right) +
\frac{\partial}{\partial x_j}
\left( a_{ij} \frac{\partial u}{\partial x_i} \right)
\right]
\frac{\partial u}{\partial x_k} x_k \, dx$$
Integrating by parts (for integral theorems see~\cite[p. 20]
{zeidler:nfa88IIa})
gives
$$= \frac{1}{2} \int_{\Omega}
\sum_{i,j} a_{ij} \left[
\frac{\partial u}{\partial x_j}
\frac{\partial^2 u}{\partial x_k\partial x_i} +
\frac{\partial u}{\partial x_i}
\frac{\partial^2 u}{\partial x_k\partial x_j}
\right] x_k \, dx
$$
$$
+
\frac{1}{2} \int_{\Omega}
\sum_{i,j} a_{ij} \left[
\frac{\partial u}{\partial x_j} \delta_{ik} +
\frac{\partial u}{\partial x_i} \delta_{jk}
\right] \frac{\partial u}{\partial x_k} \, dx
$$
$$- \frac{1}{2} \int_{\partial\Omega}
\sum_{i,j} a_{ij} \left[
\frac{\partial u}{\partial x_j} \nu_{i} +
\frac{\partial u}{\partial x_i} \nu_{j}
\right] \frac{\partial u}{\partial x_k} x_k \, dx
$$
= $I_1 + I_2 + I_3$, where the unit normal vector is $\nu$.
One may rewrite $I_1$ as
$$I_1 = \frac{1}{2} \int_{\Omega}
\sum_{i,j} a_{ij} \frac{\partial}{\partial x_k}\left(
\frac{\partial u}{\partial x_i}
\frac{\partial u}{\partial x_j}
\right) x_k \, dx
$$
Integrating the first term by parts again yields
$$I_1 = -\frac{1}{2} \int_{\Omega}
\sum_{i,j} a_{ij} \left(
\frac{\partial u}{\partial x_i}
\frac{\partial u}{\partial x_j}
\right) \, dx
+
\frac{1}{2} \int_{\partial\Omega}
\sum_{i,j} a_{ij} \left(
\frac{\partial u}{\partial x_i}
\frac{\partial u}{\partial x_j}
\right) x_k \nu_k \, dS
$$
$$
-
\frac{1}{2} \int_{\Omega}
\sum_{i,j} \left(
\frac{\partial u}{\partial x_i}
\frac{\partial u}{\partial x_j}
\right) x_k \frac{\partial a_{ij}}{\partial x_k}\, dx.
$$
Summing over $k$ gives
$$\int_{\Omega}(-Lu)(\nabla u\cdot x)\, dx =
-\frac{n}{2} \int_{\Omega}
\sum_{i,j} a_{ij} \left(
\frac{\partial u}{\partial x_i}
\frac{\partial u}{\partial x_j}
\right) \, dx
$$
$$
+
\frac{1}{2} \int_{\partial\Omega}
\sum_{i,j} a_{ij} \left(
\frac{\partial u}{\partial x_i}
\frac{\partial u}{\partial x_j}
\right) (x\cdot \nu ) \, dS
-
\frac{1}{2} \int_{\Omega}
\sum_{i,j} \left(
\frac{\partial u}{\partial x_i}
\frac{\partial u}{\partial x_j}
\right) (x\cdot  \nabla a_{ij}) \, dx
$$
$$
+
\frac{1}{2} \int_{\Omega}
\sum_{i,j,k} a_{ij} \left[
\frac{\partial u}{\partial x_j}
\frac{\partial u}{\partial x_k} \delta_{ik} +
\frac{\partial u}{\partial x_i}
\frac{\partial u}{\partial x_k} \delta_{jk}
\right] \, dx
$$
$$- \frac{1}{2} \int_{\partial\Omega}
\sum_{i,j} a_{ij} \left[
\frac{\partial u}{\partial x_j} \nu_{i} +
\frac{\partial u}{\partial x_i} \nu_{j}
\right] (\nabla u\cdot x) \, dS.
$$
Combining the first and fourth term on the right-hand side
simplifies the expression
$$\int_{\Omega}(-Lu)(\nabla u\cdot x)\, dx
=
(1-\frac{n}{2}) \int_{\Omega}
\sum_{i,j} a_{ij} \left(
\frac{\partial u}{\partial x_i}
\frac{\partial u}{\partial x_j}
\right) \, dx
$$
$$
+
\frac{1}{2} \int_{\partial\Omega}
\sum_{i,j} a_{ij} \left(
\frac{\partial u}{\partial x_i}
\frac{\partial u}{\partial x_j}
\right) (x\cdot \nu ) \, dS
-
\frac{1}{2} \int_{\Omega}
\sum_{i,j} \left(
\frac{\partial u}{\partial x_i}
\frac{\partial u}{\partial x_j}
\right) (x\cdot  \nabla a_{ij}) \, dx
$$
$$
-
\frac{1}{2} \int_{\partial\Omega}
\sum_{i,j} a_{ij} \left[
\frac{\partial u}{\partial x_j} \nu_{i} +
\frac{\partial u}{\partial x_i} \nu_{j}
\right] (\nabla u\cdot x) \, dS.
$$
Using the notation defined above, the result follows.


           % Complex Equations from the UW Math Department
% acrobat.tex
%
% This file explains how to generate Adobe Acrobat files
%
% Eric Benedict, July 2000
%
% It is provided without warranty on an AS IS basis.

\newcommand{\pdf}{\mbox{\tt *.pdf}}

\chapter{Adobe Acrobat (\pdf ) Files}
The Adobe Acrobat file format has pretty much become the {\em de facto}
standard for document sharing.  As such, some faculty members and/or
departments may be requiring a final copy of the thesis in Acrobat format
(\pdf ).

There are several different methods of obtaining a \pdf\ file from a \LaTeX{}
thesis; however, they are all very site specific.  A couple of different
methods which have been found to work are mentioned as suggested ideas to try
as a starting point.  Depending on what is installed at your site/location some
of these may be applicable.

\section{Converting from {\tt *.ps} to \pdf}
One option to obtain the \pdf\ file would be to generate the thesis in a normal
manner and then use the Acrobat\ {\tt Distiller}\ to convert the postscript file into a
\pdf\ file.

If the\ {\tt Distiller}\ program is available and convenient to use, then this is quite easy to do.

Depending on the choice of document fonts, the results may not be satisfactory since
some of the fonts may end up as bit-mapped fonts and will display poorly at any resolution
other than what they were sampled on.  Also, since the\ {\tt Distiller}\ program is an expensive
program to obtain, it is not always available.

An alternative to the Adobe\ {\tt Distiller}\ program is the Alladin\ {\tt Ghostscript}\ program.  This is
available for free from

{\tt \verb|   http://www.cs.wisc.edu/~ghost/index.html|}

This program is available for most common operating systems as a compiled binary, but the source code
is available for other systems.  One drawback is that this conversion must be performed as a
command line invocation and isn't very user friendly.  This may be addressed in a future version of\
{\tt Ghostview}, the program which provides a nice user interface to\ {\tt Ghostscript}.

\section{Converting from {\tt *.dvi} to \pdf}
There are two programs available which will convert from {\tt *.dvi} to \pdf,\ {\tt dvipdf}\ and\
{\tt dvipdfm}.  The\ {\tt dvipdfm}\ program  will be discussed here.  In version 0.12, it can generate
bookmarks, thumbnails (with assistance from\ {\tt Ghostscript}), scaling and rotation, JPEG and
PNG bitmaps and font encoding and re-encoding (to support fonts which aren't fully supported by
the Acrobat suite).  When\ {\tt Ghostscript}\ is properly installed,\ {\tt dvipdfm}\ will automatically
convert any encapsulated PostScript figures into the required \pdf{} format.  This program behaved in a
similar manner to the {\tt dvips} program and was used to produce the \pdf{} format of this document.



\section{Generating \pdf{} Initially}
There are now some programs which are similar to \TeX{} but instead of producing a\ {\tt .dvi}\ output,
they produce \pdf as a native output.  One such program, {\sc pdf}\TeX{} / {\sc pdf}\LaTeX{},
is available from

{\tt \verb|   http://www.tug.org/applications/pdftex|}

Note that as of this date, July 2000, {\sc pdf}\TeX{} / {\sc pdf}\LaTeX{} while currently quite usable, it
is still in a beta version.  Look at the web site for more current information.

The present version was able to produce a \pdf{} file of this document without any required
changes, except for the Postscript figure inclusion  (Figure~\ref{vwcontr}).  To properly include
this figure, requires the conversion of the postscript figure into a \pdf{} figure.  The procedure
is described in the manual for {\sc pdf}\TeX{} / {\sc pdf}\LaTeX{}.  Note that the figure conversion will
require either\ {\tt Distiller}\ or\ {\tt Ghostscript}.
           % A discussion on generating PDF files.
\end{appendices}         % End of the Appendix Chapters.  ibid on \end{appendix}
%\include{vita}          % Optional Vita, use \begin{vita} vita text \end{vita}
\end{document}
\end{verbatim}
\end{quote}

\section{Prelude}
After the {\tt \verb|\begin{document}|} comes the preliminary information found in
theses.  In this manual, the information is kept in the file {\tt prelude.tex} (see
above).  These pages will need to be numbered with roman numerals, so use
\begin{quote}\tt\singlespace\begin{verbatim}
\clearpage\pagenumbering{roman}
\end{verbatim}\end{quote}

Next, comes your thesis or dissertation title, your name, date of graduation, department
and degree.
\begin{quote}\tt\singlespace\begin{verbatim}
\title{How to \LaTeX\ a Thesis}
\author{Eric R. Benedict}
\date{2000}
%   - The default degree is ``Doctor of Philosophy''
%     Degree can be changed using the command \degree{}
%\degree{New Degree}
%   - for a PhD dissertation (default), specify \dissertation
%\dissertation
%   - for a masters project report, specify \project
%\project
%   - for a preliminary report, specify \prelim
%\prelim
%   - for a masters thesis, specify \thesis
%\thesis
%   - The default department is ``Electrical Engineering''
%     The department can be changed using the command \department{}
%\department{New Department}
\end{verbatim}\end{quote}

If you specified the class option {\tt msthesis}, then the degree is changed to
{\em Master \break of Science} and the {\tt \verb|\thesis|} option is specified.  If you
want to have the masters margins with another document, then the {\tt \verb|\degree|}
and {\tt \verb|\dissertation|},  {\tt \verb|\project|}, {\em etc.\/} can be specified
as needed.

Once the
above are all defined, use  {\tt \verb|\maketitle|} to generate the title page.
\begin{quote}\tt\singlespace\begin{verbatim}
\maketitle
\end{verbatim}\end{quote}

If you wish to include a copyright page (see Section~\ref{copyright} for
information on registering the copyright.), then add the command
\begin{quote}\tt\singlespace\begin{verbatim}
\copyrightpage
\end{verbatim}\end{quote}
This will generate the proper copyright page and will use the name and date specified
in {\tt \verb|\author{}|} and {\tt \verb|\date{}|}.

Next are the dedications and acknowledgements:
\begin{quote}\tt\singlespace\begin{verbatim}
\begin{dedication}
To my pet rock, Skippy.
\end{dedication}

\begin{acknowledgments}
I thank the many people who have done lots of nice things for me.
\end{acknowledgments}
\end{verbatim}\end{quote}

You must tell \LaTeX{} to generate a table of contents, a list of tables and a list of
figures:
\begin{quote}\tt\singlespace\begin{verbatim}
\tableofcontents
\listoftables
\listoffigures
\end{verbatim}\end{quote}

If you wish to have a nomenclature, list of symbols or glossary it can go here.
\begin{quote}\tt\singlespace\begin{verbatim}
\begin{nomenclature}
%\begin{listofsymbols}
%\begin{glossary}
\begin{tabular}{ll}
$C_1$ & Constant 1\\
\ldots
\end{tabular}
%\end{glossary}
%\end{listofsymbols}
\end{nomenclature}
\end{verbatim}\end{quote}

If your abstract will be microfilmed by Bell and Howell (formerly UMI), then you
will need to generate an abstract of less than 350 words.  This abstract can be created
using the {\tt umiabstract} environment.  This environment requires that you define your
advisor and your advisor's title using {\tt \verb|\advisorname{}|} and
{\tt \verb|\advisortitle{}|}.
\begin{quote}\tt\singlespace\begin{verbatim}
\advisorname{Bucky J. Badger}
\advisortitle{Assistant Professor}
% ABSTRACT
\begin{umiabstract}
\noindent       % Don't indent first paragraph.
This explains the basics for using \LaTeX\ to typeset a
dissertation, thesis or project report for the University of
Wisconsin-Madison.

...

\end{umiabstract}
\end{verbatim}\end{quote}
This will place your name, title and required text at the top of the page and follow the
abstract text with your advisor's name at the bottom for your advisor's signature.  This
page is not numbered and would be submitted separately.

If you will have an abstract as part of your document, then the {\tt abstract} environment
should be used.
\begin{quote}\tt\singlespace\begin{verbatim}
\begin{abstract}
\noindent       % Don't indent first paragraph.
This explains the basics for using \LaTeX\ to typeset a
dissertation, thesis or project report for the University of
Wisconsin-Madison.

...

\end{abstract}
\end{verbatim}\end{quote}
This will generate a page number and it will be included in the Table
of Contents.  

If you will have both the UMI and regular abstracts like this document, then
you will probably want to write the abstract once and save it in a seperate
file such as {\tt abstract.tex}.  Then, you can use the same abstract for
both purposes.

\begin{quote}\begin{verbatim}
\begin{umiabstract}
  % abstract.tex
%
% This file has the abstract for the withesis style documentation
%
% Eric Benedict, Aug 2000
%
% It is provided without warranty on an AS IS basis.

\noindent       % Don't indent this paragraph.
The Human Genome Project (HGP), completed in 2003, is considered one of the greatest accomplishments of exploration in history of science. Since then thousands of genomes have been sequenced. However, no individual human genome has been annotated to completion. Nanocoding \cite{Jo_etal_2007_PNAS} (PNAS, 2007), developed by Laboratory of Molecular and Computational Genomics (LMCG), UW Madison , is a novel system for physically mapping genomes, using measurements of single DNA molecules to construct a high-resolution genome-wide restriction map, whose representation of genome structure complements genome sequences to yield biological insight. Staining the DNA molecules with cyanine dyes and imaging them is a critical step of nanocoding. It turns out that the quantum yield of the fluorescence intensity of these stained molecules are sequence dependent. In fact, for YO complexed with GC-rich DNA sequences the quantum yield are about twice as large as for YO complexed with AT-rich sequences. Hence, regions with distinct sequence compositions should exhibit unique fluorescence intensity profiles. Establishing the fluorescence intensity profiles of a genome would provide invaluable insights into its sequence compositions without having to sequence it. We name this technique ``Fluoroscanning''. Imaged DNA molecules from the same region on a genome should exhibit similar intensity profiles, unless there has been a modification in the underlying genomic sequence. Fluoroscanning can be used to identify heterozygotes and detect large scale structural variations as a result of cancer or other diseases.  

%\vspace*{0.5em}
%\noindent       % Don't indent this paragraph.


\end{umiabstract}

\begin{abstract}
  % abstract.tex
%
% This file has the abstract for the withesis style documentation
%
% Eric Benedict, Aug 2000
%
% It is provided without warranty on an AS IS basis.

\noindent       % Don't indent this paragraph.
The Human Genome Project (HGP), completed in 2003, is considered one of the greatest accomplishments of exploration in history of science. Since then thousands of genomes have been sequenced. However, no individual human genome has been annotated to completion. Nanocoding \cite{Jo_etal_2007_PNAS} (PNAS, 2007), developed by Laboratory of Molecular and Computational Genomics (LMCG), UW Madison , is a novel system for physically mapping genomes, using measurements of single DNA molecules to construct a high-resolution genome-wide restriction map, whose representation of genome structure complements genome sequences to yield biological insight. Staining the DNA molecules with cyanine dyes and imaging them is a critical step of nanocoding. It turns out that the quantum yield of the fluorescence intensity of these stained molecules are sequence dependent. In fact, for YO complexed with GC-rich DNA sequences the quantum yield are about twice as large as for YO complexed with AT-rich sequences. Hence, regions with distinct sequence compositions should exhibit unique fluorescence intensity profiles. Establishing the fluorescence intensity profiles of a genome would provide invaluable insights into its sequence compositions without having to sequence it. We name this technique ``Fluoroscanning''. Imaged DNA molecules from the same region on a genome should exhibit similar intensity profiles, unless there has been a modification in the underlying genomic sequence. Fluoroscanning can be used to identify heterozygotes and detect large scale structural variations as a result of cancer or other diseases.  

%\vspace*{0.5em}
%\noindent       % Don't indent this paragraph.


\end{abstract}
\end{verbatim}\end{quote}

Finally, the page numbers must be changed to arabic numbers to conclude the preliminary
portion of the document.
\begin{quote}\tt\singlespace\begin{verbatim}
\clearpage\pagenumbering{arabic}
\end{verbatim}\end{quote}

\section{The Body}
At the beginning of {\tt intro.tex} there is the following command:
\begin{quote}\tt\singlespace\begin{verbatim}
\chapter{Introducing the {\tt withesis} \LaTeX{} Style Guide}
\end{verbatim}\end{quote}
Following that is the text of the chapter.  The body of your thesis is separated by
sectioning commands like {\tt \verb|\chapter{}|}.  For more information on the sectioning
commands, see Section~\ref{ess:sectioning}.

Remember the basic rule of outlining you learned in grammar school:
\begin{quote}
You cannot have an `A' if you do not have a `B'
\end{quote}
Take care to have at least two {\tt \verb|\section|}s if you use the command; have
two {\tt \verb|\subsection|}s, {\em etc}.



\section{Additional Theorem Like Environments}
The {\tt withesis} style adds numerous additional theorem like environments.  These
environments were included to allow compatibility with the University of Wisconsin's
Math Department's style file.  These environments are
{\tt theorem}, {\tt assertion}, {\tt claim}, {\tt conjecture}, {\tt corollary},
{\tt definition}, {\tt example}, {\tt figger}, {\tt lemma}, {\tt prop} and {\tt remark}.

As an example, consider the following.
\begin{lemma}
Assuming that $\partial\Omega_2 = \emptyset$ and that $h(t) = 1$, we
have $$
\begin{array}{lr}
\Delta u = f, &  x\in\Omega ,\\[2pt]
u =  g_1, &  x\in\partial\Omega .
\end{array}
$$
\end{lemma}
which was produced with the following:
\begin{quote}\tt\singlespace\begin{verbatim}
\begin{lemma}
Assuming that $\partial\Omega_2 = \emptyset$ and that $h(t) = 1$, we
have $$
\begin{array}{lr}
\Delta u = f, &  x\in\Omega ,\\[2pt]
 u =  g_1, &  x\in\partial\Omega .
\end{array}
$$
\end{lemma}
\end{verbatim}\end{quote}

\section{Bibliography or References}
As a final note, the default title for the references chapter is ``LIST OF REFERENCES.''
Since some people may prefer ``BIBLIOGRAPHY'', the command
\break{\tt \verb|\altbibtitle|}
has been added to change the chapter title.

\section{Appendices}
There are two commands which are available to suppress the writing of the auxiliary information
(to the {\tt .lot} and {\tt .lof} files).  They are:
\begin{quote}\tt\singlespace\begin{verbatim}
\noappendixtables                % Don't have appendix tables
\noappendixfigures               % Don't have appendix figures
\end{verbatim}\end{quote}
These commands should be in the preamble.  See Section~\ref{usage:noapp}.

There are two environments for doing the appendix chapter: {\tt appendix} and
\break {\tt appendices}.  If you have only one chapter in the appendix, use the {\tt appendix}
environment.  If you have more than one chapter, like this manual, use the
{\tt appendices} environment.
\begin{quote}\tt\singlespace\footnotesize\begin{verbatim}
\begin{appendices}  % Start of the Appendix Chapters.  If there is only
                    % one Appendix Chapter, then use \begin{appendix}
% code.tex
% this file is part of the example UW-Madison Thesis document
% It demonstrates one method for incorporating program listings
% into a document.

\chapter{Matlab Code} \label{matlab}
This is an example of a Matlab m-file.
\verbatimfile{derivs.m}
      % Including computer code listings
\chapter{Bib\TeX\ Entries}
\label{bibrefs}
The following shows the fields required in all types of Bib\TeX\ entries.
Fields with {\tt OPT} prefixed are optional (the three letters {\tt OPT} should 
not be used).  If an optional field is not used, then the entire field can be deleted.

{\tt
\singlespace
\begin{verbatim}

@Unpublished{,                            @Manual{,
  author =      "",                         title =           "",
  title =       "",                         OPTauthor =       "",
  note =        "",                         OPTorganization = "",
  OPTyear =     "",                         OPTaddress =      "",
  OPTmonth =    ""                          OPTedition =      "",
}                                           OPTyear =         "",
                                            OPTmonth =        "",
@TechReport{,                               OPTnote =         "" 
  author =      "",                       }
  title =       "",
  institution = "",                       @InProceedings{,
  year =        "",                         author =          "",
  OPTtype =     "",                         title =           "",
  OPTnumber =   "",                         booktitle =       "",
  OPTaddress =  "",                         year =            "",
  OPTmonth =    "",                         OPTeditor =       "",
  OPTnote =     ""                          OPTpages =        "",
}                                           OPTorganization = "",
                                            OPTpublisher =    "",
@Proceedings{,                              OPTaddress =      "",
  title =           "",                     OPTmonth =        "",
  year =            "",                     OPTnote =         "" 
  OPTeditor =       "",                   }
  OPTpublisher =    "",
  OPTorganization = "",
  OPTaddress =      "",
  OPTmonth =        "",
  OPTnote =         "" 
}



@PhDThesis{,                              @InCollection{,
  author =      "",                         author =          "",
  title =       "",                         title =           "",
  school =      "",                         booktitle =       "",
  year =        "",                         publisher =       "",
  OPTaddress =  "",                         year =            "",
  OPTmonth =    "",                         OPTeditor =       "",
  OPTnote =     ""                          OPTchapter =      "",
}                                           OPTpages =        "",
                                            OPTaddress =      "",
                                            OPTmonth =        "",
                                            OPTnote =         ""
                                          }

 
@Misc{,                                   @InCollection{,
  OPTauthor =       "",                     author =          "",
  OPTtitle =        "",                     title =           "",
  OPThowpublished = "",                     chapter =         "",
  OPTyear =         "",                     publisher =       "",
  OPTmonth =        "",                     year =            "",
  OPTnote =         ""                      OPTeditor =       "",
}                                           OPTpages =        "",
}                                           OPTvolume =       "",
                                            OPTseries =       "",
                                            OPTaddress =      "",
                                            OPTedition =      "",
                                            OPTmonth =        "",
                                            OPTnote =         ""
                                          }

@MastersThesis{,                          @Article{,
  author =      "",                         author =          "",
  title =       "",                         title =           "",
  school =      "",                         journal =         "",
  year =        "",                         year =            "",
  OPTaddress =  "",                         OPTvolume =       "",
  OPTmonth =    "",                         OPTnumber =       "",
  OPTnote =     ""                          OPTpages =        "",
}                                           OPTmonth =        "",
                                            OPTnote =         ""
                                           }\end{verbatim} }
    % a BibTeX reference
\chapter{Mathematics Examples}
This appendix provides an example of \LaTeX's typesetting
capabilities.  Most of text was obtained from the University of
Wisconsin-Madison Math Department's example thesis file.

\section{Matrices}
The equations for the {\em dq}-model of an induction machine in the
synchronous reference frame are
\begin{eqnarray}
 \left[\begin{array}{c} v_{qs}^e\\v_{ds}^e\\v_{qr}^e\\v_{dr}^e  \end{array}\right]                                                                                                                                                                                                                                                                                                                                                                                                                                                                                                              
 &=& \left[ \begin{array}{cccc}
 r_s + x_s\frac{\rho}{\omega_b} & \frac{\omega_e}{\omega_b}x_s & x_m\frac{\rho}{\omega_b} & \frac{\omega_e}{\omega_b}x_m \\
 -\frac{\omega_e}{\omega_b}x_s & r_s + x_s\frac{\rho}{\omega_b} & -\frac{\omega_e}{\omega_b}x_m & x_m\frac{\rho}{\omega_b} \\
 x_m\frac{\rho}{\omega_b} & \frac{\omega_e -\omega_r}{\omega_b}x_m & r_r'+x_r'\frac{\rho}{\omega_b} & \frac{\omega_e - \omega_r}{\omega_b}x_r' \\
 -\frac{\omega_e -\omega_r}{\omega_b}x_m & x_m\frac{\rho}{\omega_b} & -\frac{\omega_e - \omega_r}{\omega_b}x_r' & r_r' + x_r'\frac{\rho}{\omega_b}
 \end{array} \right]
 \left[\begin{array}{c} i_{qs}^e\\i_{ds}^e\\i_{qr}^e\\i_{dr}^e\end{array} \right] \label{volteq}\\
 T_e&=&\frac{3}{2}\frac{P}{2}\frac{x_m}{\omega_b}\left(i_{qs}^ei_{dr}^e - i_{ds}^ei_{qr}^e\right) \label{torqueeq}\\
 T_e-T_l&=&\frac{2J\omega_b}{P}\frac{d}{dt}\left(\frac{\omega_r}{\omega_b}\right) \label{mecheq}.
\end{eqnarray}

\section{Multi-line Equations}

\LaTeX{} has a built-in equation array feature, however the
equation numbers must be on the same line as an equation.  For example:
\begin{eqnarray}
\Delta u + \lambda e^u &= 0&u\in \Omega,  \nonumber \\
u&=0&u\in\partial\Omega.
\end{eqnarray}

Alternatively, the number can be centered in the equation using the
following method.
%
% The equation-array feature in LaTeX is a bad idea.  For centered
% numbers you should set your own equations and arrays as follows:
%
\def\dd{\displaystyle}
\begin{equation}\label{gelfand}
\begin{array}{rl}
\dd \Delta u + \lambda e^u = 0, &
\dd u\in \Omega,\\[8pt] % add 8pt extra vertical space. 1 line=23pt
\dd u=0, & \dd u\in\partial\Omega.
\end{array}
\end{equation}
The previous equation had a label.  It may be referenced as
equation~(\ref{gelfand}).


\section{More Complicated Equations}
\section*{Rellich's identity}\label{rellich.section}
\setcounter{theorem}{0}
%
%

Standard developments of Pohozaev's identity used an identity by
Rellich~\cite{rellich:der40}, reproduced here.

\begin{lemma}[Rellich]
Given $L$ in divergence form and $a,d$ defined above, $u\in C^2
(\Omega )$, we have
\begin{equation}\label{rellich}
\int_{\Omega}(-Lu)\nabla u\cdot (x-\overline{x})\, dx
= (1-\frac{n}{2}) \int_{\Omega} a(\nabla u,\nabla u) \, dx
-
\frac{1}{2} \int_{\Omega}
d(\nabla u, \nabla u) \, dx
\end{equation}
$$
+
\frac{1}{2} \int_{\partial\Omega} a(\nabla u,\nabla u)(x-\overline{x})
\cdot \nu  \, dS
-
\int_{\partial\Omega}
a(\nabla u,\nu )\nabla u\cdot (x-\overline{x}) \, dS.
$$
\end{lemma}
{\bf Proof:}\\
There is no loss in generality to take $\overline{x} = 0$. First
rewrite $L$:
$$Lu = \frac{1}{2}\left[ \sum_{i}\sum_{j}
\frac{\partial}{\partial x_i}
\left( a_{ij} \frac{\partial u}{\partial x_j} \right) +
\sum_{i}\sum_{j}
\frac{\partial}{\partial x_i}
\left( a_{ij} \frac{\partial u}{\partial x_j} \right)
\right]$$
Switching the order of summation on the second term and relabeling
subscripts, $j \rightarrow i$ and $i \rightarrow j$, then using the fact
that $a_{ij}(x)$ is a symmetric matrix,
gives the symmetric form needed to derive Rellich's identity.
\begin{equation}
Lu = \frac{1}{2} \sum_{i,j}\left[
\frac{\partial}{\partial x_i}
\left( a_{ij} \frac{\partial u}{\partial x_j} \right) +
\frac{\partial}{\partial x_j}
\left( a_{ij} \frac{\partial u}{\partial x_i} \right)
\right].
\end{equation}

Multiplying $-Lu$ by $\frac{\partial u}{\partial x_k} x_k$ and integrating
over $\Omega$, yields
$$\int_{\Omega}(-Lu)\frac{\partial u}{\partial x_k} x_k \, dx=
-\frac{1}{2} \int_{\Omega}
\sum_{i,j}\left[
\frac{\partial}{\partial x_i}
\left( a_{ij} \frac{\partial u}{\partial x_j} \right) +
\frac{\partial}{\partial x_j}
\left( a_{ij} \frac{\partial u}{\partial x_i} \right)
\right]
\frac{\partial u}{\partial x_k} x_k \, dx$$
Integrating by parts (for integral theorems see~\cite[p. 20]
{zeidler:nfa88IIa})
gives
$$= \frac{1}{2} \int_{\Omega}
\sum_{i,j} a_{ij} \left[
\frac{\partial u}{\partial x_j}
\frac{\partial^2 u}{\partial x_k\partial x_i} +
\frac{\partial u}{\partial x_i}
\frac{\partial^2 u}{\partial x_k\partial x_j}
\right] x_k \, dx
$$
$$
+
\frac{1}{2} \int_{\Omega}
\sum_{i,j} a_{ij} \left[
\frac{\partial u}{\partial x_j} \delta_{ik} +
\frac{\partial u}{\partial x_i} \delta_{jk}
\right] \frac{\partial u}{\partial x_k} \, dx
$$
$$- \frac{1}{2} \int_{\partial\Omega}
\sum_{i,j} a_{ij} \left[
\frac{\partial u}{\partial x_j} \nu_{i} +
\frac{\partial u}{\partial x_i} \nu_{j}
\right] \frac{\partial u}{\partial x_k} x_k \, dx
$$
= $I_1 + I_2 + I_3$, where the unit normal vector is $\nu$.
One may rewrite $I_1$ as
$$I_1 = \frac{1}{2} \int_{\Omega}
\sum_{i,j} a_{ij} \frac{\partial}{\partial x_k}\left(
\frac{\partial u}{\partial x_i}
\frac{\partial u}{\partial x_j}
\right) x_k \, dx
$$
Integrating the first term by parts again yields
$$I_1 = -\frac{1}{2} \int_{\Omega}
\sum_{i,j} a_{ij} \left(
\frac{\partial u}{\partial x_i}
\frac{\partial u}{\partial x_j}
\right) \, dx
+
\frac{1}{2} \int_{\partial\Omega}
\sum_{i,j} a_{ij} \left(
\frac{\partial u}{\partial x_i}
\frac{\partial u}{\partial x_j}
\right) x_k \nu_k \, dS
$$
$$
-
\frac{1}{2} \int_{\Omega}
\sum_{i,j} \left(
\frac{\partial u}{\partial x_i}
\frac{\partial u}{\partial x_j}
\right) x_k \frac{\partial a_{ij}}{\partial x_k}\, dx.
$$
Summing over $k$ gives
$$\int_{\Omega}(-Lu)(\nabla u\cdot x)\, dx =
-\frac{n}{2} \int_{\Omega}
\sum_{i,j} a_{ij} \left(
\frac{\partial u}{\partial x_i}
\frac{\partial u}{\partial x_j}
\right) \, dx
$$
$$
+
\frac{1}{2} \int_{\partial\Omega}
\sum_{i,j} a_{ij} \left(
\frac{\partial u}{\partial x_i}
\frac{\partial u}{\partial x_j}
\right) (x\cdot \nu ) \, dS
-
\frac{1}{2} \int_{\Omega}
\sum_{i,j} \left(
\frac{\partial u}{\partial x_i}
\frac{\partial u}{\partial x_j}
\right) (x\cdot  \nabla a_{ij}) \, dx
$$
$$
+
\frac{1}{2} \int_{\Omega}
\sum_{i,j,k} a_{ij} \left[
\frac{\partial u}{\partial x_j}
\frac{\partial u}{\partial x_k} \delta_{ik} +
\frac{\partial u}{\partial x_i}
\frac{\partial u}{\partial x_k} \delta_{jk}
\right] \, dx
$$
$$- \frac{1}{2} \int_{\partial\Omega}
\sum_{i,j} a_{ij} \left[
\frac{\partial u}{\partial x_j} \nu_{i} +
\frac{\partial u}{\partial x_i} \nu_{j}
\right] (\nabla u\cdot x) \, dS.
$$
Combining the first and fourth term on the right-hand side
simplifies the expression
$$\int_{\Omega}(-Lu)(\nabla u\cdot x)\, dx
=
(1-\frac{n}{2}) \int_{\Omega}
\sum_{i,j} a_{ij} \left(
\frac{\partial u}{\partial x_i}
\frac{\partial u}{\partial x_j}
\right) \, dx
$$
$$
+
\frac{1}{2} \int_{\partial\Omega}
\sum_{i,j} a_{ij} \left(
\frac{\partial u}{\partial x_i}
\frac{\partial u}{\partial x_j}
\right) (x\cdot \nu ) \, dS
-
\frac{1}{2} \int_{\Omega}
\sum_{i,j} \left(
\frac{\partial u}{\partial x_i}
\frac{\partial u}{\partial x_j}
\right) (x\cdot  \nabla a_{ij}) \, dx
$$
$$
-
\frac{1}{2} \int_{\partial\Omega}
\sum_{i,j} a_{ij} \left[
\frac{\partial u}{\partial x_j} \nu_{i} +
\frac{\partial u}{\partial x_i} \nu_{j}
\right] (\nabla u\cdot x) \, dS.
$$
Using the notation defined above, the result follows.


      % Complex Equations from the UW Math Department

\end{appendices}    % End of the Appendix Chapters. ibid on \end{appendix}
\end{verbatim}\end{quote}
The difference between these two environments is the way that the chapter header is
created and how this is listed in the table of contents.
                 % Chapter 5 Strongly based on similar by J.D. McCauley

%\bibliography{refs}             % Make the bibliography
\bibliography{bibTex_Reference}  % Make the bibliography

%\begin{appendices}              % Start of the Appendix Chapters.  If there is only
                                 % one Appendix Chapter, then use \begin{appendix}
%% code.tex
% this file is part of the example UW-Madison Thesis document
% It demonstrates one method for incorporating program listings
% into a document.

\chapter{Matlab Code} \label{matlab}
This is an example of a Matlab m-file.
\verbatimfile{derivs.m}
                  % Including computer code listings
%\chapter{Bib\TeX\ Entries}
\label{bibrefs}
The following shows the fields required in all types of Bib\TeX\ entries.
Fields with {\tt OPT} prefixed are optional (the three letters {\tt OPT} should 
not be used).  If an optional field is not used, then the entire field can be deleted.

{\tt
\singlespace
\begin{verbatim}

@Unpublished{,                            @Manual{,
  author =      "",                         title =           "",
  title =       "",                         OPTauthor =       "",
  note =        "",                         OPTorganization = "",
  OPTyear =     "",                         OPTaddress =      "",
  OPTmonth =    ""                          OPTedition =      "",
}                                           OPTyear =         "",
                                            OPTmonth =        "",
@TechReport{,                               OPTnote =         "" 
  author =      "",                       }
  title =       "",
  institution = "",                       @InProceedings{,
  year =        "",                         author =          "",
  OPTtype =     "",                         title =           "",
  OPTnumber =   "",                         booktitle =       "",
  OPTaddress =  "",                         year =            "",
  OPTmonth =    "",                         OPTeditor =       "",
  OPTnote =     ""                          OPTpages =        "",
}                                           OPTorganization = "",
                                            OPTpublisher =    "",
@Proceedings{,                              OPTaddress =      "",
  title =           "",                     OPTmonth =        "",
  year =            "",                     OPTnote =         "" 
  OPTeditor =       "",                   }
  OPTpublisher =    "",
  OPTorganization = "",
  OPTaddress =      "",
  OPTmonth =        "",
  OPTnote =         "" 
}



@PhDThesis{,                              @InCollection{,
  author =      "",                         author =          "",
  title =       "",                         title =           "",
  school =      "",                         booktitle =       "",
  year =        "",                         publisher =       "",
  OPTaddress =  "",                         year =            "",
  OPTmonth =    "",                         OPTeditor =       "",
  OPTnote =     ""                          OPTchapter =      "",
}                                           OPTpages =        "",
                                            OPTaddress =      "",
                                            OPTmonth =        "",
                                            OPTnote =         ""
                                          }

 
@Misc{,                                   @InCollection{,
  OPTauthor =       "",                     author =          "",
  OPTtitle =        "",                     title =           "",
  OPThowpublished = "",                     chapter =         "",
  OPTyear =         "",                     publisher =       "",
  OPTmonth =        "",                     year =            "",
  OPTnote =         ""                      OPTeditor =       "",
}                                           OPTpages =        "",
}                                           OPTvolume =       "",
                                            OPTseries =       "",
                                            OPTaddress =      "",
                                            OPTedition =      "",
                                            OPTmonth =        "",
                                            OPTnote =         ""
                                          }

@MastersThesis{,                          @Article{,
  author =      "",                         author =          "",
  title =       "",                         title =           "",
  school =      "",                         journal =         "",
  year =        "",                         year =            "",
  OPTaddress =  "",                         OPTvolume =       "",
  OPTmonth =    "",                         OPTnumber =       "",
  OPTnote =     ""                          OPTpages =        "",
}                                           OPTmonth =        "",
                                            OPTnote =         ""
                                           }\end{verbatim} }
                % a BibTeX reference
%\chapter{Mathematics Examples}
This appendix provides an example of \LaTeX's typesetting
capabilities.  Most of text was obtained from the University of
Wisconsin-Madison Math Department's example thesis file.

\section{Matrices}
The equations for the {\em dq}-model of an induction machine in the
synchronous reference frame are
\begin{eqnarray}
 \left[\begin{array}{c} v_{qs}^e\\v_{ds}^e\\v_{qr}^e\\v_{dr}^e  \end{array}\right]                                                                                                                                                                                                                                                                                                                                                                                                                                                                                                              
 &=& \left[ \begin{array}{cccc}
 r_s + x_s\frac{\rho}{\omega_b} & \frac{\omega_e}{\omega_b}x_s & x_m\frac{\rho}{\omega_b} & \frac{\omega_e}{\omega_b}x_m \\
 -\frac{\omega_e}{\omega_b}x_s & r_s + x_s\frac{\rho}{\omega_b} & -\frac{\omega_e}{\omega_b}x_m & x_m\frac{\rho}{\omega_b} \\
 x_m\frac{\rho}{\omega_b} & \frac{\omega_e -\omega_r}{\omega_b}x_m & r_r'+x_r'\frac{\rho}{\omega_b} & \frac{\omega_e - \omega_r}{\omega_b}x_r' \\
 -\frac{\omega_e -\omega_r}{\omega_b}x_m & x_m\frac{\rho}{\omega_b} & -\frac{\omega_e - \omega_r}{\omega_b}x_r' & r_r' + x_r'\frac{\rho}{\omega_b}
 \end{array} \right]
 \left[\begin{array}{c} i_{qs}^e\\i_{ds}^e\\i_{qr}^e\\i_{dr}^e\end{array} \right] \label{volteq}\\
 T_e&=&\frac{3}{2}\frac{P}{2}\frac{x_m}{\omega_b}\left(i_{qs}^ei_{dr}^e - i_{ds}^ei_{qr}^e\right) \label{torqueeq}\\
 T_e-T_l&=&\frac{2J\omega_b}{P}\frac{d}{dt}\left(\frac{\omega_r}{\omega_b}\right) \label{mecheq}.
\end{eqnarray}

\section{Multi-line Equations}

\LaTeX{} has a built-in equation array feature, however the
equation numbers must be on the same line as an equation.  For example:
\begin{eqnarray}
\Delta u + \lambda e^u &= 0&u\in \Omega,  \nonumber \\
u&=0&u\in\partial\Omega.
\end{eqnarray}

Alternatively, the number can be centered in the equation using the
following method.
%
% The equation-array feature in LaTeX is a bad idea.  For centered
% numbers you should set your own equations and arrays as follows:
%
\def\dd{\displaystyle}
\begin{equation}\label{gelfand}
\begin{array}{rl}
\dd \Delta u + \lambda e^u = 0, &
\dd u\in \Omega,\\[8pt] % add 8pt extra vertical space. 1 line=23pt
\dd u=0, & \dd u\in\partial\Omega.
\end{array}
\end{equation}
The previous equation had a label.  It may be referenced as
equation~(\ref{gelfand}).


\section{More Complicated Equations}
\section*{Rellich's identity}\label{rellich.section}
\setcounter{theorem}{0}
%
%

Standard developments of Pohozaev's identity used an identity by
Rellich~\cite{rellich:der40}, reproduced here.

\begin{lemma}[Rellich]
Given $L$ in divergence form and $a,d$ defined above, $u\in C^2
(\Omega )$, we have
\begin{equation}\label{rellich}
\int_{\Omega}(-Lu)\nabla u\cdot (x-\overline{x})\, dx
= (1-\frac{n}{2}) \int_{\Omega} a(\nabla u,\nabla u) \, dx
-
\frac{1}{2} \int_{\Omega}
d(\nabla u, \nabla u) \, dx
\end{equation}
$$
+
\frac{1}{2} \int_{\partial\Omega} a(\nabla u,\nabla u)(x-\overline{x})
\cdot \nu  \, dS
-
\int_{\partial\Omega}
a(\nabla u,\nu )\nabla u\cdot (x-\overline{x}) \, dS.
$$
\end{lemma}
{\bf Proof:}\\
There is no loss in generality to take $\overline{x} = 0$. First
rewrite $L$:
$$Lu = \frac{1}{2}\left[ \sum_{i}\sum_{j}
\frac{\partial}{\partial x_i}
\left( a_{ij} \frac{\partial u}{\partial x_j} \right) +
\sum_{i}\sum_{j}
\frac{\partial}{\partial x_i}
\left( a_{ij} \frac{\partial u}{\partial x_j} \right)
\right]$$
Switching the order of summation on the second term and relabeling
subscripts, $j \rightarrow i$ and $i \rightarrow j$, then using the fact
that $a_{ij}(x)$ is a symmetric matrix,
gives the symmetric form needed to derive Rellich's identity.
\begin{equation}
Lu = \frac{1}{2} \sum_{i,j}\left[
\frac{\partial}{\partial x_i}
\left( a_{ij} \frac{\partial u}{\partial x_j} \right) +
\frac{\partial}{\partial x_j}
\left( a_{ij} \frac{\partial u}{\partial x_i} \right)
\right].
\end{equation}

Multiplying $-Lu$ by $\frac{\partial u}{\partial x_k} x_k$ and integrating
over $\Omega$, yields
$$\int_{\Omega}(-Lu)\frac{\partial u}{\partial x_k} x_k \, dx=
-\frac{1}{2} \int_{\Omega}
\sum_{i,j}\left[
\frac{\partial}{\partial x_i}
\left( a_{ij} \frac{\partial u}{\partial x_j} \right) +
\frac{\partial}{\partial x_j}
\left( a_{ij} \frac{\partial u}{\partial x_i} \right)
\right]
\frac{\partial u}{\partial x_k} x_k \, dx$$
Integrating by parts (for integral theorems see~\cite[p. 20]
{zeidler:nfa88IIa})
gives
$$= \frac{1}{2} \int_{\Omega}
\sum_{i,j} a_{ij} \left[
\frac{\partial u}{\partial x_j}
\frac{\partial^2 u}{\partial x_k\partial x_i} +
\frac{\partial u}{\partial x_i}
\frac{\partial^2 u}{\partial x_k\partial x_j}
\right] x_k \, dx
$$
$$
+
\frac{1}{2} \int_{\Omega}
\sum_{i,j} a_{ij} \left[
\frac{\partial u}{\partial x_j} \delta_{ik} +
\frac{\partial u}{\partial x_i} \delta_{jk}
\right] \frac{\partial u}{\partial x_k} \, dx
$$
$$- \frac{1}{2} \int_{\partial\Omega}
\sum_{i,j} a_{ij} \left[
\frac{\partial u}{\partial x_j} \nu_{i} +
\frac{\partial u}{\partial x_i} \nu_{j}
\right] \frac{\partial u}{\partial x_k} x_k \, dx
$$
= $I_1 + I_2 + I_3$, where the unit normal vector is $\nu$.
One may rewrite $I_1$ as
$$I_1 = \frac{1}{2} \int_{\Omega}
\sum_{i,j} a_{ij} \frac{\partial}{\partial x_k}\left(
\frac{\partial u}{\partial x_i}
\frac{\partial u}{\partial x_j}
\right) x_k \, dx
$$
Integrating the first term by parts again yields
$$I_1 = -\frac{1}{2} \int_{\Omega}
\sum_{i,j} a_{ij} \left(
\frac{\partial u}{\partial x_i}
\frac{\partial u}{\partial x_j}
\right) \, dx
+
\frac{1}{2} \int_{\partial\Omega}
\sum_{i,j} a_{ij} \left(
\frac{\partial u}{\partial x_i}
\frac{\partial u}{\partial x_j}
\right) x_k \nu_k \, dS
$$
$$
-
\frac{1}{2} \int_{\Omega}
\sum_{i,j} \left(
\frac{\partial u}{\partial x_i}
\frac{\partial u}{\partial x_j}
\right) x_k \frac{\partial a_{ij}}{\partial x_k}\, dx.
$$
Summing over $k$ gives
$$\int_{\Omega}(-Lu)(\nabla u\cdot x)\, dx =
-\frac{n}{2} \int_{\Omega}
\sum_{i,j} a_{ij} \left(
\frac{\partial u}{\partial x_i}
\frac{\partial u}{\partial x_j}
\right) \, dx
$$
$$
+
\frac{1}{2} \int_{\partial\Omega}
\sum_{i,j} a_{ij} \left(
\frac{\partial u}{\partial x_i}
\frac{\partial u}{\partial x_j}
\right) (x\cdot \nu ) \, dS
-
\frac{1}{2} \int_{\Omega}
\sum_{i,j} \left(
\frac{\partial u}{\partial x_i}
\frac{\partial u}{\partial x_j}
\right) (x\cdot  \nabla a_{ij}) \, dx
$$
$$
+
\frac{1}{2} \int_{\Omega}
\sum_{i,j,k} a_{ij} \left[
\frac{\partial u}{\partial x_j}
\frac{\partial u}{\partial x_k} \delta_{ik} +
\frac{\partial u}{\partial x_i}
\frac{\partial u}{\partial x_k} \delta_{jk}
\right] \, dx
$$
$$- \frac{1}{2} \int_{\partial\Omega}
\sum_{i,j} a_{ij} \left[
\frac{\partial u}{\partial x_j} \nu_{i} +
\frac{\partial u}{\partial x_i} \nu_{j}
\right] (\nabla u\cdot x) \, dS.
$$
Combining the first and fourth term on the right-hand side
simplifies the expression
$$\int_{\Omega}(-Lu)(\nabla u\cdot x)\, dx
=
(1-\frac{n}{2}) \int_{\Omega}
\sum_{i,j} a_{ij} \left(
\frac{\partial u}{\partial x_i}
\frac{\partial u}{\partial x_j}
\right) \, dx
$$
$$
+
\frac{1}{2} \int_{\partial\Omega}
\sum_{i,j} a_{ij} \left(
\frac{\partial u}{\partial x_i}
\frac{\partial u}{\partial x_j}
\right) (x\cdot \nu ) \, dS
-
\frac{1}{2} \int_{\Omega}
\sum_{i,j} \left(
\frac{\partial u}{\partial x_i}
\frac{\partial u}{\partial x_j}
\right) (x\cdot  \nabla a_{ij}) \, dx
$$
$$
-
\frac{1}{2} \int_{\partial\Omega}
\sum_{i,j} a_{ij} \left[
\frac{\partial u}{\partial x_j} \nu_{i} +
\frac{\partial u}{\partial x_i} \nu_{j}
\right] (\nabla u\cdot x) \, dS.
$$
Using the notation defined above, the result follows.


                  % Complex Equations from the UW Math Department
%% acrobat.tex
%
% This file explains how to generate Adobe Acrobat files
%
% Eric Benedict, July 2000
%
% It is provided without warranty on an AS IS basis.

\newcommand{\pdf}{\mbox{\tt *.pdf}}

\chapter{Adobe Acrobat (\pdf ) Files}
The Adobe Acrobat file format has pretty much become the {\em de facto}
standard for document sharing.  As such, some faculty members and/or
departments may be requiring a final copy of the thesis in Acrobat format
(\pdf ).

There are several different methods of obtaining a \pdf\ file from a \LaTeX{}
thesis; however, they are all very site specific.  A couple of different
methods which have been found to work are mentioned as suggested ideas to try
as a starting point.  Depending on what is installed at your site/location some
of these may be applicable.

\section{Converting from {\tt *.ps} to \pdf}
One option to obtain the \pdf\ file would be to generate the thesis in a normal
manner and then use the Acrobat\ {\tt Distiller}\ to convert the postscript file into a
\pdf\ file.

If the\ {\tt Distiller}\ program is available and convenient to use, then this is quite easy to do.

Depending on the choice of document fonts, the results may not be satisfactory since
some of the fonts may end up as bit-mapped fonts and will display poorly at any resolution
other than what they were sampled on.  Also, since the\ {\tt Distiller}\ program is an expensive
program to obtain, it is not always available.

An alternative to the Adobe\ {\tt Distiller}\ program is the Alladin\ {\tt Ghostscript}\ program.  This is
available for free from

{\tt \verb|   http://www.cs.wisc.edu/~ghost/index.html|}

This program is available for most common operating systems as a compiled binary, but the source code
is available for other systems.  One drawback is that this conversion must be performed as a
command line invocation and isn't very user friendly.  This may be addressed in a future version of\
{\tt Ghostview}, the program which provides a nice user interface to\ {\tt Ghostscript}.

\section{Converting from {\tt *.dvi} to \pdf}
There are two programs available which will convert from {\tt *.dvi} to \pdf,\ {\tt dvipdf}\ and\
{\tt dvipdfm}.  The\ {\tt dvipdfm}\ program  will be discussed here.  In version 0.12, it can generate
bookmarks, thumbnails (with assistance from\ {\tt Ghostscript}), scaling and rotation, JPEG and
PNG bitmaps and font encoding and re-encoding (to support fonts which aren't fully supported by
the Acrobat suite).  When\ {\tt Ghostscript}\ is properly installed,\ {\tt dvipdfm}\ will automatically
convert any encapsulated PostScript figures into the required \pdf{} format.  This program behaved in a
similar manner to the {\tt dvips} program and was used to produce the \pdf{} format of this document.



\section{Generating \pdf{} Initially}
There are now some programs which are similar to \TeX{} but instead of producing a\ {\tt .dvi}\ output,
they produce \pdf as a native output.  One such program, {\sc pdf}\TeX{} / {\sc pdf}\LaTeX{},
is available from

{\tt \verb|   http://www.tug.org/applications/pdftex|}

Note that as of this date, July 2000, {\sc pdf}\TeX{} / {\sc pdf}\LaTeX{} while currently quite usable, it
is still in a beta version.  Look at the web site for more current information.

The present version was able to produce a \pdf{} file of this document without any required
changes, except for the Postscript figure inclusion  (Figure~\ref{vwcontr}).  To properly include
this figure, requires the conversion of the postscript figure into a \pdf{} figure.  The procedure
is described in the manual for {\sc pdf}\TeX{} / {\sc pdf}\LaTeX{}.  Note that the figure conversion will
require either\ {\tt Distiller}\ or\ {\tt Ghostscript}.
                  % A discussion on generating PDF files.
%\end{appendices}                % End of the Appendix Chapters.  ibid on \end{appendix}
%\include{vita}                  % Optional Vita, use \begin{vita} vita text \end{vita}
\end{document}
