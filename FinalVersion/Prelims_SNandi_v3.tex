%=====================================================================
% Document Style
%=====================================================================
% Choose only one of the following document classes:
%
% for a 12 Point UW PhD Thesis without Margin Check
\documentclass[12pt]{article}
%
%%%%%%%%%%%%%%%%%%%%%%%%%%%%%%%%%%%%%%%%%%%%%%%%%%%%%%%%%%%%%%%%%%%%%%
%% Usepackages
%%%%%%%%%%%%%%%%%%%%%%%%%%%%%%%%%%%%%%%%%%%%%%%%%%%%%%%%%%%%%%%%%%%%%%
\input{HeaderfileTexDocs}

%%%%%%%%%%%%%%%%%%% To change the margins and stuff %%%%%%%%%%%%%%%%%%%
\geometry{left=1.0in, right=1.0in, top=1.0in, bottom=1.0in}
%\setlength{\voffset}{0.5in}
%\setlength{\hoffset}{-0.4in}
%\setlength{\textwidth}{7.6in}
%\setlength{\textheight}{10in}
%%%%%%%%%%%%%%%%%%% To change the margins and stuff %%%%%%%%%%%%%%%%%%%

\begin{document}

\title{Statistical methods for analyzing images of fluorochrome stained DNA molecules}
\author{Subhrangshu Nandi}
\date{Mar 1, 2016}

\maketitle

\begin{abstract}
One of the most important components of precision medicine is analysis and use of an individual's genomic information and development of targeted intervention, tailored to the individual. This will entail building systems that will allow researchers and practitioners to extract important information from a genome, in a parsimonious manner. Next generation sequencing (NGS) systems are enhancing molecular biology in the right direction, but show little promise towards sequencing every human being's genome. In addition, NGS still have difficulty inferring repetitive structures and duplications \cite{Lander_etal_2001_Nature}, further complicated by gaps and errors in the reference genome. Pioneered by laboratory of computational genomics (LMCG) at UW-Madison, {\emph{Fluoroscanning}} is a single-molecule-based whole genome analysis system that can glean extract important information out of a genome in a swift, economical manner. The aims of this project are developing statistical methods for fluoroscanning - analysis of intensity profiles measured from imaged fluorochrome stained DNA molecules. Aim 1 is to establish that fluorescence intensity profiles measured at different regions have discernible differences if the underlying sequence compositions are distinct. Aim 2 is to detect two sets of fluorescence intensity profiles in heterozygous genomic loci. These analyses will be performed on experiments conducted at LMCG, on both ${\emph{M. florum}}$ and a human genome. 
\end{abstract}

% Choose your bibliography style
% plain is the basic style, others include ieeetr, siam, asm, etc
\bibliographystyle{apalike}

\newpage
%\include{prelude}                % Title page, abstract, table of contents, etc
\section{Introduction}

\subsection{Motivation}
The Human Genome Project (HGP), completed in 2003, is considered one of the greatest accomplishments of exploration in history of science. Since then thousands of genomes have been sequenced. However, no individual human genome has been annotated to completion. DNA sequencing-based genomic analysis continue to evolve, but their abilities to detect large scale structural variations, or heterozygosity in diploid genomes, remain limited. Next generation sequencing (NGS) is considerable more cost effective, with longer reads, but still have difficulty inferring repetitive structures and duplications \cite{Lander_etal_2001_Nature}, further complicated by gaps and errors in the reference genome. NGS also has inferior sensitivity of detecting heterozygotes \cite{Wheeler_etal_2008_Nature}. In addition, the sheer size of the diploid human genome (6 gigabases) presents multiple challenges of just using NGS technologies to analyze its complexities. The analysis of cancer genomes is made even more complex by the accumulation of large and small-scale structural variations (SVs), and genotype heterogeneity fostered by on-going mutagenesis processes, especially apparent in in solid tumor. The sequencing technologies currently being used were developed  primarily for characterization of single genes, not entire genomes and, as such, are not ideal to analyze polygenic diseases, complex trait inheritance, and population-based molecular genetics \cite{Samad_etal_1995_GenomeResearch}. Given the current need for comprehensively analyzed human and cancer genomes that are readily created, contemporary sequencing and mapping approaches are insufficient to meet these challenges. In order to achieve our dreams of improving healthcare by precision genomics we need techniques that can overcome the shortcomings of NGS, yet, maintaining economic viability. The answer lies in the latest developments of single molecule genome mapping techniques like optical mapping and nanocoding. 

\subsection{Background}
\subsubsection*{Optical Mapping}
Pioneered by LMCG, single molecule genome mapping techniques like optical mapping \cite{Schwartz_etal_1993_Science}, \cite{Dimalanta_etal_2004_AnalChem}, and nanocoding \cite{Jo_etal_2007_PNAS} have changed the landscape of whole genome analysis. Optical Mapping is a novel platform for analyzing genomes: it uses measurements of single DNA molecules to infer a high-resolution genome-wide restriction map, whose representation of genome structure complements genome sequence to yield biological insight. Briefly, DNA from thousands of cells in solution is randomly sheared to produce pieces that are around 500 Kb long. The solution is then passed through a micro-channel, where the DNA molecules are stretched and then attached to a positively charged glass support. A restriction enzyme is then applied, cleaving the DNA at corresponding restriction sites. The DNA molecules remain attached to the surface. The surface is photographed under a microscope after being stained with a fluorochrome. The cleavage sites show up in the image as tiny gaps in the fluorescent line of the molecule, giving an snapshot of the full restriction map. Even though these molecules are large by many standards, they may still represent only a small fraction of the chromosome they come from. Naturally, the amount of information in an optical map data set is related to the size of the underlying genome. 

Information about genomic variation can thus be obtained from these restriction maps, that do not record the full nucleotide sequence. A physical map is a listing of the locations along the genome where certain markers occur. A restriction map is a physical map induced by restriction enzymes. The ordered sequence of distances in base pairs between successive marker positions summarizes the genome sequence and can be viewed as a sort of bar code of the genome. Genomic differences can affect the presence or absence of markers, the distances between them and their orientation, inducing analogous changes in the bar code. Having a reference copy of the human genome allows us to perform such {\emph{in silico}} experiments. The availability of in silico reference maps can be extremely helpful.

\subsubsection*{Nanocoding}
Nanocoding \cite{Jo_etal_2007_PNAS}, \cite{Jo_etal_2009}, also invented in LMCG, is a more advanced DNA barcoding technique, to obtain genome-wide restriction maps. In nanocoding, the restriction enzyme used is Nb.BbvCI, which cleaves the sequence GC\^\ TGAGG on single strands of double-stranded DNA molecules. The cleaved sites are made detectable by nick translation using fluorochrome-labelled nucleotides. The cleaved sites are called punctates. The biggest advantage over optical mapping is being able to retain long DNA molecules intact and hence improving labeling efficiency. DNA molecules are stained with an intercalating dye YOYO-1. It is a green fluorescent dye which belongs to the family of monomethine cyanine dyes and is a tetracationic homodimer of Oxazole Yellow (abbreviated YO, hence the name YOYO). The labelled molecules are imaged by microscopes equipped with two CCD cameras, that have two filtering optics for green and red colors. The green channel acquires images of DNA backbone stained with YOYO-1, and the red channel acquires images of sequence-specific decorations of Alexa Fluor 547 punctuates via fluorescence resonance energy transfer (7). The restriction maps are obtained by meticulous image processing software that analyze the images from the green and red channels, to measure the lengths between two consecutive punctates.  \\

\subsubsection*{Dye - DNA interaction} 
The dye that is used to stain the DNA molecules is a tetracationic homodimer of Oxazole Yellow (abbreviated YO, hence the name YOYO). It is a green fluorescent dye which belongs to the family of monomethine cyanine dyes. Cyanine dyes have a rich history in the photographic industry. Cationic  cyanine dyes  exhibit very large degrees of fluorescence enhancement on binding to nucleic acids \cite{Rye_etal_1992_NAR}, \cite{Lee_etal_1986_Cytometry}. These  characteristics of fluorescence enhancement and  high binding affinity are crucial for nanocoding. Netzel, et al, 1995 \cite{Netzel_etal_1995_JPC} observe that there are differences in emission quantum yield between dyes with pyridinium and quinolinium structural components, such as YOYO-1, when bound to $(\text{dAdT})_{10}$ and $(\text{dGdC})_6$ duplexes. In fact, they observe a 2-fold quantum yield increase when switching from AT-rich regions to GC-rich regions. Larsson, et al, 1994, (\cite{Larsson_etal_1994_JACS}) also observe that the fluorescence intensity of YO depends on the base sequence. In fact, their observation suggests that the quantum yield and fluorescence lifetime for YO complexed with GC-rich DNA sequences are about twice as large as for YO complexed with AT-rich sequences. Netzel, et al, 1995 (10) also note that one of the reasons that could differentially alter emission enhancements for cyanine dyes bound to DNA  duplexes is excited-state electron transfer quenching by DNA nucleotides. They argue that Guanosine is the easiest nucleotide to oxidize, while thymidine and cytidine are the easiest nucleotides to reduce. Adenosine is difficult to oxidize and is also the nucleotide least easy to reduce. These chemical properties could be responsible for higher fluorescence intensity observed from GC-rich regions compared to AT-rich ones. 

\subsection{Discovery of a new effect - Fluoroscanning}
In nanocoding, the microscope, equipped with CCD cameras, capture the images of fluorescent DNA molecules, dyed with YOYO-1. From the physical and chemical properties of the differential interaction between the dye molecules and the nucleotides mentioned above, it is clear that the fluorescence intensity profiles of the DNA molecules will be strongly dependent on the underlying nucleotide sequence composition. In fact, DNA molecules from the same region on the genome should have similar fluorescence intensity profiles. In addition, different regions on the genome, with discernible differences between their sequence compositions, should exhibit discernible difference in their corresponding fluorescence intensity profiles. This novel technique of inferring about genomic sequence compositions from analyzing fluorescent intensity profiles of imaged DNA molecules is termed {\bf{Fluoroscanning}}. \\

This dissertation addresses the statistical and computational challenges associated with this newly discovered effect, fluoroscanning. 


              % Chapter 1
\section{Data Structure} \label{Ch2}

\subsection{Fluorescence Intensity Profiles}
Nanocoding \cite{Jo_etal_2007_PNAS}, \cite{Jo_etal_2009}, invented in LMCG, is a more advanced DNA barcoding technique, to obtain genome-wide restriction maps. In nanocoding, molecules are labeled with the restriction enzyme Nb.BbvCI, which cleaves the sequence GC\^\ TGAGG on single strands of double-stranded DNA molecules. The cleaved sites are made detectable by nick translation using fluorochrome-labeled nucleotides. The cleaved sites are called punctuates. The labeling is done in test tubes and then presented on nanoscale tubes (fig \ref{fig:Fig2_NanoSlit}) for imaging and subsequent measurements. 
%\begin{wrapfigure}{l}{0.5\textwidth}
%\includegraphics[width=0.9\linewidth]{Images/Image_Nanoslit.jpg} 
\begin{figure}[H]
\begin{center}
\includegraphics[scale = 0.3]{Images/Image_Nanoslit.pdf}
\end{center}
\caption{Nanoslits where DNA molecules are presented}
\label{fig:Fig2_NanoSlit}
\end{figure}
The negatively charged DNA molecules are deposited onto positively charged glass surface. The surface with DNA are then stained with YOYO-I dye, so they can be imaged by fluorescence microscopy. An integrated image acquisition and machine vision system was developed in LMCG to automatically detect and analyze molecule within the collected image data. Details of the image processing steps can be found in \cite{Dimalanta_etal_2004_AnalChem}, \cite{Jo_etal_2007_PNAS}, \cite{Ravindran_Gupta_2015_GigaScience}. The image processing softwares scans through images of surfaces presented with DNA molecules, as shown below in fig \ref{fig:Fig2_FrameImage}. It shows images of multiple straight DNA molecules which are stained with the YOYO-I dye which fluoresce when excited by the laser-illuminated microscope.  
\begin{figure}[H]
\begin{center}
%\includegraphics[width=0.55\textwidth, bb=0 0 700 350]{Images/Image4_FrameImage.png}
\includegraphics[scale = 0.45]{Images/Image4_FrameImage.pdf}
\end{center}
\caption{Image of a surface with stained DNA molecules}
\label{fig:Fig2_FrameImage}
\end{figure}
The restriction maps (named Nmaps, for ``nanocoded'' maps) are obtained by meticulous image processing software that analyze the images, to measure the lengths between two consecutive punctuates. 
\begin{figure}[H]
\begin{center}
\includegraphics[scale=1]{Images/Image5_NMap.png}
\end{center}
\caption{5 DNA molecules aligned to reference}
\label{fig:Fig2_NMap}
\end{figure}

These Nmaps are used to align the molecules to an {\emph{in silicon}} reference map of a reference genome. Fig \ref{fig:Fig2_NMap} shows how five DNA molecules are nanocoded and aligned to a reference genome. Notice that intervals 5 and 6 (from the left) of the reference genome have a depth of 5, i.e., 5 molecular fragments are aligned to that interval. However, interval 4 has a depth of 4. \\

The YOYO-I dye, used to stain the DNA molecules. The dyes exhibit very large degrees of fluorescence enhancement on binding to nucleic acids \cite{Rye_etal_1992_NAR}, \cite{Lee_etal_1986_Cytometry}. These characteristics of fluorescence enhancement and high binding affinity are crucial for nanocoding. Netzel, et al, 1995 \cite{Netzel_etal_1995_JPC} observed a 2-fold quantum yield increase when switching from AT-rich regions to GC-rich regions. Larsson, et al, 1994, (\cite{Larsson_etal_1994_JACS}) also observed that the fluorescence intensity of YO depends on the base sequence. This is fundamental to fluoroscanning.  

\subsection{Data description} \label{Ch2_data}
Throughout this thesis, we will use images collected during nanocoding {\emph{Mesoplasma florum (M. florum)}} genome and a multiple myeloma (MM) genome. Below are brief data descriptions of the two datasets.
\subsubsection{\emph{Mesoplasma florum (M. florum)}}
{\emph{M. florums}} are members of the class Mollicutes, a large group of bacteria that lack a cell wall and have a characteristically low GC content (\cite{Razin_etal_1998_MMBR}). These diverse organisms are parasites in a wide range of hosts, including humans, animals, insects, plants, and cells grown in tissue culture (\cite{Razin_etal_1998_MMBR}). Aside from their role as potential pathogens, {\emph{M. florums}} are of interest because of their extremely small genome size. The {\emph{M. florum}} has only one chromosome, and its NMap consists of 39 intervals. This implies, the reference genome has 38 restriction sites. The whole genome is approximately 793kb long. The intervals are between 2.111kb and 81.621kb. One of the goals of fluoroscanning it to analyze how similar the fluorescence intensity profiles of segments of different molecules aligned to the same location on a genome are. 
\begin{figure}[H]
\begin{center}
\includegraphics[scale=0.42,page=1]{Plots/MF_Frag15.pdf}
\includegraphics[scale=0.42,page=2]{Plots/MF_Frag15.pdf}
\end{center}
\caption{NMaps aligned to {\emph{M. florum}} Interval 15}
\label{fig:Fig2_MF_Frag15}
\end{figure}

Fig \ref{fig:Fig2_MF_Frag15} shows fluorescence intensity profiles of fragments of 12 DNA molecules that have been aligned to interval 15 of {\emph{M. florum}} genome. The interval is 11.119kb long and each pixel of the captured images correspond to around 209 base pairs on the genome. So, under perfect experimental conditions, each of these fragments should be 53 ($\frac{11119}{209} = 53.2$) pixels long. However, due to several reasons the molecule lengths do not perfectly align with that of the reference. As the dye molecules intercalate between the bases of the DNA molecules, there could be local deformation of the molecules, resulting in more dye molecules to seep in. This could result in slight elongation of the DNA molecule and hence it captures more pixels in the image. This phenomenon of ``stretching'' of molecules is hard to control experimentally, as it involves meticulous measurement of dye and DNA molecule concentrations. Furthermore, when one end of a negatively charged DNA molecule attaches to the positively charged glass surface and the rest of the molecule elongates in the direction of the analyte flow, it also causes differential stretching. 

In the {\emph{M. florum}} dataset, we have molecular fragments aligned to 39 intervals on the reference genome. The lengths of these intervals vary from 2.111kb - 81.620kb, with the average lengths of these intervals being 20.34kb. 
% latex table generated in R 3.2.2 by xtable 1.8-0 package
% Tue Feb 16 15:11:00 2016
\begin{table}[H]
\centering
%\begin{tabular}{l*{7}{c}}
\begin{tabular}{lrrr|rrr}
  \hline
  \hline
  \multicolumn{4}{c}{Reference Interval} & \multicolumn{3}{c}{Molecular Fragment lengths} \\
  \hline
   Int  & molecules & pixels & length(kb) & min (kb) & avg (kb) & max (kb)\\ 
  \hline
  \hline
    0 &  66 & 391 & 81.62 & 65.67 & 81.07 & 92.79 \\ 
    1 & 208 & 89 & 18.68 & 13.27 & 18.64 & 21.55 \\ 
    2 & 467 & 284 & 59.40 & 43.92 & 59.24 & 69.39 \\ 
    3 & 734 & 67 & 13.94 & 9.59 & 13.86 & 17.34 \\ 
    4 & 895 & 43 & 9.03 & 6.47 & 8.99 & 11.48 \\ 
    5 & 849 & 24 & 5.04 & 2.14 & 5.02 & 5.90 \\ 
    6 & 939 & 59 & 12.34 & 6.58 & 12.29 & 15.55 \\ 
    7 & 1200 & 49 & 10.24 & 6.74 & 10.20 & 12.22 \\ 
  \hline
  \hline
\end{tabular}
\caption{Coverage of {\emph{M.florum}} data}
\label{tab:mftable}
\end{table}

Table \ref{tab:mftable} lists the coverage of the {\emph{M.florum}} data from interval 0 - 7. The rest of the intervals are in \ref{tab:App_mftable}. For example, interval 7 has 1200 molecular fragments aligned to it. Ideally, all of them should be of length 10.24 kb, but in reality the lengths of these fragments lie between 6.74 kb and 12.22 kb, the average being 10.20 kb. 

\subsubsection{Human genome - multiple myeloma}
We have nanocoded data of a human genome from DNA samples were prepared from purified CD138 plasma cells (MM-S and MM-R sample) and paired cultured stromal cells (normal) from a 58-y-old male MM patient with International Staging System (ISS) Stage IIIb disease. Multiple myeloma is the malignancy of B lymphocytes that terminally differentiate into long-lived, antibody-producing plasma cells. Although it is a cancer genome, substantial portions of it are still identical to the reference human genome. This genome has been comprehensively analyzed to characterize its genome structure and variation by integrating findings from optical mapping with those from DNA sequencing-based genomic analysis \cite{Gupta_etal_2015_PNAS}. While the {\emph{M. florum}} genome only had 39 intervals, the table below (table \ref{tab:mm52intervals})lists the number of intervals each chromosome of the human genome contains.

\begin{table}[H]
\centering
\begin{tabular}{c | r || c | r}
  \hline
  \hline
  Chromosome & Number of intervals & Chromosome & Number of intervals \\ 
  \hline
  1 & 26069  & 13 & 9916 \\
  2 & 26772  & 14 & 9764 \\
  3 & 21334  & 15 & 9701 \\
  4 & 19406  & 16 & 9106 \\
  5 & 19359  & 17 & 9291 \\
  6 & 18504  & 18 & 8306 \\
  7 & 16797  & 19 & 5763 \\
  8 & 15828  & 20 & 7311 \\
  9 & 13553  & 21 & 3900 \\
  10 & 15053  & 22 & 4326 \\
  11 & 15212  & X & 15953 \\
  12 & 14371  & Y & 2582 \\
  \hline
  \hline
\end{tabular}
\caption{Number of intervals in each human chromosome}
\label{tab:mm52intervals}
\end{table}
Although there are many more intervals in the MM dataset, the coverage is not as deep as that of the {\emph{M. florum}} dataset. Below are some plots of the coverage of 4 of the chromosomes
\begin{figure}[H]
\begin{center}
\includegraphics[scale = 0.46, page = 1]{Plots/mm52_Coverage.pdf}
\includegraphics[scale = 0.46, page = 6]{Plots/mm52_Coverage.pdf}
\end{center}
\caption{Coverage of intervals (at least 6 kb long)}
\label{fig:Coverage_mm1}
\end{figure}

\begin{figure}[H]
\begin{center}
\includegraphics[scale = 0.46, page = 10]{Plots/mm52_Coverage.pdf}
\includegraphics[scale = 0.46, page = 13]{Plots/mm52_Coverage.pdf}
\end{center}
\caption{Coverage of intervals (at least 6 kb long)}
\label{fig:Coverage_mm2}
\end{figure}

Notice in fig \ref{fig:Coverage_mm1} that in chromosome 1, out of 26,069 intervals, only 12,701 are at least 6kb long, and on an average have a coverage of 30 molecules per interval.

\section{Research questions}
{\bf{Uniqueness}}: The features of the fluorescence intensity profiles of imaged DNA molecules are believed to be influenced by the nucleotide composition of the genome. Hence, molecules aligned to the same location of the genome should exhibit ``similar'' fluorescence intensity profiles. And, consequently, molecules aligned to different regions of the genome, with distinct sequence composition should exhibit discernible fluorescence intensity profiles. This will have significant impact on the future of whole genome analysis using single molecule techniques. For example, the nanocoding system could give way to the fluoroscanning system, by obviating the punctuates chemistry and hence improving the throughput. Using nanocoding and fluoroscanning information, a whole genome could be represented by its fluorescence intensity profile. This is similar to representing a melody in mp3 version. It will not have the exact nucleotide sequence, but will contain enough important information to analyze whole genomes in an extremely economically and swiftly. This technique promises to be the future of precision genomics, if every human being's genome is to be analyzed multiple times over the course of their lives. Just like the mp3 file format revolutionized the music industry, fluoroscanning will do the same to precision genomics. 

\begin{tcolorbox}[colback=green!5,colframe=green!40!black,title=Statistical challenge] %green!40!black=40%green and 60%black
The primary challenge is to analyze the fluorescence intensity profiles of different regions on both {\emph{M. florum}} and MM genomes. Since the molecular fragments are differentially stretched the features of their intensity profiles are misaligned. Before comparing and contrasting intensity profiles between different regions on the genomes, it is critical that these profiles are aligned to each other. 
\end{tcolorbox}

{\bf{Heterozygosity}}: A diploid organism (like human being) is ``heterozygous'' at a genomic locus when its cells contain two different alleles of that locus. Two different alleles would mean the same region on the genome will have two distint nucleotide sequence composition, and consequently (fluoroscanning) two distint sets of fluorescence intensity profiles. By nanocoding, the molecules can be aligned to these loci, and by fluoroscanning we would like to detect heterozygous regions on the MM genome. 

\begin{tcolorbox}[colback=green!5,colframe=green!40!black,title=Statistical challenge] %green!40!black=40%green and 60%black
This will entail simultaneously aligning the curve features and testing the presence of multiple clusters in the fluorescence intensity profiles of molecules aligned to the same location on a genome.
\end{tcolorbox}

{\bf{Quantify Intensity from Sequence Composition}}: We would like to quantify the relationship between sequence composition and fluorescence profiles.

\begin{tcolorbox}[colback=green!5,colframe=green!40!black,title=Statistical challenge] %green!40!black=40%green and 60%black
This will entail creating a probabilistic model between nucleotide sequence composition and the features of the fluorescence intensity profiles.
\end{tcolorbox}

%\section{Preliminary Analysis}
%\begin{tcolorbox}[colback=red!5,colframe=red!40!black,title=Work in progress] %green!40!black=40%green and 60%black
%I feel like I should revisit the ANOVA, and post-hoc tests that I had done with {\emph{M.florum}} data, and use the quality scores to eliminate outliers detected by analyzing the morphological features.
%\end{tcolorbox}


          % Chapter 2
\include{Sec3_Registration}       % Chapter 3
\section{Application to human genome}

We applied the iterated registration technique to fluoroscanning data from the human genome. From previous experiments conducted on this genome, we know that chromosome 13 of this genome is homozygous and we expect to find only one cluster of intensity of profiles. Below is a step-by-step application of the statistical techniques described in chapter 3, to interval 7491 of chromosome 13.
\begin{itemize}
\item There are 30 DNA molecular fragments that are aligned to this interval
\begin{figure}[H]
\begin{center}
\includegraphics[scale = 0.45, page = 2]{Plots/chr13_frag7491_registered.pdf}
\end{center}
\caption{30 fluorescence intensity profiles of DNA molecular fragments}
\label{fig:Frag7491_Orig}
\end{figure}

\item After normalizing the fluorescence intensity profiles by the median values of each fragment
\begin{figure}[H]
\begin{center}
\includegraphics[scale = 0.45, page = 3]{Plots/chr13_frag7491_registered.pdf}
\end{center}
\caption{30 normalized fluorescence intensity profiles of DNA molecular fragments}
\label{fig:Frag7491_Norm}
\end{figure}

\item After smoothing the normalized the fluorescence intensity profiles by B-spline and evaluating them at equidistant points
\begin{figure}[H]
\begin{center}
\includegraphics[scale = 0.45, page = 4]{Plots/chr13_frag7491_registered.pdf}
\end{center}
\caption{30 smoothed, normalized fluorescence intensity profiles of DNA molecular fragments}
\label{fig:Frag7491_Smooth}
\end{figure}

\item After iterated registration
\begin{figure}[H]
\begin{center}
\includegraphics[scale = 0.45, page = 6]{Plots/chr13_frag7491_registered.pdf}
\end{center}
\caption{Registered values of fluorescence intensity profiles}
\label{fig:Frag7491_Regist}
\end{figure}

\item Comparing final consensus intensity profile estimated by registration, with GCAT composition of the interval
\begin{figure}[H]
\begin{center}
\includegraphics[scale = 0.6, page = 7]{Plots/chr13_frag7491_registered.pdf}
\end{center}
\caption{Comparison of GCAT composition with consensus intensity profile of the interval}
\label{fig:Frag7491_Compare}
\end{figure}
\end{itemize}

\newpage
Below is the application of the methodology to interval 9774 of chromosome 13
\begin{figure}[H]
\begin{center}
\includegraphics[scale = 0.42, page = 2]{Plots/chr13_frag9774_registered.pdf}
\includegraphics[scale = 0.42, page = 3]{Plots/chr13_frag9774_registered.pdf} \\
\includegraphics[scale = 0.42, page = 6]{Plots/chr13_frag9774_registered.pdf} 
\includegraphics[scale = 0.42, page = 7]{Plots/chr13_frag9774_registered.pdf}
\end{center}
\caption{Fragment 9774, of human chromosome 13}
\label{fig:Frag9774_All}
\end{figure}

\subsection{Reproducible Fluorescence Intensity signals}
\begin{tcolorbox}[colback=green!5,colframe=green!40!black,title=Work in progress] %green!40!black=40%green and 60%black
The human genome consists of Ribosomal repeat regions which have the same nucleotide composition. Currently, we are working on acquiring the data to test if intensity profiles aligned to intervals of those repeat regions exhibit similar structure.
\end{tcolorbox}





        % Chapter 4
\section{Future work}

\subsection{Relate nucleotide sequence composition to the features of fluorescence intensity profiles}
We want to build a regression type model, relating features of fluorescence intensity profiles to nucleotide sequence composition. 

\subsection{Simultaneous clustering and registration}
We want to extend the iterated registration algorithm to simultaneously register and cluster a set of fluorescence intensity profiles. Then we want to test how much of a difference in sequence composition is translated to detectable differences in intensity profiles. 

\subsection{Prove that iterated registration increases power more than any other existing algorithm}
This is the first time a registration algorithm has been put through a rigorous procedure to test improvement in power. This simulation study is considerably more complex than the ones used in the literature. We want to compare the power improvement of all the existing registration techniques with the proposed iterated registration algorithm. 

\subsection{Relate relationship between warping function and sequence composition}
In addition to building a model between fluorescence intensity profiles and nucleotide sequence composition, we also want to fit the features of the warping functions with sequence composition. This will answer if certain sequences are more susceptible to warping than others, when presented on a glass surface.

            % Chapter 5
\section{Appendix}

% latex table generated in R 3.2.2 by xtable 1.8-0 package
% Tue Feb 16 15:11:00 2016
\begin{table}[t]
\centering
%\begin{tabular}{l*{7}{c}}
\begin{tabular}{lrrr|rrr}
  \hline
  \hline
  \multicolumn{4}{c}{Reference Interval} & \multicolumn{3}{c}{Molecular Fragment lengths} \\
  \hline
   Int  & molecules & pixels & length(kb) & min (kb) & avg (kb) & max (kb)\\ 
  \hline
  \hline
    0 &  66 & 391 & 81.62 & 65.67 & 81.07 & 92.79 \\ 
    1 & 208 & 89 & 18.68 & 13.27 & 18.64 & 21.55 \\ 
    2 & 467 & 284 & 59.40 & 43.92 & 59.24 & 69.39 \\ 
    3 & 734 & 67 & 13.94 & 9.59 & 13.86 & 17.34 \\ 
    4 & 895 & 43 & 9.03 & 6.47 & 8.99 & 11.48 \\ 
    5 & 849 & 24 & 5.04 & 2.14 & 5.02 & 5.90 \\ 
    6 & 939 & 59 & 12.34 & 6.58 & 12.29 & 15.55 \\ 
    7 & 1200 & 49 & 10.24 & 6.74 & 10.20 & 12.22 \\ 
    8 & 965 & 72 & 15.02 & 11.13 & 15.00 & 19.48 \\ 
    9 & 751 & 122 & 25.45 & 20.52 & 25.45 & 30.91 \\ 
   10 & 784 & 19 & 3.89 & 2.40 & 3.90 & 4.94 \\ 
   11 & 898 & 100 & 20.89 & 14.35 & 20.83 & 26.42 \\ 
   12 & 883 & 75 & 15.57 & 9.97 & 15.43 & 19.24 \\ 
   13 & 855 & 49 & 10.21 & 6.21 & 9.98 & 13.72 \\ 
   14 & 731 & 45 & 9.47 & 6.84 & 9.19 & 12.79 \\ 
   15 & 631 & 53 & 11.12 & 5.69 & 10.42 & 13.94 \\ 
   16 & 203 & 24 & 4.99 & 1.46 & 4.24 & 7.99 \\ 
   17 & 151 & 66 & 13.73 & 8.29 & 12.97 & 16.76 \\ 
   18 & 377 & 126 & 26.28 & 21.17 & 25.66 & 31.02 \\ 
   19 & 551 & 183 & 38.28 & 29.91 & 38.14 & 43.33 \\ 
   20 & 488 & 10 & 2.11 & 1.46 & 2.14 & 3.18 \\ 
   21 & 572 & 148 & 31.02 & 18.48 & 31.12 & 35.62 \\ 
   22 & 712 & 91 & 19.10 & 14.66 & 19.12 & 24.44 \\ 
   23 & 918 & 17 & 3.62 & 1.04 & 3.61 & 6.37 \\ 
   24 & 947 & 154 & 32.19 & 25.89 & 32.24 & 37.16 \\ 
   25 & 876 & 198 & 41.30 & 30.39 & 41.20 & 48.77 \\ 
   26 & 824 & 47 & 9.76 & 4.62 & 9.74 & 13.15 \\ 
   27 & 835 & 78 & 16.38 & 10.50 & 16.34 & 20.35 \\ 
   28 & 666 & 75 & 15.69 & 11.18 & 15.96 & 18.90 \\ 
   29 & 653 & 30 & 6.28 & 4.07 & 5.86 & 7.36 \\ 
   30 & 881 & 175 & 36.50 & 29.11 & 36.34 & 42.61 \\ 
   31 & 795 & 88 & 18.31 & 12.95 & 18.24 & 21.90 \\ 
   32 & 668 & 153 & 32.07 & 25.75 & 31.81 & 38.11 \\ 
   33 & 431 & 100 & 20.95 & 15.15 & 20.86 & 23.97 \\ 
   34 & 334 & 16 & 3.28 & 1.25 & 3.03 & 4.60 \\ 
   35 & 295 & 68 & 14.26 & 11.32 & 14.16 & 16.37 \\ 
   36 & 191 & 245 & 51.31 & 36.60 & 50.81 & 59.52 \\ 
   37 & 103 & 77 & 15.99 & 12.06 & 15.90 & 18.12 \\ 
   38 &  68 & 86 & 17.88 & 15.04 & 17.68 & 20.14 \\ 
  \hline
  \hline
\end{tabular}
\caption{Coverage of {\emph{M.florum}} data}
\label{tab:App_mftable}
\end{table}

           % Chapter 6

%\include{essentials}            % Edited ``Essential LaTeX'' by Jon Warbrick
%\include{figs}                  % Chapter 3 Edited from UW Math Dept's Sample Thesis
%\include{bibs}                  % Chapter 4 From PU Thesis styles, by J.D. McCauley
%\include{usage}                 % Chapter 5 Strongly based on similar by J.D. McCauley

%\bibliography{refs}             % Make the bibliography
\bibliography{bibTex_Reference}  % Make the bibliography

%\begin{appendices}              % Start of the Appendix Chapters.  If there is only
                                 % one Appendix Chapter, then use \begin{appendix}
%\include{code}                  % Including computer code listings
%\include{bibref}                % a BibTeX reference
%\include{math}                  % Complex Equations from the UW Math Department
%\include{acro}                  % A discussion on generating PDF files.
%\end{appendices}                % End of the Appendix Chapters.  ibid on \end{appendix}
%\include{vita}                  % Optional Vita, use \begin{vita} vita text \end{vita}
\end{document}
