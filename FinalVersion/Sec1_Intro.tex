\section{Introduction}

\subsection{Motivation}
An important component of precision medicine centers on obtaining and interpreting an individual's genomic background for the development of targeted interventions, tailored to that individual. Much of this effort will center on building large databases using critically developed approaches, both ``wet'' and computational that will allow researchers and clinicians to wield detailed, molecular insights into disease that are specifically tailored to the individual. Although contemporary sequencing approaches have greatly accelerated the pace of discovery in the biosciences, current approaches show little promise towards achieving comprehensive analysis of large, human populations. Accordingly, the issues surrounding ``universal genome analysis'' present challenges across a broad range of scientific disciplines and engineering domains.

Our own thinking about the problems of how to build genome analysis systems, appropriate for large human populations, are centered on parsimonious considerations that take into account the structure of genomes across human populations and the intrinsic advantages that single molecule approaches provide for the creation and interpretation of very large datasets.

Pioneered by the Laboratory of Computational Genomics (LMCG), Fluoroscanning is a single-molecule-based system conceived for whole genome analysis that will provide information that is complementary to DNA sequencing and in many applications, we believe it may supplant it. The basic principles of Fluoroscanning are simple: very long DNA molecules (as long as 1,000 microns) are stained with a fluorochrome (a dye molecule that fluoresces upon excitation with light) and imaged by fluorescence microscopy. Such molecules look like long strings, as shown in Fig () and the amount of
fluorescence (light given-off by the stained molecule) measured along its backbone varies. Our hypothesis is that this variation in measured fluorescence varies according to the DNA sequence imaged at each point along a DNA molecule. Because light microscopy has a spatial resolution of about 200 nm, we do not enjoy single base resolution. Instead, fluorescence measurements locally average about 600 bp of DNA, which are collectively based on the optical characteristics of the imaging system (objective, lenses, cameras, dyes, etc.) and how DNA molecules are presented for imaging. Lastly, a slew of automation approaches (workflows) enable creation of large datasets and analyses comprising individually analyzed DNA molecules.

The aims of my project are develop statistical methods for Fluoroscanning, predicated on the analysis of fluorescence intensity profiles of single DNA molecules that reside within large machine vision datasets.\\
\noindent
{\bf{Aim 1}} is to establish that fluorescence intensity profiles vary, in a measurable way, according to the composition of the underlying sequence. \\
\noindent
{\bf{Aim 2}} is to confidently infer detect two sets of fluorescence intensity profiles that will report the presence of genomic heterozygosity (two genotypes).

\subsection{Background}
\subsubsection*{Optical Mapping}
Pioneered by LMCG, single molecule genome analysis system like optical mapping \cite{Schwartz_etal_1993_Science}, \cite{Dimalanta_etal_2004_AnalChem}, \cite{Aston_etal_1999_Optical}, \cite{Teague_etal_2010_PNAS}, \cite{Yokota_etal_1997_NAR} and nanocoding \cite{Jo_etal_2007_PNAS}, \cite{Chen_etal_2005_Macromolecules} have changed the landscape of whole genome analysis. Optical Mapping is a novel platform for analyzing genomes: it uses measurements of single DNA molecules to infer a high-resolution genome-wide restriction map, whose representation of genome structure complements genome sequence to yield biological insight. Information about genomic variation can be obtained from these restriction maps, that do not record the full nucleotide sequence. A physical map is a listing of the locations along the genome where certain markers occur. A restriction map is a physical map induced by restriction enzymes. The ordered sequence of distances in base pairs between successive marker positions summarizes the genome sequence and can be viewed as a sort of bar code of the genome. Genomic differences can affect the presence or absence of markers, the distances between them and their orientation, inducing analogous changes in the bar code. Having a reference copy of the human genome allows us to perform such {\emph{in silico}} experiments. 

\subsubsection*{Nanocoding}
Nanocoding \cite{Jo_etal_2007_PNAS}, \cite{Jo_etal_2009}, also invented in LMCG, is a more advanced DNA barcoding technique, to obtain genome-wide restriction maps. In nanocoding, the restriction enzyme cleaves the single strands of double-stranded DNA molecules. The cleaved sites are made detectable by nick translation using fluorochrome-labelled nucleotides. The cleaved sites are called punctates. The biggest advantage over optical mapping is being able to retain long DNA molecules intact and hence improving labeling efficiency. DNA molecules are stained with an intercalating dye YOYO-1 (Oxazole Yellow - abbreviated YO, hence the name YOYO). The labelled molecules are imaged by microscopes equipped with two CCD cameras. The restriction maps are obtained by meticulous image processing software that analyze the images to measure the lengths between two consecutive punctates. 

\subsubsection*{Dye - DNA interaction} 
The dye that is used to stain the DNA molecules, YOYO, exhibit very large degrees of fluorescence enhancement on binding to nucleic acids \cite{Rye_etal_1992_NAR}, \cite{Lee_etal_1986_Cytometry}. These  characteristics of fluorescence enhancement and  high binding affinity are crucial for nanocoding. Netzel, et al, 1995 \cite{Netzel_etal_1995_JPC} observe a 2-fold quantum yield increase when switching from AT-rich regions to GC-rich regions. Larsson, et al, 1994, (\cite{Larsson_etal_1994_JACS}) also observe that the fluorescence intensity of YO depends on the base sequence. In fact, their observation suggests that the quantum yield and fluorescence lifetime for YO complexed with GC-rich DNA sequences are about twice as large as for YO complexed with AT-rich sequences. 

\subsection{Discovery of a new effect - Fluoroscanning}
In nanocoding, the microscope, equipped with CCD cameras, capture the images of fluorescent DNA molecules, dyed with YOYO-1. From the physical and chemical properties of the differential interaction between the dye molecules and the nucleotides mentioned above, it is clear that the fluorescence intensity profiles of the DNA molecules will be strongly dependent on the underlying nucleotide sequence composition. In fact, DNA molecules from the same region on the genome should have similar fluorescence intensity profiles. In addition, different regions on the genome, with discernible differences between their sequence compositions, should exhibit discernible difference in their corresponding fluorescence intensity profiles. This novel technique of inferring about genomic sequence compositions from analyzing fluorescent intensity profiles of imaged DNA molecules is termed {\bf{Fluoroscanning}}. \\

The aim of this project is to develop statistical methods for fluoroscanning data. 
