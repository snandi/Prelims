\section{Introduction}

\subsection{Motivation}
The Human Genome Project (HGP), completed in 2003, is considered one of the greatest accomplishments of exploration in history of science. Since then thousands of genomes have been sequenced. However, no individual human genome has been annotated to completion. DNA sequencing-based genomic analysis continue to evolve, but their abilities to detect large scale structural variations, or heterozygosity in diploid genomes, remain limited. Next generation sequencing (NGS) is considerable more cost effective, with longer reads, but still have difficulty inferring repetitive structures and duplications \cite{Lander_etal_2001_Nature}, further complicated by gaps and errors in the reference genome. NGS also has inferior sensitivity of detecting heterozygotes \cite{Wheeler_etal_2008_Nature}. In addition, the sheer size of the diploid human genome (6 gigabases) presents multiple challenges of just using NGS technologies to analyze its complexities. The analysis of cancer genomes is made even more complex by the accumulation of large and small-scale structural variations (SVs), and genotype heterogeneity fostered by on-going mutagenesis processes, especially apparent in in solid tumor. The sequencing technologies currently being used were developed  primarily for characterization of single genes, not entire genomes and, as such, are not ideal to analyze polygenic diseases, complex trait inheritance, and population-based molecular genetics \cite{Samad_etal_1995_GenomeResearch}. Given the current need for comprehensively analyzed human and cancer genomes that are readily created, contemporary sequencing and mapping approaches are insufficient to meet these challenges. In order to achieve our dreams of improving healthcare by precision genomics we need techniques that can overcome the shortcomings of NGS, yet, maintaining economic viability. The answer lies in the latest developments of single molecule genome mapping techniques like optical mapping and nanocoding. 

\subsection{Background}
\subsubsection*{Optical Mapping}
Pioneered by LMCG, single molecule genome mapping techniques like optical mapping \cite{Schwartz_etal_1993_Science}, \cite{Dimalanta_etal_2004_AnalChem}, and nanocoding \cite{Jo_etal_2007_PNAS} have changed the landscape of whole genome analysis. Optical Mapping is a novel platform for analyzing genomes: it uses measurements of single DNA molecules to infer a high-resolution genome-wide restriction map, whose representation of genome structure complements genome sequence to yield biological insight. Briefly, DNA from thousands of cells in solution is randomly sheared to produce pieces that are around 500 Kb long. The solution is then passed through a micro-channel, where the DNA molecules are stretched and then attached to a positively charged glass support. A restriction enzyme is then applied, cleaving the DNA at corresponding restriction sites. The DNA molecules remain attached to the surface. The surface is photographed under a microscope after being stained with a fluorochrome. The cleavage sites show up in the image as tiny gaps in the fluorescent line of the molecule, giving an snapshot of the full restriction map. Even though these molecules are large by many standards, they may still represent only a small fraction of the chromosome they come from. Naturally, the amount of information in an optical map data set is related to the size of the underlying genome. 

Information about genomic variation can thus be obtained from these restriction maps, that do not record the full nucleotide sequence. A physical map is a listing of the locations along the genome where certain markers occur. A restriction map is a physical map induced by restriction enzymes. The ordered sequence of distances in base pairs between successive marker positions summarizes the genome sequence and can be viewed as a sort of bar code of the genome. Genomic differences can affect the presence or absence of markers, the distances between them and their orientation, inducing analogous changes in the bar code. Having a reference copy of the human genome allows us to perform such {\emph{in silico}} experiments. The availability of in silico reference maps can be extremely helpful.

\subsubsection*{Nanocoding}
Nanocoding \cite{Jo_etal_2007_PNAS}, \cite{Jo_etal_2009}, also invented in LMCG, is a more advanced DNA barcoding technique, to obtain genome-wide restriction maps. In nanocoding, the restriction enzyme used is Nb.BbvCI, which cleaves the sequence GC\^\ TGAGG on single strands of double-stranded DNA molecules. The cleaved sites are made detectable by nick translation using fluorochrome-labelled nucleotides. The cleaved sites are called punctates. The biggest advantage over optical mapping is being able to retain long DNA molecules intact and hence improving labeling efficiency. DNA molecules are stained with an intercalating dye YOYO-1. It is a green fluorescent dye which belongs to the family of monomethine cyanine dyes and is a tetracationic homodimer of Oxazole Yellow (abbreviated YO, hence the name YOYO). The labelled molecules are imaged by microscopes equipped with two CCD cameras, that have two filtering optics for green and red colors. The green channel acquires images of DNA backbone stained with YOYO-1, and the red channel acquires images of sequence-specific decorations of Alexa Fluor 547 punctuates via fluorescence resonance energy transfer (7). The restriction maps are obtained by meticulous image processing software that analyze the images from the green and red channels, to measure the lengths between two consecutive punctates.  \\

\subsubsection*{Dye - DNA interaction} 
The dye that is used to stain the DNA molecules is a tetracationic homodimer of Oxazole Yellow (abbreviated YO, hence the name YOYO). It is a green fluorescent dye which belongs to the family of monomethine cyanine dyes. Cyanine dyes have a rich history in the photographic industry. Cationic  cyanine dyes  exhibit very large degrees of fluorescence enhancement on binding to nucleic acids \cite{Rye_etal_1992_NAR}, \cite{Lee_etal_1986_Cytometry}. These  characteristics of fluorescence enhancement and  high binding affinity are crucial for nanocoding. Netzel, et al, 1995 \cite{Netzel_etal_1995_JPC} observe that there are differences in emission quantum yield between dyes with pyridinium and quinolinium structural components, such as YOYO-1, when bound to $(\text{dAdT})_{10}$ and $(\text{dGdC})_6$ duplexes. In fact, they observe a 2-fold quantum yield increase when switching from AT-rich regions to GC-rich regions. Larsson, et al, 1994, (\cite{Larsson_etal_1994_JACS}) also observe that the fluorescence intensity of YO depends on the base sequence. In fact, their observation suggests that the quantum yield and fluorescence lifetime for YO complexed with GC-rich DNA sequences are about twice as large as for YO complexed with AT-rich sequences. Netzel, et al, 1995 (10) also note that one of the reasons that could differentially alter emission enhancements for cyanine dyes bound to DNA  duplexes is excited-state electron transfer quenching by DNA nucleotides. They argue that Guanosine is the easiest nucleotide to oxidize, while thymidine and cytidine are the easiest nucleotides to reduce. Adenosine is difficult to oxidize and is also the nucleotide least easy to reduce. These chemical properties could be responsible for higher fluorescence intensity observed from GC-rich regions compared to AT-rich ones. 

\subsection{Discovery of a new effect - Fluoroscanning}
In nanocoding, the microscope, equipped with CCD cameras, capture the images of fluorescent DNA molecules, dyed with YOYO-1. From the physical and chemical properties of the differential interaction between the dye molecules and the nucleotides mentioned above, it is clear that the fluorescence intensity profiles of the DNA molecules will be strongly dependent on the underlying nucleotide sequence composition. In fact, DNA molecules from the same region on the genome should have similar fluorescence intensity profiles. In addition, different regions on the genome, with discernible differences between their sequence compositions, should exhibit discernible difference in their corresponding fluorescence intensity profiles. This novel technique of inferring about genomic sequence compositions from analyzing fluorescent intensity profiles of imaged DNA molecules is termed {\bf{Fluoroscanning}}. \\

This dissertation addresses the statistical and computational challenges associated with this newly discovered effect, fluoroscanning. 


